\subsection{Ranking}
\label{subsec:experiments-ranking}

We conclude the experimental part of this paper with a ranking with respect to \ARec and \AUE
 -- reflecting the objective used for parameter optimization. Unfortunately, the high number
of algorithms as well as the low number of datasets prohibits using statistical
tests to extract rankings, as done in other benchmarks (\eg \cite{Demsar:2006,DollarWojekSchielePerona:2009}).
Therefore, Table \ref{table:ranking} presents average \ARec and \AUE, average ranks
as well as the corresponding rank matrix.
On each dataset, the algorithms were ranked according to $\ARec + \AUE$ where
lowest $\ARec + \AUE$ corresponds to the best rank, \ie rank one.
%The average
%rank is then computed by dividing through the number of datasets each algorithm was evaluated on.
The corresponding rank matrix represents the rank distribution
(\ie the frequencies of the attained ranks) for each algorithm.
We find that the presented average ranks provide a founded overview of the evaluated 
algorithms, summarizing many of the observations discussed before.
In the absence of additional constraints, Table \ref{table:ranking}
may be used to select suitable superpixel algorithms.