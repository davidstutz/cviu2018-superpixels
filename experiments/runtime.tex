\subsection{Runtime}
\label{subsec:experiments-runtime}

Considering runtime, we report CPU time\footnote{\label{foot:runtime}
    Runtimes have been taken on an Intel$^\text{\textregistered}$ Core\texttrademark\xspace i7-3770 @ $3.4GHz$, 64bit with 32GB RAM.
} excluding connected components but including color space conversions if applicable.
We made sure that no multi-threading or GPU computation were used. We begin by considering
runtime in general, with a glimpse on realtime applications, before considering iterative algorithms.

We find that some well performing algorithms can be run at (near) realtime. We
consider Figure \ref{fig:experiments-runtime}
showing runtime in seconds $t$ on the \BSDS (image size $481 \times 321$) and \NYU (image size $608 \times 448$) datasets.
Concretely, considering the watershed-based algorithms \Wr and \CWr, we can report runtimes below $10\text{ms}$
on both datasets, corresponding to roughly~$100\text{fps}$. Similarly, \PFr
runs at below $10\text{ms}$. Furthermore, several algorithms, such as
\SLICr, \ERGCr, \FHr, \PBr, \MSSr and \preSLICr provide runtimes below $80\text{ms}$
and some of them are iterative, \ie reducing the number of iterations may further
reduce runtime. However, using the convention that realtime corresponds to
roughly $30\text{fps}$, this leaves \preSLIC and \MSS on the larger images
of the \NYU dataset. However, even without explicit runtime requirements, we find runtimes
below $1\text{s}$ per image to be beneficial for using superpixel algorithms as
pre-processing tool, ruling out \TPS, \CIS, \SEAW, \RW and \NC.
Overall, several superpixel algorithms provide runtimes appropriate for pre-processing tools;
realtime applications are still limited to a few fast algorithms.

\begin{figure}[t]
    \centering
    \begin{subfigure}[b]{\halftwoone\textwidth}\phantomsubcaption\label{subfig:experiments-runtime-bsds500}
	\begin{tikzpicture}
		\begin{axis}[EQBSDS500t,ymode=log]
			% CCS %%%%%%%%%%%%%%%%%%%%%%%%%%%%%%%%%%%%%%%%%%%%%%%%%%%%%%%%%%%%
			\addplot[CCS] coordinates{
				(453.82,0.21183333333333)
				(1438.62,0.19863333333333)
				(3406.7,0.19163333333333)
			};

				% CIS %%%%%%%%%%%%%%%%%%%%%%%%%%%%%%%%%%%%%%%%%%%%%%%%%%%%%%%%%%%%
			\addplot[CIS] coordinates{
				(472.685,2.619925)
				(1150.5,3.08015)
				(3625.6,2.982875)
			};

				% CRS %%%%%%%%%%%%%%%%%%%%%%%%%%%%%%%%%%%%%%%%%%%%%%%%%%%%%%%%%%%%
			\addplot[CRS] coordinates{
				(635.865,0.436475)
				(1758.22,0.665825)
				(5208.23,1.062525)
			};

				% CW %%%%%%%%%%%%%%%%%%%%%%%%%%%%%%%%%%%%%%%%%%%%%%%%%%%%%%%%%%%%
			\addplot[CW] coordinates{
				(405.935,0.003624995)
				(1252.72,0.003549995)
				(4023.98,0.003275)
			};

				% EAMS %%%%%%%%%%%%%%%%%%%%%%%%%%%%%%%%%%%%%%%%%%%%%%%%%%%%%%%%%%%%
			\addplot[EAMS] coordinates{
				(309.87,0.1219465)
				(1417.35,0.121738)
				(3640.44,0.1721875)
			};

				% ERGC %%%%%%%%%%%%%%%%%%%%%%%%%%%%%%%%%%%%%%%%%%%%%%%%%%%%%%%%%%%%
			\addplot[ERGC] coordinates{
				(400,0.05225)
				(1224,0.053425)
				(3840,0.054875)
			};

				% ERS %%%%%%%%%%%%%%%%%%%%%%%%%%%%%%%%%%%%%%%%%%%%%%%%%%%%%%%%%%%%
			\addplot[ERS] coordinates{
				(400,0.311775)
				(1200,0.3269)
				(3600,0.335825)
			};

				% ETPS %%%%%%%%%%%%%%%%%%%%%%%%%%%%%%%%%%%%%%%%%%%%%%%%%%%%%%%%%%%%
			\addplot[ETPS] coordinates{
				(425,0.326775)
				(1320,0.396575)
				(3174,0.3977)
			};

				% FH %%%%%%%%%%%%%%%%%%%%%%%%%%%%%%%%%%%%%%%%%%%%%%%%%%%%%%%%%%%%
			\addplot[FH] coordinates{
				(782.42,0.03667495)
				%(799.09,0.034075)
				(3219.63,0.02397505)
			};

				% LSC %%%%%%%%%%%%%%%%%%%%%%%%%%%%%%%%%%%%%%%%%%%%%%%%%%%%%%%%%%%%
			\addplot[LSC] coordinates{
				(1045.31,0.073775)
				(2055.98,0.07615)
				(3762.88,0.07905)
			};

				% MSS %%%%%%%%%%%%%%%%%%%%%%%%%%%%%%%%%%%%%%%%%%%%%%%%%%%%%%%%%%%%
			\addplot[MSS] coordinates{
				(420.66,0.02530005)
				(1422.34,0.02640005)
				(3669.92,0.02822505)
			};

				% PB %%%%%%%%%%%%%%%%%%%%%%%%%%%%%%%%%%%%%%%%%%%%%%%%%%%%%%%%%%%%
			\addplot[PB] coordinates{
				(494.17,0.02645005)
				(1323.88,0.02657505)
				(3126.91,0.02667505)
			};

				% PF %%%%%%%%%%%%%%%%%%%%%%%%%%%%%%%%%%%%%%%%%%%%%%%%%%%%%%%%%%%%
			\addplot[PF] coordinates{
				(585.08,0.0035127610425)
				(1861.72,0.0036465155775)
				(22244.1,0.0040697705125)
			};

				% POISE %%%%%%%%%%%%%%%%%%%%%%%%%%%%%%%%%%%%%%%%%%%%%%%%%%%%%%%%%%%%
			\addplot[POISE] coordinates{
				(408.385,0.371509)
				(1216.26,0.3912085)
				(2287.5,0.3209995)
			};

				% PRESLIC %%%%%%%%%%%%%%%%%%%%%%%%%%%%%%%%%%%%%%%%%%%%%%%%%%%%%%%%%%%%
			\addplot[PRESLIC] coordinates{
				(369,0.01435)
				(1229.22,0.01415)
				(3059.43,0.01335)
			};

				% QS %%%%%%%%%%%%%%%%%%%%%%%%%%%%%%%%%%%%%%%%%%%%%%%%%%%%%%%%%%%%
			\addplot[QS] coordinates{
				(468.655,2.227445)
				(1111.34,0.756621)
				(3646.03,0.291654)
			};

				% RESEEDS %%%%%%%%%%%%%%%%%%%%%%%%%%%%%%%%%%%%%%%%%%%%%%%%%%%%%%%%%%%%
			\addplot[RESEEDS] coordinates{
				(401.465,0.03832495)
				(1201.34,0.03862495)
				(3840.22,0.03812495)
			};

				% RW %%%%%%%%%%%%%%%%%%%%%%%%%%%%%%%%%%%%%%%%%%%%%%%%%%%%%%%%%%%%
			\addplot[RW] coordinates{
				(407.125,8.278541)
				(1280.94,27.956049)
				(3882.85,68.198449)
			};

			% NC %%%%%%%%%%%%%%%%%%%%%%%%%%%%%%%%%%%%%%%%%%%%%%%%%%%%%%%%%%%%
			\addplot[NC] coordinates{
				(417.84,58.822008) % 7.352751
				(1025.88,75.109984) % 9.388748
				(2930.68,174.019264) % 21.752408
			};

				% SEAW %%%%%%%%%%%%%%%%%%%%%%%%%%%%%%%%%%%%%%%%%%%%%%%%%%%%%%%%%%%%
			\addplot[SEAW] coordinates{
				(356.805,0.3441125)
				(923.49,1.2333205)
				(2933.9,4.5641455)
			};

				% SEEDS %%%%%%%%%%%%%%%%%%%%%%%%%%%%%%%%%%%%%%%%%%%%%%%%%%%%%%%%%%%%
			\addplot[SEEDS] coordinates{
				(468.81,0.450675)
				(1270.11,0.45755)
				(3895.78,0.452425)
			};

				% SLIC %%%%%%%%%%%%%%%%%%%%%%%%%%%%%%%%%%%%%%%%%%%%%%%%%%%%%%%%%%%%
			\addplot[SLIC] coordinates{
				(368.57,0.072925)
				(1203.61,0.07685)
				(3038.13,0.078375)
			};

				% TP %%%%%%%%%%%%%%%%%%%%%%%%%%%%%%%%%%%%%%%%%%%%%%%%%%%%%%%%%%%%
			\addplot[TP] coordinates{
				(555.1,2.2396015)
				(1495.14,2.3026)
				(2831,2.282673)
			};

				% TPS %%%%%%%%%%%%%%%%%%%%%%%%%%%%%%%%%%%%%%%%%%%%%%%%%%%%%%%%%%%%
			\addplot[TPS] coordinates{
				(445.405,0.998402)
				(1332.49,1.1911175)
				(4030.29,1.718283)
			};

				% VC %%%%%%%%%%%%%%%%%%%%%%%%%%%%%%%%%%%%%%%%%%%%%%%%%%%%%%%%%%%%
			\addplot[VC] coordinates{
				(4787.61,0.72555)
				(6798.92,0.5635)
				(9919.31,0.7189)
			};

				% VLSLIC %%%%%%%%%%%%%%%%%%%%%%%%%%%%%%%%%%%%%%%%%%%%%%%%%%%%%%%%%%%%
			\addplot[VLSLIC] coordinates{
				(763.62,0.0588)
				(1348.08,0.05875)
				(2954.18,0.05895)
			};

				% W %%%%%%%%%%%%%%%%%%%%%%%%%%%%%%%%%%%%%%%%%%%%%%%%%%%%%%%%%%%%
			\addplot[W] coordinates{
				(387.92,0.003824995)
				(1302.83,0.004249995)
				(3292.41,0.004199995)
			};

				% WP %%%%%%%%%%%%%%%%%%%%%%%%%%%%%%%%%%%%%%%%%%%%%%%%%%%%%%%%%%%%
			\addplot[WP] coordinates{
				(384,0.12139303482587)
				(1320,0.15840796019901)
				(4320,0.25457711442786)
			};

			\end{axis}
	\end{tikzpicture}
\end{subfigure}
\begin{subfigure}[b]{\halftwoone\textwidth}\phantomsubcaption\label{subfig:experiments-runtime-nyuv2}
	\begin{tikzpicture}
		\begin{axis}[EQNYUV2t,ymode=log]
			% CCS %%%%%%%%%%%%%%%%%%%%%%%%%%%%%%%%%%%%%%%%%%%%%%%%%%%%%%%%%%%%
			\addplot[CCS] coordinates{
				(397.098,0.35497066666667)
				(1177.26,0.34076833333333)
				(3328.01,0.33)
			};

				% CIS %%%%%%%%%%%%%%%%%%%%%%%%%%%%%%%%%%%%%%%%%%%%%%%%%%%%%%%%%%%%
			\addplot[CIS] coordinates{
				(436.398,5.12625)
				(963.875,5.56715)
				(2717.81,5.353)
			};

				% CRS %%%%%%%%%%%%%%%%%%%%%%%%%%%%%%%%%%%%%%%%%%%%%%%%%%%%%%%%%%%%
			\addplot[CRS] coordinates{
				(514.895,0.55297)
				(1498.25,0.843745)
				(4055.02,1.279475)
			};

				% CW %%%%%%%%%%%%%%%%%%%%%%%%%%%%%%%%%%%%%%%%%%%%%%%%%%%%%%%%%%%%
			\addplot[CW] coordinates{
				(407.073,0.00654135)
				(1266,0.0065915)
				(3707.85,0.00652885)
			};

				% DASP %%%%%%%%%%%%%%%%%%%%%%%%%%%%%%%%%%%%%%%%%%%%%%%%%%%%%%%%%%%%
			\addplot[DASP] coordinates{
				(620.704,0.274273)
				(1388.06,0.3037345)
				(3564.56,0.361529)
			};

				% EAMS %%%%%%%%%%%%%%%%%%%%%%%%%%%%%%%%%%%%%%%%%%%%%%%%%%%%%%%%%%%%
			\addplot[EAMS] coordinates{
				(499.416,0.20329)
				(2584.6,0.499207)
				(4448.4,1.182145)
			};

				% ERGC %%%%%%%%%%%%%%%%%%%%%%%%%%%%%%%%%%%%%%%%%%%%%%%%%%%%%%%%%%%%
			\addplot[ERGC] coordinates{
				(400,0.0973685)
				(1224,0.1017415)
				(3520,0.106717)
			};

				% ERS %%%%%%%%%%%%%%%%%%%%%%%%%%%%%%%%%%%%%%%%%%%%%%%%%%%%%%%%%%%%
			\addplot[ERS] coordinates{
				(400,0.581565)
				(1200,0.57872)
				(3600,0.66461)
			};

				% ETPS %%%%%%%%%%%%%%%%%%%%%%%%%%%%%%%%%%%%%%%%%%%%%%%%%%%%%%%%%%%%
			\addplot[ETPS] coordinates{
				(432,0.416629)
				(1230,0.4826945)
				(3400,0.83796)
			};

				% FH %%%%%%%%%%%%%%%%%%%%%%%%%%%%%%%%%%%%%%%%%%%%%%%%%%%%%%%%%%%%
			\addplot[FH] coordinates{
				(989.331,0.0674935)
				%(962.526,0.065426)
				(4594.6,0.061817)
			};

				% LSC %%%%%%%%%%%%%%%%%%%%%%%%%%%%%%%%%%%%%%%%%%%%%%%%%%%%%%%%%%%%
			\addplot[LSC] coordinates{
				(718.163,0.1265165)
				(2009.32,0.1296865)
				(3802.19,0.1339595)
			};

				% MSS %%%%%%%%%%%%%%%%%%%%%%%%%%%%%%%%%%%%%%%%%%%%%%%%%%%%%%%%%%%%
			\addplot[MSS] coordinates{
				(436.368,0.0347869)
				(1363.6,0.0357894)
				(4078.85,0.0394486)
			};

				% PB %%%%%%%%%%%%%%%%%%%%%%%%%%%%%%%%%%%%%%%%%%%%%%%%%%%%%%%%%%%%
			\addplot[PB] coordinates{
				(463.193,0.04695495)
				(1217.27,0.04713035)
				(3316.29,0.0475063)
			};

				% PF %%%%%%%%%%%%%%%%%%%%%%%%%%%%%%%%%%%%%%%%%%%%%%%%%%%%%%%%%%%%
			\addplot[PF] coordinates{
				(586.371,0.0067319371478697)
				(1718.72,0.007116080735589)
				(23725.5,0.0075440853170426)
			};

				% POISE %%%%%%%%%%%%%%%%%%%%%%%%%%%%%%%%%%%%%%%%%%%%%%%%%%%%%%%%%%%%
			\addplot[POISE] coordinates{
				(408.614,0.4976745)
				(1222.5,0.5144655)
				(3466.96,0.4675375)
			};

				% PRESLIC %%%%%%%%%%%%%%%%%%%%%%%%%%%%%%%%%%%%%%%%%%%%%%%%%%%%%%%%%%%%
			\addplot[PRESLIC] coordinates{
				(388.539,0.0283709)
				(1188.65,0.0302631)
				(3422.72,0.02922305)
			};

			% QS %%%%%%%%%%%%%%%%%%%%%%%%%%%%%%%%%%%%%%%%%%%%%%%%%%%%%%%%%%%%
			\addplot[QS] coordinates{
				(414.609,1.61712)
				(1252.44,0.717436)
				(3460.91,0.462995)
			};

				% RESEEDS %%%%%%%%%%%%%%%%%%%%%%%%%%%%%%%%%%%%%%%%%%%%%%%%%%%%%%%%%%%%
			\addplot[RESEEDS] coordinates{
				(400.323,0.052832)
				(1200.06,0.05302)
				(3648.12,0.072757)
			};

				% RW %%%%%%%%%%%%%%%%%%%%%%%%%%%%%%%%%%%%%%%%%%%%%%%%%%%%%%%%%%%%
			\addplot[RW] coordinates{
				(440.393,21.302869)
				(1306.99,88.026016)
				(3967.51,98.023395)
			};

			% NC %%%%%%%%%%%%%%%%%%%%%%%%%%%%%%%%%%%%%%%%%%%%%%%%%%%%%%%%%%%%
			\addplot[NC] coordinates{
				(436.48,184.74016008) % 11.575198
				(1049.4,193.50751476) % 12.124531
				(2992.64,417.49685796) % 26.158951
			};

			% SEAW %%%%%%%%%%%%%%%%%%%%%%%%%%%%%%%%%%%%%%%%%%%%%%%%%%%%%%%%%%%%
			\addplot[SEAW] coordinates{
				(349.414,0.8460285)
				(1192.36,3.3036935)
				%(0,12.9707715)
			};

				% SEEDS %%%%%%%%%%%%%%%%%%%%%%%%%%%%%%%%%%%%%%%%%%%%%%%%%%%%%%%%%%%%
			\addplot[SEEDS] coordinates{
				(450.607,0.747595)
				(1258.91,0.68287)
				(3711.17,2.006325)
			};

				% SLIC %%%%%%%%%%%%%%%%%%%%%%%%%%%%%%%%%%%%%%%%%%%%%%%%%%%%%%%%%%%%
			\addplot[SLIC] coordinates{
				(385.211,0.077268)
				(1211.08,0.0787595)
				(3441.91,0.0807015)
			};

			% TP %%%%%%%%%%%%%%%%%%%%%%%%%%%%%%%%%%%%%%%%%%%%%%%%%%%%%%%%%%%%
			\addplot[TP] coordinates{
				(551.496,9.5313755)
				(1401.47,6.669046)
				(3432.31,4.0482035)
			};

				% TPS %%%%%%%%%%%%%%%%%%%%%%%%%%%%%%%%%%%%%%%%%%%%%%%%%%%%%%%%%%%%
			\addplot[TPS] coordinates{
				(450.436,1.52772)
				(1317.64,1.73017)
				(3924.13,2.17962)
			};

				% VC %%%%%%%%%%%%%%%%%%%%%%%%%%%%%%%%%%%%%%%%%%%%%%%%%%%%%%%%%%%%
			\addplot[VC] coordinates{
				(1593.27,1.515665)
				(2870.03,1.11946)
				(5672.72,1.055615)
			};

				% VCCS %%%%%%%%%%%%%%%%%%%%%%%%%%%%%%%%%%%%%%%%%%%%%%%%%%%%%%%%%%%%
			\addplot[VCCS] coordinates{
				(1179.74,0.75426)
				(1526.22,0.50258)
				(3044.22,0.317707)
			};

				% VLSLIC %%%%%%%%%%%%%%%%%%%%%%%%%%%%%%%%%%%%%%%%%%%%%%%%%%%%%%%%%%%%
			%\addplot[VLSLIC] coordinates{
			%	(0,0.1012405)
			%	(0,0.1022055)
			%	(0,0.104511)
			%};

				% W %%%%%%%%%%%%%%%%%%%%%%%%%%%%%%%%%%%%%%%%%%%%%%%%%%%%%%%%%%%%
			\addplot[W] coordinates{
				(397.103,0.0066792)
				(1234.38,0.0071429)
				(3518.42,0.0076441)
			};

				% WP %%%%%%%%%%%%%%%%%%%%%%%%%%%%%%%%%%%%%%%%%%%%%%%%%%%%%%%%%%%%
			\addplot[WP] coordinates{
				(432,0.159675)
				(1230,0.189425)
				(3400,0.26395)
			};

			\end{axis}
	\end{tikzpicture}
\end{subfigure}

    \caption{Runtime in seconds on the \BSDS and \NYU datasets.
    Watershed-based, some clustering-based algorithms as well as \PF offer
    runtimes below $100ms$. In the light of realtime applications, \CW, \W
    and \PF even provide runtimes below $10ms$. However, independent of the
    application at hand, we find runtimes below $1s$ beneficial for using
    superpixel algorithms as pre-processing tools.
    \textbf{Best viewed in color.}}
    \label{fig:experiments-runtime}
\end{figure}
\begin{figure}[t]
    \centering
    \begin{subfigure}[b]{\halfthreeone\textwidth}\phantomsubcaption\label{subfig:experiments-iterations-bsds500-rec}
	\begin{tikzpicture}
		%%%%%%%%%%%%%%%%%%%%%%%%%%%%%%%%%%%%%%%%%%%%%%%%%%%%%%%%%%%%%%%%%%%%%%%%%%%%%%%%
		% 400 rec.mean_min
		%%%%%%%%%%%%%%%%%%%%%%%%%%%%%%%%%%%%%%%%%%%%%%%%%%%%%%%%%%%%%%%%%%%%%%%%%%%%%%%%
		\begin{axis}[EQBSDS500ItRec]
			% CCS rec.mean_min %%%%%%%%%%%%%%%%%%%%%%%%%%%%%%%%%%%%%%%%%%%%%%%%%%%%%%%%%%%%
			\addplot[CCS] coordinates{
				(1,0.405664)
				(5,0.599782)
				(10,0.671463)
				(25,0.696412)
			};
			% CIS rec.mean_min %%%%%%%%%%%%%%%%%%%%%%%%%%%%%%%%%%%%%%%%%%%%%%%%%%%%%%%%%%%%
			\addplot[CIS] coordinates{
				(1,0.668558)
				(3,0.636139)
				(6,0.635233)
			};
			% CRS rec.mean_min %%%%%%%%%%%%%%%%%%%%%%%%%%%%%%%%%%%%%%%%%%%%%%%%%%%%%%%%%%%%
			\addplot[CRS] coordinates{
				(1,0.818608)
				(3,0.833132)
				(6,0.834531)
			};
			% ETPS rec.mean_min %%%%%%%%%%%%%%%%%%%%%%%%%%%%%%%%%%%%%%%%%%%%%%%%%%%%%%%%%%%%
			\addplot[ETPS] coordinates{
				(1,0.897415)
				(5,0.901659)
				(10,0.901555)
				(25,0.901663)
			};
			% LSC rec.mean_min %%%%%%%%%%%%%%%%%%%%%%%%%%%%%%%%%%%%%%%%%%%%%%%%%%%%%%%%%%%%
			\addplot[LSC] coordinates{
				(1,0.833689)
				(5,0.843883)
				(10,0.843333)
				(25,0.841751)
			};
			% SLIC rec.mean_min %%%%%%%%%%%%%%%%%%%%%%%%%%%%%%%%%%%%%%%%%%%%%%%%%%%%%%%%%%%%
			\addplot[SLIC] coordinates{
				(1,0.703309)
				(5,0.722822)
				(10,0.725153)
				(25,0.726167)
			};
			% PRESLIC rec.mean_min %%%%%%%%%%%%%%%%%%%%%%%%%%%%%%%%%%%%%%%%%%%%%%%%%%%%%%%%%%%%
			\addplot[PRESLIC] coordinates{
				(1,0.694801)
				(5,0.704329)
				(10,0.704329)
				(25,0.704329)
			};
			% RESEEDS rec.mean_min %%%%%%%%%%%%%%%%%%%%%%%%%%%%%%%%%%%%%%%%%%%%%%%%%%%%%%%%%%%%
			\addplot[RESEEDS] coordinates{
				(1,0.862949)
				(5,0.90843)
				(10,0.916129)
				(25,0.918462)
			};
			% SEEDS rec.mean_min %%%%%%%%%%%%%%%%%%%%%%%%%%%%%%%%%%%%%%%%%%%%%%%%%%%%%%%%%%%%
			\addplot[SEEDS] coordinates{
				(1,0.771216)
				(5,0.872416)
				(10,0.901845)
				(25,0.922132)
			};
			% VLSLIC rec.mean_min %%%%%%%%%%%%%%%%%%%%%%%%%%%%%%%%%%%%%%%%%%%%%%%%%%%%%%%%%%%%
			\addplot[VLSLIC] coordinates{
				(1,0.793577)
				(5,0.819088)
				(10,0.815093)
				(25,0.813072)
			};
		\end{axis}
	\end{tikzpicture}
\end{subfigure}
\begin{subfigure}[b]{\halfthreeone\textwidth}\phantomsubcaption\label{subfig:experiments-iterations-bsds500-ue}
	\begin{tikzpicture}
		%%%%%%%%%%%%%%%%%%%%%%%%%%%%%%%%%%%%%%%%%%%%%%%%%%%%%%%%%%%%%%%%%%%%%%%%%%%%%%%%
		% 400 ue_np.mean_max
		%%%%%%%%%%%%%%%%%%%%%%%%%%%%%%%%%%%%%%%%%%%%%%%%%%%%%%%%%%%%%%%%%%%%%%%%%%%%%%%%
		\begin{axis}[EQBSDS500ItUE]
			% CCS ue_np.mean_max %%%%%%%%%%%%%%%%%%%%%%%%%%%%%%%%%%%%%%%%%%%%%%%%%%%%%%%%%%%%
			\addplot[CCS] coordinates{
				(1,0.204253)
				(5,0.162655)
				(10,0.143563)
				(25,0.133202)
			};
			% CIS ue_np.mean_max %%%%%%%%%%%%%%%%%%%%%%%%%%%%%%%%%%%%%%%%%%%%%%%%%%%%%%%%%%%%
			\addplot[CIS] coordinates{
				(1,0.123373)
				(3,0.132016)
				(6,0.132153)
			};
			% CRS ue_np.mean_max %%%%%%%%%%%%%%%%%%%%%%%%%%%%%%%%%%%%%%%%%%%%%%%%%%%%%%%%%%%%
			\addplot[CRS] coordinates{
				(1,0.111657)
				(3,0.109511)
				(6,0.109207)
			};
			% ETPS ue_np.mean_max %%%%%%%%%%%%%%%%%%%%%%%%%%%%%%%%%%%%%%%%%%%%%%%%%%%%%%%%%%%%
			\addplot[ETPS] coordinates{
				(1,0.120458)
				(5,0.120686)
				(10,0.120595)
				(25,0.120575)
			};
			% LSC ue_np.mean_max %%%%%%%%%%%%%%%%%%%%%%%%%%%%%%%%%%%%%%%%%%%%%%%%%%%%%%%%%%%%
			\addplot[LSC] coordinates{
				(1,0.110857)
				(5,0.107074)
				(10,0.106936)
				(25,0.107473)
			};
			% SLIC ue_np.mean_max %%%%%%%%%%%%%%%%%%%%%%%%%%%%%%%%%%%%%%%%%%%%%%%%%%%%%%%%%%%%
			\addplot[SLIC] coordinates{
				(1,0.14737)
				(5,0.136587)
				(10,0.134108)
				(25,0.133617)
			};
			% PRESLIC ue_np.mean_max %%%%%%%%%%%%%%%%%%%%%%%%%%%%%%%%%%%%%%%%%%%%%%%%%%%%%%%%%%%%
			\addplot[PRESLIC] coordinates{
				(1,0.156342)
				(5,0.15315)
				(10,0.15315)
				(25,0.15315)
			};
			% RESEEDS ue_np.mean_max %%%%%%%%%%%%%%%%%%%%%%%%%%%%%%%%%%%%%%%%%%%%%%%%%%%%%%%%%%%%
			\addplot[RESEEDS] coordinates{
				(1,0.135469)
				(5,0.132502)
				(10,0.138297)
				(25,0.138375)
			};
			% SEEDS ue_np.mean_max %%%%%%%%%%%%%%%%%%%%%%%%%%%%%%%%%%%%%%%%%%%%%%%%%%%%%%%%%%%%
			\addplot[SEEDS] coordinates{
				(1,0.156297)
				(5,0.162523)
				(10,0.165167)
				(25,0.166744)
			};
			% VLSLIC ue_np.mean_max %%%%%%%%%%%%%%%%%%%%%%%%%%%%%%%%%%%%%%%%%%%%%%%%%%%%%%%%%%%%
			\addplot[VLSLIC] coordinates{
				(1,0.13491)
				(5,0.12431)
				(10,0.125229)
				(25,0.126088)
			};
		\end{axis}
	\end{tikzpicture}
\end{subfigure}
\begin{subfigure}[b]{\halfthreeone\textwidth}\phantomsubcaption\label{subfig:experiments-iterations-bsds500-t}
	\begin{tikzpicture}
		%%%%%%%%%%%%%%%%%%%%%%%%%%%%%%%%%%%%%%%%%%%%%%%%%%%%%%%%%%%%%%%%%%%%%%%%%%%%%%%%
		% 400 runtime
		%%%%%%%%%%%%%%%%%%%%%%%%%%%%%%%%%%%%%%%%%%%%%%%%%%%%%%%%%%%%%%%%%%%%%%%%%%%%%%%%
		\begin{axis}[EQBSDS500Itt,ymode=log]
			% CCS runtime %%%%%%%%%%%%%%%%%%%%%%%%%%%%%%%%%%%%%%%%%%%%%%%%%%%%%%%%%%%%
			\addplot[CCS] coordinates{
				(1,0.0665)
				(5,0.0985002)
				(10,0.14135)
				(25,0.24505)
			};
			% CIS runtime %%%%%%%%%%%%%%%%%%%%%%%%%%%%%%%%%%%%%%%%%%%%%%%%%%%%%%%%%%%%
			\addplot[CIS] coordinates{
				(1,1.97275)
				(3,5.20055)
				(6,7.3538)
			};
			% CRS runtime %%%%%%%%%%%%%%%%%%%%%%%%%%%%%%%%%%%%%%%%%%%%%%%%%%%%%%%%%%%%
			\addplot[CRS] coordinates{
				(1,0.3295)
				(3,0.8723)
				(6,1.68255)
			};
			% ETPS runtime %%%%%%%%%%%%%%%%%%%%%%%%%%%%%%%%%%%%%%%%%%%%%%%%%%%%%%%%%%%%
			\addplot[ETPS] coordinates{
				(1,0.0583001)
				(5,0.16035)
				(10,0.28495)
				(25,0.65425)
			};
			% LSC runtime %%%%%%%%%%%%%%%%%%%%%%%%%%%%%%%%%%%%%%%%%%%%%%%%%%%%%%%%%%%%
			\addplot[LSC] coordinates{
				(1,0.1149)
				(5,0.1484)
				(10,0.19055)
				(25,0.31585)
			};
			% SLIC runtime %%%%%%%%%%%%%%%%%%%%%%%%%%%%%%%%%%%%%%%%%%%%%%%%%%%%%%%%%%%%
			\addplot[SLIC] coordinates{
				(1,0.0610501)
				(5,0.0748499)
				(10,0.0937001)
				(25,0.1465)
			};
			% PRESLIC runtime %%%%%%%%%%%%%%%%%%%%%%%%%%%%%%%%%%%%%%%%%%%%%%%%%%%%%%%%%%%%
			\addplot[PRESLIC] coordinates{
				(1,0.01265)
				(5,0.0294)
				(10,0.02955)
				(25,0.02925)
			};
			% RESEEDS runtime %%%%%%%%%%%%%%%%%%%%%%%%%%%%%%%%%%%%%%%%%%%%%%%%%%%%%%%%%%%%
			\addplot[RESEEDS] coordinates{
				(1,0.0414)
				(5,0.0555001)
				(10,0.0619001)
				(25,0.0773499)
			};
			% SEEDS runtime %%%%%%%%%%%%%%%%%%%%%%%%%%%%%%%%%%%%%%%%%%%%%%%%%%%%%%%%%%%%
			\addplot[SEEDS] coordinates{
				(1,0.0870501)
				(5,0.22575)
				(10,0.41915)
				(25,0.92205)
			};
			% VLSLIC runtime %%%%%%%%%%%%%%%%%%%%%%%%%%%%%%%%%%%%%%%%%%%%%%%%%%%%%%%%%%%%
			\addplot[VLSLIC] coordinates{
				(1,0.03525)
				(5,0.11775)
				(10,0.22025)
				(25,0.5291)
			};
		\end{axis}
	\end{tikzpicture}
\end{subfigure}

    \caption{\Rec, \UE and runtime in seconds $t$ for iterative algorithms with 
    $\K \approx 400$ on the \BSDS dataset.
    Some algorithms allow to gradually trade performance for runtime, reducing runtime 
    by several $100ms$ in some cases.
    \textbf{Best viewed in color.}}
    \label{fig:experiments-iterations}
	\vskip 12px
	% Total: 26 (+2 depth)
%\begin{mdframed}
	{\scriptsize
		\begin{tabularx}{0.475\textwidth}{X X X l}
			\ref{plot:w} \W &
			\ref{plot:eams} \EAMS &
			\ref{plot:nc} \NC &
			\ref{plot:fh} \FH \\
			\ref{plot:refh} \reFH &
			\ref{plot:rw} \RW &
			\ref{plot:qs} \QS &
			\ref{plot:pf} \PF \\
			\ref{plot:tp} \TP &
			\ref{plot:cis} \CIS &
			\ref{plot:slic} \SLIC &
			\ref{plot:vlslic} \vlSLIC \\
			\ref{plot:crs} \CRS &
			\ref{plot:ers} \ERS &
			\ref{plot:pb} \PB &
			\ref{plot:dasp} \DASP \\
			\ref{plot:seeds} \SEEDS &
			\ref{plot:reseeds} \reSEEDS &
			\ref{plot:tps} \TPS &
			\ref{plot:vc} \VC \\
			\ref{plot:ccs} \CCS &
			\ref{plot:vccs} \VCCS &
			\ref{plot:cw} \CW &
			\ref{plot:ergc} \ERGC \\
			\ref{plot:mss} \MSS &
			\ref{plot:preslic} \preSLIC &
			\ref{plot:wp} \WP &
			\ref{plot:etps} \ETPS \\
			\ref{plot:lsc} \LSC &
			\ref{plot:poise} \POISE &
			\ref{plot:seaw} \SEAW &
		\end{tabularx}
	}
%\end{mdframed}

\end{figure}
\FloatBarrier

Iterative algorithms offer to reduce runtime while gradually lowering performance.
Considering Figure \ref{fig:experiments-iterations},
showing \Rec, \UE and runtime in seconds $t$ for all iterative algorithms on the \BSDS dataset,
we observe that the number of iterations can safely be reduced to decrease
runtime while lowering \Rec and increasing \UE only slightly. In the best case,
for example considering \ETPSr, reducing the number of iterations from $25$ to $1$
reduces the runtime from $680\text{ms}$ to $58\text{ms}$, while keeping \Rec und \UE
nearly constant. For other cases, such as \SEEDSr\footnote{
	We note that we use the original \SEEDS implementation instead of the
	OpenCV \cite{Bradski:2000} implementation which is reported to be more efficient, see
	\url{http://docs.opencv.org/3.0-last-rst/modules/ximgproc/doc/superpixels.html}.
}, \Rec decreases abruptly when
using less than $5$ iterations. Still, runtime can be reduced from $920\text{ms}$
to $220\text{ms}$. For \CRSr and \CISr, runtime reduction is similarly significant,
but both algorithms still exhibit higher runtimes. If post-processing
is necessary, for example for \SLICr and \preSLICr, the number of iterations has
to be fixed in advance. However, for other iterative algorithms, the number of
iterations may be adapted at runtime depending on the available time.
Overall, iterative algorithms are beneficial as they are able to gradually
trade performance for runtime.

We conclude that watershed-based as well as some path-based and clustering-based
algorithms are candidates for realtime applications. Iterative algorithms,
in particular many clustering-based and iterative energy optimization algorithms, may
further be speeded up by reducing the number of iterations and trading performance for
runtime. On a final note, we want to remind the reader that the image sizes of all
used datasets may be relatively small compared to today's applications.
However, the relative runtime comparison is still valuable.
