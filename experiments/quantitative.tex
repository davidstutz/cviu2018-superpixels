\subsection{Quantitative}
\label{subsec:experiments-quantitative}

\begin{figure*}
	\centering
	\vspace{-12px}
	\begin{subfigure}[b]{\fullthreeone\textwidth}\phantomsubcaption\label{subfig:experiments-quantitative-bsds500-rec.mean_min}
	%%%%%%%%%%%%%%%%%%%%%%%%%%%%%%%%%%%%%%%%%%%%%%%%%%%%%%%%%%%%
	% rec.mean_min
	%%%%%%%%%%%%%%%%%%%%%%%%%%%%%%%%%%%%%%%%%%%%%%%%%%%%%%%%%%%%
	\begin{tikzpicture}
		\begin{axis}[EQBSDS500Rec,xmode=log]

			% CCS rec.mean_min %%%%%%%%%%%%%%%%%%%%%%%%%%%%%%%%%%%%%%%%%%%%%%%%%%%%%%%%%%%%
			\addplot[CCS] coordinates{
				(218.865,0.592158)
				(303.21,0.643079)
				(453.82,0.706618)
				(706.35,0.773003)
				(871.12,0.801807)
				(1204.99,0.84447)
				(1438.62,0.866527)
				(1438.62,0.866527)
				(1762.49,0.891227)
				(2107.24,0.909919)
				(2107.24,0.909919)
				(2702.61,0.935207)
				(3406.7,0.955103)
				(3406.7,0.955103)
				(3406.7,0.955103)
				(4654.44,0.974977)
				(4654.44,0.974977)
				(6619.1,0.990276)
			};

			% SEEDS rec.mean_min %%%%%%%%%%%%%%%%%%%%%%%%%%%%%%%%%%%%%%%%%%%%%%%%%%%%%%%%%%%%
			\addplot[SEEDS] coordinates{
				(261.62,0.882484)
				(365.675,0.906473)
				(468.81,0.922132)
				(670.57,0.941889)
				(870.75,0.952388)
				(1087.4,0.963175)
				(1270.11,0.967167)
				(1451.85,0.973778)
				(1669.18,0.974282)
				(1873.19,0.980114)
				(2104.62,0.984773)
				(2462.77,0.98341)
				(2793.43,0.989414)
				(3260.86,0.989969)
				(3895.78,0.993267)
				(3895.78,0.993267)
				(4846.12,0.995104)
				(4846.12,0.995104)
			};

			% SLIC rec.mean_min %%%%%%%%%%%%%%%%%%%%%%%%%%%%%%%%%%%%%%%%%%%%%%%%%%%%%%%%%%%%
			\addplot[SLIC] coordinates{
				(180.32,0.649493)
				(256.725,0.687987)
				(368.57,0.726167)
				(575.335,0.778431)
				(726.36,0.807455)
				(1002.87,0.843839)
				(1203.61,0.868781)
				(1203.61,0.868781)
				(1475.69,0.887329)
				(1814.8,0.908264)
				(1814.8,0.908264)
				(2334.56,0.929101)
				(3038.13,0.949264)
				(3038.13,0.949264)
				(3038.13,0.949264)
				(4188.97,0.970933)
				(4188.97,0.970933)
				(6139.41,0.988423)
			};

			% RW rec.mean_min %%%%%%%%%%%%%%%%%%%%%%%%%%%%%%%%%%%%%%%%%%%%%%%%%%%%%%%%%%%%
			\addplot[RW] coordinates{
				(211.565,0.535574)
				(313.64,0.593389)
				(407.125,0.631774)
				(649.355,0.700438)
				(858.65,0.741309)
				(1032.29,0.768567)
				(1280.94,0.800097)
				(1505.72,0.823098)
				(1649.74,0.835613)
				(1940.98,0.859641)
				(2101.2,0.869704)
				(2570.18,0.895993)
				(2916.36,0.910657)
				(3413.13,0.930862)
				(3882.85,0.944196)
				(4457.4,0.953941)
				(5093.4,0.971407)
				(5663.8,0.979521)
			};

			% CW rec.mean_min %%%%%%%%%%%%%%%%%%%%%%%%%%%%%%%%%%%%%%%%%%%%%%%%%%%%%%%%%%%%
			\addplot[CW] coordinates{
				(197.775,0.614748)
				(309.29,0.666923)
				(405.935,0.699117)
				(609.18,0.747919)
				(807.005,0.783538)
				(1045.56,0.814313)
				(1252.72,0.834985)
				(1492.83,0.855504)
				(1649.62,0.868775)
				(1816.43,0.878199)
				(2091.79,0.896642)
				(2565.8,0.919099)
				(2917.57,0.932486)
				(3319.52,0.944674)
				(4023.98,0.964695)
				(4023.98,0.964695)
				(4639.83,0.973797)
				(5846.73,0.98843)
			};

			% TP rec.mean_min %%%%%%%%%%%%%%%%%%%%%%%%%%%%%%%%%%%%%%%%%%%%%%%%%%%%%%%%%%%%
			\addplot[TP] coordinates{
				(283.695,0.549743)
				(381.285,0.60368)
				(555.1,0.669034)
				(752.29,0.701044)
				(1007.04,0.752593)
				(1292.4,0.782178)
				(1495.14,0.799116)
				(1696.62,0.808163)
				(2017.73,0.837956)
				(2478.02,0.869353)
				(2831,0.891659)
				(3018.75,0.891316)
			};

			% POISE rec.mean_min %%%%%%%%%%%%%%%%%%%%%%%%%%%%%%%%%%%%%%%%%%%%%%%%%%%%%%%%%%%%
			\addplot[POISE] coordinates{
				(204.855,0.716777)
				(306.68,0.75173)
				(408.385,0.776481)
				(611.79,0.809652)
				(814.975,0.832115)
				(1017.48,0.847986)
				(1216.26,0.859498)
				(1409.36,0.867716)
				(1587.37,0.87367)
				(1749.07,0.877915)
				(1887.06,0.880969)
				(2090.32,0.884553)
				(2221.84,0.885999)
				(2278.76,0.886523)
				(2287.5,0.886627)
				(2288.94,0.88664)
				(2288.94,0.88664)
				(2288.94,0.88664)
			};

			% FH rec.mean_min %%%%%%%%%%%%%%%%%%%%%%%%%%%%%%%%%%%%%%%%%%%%%%%%%%%%%%%%%%%%
			\addplot[FH] coordinates{
				(628.745,0.663508)
				(799.09,0.706555)
				(963.36,0.748153)
				(1090.39,0.75093)
				(1187.04,0.782457)
				(1605.71,0.808989)
				(2533.01,0.888881)
				(3000.74,0.889299)
				(3219.63,0.904142)
				(3814.42,0.913162)
				(4746.96,0.935408)
			};

			% EAMS rec.mean_min %%%%%%%%%%%%%%%%%%%%%%%%%%%%%%%%%%%%%%%%%%%%%%%%%%%%%%%%%%%%
			\addplot[EAMS] coordinates{
				(261.12,0.765523)
				(283.33,0.775191)
				(309.87,0.785763)
				(383.52,0.807197)
				(499.435,0.835303)
				(725.36,0.87097)
				(1417.35,0.927531)
				(1473.6,0.93019)
				(1534.33,0.933122)
				(1883.59,0.928756)
				(2063.65,0.934911)
				(2346.71,0.942548)
				(2642.78,0.949301)
				(3113.25,0.950181)
				(3640.44,0.956969)
				(4914.12,0.965963)
				(8189.58,0.968896)
			};

			% CRS rec.mean_min %%%%%%%%%%%%%%%%%%%%%%%%%%%%%%%%%%%%%%%%%%%%%%%%%%%%%%%%%%%%
			\addplot[CRS] coordinates{
				(299.575,0.739735)
				(449.585,0.796324)
				(635.865,0.834591)
				(895.935,0.872104)
				(1180.31,0.89801)
				(1530.1,0.919282)
				(1758.22,0.931076)
				(2057.15,0.942825)
				(2308.66,0.949839)
				(2461.75,0.954149)
				(2755.4,0.96145)
				(3401.31,0.972214)
				(3879.51,0.978129)
				(4346.83,0.981689)
				(5208.23,0.988075)
				(5208.23,0.988075)
				(5821.7,0.99096)
				(7241.41,0.995293)
			};

			% SEAW rec.mean_min %%%%%%%%%%%%%%%%%%%%%%%%%%%%%%%%%%%%%%%%%%%%%%%%%%%%%%%%%%%%
			\addplot[SEAW] coordinates{
				(165.505,0.402476)
				(356.805,0.540193)
				(923.49,0.700691)
				(2933.9,0.884813)
				(10727.5,0.99848)
			};

			% RESEEDS rec.mean_min %%%%%%%%%%%%%%%%%%%%%%%%%%%%%%%%%%%%%%%%%%%%%%%%%%%%%%%%%%%%
			\addplot[RESEEDS] coordinates{
				(200.795,0.876167)
				(301.525,0.902782)
				(401.465,0.918521)
				(602.33,0.937366)
				(800.99,0.949909)
				(1020.12,0.957079)
				(1201.34,0.965188)
				(1378.1,0.969268)
				(1601.55,0.97409)
				(1802.11,0.976241)
				(2040.11,0.980008)
				(2402.13,0.984131)
				(2720.13,0.987033)
				(3200.2,0.988506)
				(3840.22,0.992189)
				(3840.22,0.992189)
				(4800.37,0.994628)
				(4800.37,0.994628)
			};

			% ERGC rec.mean_min %%%%%%%%%%%%%%%%%%%%%%%%%%%%%%%%%%%%%%%%%%%%%%%%%%%%%%%%%%%%
			\addplot[ERGC] coordinates{
				(196,0.655001)
				(306,0.719633)
				(400,0.756603)
				(600,0.808226)
				(792.57,0.842021)
				(1024,0.871512)
				(1224,0.889976)
				(1453.2,0.906536)
				(1600,0.916695)
				(1760,0.924107)
				(2024,0.93718)
				(2472.98,0.951575)
				(2809,0.960359)
				(3180,0.967724)
				(3840,0.978263)
				(3840,0.978263)
				(4416,0.983896)
				(5520,0.991439)
			};

			% PF rec.mean_min %%%%%%%%%%%%%%%%%%%%%%%%%%%%%%%%%%%%%%%%%%%%%%%%%%%%%%%%%%%%
			\addplot[PF] coordinates{
				(281.06,0.383318)
				(428.82,0.433408)
				(585.08,0.472897)
				(928.805,0.536394)
				(1172.76,0.573425)
				(1602.65,0.618983)
				(1861.72,0.646727)
				(2267.11,0.674311)
				(2697.08,0.704684)
				(3382.15,0.752031)
				(4207.66,0.784641)
				(6051.74,0.84243)
			};

			% TPS rec.mean_min %%%%%%%%%%%%%%%%%%%%%%%%%%%%%%%%%%%%%%%%%%%%%%%%%%%%%%%%%%%%
			\addplot[TPS] coordinates{
				(224.26,0.529525)
				(903.595,0.71643)
				(1130.53,0.749941)
				(1332.49,0.772291)
				(1556.56,0.793589)
				(1736.84,0.810049)
				(1988.95,0.827813)
				(2167.04,0.840816)
				(2656.03,0.865906)
				(2987.25,0.883462)
				(3505.61,0.902668)
				(4030.29,0.920867)
				(4568.25,0.93147)
				(5242.04,0.942233)
				(5978.67,0.953847)
			};

			% NC rec.mean_min %%%%%%%%%%%%%%%%%%%%%%%%%%%%%%%%%%%%%%%%%%%%%%%%%%%%%%%%%%%%
			\addplot[NC] coordinates{
				(223.505,0.63167)
				(321.69,0.671044)
				(416.03,0.698916)
				(594.185,0.73739)
				(763.865,0.767168)
				(922.645,0.788693)
				(1073.6,0.807789)
				(1213.8,0.822597)
				(1348.36,0.836078)
				(1473.43,0.84768)
				(1591.75,0.859031)
				(2000.27,0.887365)
			};

			% VC rec.mean_min %%%%%%%%%%%%%%%%%%%%%%%%%%%%%%%%%%%%%%%%%%%%%%%%%%%%%%%%%%%%
			\addplot[VC] coordinates{
				(351.72,0.49227)
				(498.095,0.577478)
				(712.82,0.656452)
				(994.125,0.730334)
				(1201.27,0.767069)
				(1375.27,0.793846)
				(1685.08,0.824284)
				(1968.97,0.845684)
				(2224.75,0.863491)
				(2476.43,0.877434)
				(2712.29,0.88747)
				(2948.04,0.896458)
				(3169.69,0.903176)
				(3389.95,0.909392)
				(3804.08,0.920364)
				(4196.75,0.927701)
				(4566.66,0.934096)
				(4815.37,0.938079)
				(5157.48,0.942366)
				(5401.49,0.943816)
			};

			% PB rec.mean_min %%%%%%%%%%%%%%%%%%%%%%%%%%%%%%%%%%%%%%%%%%%%%%%%%%%%%%%%%%%%
			\addplot[PB] coordinates{
				(324.39,0.574859)
				(398.425,0.615142)
				(494.17,0.653303)
				(692.845,0.716897)
				(853.905,0.75298)
				(1129.27,0.793438)
				(1323.88,0.819173)
				(1323.88,0.819173)
				(1601.37,0.846321)
				(1929.58,0.874133)
				(1929.58,0.874133)
				(2453.31,0.903297)
				(3126.91,0.931536)
				(3126.91,0.931536)
				(3126.91,0.931536)
				(4270.56,0.95966)
				(4270.56,0.95966)
				(6122.31,0.982039)
			};

			% PRESLIC rec.mean_min %%%%%%%%%%%%%%%%%%%%%%%%%%%%%%%%%%%%%%%%%%%%%%%%%%%%%%%%%%%%
			\addplot[PRESLIC] coordinates{
				(369,0.704329)
				(581.54,0.753518)
				(734.575,0.781089)
				(1020.3,0.819649)
				(1229.22,0.839652)
				(1229.22,0.839652)
				(1511.3,0.862751)
				(1838.94,0.883831)
				(1838.94,0.883831)
				(2375.99,0.911038)
				(3059.43,0.934761)
				(3059.43,0.934761)
				(3059.43,0.934761)
				(4218.88,0.962025)
				(4218.88,0.962025)
				(6137.8,0.985124)
			};

			% W rec.mean_min %%%%%%%%%%%%%%%%%%%%%%%%%%%%%%%%%%%%%%%%%%%%%%%%%%%%%%%%%%%%
			\addplot[W] coordinates{
				(188.285,0.602725)
				(296.48,0.658145)
				(387.92,0.692999)
				(608.055,0.749924)
				(794.47,0.783432)
				(1100.87,0.822392)
				(1302.83,0.843054)
				(1302.83,0.843054)
				(1572.26,0.865465)
				(1960.24,0.890398)
				(1960.24,0.890398)
				(2478.43,0.917414)
				(3292.41,0.945248)
				(3292.41,0.945248)
				(3292.41,0.945248)
				(4439.31,0.969279)
				(4439.31,0.969279)
				(6526.67,0.99025)
			};

			% LSC rec.mean_min %%%%%%%%%%%%%%%%%%%%%%%%%%%%%%%%%%%%%%%%%%%%%%%%%%%%%%%%%%%%
			\addplot[LSC] coordinates{
				(548.755,0.761641)
				(756.92,0.804947)
				(1045.31,0.843883)
				(1360.95,0.872394)
				(1665.85,0.892952)
				(1919.65,0.909249)
				(2055.98,0.914786)
				(2239.5,0.922711)
				(2376.71,0.926652)
				(2441.63,0.930021)
				(2603.55,0.934474)
				(2909.2,0.94186)
				(3147.7,0.946013)
				(3373.12,0.951264)
				(3762.88,0.955676)
				(3762.88,0.955676)
				(4057.29,0.96128)
				(4401.67,0.961159)
			};

			% WP rec.mean_min %%%%%%%%%%%%%%%%%%%%%%%%%%%%%%%%%%%%%%%%%%%%%%%%%%%%%%%%%%%%
			\addplot[WP] coordinates{
				(216,0.596779)
				(294,0.636871)
				(384,0.668597)
				(600,0.720116)
				(805,0.756131)
				(1080,0.792842)
				(1320,0.816474)
				(1536,0.835733)
				(1536,0.834598)
				(1944,0.861604)
				(1944,0.861092)
				(2400,0.886963)
				(3174,0.917725)
				(3174,0.916981)
				(4320,0.950223)
				(4320,0.950223)
				(4320,0.950223)
				(5836.85,0.9745)
			};

			% QS rec.mean_min %%%%%%%%%%%%%%%%%%%%%%%%%%%%%%%%%%%%%%%%%%%%%%%%%%%%%%%%%%%%
			\addplot[QS] coordinates{
				(324.725,0.292823)
				(468.655,0.405017)
				(615,0.481636)
				(884.79,0.577498)
				(984.505,0.603615)
				(1275.62,0.664099)
				(1500.06,0.695253)
				(1646.89,0.712214)
				(1824.2,0.729719)
				(2040.32,0.746918)
				(2303.69,0.766981)
				(2626.4,0.786206)
				(3066.62,0.808161)
				(3646.03,0.830694)
				(4444.58,0.854576)
				(5554.87,0.87988)
			};

			% VLSLIC rec.mean_min %%%%%%%%%%%%%%%%%%%%%%%%%%%%%%%%%%%%%%%%%%%%%%%%%%%%%%%%%%%%
			\addplot[VLSLIC] coordinates{
				(575.975,0.78762)
				(651.86,0.800183)
				(763.62,0.819088)
				(899,0.836123)
				(988.985,0.845203)
				(1193.67,0.860582)
				(1348.08,0.866749)
				(1349.01,0.865282)
				(1579.08,0.87797)
				(1849.65,0.891934)
				(1857.27,0.890967)
				(2307.73,0.909537)
				(2890.56,0.929862)
				(2924.52,0.929379)
				(2954.18,0.9279)
				(3858.98,0.952214)
				(3858.98,0.952214)
				(4809.57,0.969088)
			};

			% CIS rec.mean_min %%%%%%%%%%%%%%%%%%%%%%%%%%%%%%%%%%%%%%%%%%%%%%%%%%%%%%%%%%%%
			\addplot[CIS] coordinates{
				(317.63,0.58254)
				(411.595,0.617403)
				(472.685,0.636139)
				(633.345,0.674908)
				(777.05,0.701314)
				(969.085,0.730917)
				(1150.5,0.752483)
				(1168.82,0.75402)
				(1371.27,0.776382)
				(1637.39,0.798779)
				(1667.02,0.801391)
				(2056.01,0.828562)
				(3126.04,0.86339)
				(3625.6,0.874659)
				(4558.92,0.905583)
				(4558.92,0.905583)
				(6448.65,0.938776)
			};

			% ERS rec.mean_min %%%%%%%%%%%%%%%%%%%%%%%%%%%%%%%%%%%%%%%%%%%%%%%%%%%%%%%%%%%%
			\addplot[ERS] coordinates{
				(200,0.702556)
				(300,0.747987)
				(400,0.77887)
				(600,0.823611)
				(800,0.854591)
				(1000,0.877884)
				(1200,0.895901)
				(1400,0.911234)
				(1600,0.923745)
				(1800,0.934233)
				(2000,0.943065)
				(2400,0.956368)
				(2800,0.966708)
				(3200,0.974047)
				(3600,0.980268)
				(4000,0.984629)
				(4600,0.989514)
				(5200,0.992497)
			};

			% MSS rec.mean_min %%%%%%%%%%%%%%%%%%%%%%%%%%%%%%%%%%%%%%%%%%%%%%%%%%%%%%%%%%%%
			\addplot[MSS] coordinates{
				(200.765,0.648116)
				(282.33,0.680236)
				(420.66,0.730153)
				(664.115,0.777084)
				(837.885,0.796501)
				(1179.3,0.831002)
				(1422.34,0.849141)
				(1423.04,0.849163)
				(1768,0.870763)
				(2156.96,0.888925)
				(2157.01,0.889235)
				(2826.35,0.913644)
				(3669.97,0.93531)
				(3670.34,0.935078)
				(3669.92,0.935225)
				(5213.45,0.960719)
				(5213.45,0.960719)
				(7846.35,0.983439)
			};

			% ETPS rec.mean_min %%%%%%%%%%%%%%%%%%%%%%%%%%%%%%%%%%%%%%%%%%%%%%%%%%%%%%%%%%%%
			\addplot[ETPS] coordinates{
				(216,0.85347)
				(294,0.879135)
				(425,0.901663)
				(651,0.923596)
				(805,0.937546)
				(1107,0.951365)
				(1320,0.962067)
				(1320,0.962067)
				(1617,0.965753)
				(1944,0.973692)
				(1944,0.973692)
				(2501,0.979204)
				(3174,0.986005)
				(3174,0.986005)
				(3174,0.986005)
				(4374,0.991711)
				(4374,0.991711)
				(6305,0.99675)
			};

		\end{axis}
	\end{tikzpicture}
\end{subfigure}
\begin{subfigure}[b]{\fullthreeone\textwidth}\phantomsubcaption\label{subfig:experiments-quantitative-bsds500-ue_np.mean_max}
	%%%%%%%%%%%%%%%%%%%%%%%%%%%%%%%%%%%%%%%%%%%%%%%%%%%%%%%%%%%%
	% ue_np.mean_max
	%%%%%%%%%%%%%%%%%%%%%%%%%%%%%%%%%%%%%%%%%%%%%%%%%%%%%%%%%%%%
	\begin{tikzpicture}
		\begin{axis}[EQBSDS500UE,xmode=log]

			% CCS ue_np.mean_max %%%%%%%%%%%%%%%%%%%%%%%%%%%%%%%%%%%%%%%%%%%%%%%%%%%%%%%%%%%%
			\addplot[CCS] coordinates{
				(218.865,0.179397)
				(303.21,0.155508)
				(453.82,0.129268)
				(706.35,0.107203)
				(871.12,0.0988991)
				(1204.99,0.0876399)
				(1438.62,0.0821357)
				(1438.62,0.0821357)
				(1762.49,0.076469)
				(2107.24,0.0720313)
				(2107.24,0.0720313)
				(2702.61,0.0660912)
				(3406.7,0.0612614)
				(3406.7,0.0612614)
				(3406.7,0.0612614)
				(4654.44,0.0554056)
				(4654.44,0.0554056)
				(6619.1,0.0489194)
			};

			% SEEDS ue_np.mean_max %%%%%%%%%%%%%%%%%%%%%%%%%%%%%%%%%%%%%%%%%%%%%%%%%%%%%%%%%%%%
			\addplot[SEEDS] coordinates{
				(261.62,0.223076)
				(365.675,0.19367)
				(468.81,0.166744)
				(670.57,0.144635)
				(870.75,0.127624)
				(1087.4,0.120216)
				(1270.11,0.110512)
				(1451.85,0.104601)
				(1669.18,0.0958062)
				(1873.19,0.091841)
				(2104.62,0.102696)
				(2462.77,0.0825942)
				(2793.43,0.0820788)
				(3260.86,0.0738429)
				(3895.78,0.0707603)
				(3895.78,0.0707603)
				(4846.12,0.0633063)
				(4846.12,0.0633063)
			};

			% SLIC ue_np.mean_max %%%%%%%%%%%%%%%%%%%%%%%%%%%%%%%%%%%%%%%%%%%%%%%%%%%%%%%%%%%%
			\addplot[SLIC] coordinates{
				(180.32,0.174507)
				(256.725,0.14981)
				(368.57,0.133617)
				(575.335,0.113918)
				(726.36,0.10487)
				(1002.87,0.0957898)
				(1203.61,0.0912283)
				(1203.61,0.0912283)
				(1475.69,0.0857781)
				(1814.8,0.0785351)
				(1814.8,0.0785351)
				(2334.56,0.0724657)
				(3038.13,0.0657068)
				(3038.13,0.0657068)
				(3038.13,0.0657068)
				(4188.97,0.0589157)
				(4188.97,0.0589157)
				(6139.41,0.0512524)
			};

			% RW ue_np.mean_max %%%%%%%%%%%%%%%%%%%%%%%%%%%%%%%%%%%%%%%%%%%%%%%%%%%%%%%%%%%%
			\addplot[RW] coordinates{
				(211.565,0.22099)
				(313.64,0.18514)
				(407.125,0.163871)
				(649.355,0.131395)
				(858.65,0.117308)
				(1032.29,0.10929)
				(1280.94,0.0999393)
				(1505.72,0.0932428)
				(1649.74,0.0900246)
				(1940.98,0.0854435)
				(2101.2,0.0821806)
				(2570.18,0.0765988)
				(2916.36,0.0728736)
				(3413.13,0.0694277)
				(3882.85,0.0661343)
				(4457.4,0.0624712)
				(5093.4,0.0609616)
				(5663.8,0.0527803)
			};

			% CW ue_np.mean_max %%%%%%%%%%%%%%%%%%%%%%%%%%%%%%%%%%%%%%%%%%%%%%%%%%%%%%%%%%%%
			\addplot[CW] coordinates{
				(197.775,0.193019)
				(309.29,0.160417)
				(405.935,0.144929)
				(609.18,0.125212)
				(807.005,0.112944)
				(1045.56,0.10316)
				(1252.72,0.0974855)
				(1492.83,0.0916924)
				(1649.62,0.0889166)
				(1816.43,0.0858197)
				(2091.79,0.0822643)
				(2565.8,0.0769704)
				(2917.57,0.0738519)
				(3319.52,0.0706267)
				(4023.98,0.0661988)
				(4023.98,0.0661988)
				(4639.83,0.0628539)
				(5846.73,0.0586372)
			};

			% TP ue_np.mean_max %%%%%%%%%%%%%%%%%%%%%%%%%%%%%%%%%%%%%%%%%%%%%%%%%%%%%%%%%%%%
			\addplot[TP] coordinates{
				(283.695,0.170092)
				(381.285,0.146071)
				(555.1,0.121786)
				(752.29,0.110206)
				(1007.04,0.0999947)
				(1292.4,0.09124)
				(1495.14,0.0874474)
				(2017.73,0.0993508)
				(2478.02,0.0830604)
				(2831,0.0770876)
				(3018.75,0.079288)
			};

			% POISE ue_np.mean_max %%%%%%%%%%%%%%%%%%%%%%%%%%%%%%%%%%%%%%%%%%%%%%%%%%%%%%%%%%%%
			\addplot[POISE] coordinates{
				(204.855,0.165397)
				(306.68,0.140885)
				(408.385,0.128172)
				(611.79,0.112118)
				(814.975,0.103144)
				(1017.48,0.0972514)
				(1216.26,0.0924521)
				(1409.36,0.0894045)
				(1587.37,0.0869714)
				(1749.07,0.085415)
				(1887.06,0.0842683)
				(2090.32,0.08283)
				(2221.84,0.0821898)
				(2278.76,0.0819765)
				(2287.5,0.0819418)
				(2288.94,0.0819369)
				(2288.94,0.0819369)
				(2288.94,0.0819369)
			};

			% FH ue_np.mean_max %%%%%%%%%%%%%%%%%%%%%%%%%%%%%%%%%%%%%%%%%%%%%%%%%%%%%%%%%%%%
			\addplot[FH] coordinates{
				(628.745,0.182601)
				(799.09,0.150389)
				(963.36,0.140866)
				(1090.39,0.131649)
				(1187.04,0.125432)
				(1605.71,0.110407)
				(2533.01,0.0877137)
				(3000.74,0.0845623)
				(3219.63,0.0821065)
				(3814.42,0.0819198)
				(4746.96,0.0709097)
			};

			% EAMS ue_np.mean_max %%%%%%%%%%%%%%%%%%%%%%%%%%%%%%%%%%%%%%%%%%%%%%%%%%%%%%%%%%%%
			\addplot[EAMS] coordinates{
				(261.12,0.165201)
				(283.33,0.159604)
				(309.87,0.153536)
				(383.52,0.140069)
				(499.435,0.126177)
				(725.36,0.110582)
				(1417.35,0.0873781)
				(1473.6,0.0862626)
				(1534.33,0.0851062)
				(1883.59,0.0823172)
				(2063.65,0.079916)
				(2346.71,0.0765394)
				(2642.78,0.0737047)
				(3113.25,0.0724)
				(3640.44,0.0687855)
				(4914.12,0.0624623)
				(8189.58,0.0539269)
			};

			% CRS ue_np.mean_max %%%%%%%%%%%%%%%%%%%%%%%%%%%%%%%%%%%%%%%%%%%%%%%%%%%%%%%%%%%%
			\addplot[CRS] coordinates{
				(299.575,0.143179)
				(449.585,0.123216)
				(635.865,0.108806)
				(895.935,0.0960168)
				(1180.31,0.0873514)
				(1530.1,0.0799281)
				(1758.22,0.0763857)
				(2057.15,0.0722613)
				(2308.66,0.0694392)
				(2461.75,0.0678385)
				(2755.4,0.065272)
				(3401.31,0.0606573)
				(3879.51,0.057869)
				(4346.83,0.0557048)
				(5208.23,0.0521298)
				(5208.23,0.0521298)
				(5821.7,0.0499159)
				(7241.41,0.0460821)
			};

			% SEAW ue_np.mean_max %%%%%%%%%%%%%%%%%%%%%%%%%%%%%%%%%%%%%%%%%%%%%%%%%%%%%%%%%%%%
			\addplot[SEAW] coordinates{
				(165.505,0.374019)
				(356.805,0.224315)
				(923.49,0.130425)
				(2933.9,0.0762253)
				(10727.5,0.0448995)
			};

			% RESEEDS ue_np.mean_max %%%%%%%%%%%%%%%%%%%%%%%%%%%%%%%%%%%%%%%%%%%%%%%%%%%%%%%%%%%%
			\addplot[RESEEDS] coordinates{
				(200.795,0.185835)
				(301.525,0.163531)
				(401.465,0.132874)
				(602.33,0.113933)
				(800.99,0.10611)
				(1020.12,0.0994214)
				(1201.34,0.0923218)
				(1378.1,0.087877)
				(1601.55,0.0800663)
				(1802.11,0.0781975)
				(2040.11,0.0832874)
				(2402.13,0.0699599)
				(2720.13,0.0694846)
				(3200.2,0.0644983)
				(3840.22,0.0617858)
				(3840.22,0.0617858)
				(4800.37,0.0561646)
				(4800.37,0.0561646)
			};

			% ERGC ue_np.mean_max %%%%%%%%%%%%%%%%%%%%%%%%%%%%%%%%%%%%%%%%%%%%%%%%%%%%%%%%%%%%
			\addplot[ERGC] coordinates{
				(196,0.159735)
				(306,0.133944)
				(400,0.121812)
				(600,0.104949)
				(792.57,0.0954413)
				(1024,0.0878287)
				(1224,0.0831799)
				(1453.2,0.0784696)
				(1600,0.0761292)
				(1760,0.0737437)
				(2024,0.0707404)
				(2472.98,0.0666267)
				(2809,0.0641758)
				(3180,0.0612931)
				(3840,0.0581234)
				(3840,0.0581234)
				(4416,0.0553687)
				(5520,0.0518964)
			};

			% PF ue_np.mean_max %%%%%%%%%%%%%%%%%%%%%%%%%%%%%%%%%%%%%%%%%%%%%%%%%%%%%%%%%%%%
			\addplot[PF] coordinates{
				(281.06,0.333921)
				(428.82,0.293331)
				(585.08,0.265215)
				(928.805,0.227506)
				(1172.76,0.207163)
				(1602.65,0.186339)
				(1861.72,0.174457)
				(2267.11,0.16405)
				(2697.08,0.153131)
				(3382.15,0.144579)
				(4207.66,0.133761)
				(6051.74,0.116788)
			};

			% TPS ue_np.mean_max %%%%%%%%%%%%%%%%%%%%%%%%%%%%%%%%%%%%%%%%%%%%%%%%%%%%%%%%%%%%
			\addplot[TPS] coordinates{
				(224.26,0.196998)
				(903.595,0.115562)
				(1130.53,0.106762)
				(1332.49,0.100735)
				(1556.56,0.0961695)
				(1736.84,0.0913408)
				(1988.95,0.0869456)
				(2167.04,0.0838316)
				(2656.03,0.0790315)
				(2987.25,0.0744852)
				(3505.61,0.0708157)
				(4030.29,0.0663199)
				(4568.25,0.0642427)
				(5242.04,0.0611118)
				(5978.67,0.0581679)
			};

			% NC ue_np.mean_max %%%%%%%%%%%%%%%%%%%%%%%%%%%%%%%%%%%%%%%%%%%%%%%%%%%%%%%%%%%%
			\addplot[NC] coordinates{
				(223.505,0.146866)
				(321.69,0.130709)
				(416.03,0.121517)
				(594.185,0.111767)
				(763.865,0.105481)
				(922.645,0.101732)
				(1073.6,0.0986864)
				(1213.8,0.0961924)
				(1348.36,0.0944127)
				(1473.43,0.0930103)
				(1591.75,0.0915994)
				(2000.27,0.087882)
			};

			% VC ue_np.mean_max %%%%%%%%%%%%%%%%%%%%%%%%%%%%%%%%%%%%%%%%%%%%%%%%%%%%%%%%%%%%
			\addplot[VC] coordinates{
				(351.72,0.377812)
				(498.095,0.281205)
				(712.82,0.209932)
				(994.125,0.155886)
				(1201.27,0.13293)
				(1375.27,0.118151)
				(1685.08,0.101915)
				(1968.97,0.0920285)
				(2224.75,0.0855216)
				(2476.43,0.0806172)
				(2712.29,0.0769346)
				(2948.04,0.0739811)
				(3169.69,0.0713155)
				(3389.95,0.0689737)
				(3804.08,0.0655697)
				(4196.75,0.0628818)
				(4566.66,0.0605806)
				(4815.37,0.0591216)
				(5157.48,0.0573595)
				(5401.49,0.0558277)
			};

			% PB ue_np.mean_max %%%%%%%%%%%%%%%%%%%%%%%%%%%%%%%%%%%%%%%%%%%%%%%%%%%%%%%%%%%%
			\addplot[PB] coordinates{
				(324.39,0.243429)
				(398.425,0.209432)
				(494.17,0.180692)
				(692.845,0.147203)
				(853.905,0.131623)
				(1129.27,0.115345)
				(1323.88,0.10663)
				(1323.88,0.10663)
				(1601.37,0.0979916)
				(1929.58,0.0897623)
				(1929.58,0.0897623)
				(2453.31,0.0812174)
				(3126.91,0.0732926)
				(3126.91,0.0732926)
				(3126.91,0.0732926)
				(4270.56,0.0646975)
				(4270.56,0.0646975)
				(6122.31,0.0565567)
			};

			% PRESLIC ue_np.mean_max %%%%%%%%%%%%%%%%%%%%%%%%%%%%%%%%%%%%%%%%%%%%%%%%%%%%%%%%%%%%
			\addplot[PRESLIC] coordinates{
				(369,0.15315)
				(581.54,0.127686)
				(734.575,0.117583)
				(1020.3,0.103212)
				(1229.22,0.0957013)
				(1229.22,0.0957013)
				(1511.3,0.0880409)
				(1838.94,0.0822743)
				(1838.94,0.0822743)
				(2375.99,0.0748201)
				(3059.43,0.068826)
				(3059.43,0.068826)
				(3059.43,0.068826)
				(4218.88,0.0614387)
				(4218.88,0.0614387)
				(6137.8,0.0537188)
			};

			% W ue_np.mean_max %%%%%%%%%%%%%%%%%%%%%%%%%%%%%%%%%%%%%%%%%%%%%%%%%%%%%%%%%%%%
			\addplot[W] coordinates{
				(188.285,0.227112)
				(296.48,0.182095)
				(387.92,0.162436)
				(608.055,0.133801)
				(794.47,0.120306)
				(1100.87,0.10578)
				(1302.83,0.0990923)
				(1302.83,0.0990923)
				(1572.26,0.0928368)
				(1960.24,0.086063)
				(1960.24,0.086063)
				(2478.43,0.078717)
				(3292.41,0.0715909)
				(3292.41,0.0715909)
				(3292.41,0.0715909)
				(4439.31,0.0650595)
				(4439.31,0.0650595)
				(6526.67,0.0568372)
			};

			% LSC ue_np.mean_max %%%%%%%%%%%%%%%%%%%%%%%%%%%%%%%%%%%%%%%%%%%%%%%%%%%%%%%%%%%%
			\addplot[LSC] coordinates{
				(548.755,0.143054)
				(756.92,0.122502)
				(1045.31,0.107074)
				(1360.95,0.0969879)
				(1665.85,0.0900165)
				(1919.65,0.0855723)
				(2055.98,0.0836367)
				(2239.5,0.081482)
				(2376.71,0.0801275)
				(2441.63,0.0791931)
				(2603.55,0.0777619)
				(2909.2,0.0759016)
				(3147.7,0.0749653)
				(3373.12,0.0732703)
				(3762.88,0.0734845)
				(3762.88,0.0734845)
				(4057.29,0.0738871)
				(4401.67,0.0803919)
			};

			% WP ue_np.mean_max %%%%%%%%%%%%%%%%%%%%%%%%%%%%%%%%%%%%%%%%%%%%%%%%%%%%%%%%%%%%
			\addplot[WP] coordinates{
				(216,0.189755)
				(294,0.164562)
				(384,0.148291)
				(600,0.125216)
				(805,0.112475)
				(1080,0.101169)
				(1320,0.0945852)
				(1536,0.0892273)
				(1536,0.0889869)
				(1944,0.0826232)
				(1944,0.082478)
				(2400,0.0765399)
				(3174,0.0702651)
				(3174,0.0701382)
				(4320,0.0639556)
				(4320,0.0639556)
				(4320,0.0639556)
				(5836.85,0.057387)
			};

			% QS ue_np.mean_max %%%%%%%%%%%%%%%%%%%%%%%%%%%%%%%%%%%%%%%%%%%%%%%%%%%%%%%%%%%%
			\addplot[QS] coordinates{
				(324.725,0.631546)
				(468.655,0.481517)
				(615,0.381608)
				(884.79,0.27222)
				(984.505,0.24583)
				(1275.62,0.193321)
				(1500.06,0.168415)
				(1646.89,0.155863)
				(1824.2,0.143951)
				(2040.32,0.132493)
				(2303.69,0.121701)
				(2626.4,0.111059)
				(3066.62,0.10053)
				(3646.03,0.0906887)
				(4444.58,0.0812972)
				(5554.87,0.072386)
			};

			% VLSLIC ue_np.mean_max %%%%%%%%%%%%%%%%%%%%%%%%%%%%%%%%%%%%%%%%%%%%%%%%%%%%%%%%%%%%
			\addplot[VLSLIC] coordinates{
				(575.975,0.146745)
				(651.86,0.133883)
				(763.62,0.12431)
				(899,0.112635)
				(988.985,0.106043)
				(1193.67,0.0981205)
				(1348.08,0.0935383)
				(1349.01,0.0929579)
				(1579.08,0.0889043)
				(1849.65,0.0848381)
				(1857.27,0.0842664)
				(2307.73,0.0796918)
				(2890.56,0.0748332)
				(2924.52,0.0736624)
				(2954.18,0.0727692)
				(3858.98,0.0707121)
				(3858.98,0.0707121)
				(4809.57,0.077104)
			};

			% CIS ue_np.mean_max %%%%%%%%%%%%%%%%%%%%%%%%%%%%%%%%%%%%%%%%%%%%%%%%%%%%%%%%%%%%
			\addplot[CIS] coordinates{
				(317.63,0.163639)
				(411.595,0.14103)
				(472.685,0.132016)
				(633.345,0.115086)
				(777.05,0.106079)
				(969.085,0.0968127)
				(1150.5,0.0910555)
				(1168.82,0.0908149)
				(1371.27,0.085803)
				(1637.39,0.080614)
				(1667.02,0.0803034)
				(2056.01,0.0750905)
				(3126.04,0.0675849)
				(3625.6,0.0653815)
				(4558.92,0.0600433)
				(4558.92,0.0600433)
				(6448.65,0.0528125)
			};

			% ERS ue_np.mean_max %%%%%%%%%%%%%%%%%%%%%%%%%%%%%%%%%%%%%%%%%%%%%%%%%%%%%%%%%%%%
			\addplot[ERS] coordinates{
				(200,0.165125)
				(300,0.143245)
				(400,0.128528)
				(600,0.112156)
				(800,0.101957)
				(1000,0.0951928)
				(1200,0.0897034)
				(1400,0.0855294)
				(1600,0.0816775)
				(1800,0.0787337)
				(2000,0.0762525)
				(2400,0.071886)
				(2800,0.0683312)
				(3200,0.0654288)
				(3600,0.0629577)
				(4000,0.0606193)
				(4600,0.0576594)
				(5200,0.0552272)
			};

			% MSS ue_np.mean_max %%%%%%%%%%%%%%%%%%%%%%%%%%%%%%%%%%%%%%%%%%%%%%%%%%%%%%%%%%%%
			\addplot[MSS] coordinates{
				(200.765,0.191002)
				(282.33,0.169581)
				(420.66,0.135999)
				(664.115,0.114166)
				(837.885,0.107277)
				(1179.3,0.0953561)
				(1422.34,0.0892904)
				(1423.04,0.08919)
				(1768,0.0832694)
				(2156.96,0.079052)
				(2157.01,0.0788515)
				(2826.35,0.0723073)
				(3669.97,0.0682022)
				(3670.34,0.0682585)
				(3669.92,0.0681392)
				(5213.45,0.0616563)
				(5213.45,0.0616563)
				(7846.35,0.0546919)
			};

			% ETPS ue_np.mean_max %%%%%%%%%%%%%%%%%%%%%%%%%%%%%%%%%%%%%%%%%%%%%%%%%%%%%%%%%%%%
			\addplot[ETPS] coordinates{
				(216,0.155646)
				(294,0.136541)
				(425,0.120575)
				(651,0.102522)
				(805,0.0941807)
				(1107,0.0842104)
				(1320,0.0803088)
				(1320,0.0803088)
				(1617,0.0745378)
				(1944,0.0703869)
				(1944,0.0703869)
				(2501,0.0644105)
				(3174,0.059511)
				(3174,0.059511)
				(3174,0.059511)
				(4374,0.0532)
				(4374,0.0532)
				(6305,0.047044)
			};

		\end{axis}
	\end{tikzpicture}
\end{subfigure}
\begin{subfigure}[b]{\fullthreeone\textwidth}\phantomsubcaption\label{subfig:experiments-quantitative-bsds500-ev.mean_min}
	%%%%%%%%%%%%%%%%%%%%%%%%%%%%%%%%%%%%%%%%%%%%%%%%%%%%%%%%%%%%
	% ev.mean_min
	%%%%%%%%%%%%%%%%%%%%%%%%%%%%%%%%%%%%%%%%%%%%%%%%%%%%%%%%%%%%
	\begin{tikzpicture}
		\begin{axis}[EQBSDS500EV,xmode=log]

			% CCS ev.mean_min %%%%%%%%%%%%%%%%%%%%%%%%%%%%%%%%%%%%%%%%%%%%%%%%%%%%%%%%%%%%
			\addplot[CCS] coordinates{
				(218.865,0.809137)
				(303.21,0.829039)
				(453.82,0.853854)
				(706.35,0.874843)
				(871.12,0.882958)
				(1204.99,0.895677)
				(1438.62,0.902221)
				(1438.62,0.902221)
				(1762.49,0.909247)
				(2107.24,0.914266)
				(2107.24,0.914266)
				(2702.61,0.921771)
				(3406.7,0.927677)
				(3406.7,0.927677)
				(3406.7,0.927677)
				(4654.44,0.935133)
				(4654.44,0.935133)
				(6619.1,0.943757)
			};

			% SEEDS ev.mean_min %%%%%%%%%%%%%%%%%%%%%%%%%%%%%%%%%%%%%%%%%%%%%%%%%%%%%%%%%%%%
			\addplot[SEEDS] coordinates{
				(261.62,0.864326)
				(365.675,0.878133)
				(468.81,0.889903)
				(670.57,0.901613)
				(870.75,0.90727)
				(1087.4,0.908103)
				(1270.11,0.917012)
				(1451.85,0.915435)
				(1669.18,0.925402)
				(1873.19,0.924017)
				(2104.62,0.917357)
				(2462.77,0.935122)
				(2793.43,0.929949)
				(3260.86,0.938591)
				(3895.78,0.940368)
				(3895.78,0.940368)
				(4846.12,0.946842)
				(4846.12,0.946842)
			};

			% SLIC ev.mean_min %%%%%%%%%%%%%%%%%%%%%%%%%%%%%%%%%%%%%%%%%%%%%%%%%%%%%%%%%%%%
			\addplot[SLIC] coordinates{
				(180.32,0.791807)
				(256.725,0.81628)
				(368.57,0.828517)
				(575.335,0.847241)
				(726.36,0.857305)
				(1002.87,0.866695)
				(1203.61,0.869712)
				(1203.61,0.869712)
				(1475.69,0.876552)
				(1814.8,0.887267)
				(1814.8,0.887267)
				(2334.56,0.894955)
				(3038.13,0.906925)
				(3038.13,0.906925)
				(3038.13,0.906925)
				(4188.97,0.916459)
				(4188.97,0.916459)
				(6139.41,0.9292)
			};

			% RW ev.mean_min %%%%%%%%%%%%%%%%%%%%%%%%%%%%%%%%%%%%%%%%%%%%%%%%%%%%%%%%%%%%
			\addplot[RW] coordinates{
				(211.565,0.72753)
				(313.64,0.758186)
				(407.125,0.779485)
				(649.355,0.812984)
				(858.65,0.831041)
				(1032.29,0.841654)
				(1280.94,0.854335)
				(1505.72,0.863765)
				(1649.74,0.868958)
				(1940.98,0.875739)
				(2101.2,0.880385)
				(2570.18,0.890515)
				(2916.36,0.896794)
				(3413.13,0.903301)
				(3882.85,0.908869)
				(4457.4,0.915601)
				(5093.4,0.918598)
				(5663.8,0.920002)
			};

			% CW ev.mean_min %%%%%%%%%%%%%%%%%%%%%%%%%%%%%%%%%%%%%%%%%%%%%%%%%%%%%%%%%%%%
			\addplot[CW] coordinates{
				(197.775,0.741844)
				(309.29,0.771792)
				(405.935,0.785945)
				(609.18,0.809648)
				(807.005,0.822932)
				(1045.56,0.834955)
				(1252.72,0.842364)
				(1492.83,0.849839)
				(1649.62,0.853315)
				(1816.43,0.857478)
				(2091.79,0.862108)
				(2565.8,0.869358)
				(2917.57,0.873301)
				(3319.52,0.877607)
				(4023.98,0.883105)
				(4023.98,0.883105)
				(4639.83,0.887882)
				(5846.73,0.893661)
			};

			% TP ev.mean_min %%%%%%%%%%%%%%%%%%%%%%%%%%%%%%%%%%%%%%%%%%%%%%%%%%%%%%%%%%%%
			\addplot[TP] coordinates{
				(283.695,0.77349)
				(381.285,0.797874)
				(555.1,0.822437)
				(752.29,0.835191)
				(1007.04,0.846467)
				(1292.4,0.857764)
				(1495.14,0.8623)
				(2017.73,0.844457)
				(2478.02,0.861952)
				(2831,0.870847)
				(3018.75,0.865151)
			};

			% POISE ev.mean_min %%%%%%%%%%%%%%%%%%%%%%%%%%%%%%%%%%%%%%%%%%%%%%%%%%%%%%%%%%%%
			\addplot[POISE] coordinates{
				(204.855,0.80409)
				(306.68,0.823027)
				(408.385,0.832908)
				(611.79,0.844348)
				(814.975,0.850242)
				(1017.48,0.854007)
				(1216.26,0.856647)
				(1409.36,0.858246)
				(1587.37,0.859526)
				(1749.07,0.860502)
				(1887.06,0.861133)
				(2090.32,0.861845)
				(2221.84,0.862198)
				(2278.76,0.862301)
				(2287.5,0.862316)
				(2288.94,0.862321)
				(2288.94,0.862321)
				(2288.94,0.862321)
			};

			% FH ev.mean_min %%%%%%%%%%%%%%%%%%%%%%%%%%%%%%%%%%%%%%%%%%%%%%%%%%%%%%%%%%%%
			\addplot[FH] coordinates{
				(628.745,0.750581)
				(782.42,0.778063)
				(799.09,0.80952)
				(963.36,0.802892)
				(1090.39,0.831299)
				(1187.04,0.82472)
				(1605.71,0.85807)
				(2533.01,0.888635)
				(3000.74,0.898733)
				(3219.63,0.89966)
				(3814.42,0.913759)
				(4746.96,0.928309)
			};

			% EAMS ev.mean_min %%%%%%%%%%%%%%%%%%%%%%%%%%%%%%%%%%%%%%%%%%%%%%%%%%%%%%%%%%%%
			\addplot[EAMS] coordinates{
				(261.12,0.855933)
				(283.33,0.860085)
				(309.87,0.864383)
				(383.52,0.874373)
				(499.435,0.885485)
				(725.36,0.900102)
				(1417.35,0.922866)
				(1473.6,0.924071)
				(1534.33,0.925302)
				(1883.59,0.924873)
				(2063.65,0.927575)
				(2346.71,0.93132)
				(2642.78,0.934711)
				(3113.25,0.939055)
				(3640.44,0.943368)
				(4914.12,0.951086)
				(8189.58,0.96371)
			};

			% CRS ev.mean_min %%%%%%%%%%%%%%%%%%%%%%%%%%%%%%%%%%%%%%%%%%%%%%%%%%%%%%%%%%%%
			\addplot[CRS] coordinates{
				(299.575,0.816594)
				(449.585,0.835927)
				(635.865,0.852047)
				(895.935,0.864847)
				(1180.31,0.874129)
				(1530.1,0.884079)
				(1758.22,0.887991)
				(2057.15,0.892804)
				(2308.66,0.896625)
				(2461.75,0.89816)
				(2755.4,0.90097)
				(3401.31,0.907102)
				(3879.51,0.910772)
				(4346.83,0.91374)
				(5208.23,0.918765)
				(5208.23,0.918765)
				(5821.7,0.921736)
				(7241.41,0.92709)
			};

			% SEAW ev.mean_min %%%%%%%%%%%%%%%%%%%%%%%%%%%%%%%%%%%%%%%%%%%%%%%%%%%%%%%%%%%%
			\addplot[SEAW] coordinates{
				(165.505,0.691088)
				(356.805,0.785475)
				(923.49,0.855072)
				(2933.9,0.908834)
				(10727.5,0.950532)
			};

			% RESEEDS ev.mean_min %%%%%%%%%%%%%%%%%%%%%%%%%%%%%%%%%%%%%%%%%%%%%%%%%%%%%%%%%%%%
			\addplot[RESEEDS] coordinates{
				(200.795,0.875471)
				(301.525,0.885472)
				(401.465,0.898927)
				(602.33,0.909438)
				(800.99,0.915782)
				(1020.12,0.919815)
				(1201.34,0.923969)
				(1378.1,0.926092)
				(1601.55,0.931503)
				(1802.11,0.933054)
				(2040.11,0.929723)
				(2402.13,0.939844)
				(2720.13,0.939208)
				(3200.2,0.945133)
				(3840.22,0.947216)
				(3840.22,0.947216)
				(4800.37,0.95236)
				(4800.37,0.95236)
			};

			% ERGC ev.mean_min %%%%%%%%%%%%%%%%%%%%%%%%%%%%%%%%%%%%%%%%%%%%%%%%%%%%%%%%%%%%
			\addplot[ERGC] coordinates{
				(196,0.810607)
				(306,0.836508)
				(400,0.850882)
				(600,0.871094)
				(792.57,0.883383)
				(1024,0.893615)
				(1224,0.901017)
				(1453.2,0.907519)
				(1600,0.910855)
				(1760,0.914309)
				(2024,0.919185)
				(2472.98,0.925411)
				(2809,0.929305)
				(3180,0.933208)
				(3840,0.938679)
				(3840,0.938679)
				(4416,0.942654)
				(5520,0.948197)
			};

			% PF ev.mean_min %%%%%%%%%%%%%%%%%%%%%%%%%%%%%%%%%%%%%%%%%%%%%%%%%%%%%%%%%%%%
			\addplot[PF] coordinates{
				(281.06,0.616952)
				(428.82,0.649894)
				(585.08,0.673819)
				(928.805,0.707895)
				(1172.76,0.725262)
				(1602.65,0.746248)
				(1861.72,0.756351)
				(2267.11,0.767953)
				(2697.08,0.780059)
				(3382.15,0.795306)
				(4207.66,0.808168)
				(6051.74,0.829325)
			};

			% TPS ev.mean_min %%%%%%%%%%%%%%%%%%%%%%%%%%%%%%%%%%%%%%%%%%%%%%%%%%%%%%%%%%%%
			\addplot[TPS] coordinates{
				(224.26,0.708985)
				(903.595,0.798748)
				(1130.53,0.809869)
				(1332.49,0.819166)
				(1556.56,0.826064)
				(1736.84,0.832515)
				(1988.95,0.838637)
				(2167.04,0.843343)
				(2656.03,0.851562)
				(2987.25,0.857993)
				(3505.61,0.863974)
				(4030.29,0.870507)
				(4568.25,0.873736)
				(5242.04,0.879277)
				(5978.67,0.883218)
			};

			% NC ev.mean_min %%%%%%%%%%%%%%%%%%%%%%%%%%%%%%%%%%%%%%%%%%%%%%%%%%%%%%%%%%%%
			\addplot[NC] coordinates{
				(223.505,0.767223)
				(321.69,0.786324)
				(416.03,0.797655)
				(594.185,0.81064)
				(763.865,0.818572)
				(922.645,0.8235)
				(1073.6,0.827505)
				(1213.8,0.830478)
				(1348.36,0.832702)
				(1473.43,0.834559)
				(1591.75,0.836164)
				(2000.27,0.840558)
			};

			% VC ev.mean_min %%%%%%%%%%%%%%%%%%%%%%%%%%%%%%%%%%%%%%%%%%%%%%%%%%%%%%%%%%%%
			\addplot[VC] coordinates{
				(351.72,0.678482)
				(498.095,0.737636)
				(712.82,0.787983)
				(994.125,0.830143)
				(1201.27,0.850894)
				(1375.27,0.863532)
				(1685.08,0.880072)
				(1968.97,0.88987)
				(2224.75,0.897604)
				(2476.43,0.90349)
				(2712.29,0.907597)
				(2948.04,0.911978)
				(3169.69,0.915256)
				(3389.95,0.918398)
				(3804.08,0.922972)
				(4196.75,0.927163)
				(4566.66,0.930878)
				(4815.37,0.932959)
				(5157.48,0.936545)
				(5401.49,0.938659)
			};

			% PB ev.mean_min %%%%%%%%%%%%%%%%%%%%%%%%%%%%%%%%%%%%%%%%%%%%%%%%%%%%%%%%%%%%
			\addplot[PB] coordinates{
				(324.39,0.674466)
				(398.425,0.70544)
				(494.17,0.731343)
				(692.845,0.766402)
				(853.905,0.784059)
				(1129.27,0.806268)
				(1323.88,0.816981)
				(1323.88,0.816981)
				(1601.37,0.830332)
				(1929.58,0.842179)
				(1929.58,0.842179)
				(2453.31,0.85772)
				(3126.91,0.871523)
				(3126.91,0.871523)
				(3126.91,0.871523)
				(4270.56,0.889069)
				(4270.56,0.889069)
				(6122.31,0.907693)
			};

			% PRESLIC ev.mean_min %%%%%%%%%%%%%%%%%%%%%%%%%%%%%%%%%%%%%%%%%%%%%%%%%%%%%%%%%%%%
			\addplot[PRESLIC] coordinates{
				(369,0.809835)
				(581.54,0.834095)
				(734.575,0.843746)
				(1020.3,0.860065)
				(1229.22,0.868563)
				(1229.22,0.868563)
				(1511.3,0.878709)
				(1838.94,0.885851)
				(1838.94,0.885851)
				(2375.99,0.896889)
				(3059.43,0.904858)
				(3059.43,0.904858)
				(3059.43,0.904858)
				(4218.88,0.916974)
				(4218.88,0.916974)
				(6137.8,0.929625)
			};

			% W ev.mean_min %%%%%%%%%%%%%%%%%%%%%%%%%%%%%%%%%%%%%%%%%%%%%%%%%%%%%%%%%%%%
			\addplot[W] coordinates{
				(188.285,0.717307)
				(296.48,0.754624)
				(387.92,0.773384)
				(608.055,0.802638)
				(794.47,0.816675)
				(1100.87,0.833498)
				(1302.83,0.841242)
				(1302.83,0.841242)
				(1572.26,0.849824)
				(1960.24,0.858543)
				(1960.24,0.858543)
				(2478.43,0.867261)
				(3292.41,0.876992)
				(3292.41,0.876992)
				(3292.41,0.876992)
				(4439.31,0.885738)
				(4439.31,0.885738)
				(6526.67,0.896176)
			};

			% LSC ev.mean_min %%%%%%%%%%%%%%%%%%%%%%%%%%%%%%%%%%%%%%%%%%%%%%%%%%%%%%%%%%%%
			\addplot[LSC] coordinates{
				(548.755,0.852346)
				(756.92,0.866151)
				(1045.31,0.875002)
				(1360.95,0.882822)
				(1665.85,0.887123)
				(1919.65,0.887127)
				(2055.98,0.888295)
				(2239.5,0.887957)
				(2376.71,0.888437)
				(2441.63,0.889243)
				(2603.55,0.889936)
				(2909.2,0.88941)
				(3147.7,0.889376)
				(3373.12,0.89085)
				(3762.88,0.888239)
				(3762.88,0.888239)
				(4057.29,0.886462)
				(4401.67,0.87346)
			};

			% WP ev.mean_min %%%%%%%%%%%%%%%%%%%%%%%%%%%%%%%%%%%%%%%%%%%%%%%%%%%%%%%%%%%%
			\addplot[WP] coordinates{
				(216,0.756107)
				(294,0.780453)
				(384,0.797386)
				(600,0.824106)
				(805,0.839082)
				(1080,0.853854)
				(1320,0.862917)
				(1536,0.870111)
				(1536,0.87012)
				(1944,0.879776)
				(1944,0.879752)
				(2400,0.887579)
				(3174,0.897171)
				(3174,0.896743)
				(4320,0.905927)
				(4320,0.905927)
				(4320,0.905927)
				(5836.85,0.918459)
			};

			% QS ev.mean_min %%%%%%%%%%%%%%%%%%%%%%%%%%%%%%%%%%%%%%%%%%%%%%%%%%%%%%%%%%%%
			\addplot[QS] coordinates{
				(324.725,0.415947)
				(468.655,0.572627)
				(615,0.669826)
				(884.79,0.763505)
				(984.505,0.785499)
				(1275.62,0.828283)
				(1500.06,0.848396)
				(1646.89,0.85898)
				(1824.2,0.86917)
				(2040.32,0.879482)
				(2303.69,0.889904)
				(2626.4,0.900262)
				(3066.62,0.910971)
				(3646.03,0.921853)
				(4444.58,0.933269)
				(5554.87,0.944553)
			};

			% VLSLIC ev.mean_min %%%%%%%%%%%%%%%%%%%%%%%%%%%%%%%%%%%%%%%%%%%%%%%%%%%%%%%%%%%%
			\addplot[VLSLIC] coordinates{
				(575.975,0.860969)
				(651.86,0.862461)
				(763.62,0.864765)
				(899,0.866607)
				(988.985,0.869824)
				(1193.67,0.874034)
				(1348.08,0.878199)
				(1349.01,0.879457)
				(1579.08,0.88191)
				(1849.65,0.884423)
				(1857.27,0.885815)
				(2307.73,0.88957)
				(2890.56,0.893298)
				(2924.52,0.896099)
				(2954.18,0.898575)
				(3858.98,0.895584)
				(3858.98,0.895584)
				(4809.57,0.877434)
			};

			% CIS ev.mean_min %%%%%%%%%%%%%%%%%%%%%%%%%%%%%%%%%%%%%%%%%%%%%%%%%%%%%%%%%%%%
			\addplot[CIS] coordinates{
				(317.63,0.813648)
				(411.595,0.833303)
				(472.685,0.839585)
				(633.345,0.854343)
				(777.05,0.864218)
				(969.085,0.873552)
				(1150.5,0.884972)
				(1168.82,0.885765)
				(1371.27,0.890196)
				(1637.39,0.898343)
				(1667.02,0.89935)
				(2056.01,0.904708)
				(3126.04,0.923756)
				(3625.6,0.929426)
				(4558.92,0.934446)
				(4558.92,0.934446)
				(6448.65,0.950469)
			};

			% ERS ev.mean_min %%%%%%%%%%%%%%%%%%%%%%%%%%%%%%%%%%%%%%%%%%%%%%%%%%%%%%%%%%%%
			\addplot[ERS] coordinates{
				(200,0.764612)
				(300,0.78696)
				(400,0.801407)
				(600,0.820992)
				(800,0.833156)
				(1000,0.842564)
				(1200,0.84967)
				(1400,0.855317)
				(1600,0.860317)
				(1800,0.864636)
				(2000,0.86831)
				(2400,0.874652)
				(2800,0.879858)
				(3200,0.884819)
				(3600,0.888949)
				(4000,0.892749)
				(4600,0.897743)
				(5200,0.902102)
			};

			% MSS ev.mean_min %%%%%%%%%%%%%%%%%%%%%%%%%%%%%%%%%%%%%%%%%%%%%%%%%%%%%%%%%%%%
			\addplot[MSS] coordinates{
				(200.765,0.765432)
				(282.33,0.784005)
				(420.66,0.816051)
				(664.115,0.838138)
				(837.885,0.844171)
				(1179.3,0.857181)
				(1422.34,0.863212)
				(1423.04,0.86313)
				(1768,0.869476)
				(2156.96,0.872884)
				(2157.01,0.87296)
				(2826.35,0.879769)
				(3669.97,0.884159)
				(3670.34,0.883979)
				(3669.92,0.884121)
				(5213.45,0.891472)
				(5213.45,0.891472)
				(7846.35,0.898827)
			};

			% ETPS ev.mean_min %%%%%%%%%%%%%%%%%%%%%%%%%%%%%%%%%%%%%%%%%%%%%%%%%%%%%%%%%%%%
			\addplot[ETPS] coordinates{
				(216,0.898375)
				(294,0.907102)
				(425,0.916049)
				(651,0.924757)
				(805,0.92899)
				(1107,0.936142)
				(1320,0.938489)
				(1320,0.938489)
				(1617,0.942479)
				(1944,0.945321)
				(1944,0.945321)
				(2501,0.950012)
				(3174,0.953272)
				(3174,0.953272)
				(3174,0.953272)
				(4374,0.959285)
				(4374,0.959285)
				(6305,0.964332)
			};

		\end{axis}
	\end{tikzpicture}
\end{subfigure}\\[-4px]
\begin{subfigure}[b]{\fullthreeone\textwidth}\phantomsubcaption\label{subfig:experiments-quantitative-bsds500-rec.min_min}
	%%%%%%%%%%%%%%%%%%%%%%%%%%%%%%%%%%%%%%%%%%%%%%%%%%%%%%%%%%%%
	% rec.min_min
	%%%%%%%%%%%%%%%%%%%%%%%%%%%%%%%%%%%%%%%%%%%%%%%%%%%%%%%%%%%%
	\begin{tikzpicture}
		\begin{axis}[EQBSDS500RecMin,xmode=log]

			% CCS rec.min_min %%%%%%%%%%%%%%%%%%%%%%%%%%%%%%%%%%%%%%%%%%%%%%%%%%%%%%%%%%%%
			\addplot[CCS] coordinates{
				(218.865,0.367358)
				(303.21,0.410731)
				(453.82,0.479186)
				(706.35,0.590753)
				(871.12,0.636807)
				(1204.99,0.7067)
				(1438.62,0.741557)
				(1438.62,0.741557)
				(1762.49,0.762023)
				(2107.24,0.819397)
				(2107.24,0.819397)
				(2702.61,0.840347)
				(3406.7,0.884053)
				(3406.7,0.884053)
				(3406.7,0.884053)
				(4654.44,0.925953)
				(4654.44,0.925953)
				(6619.1,0.965112)
			};

			% SEEDS rec.min_min %%%%%%%%%%%%%%%%%%%%%%%%%%%%%%%%%%%%%%%%%%%%%%%%%%%%%%%%%%%%
			\addplot[SEEDS] coordinates{
				(261.62,0.715561)
				(365.675,0.767947)
				(468.81,0.748933)
				(670.57,0.863522)
				(870.75,0.886587)
				(1087.4,0.917168)
				(1270.11,0.925411)
				(1451.85,0.941632)
				(1669.18,0.927697)
				(1873.19,0.956133)
				(2104.62,0.96328)
				(2462.77,0.944395)
				(2793.43,0.969632)
				(3260.86,0.97492)
				(3895.78,0.974483)
				(3895.78,0.974483)
				(4846.12,0.985926)
				(4846.12,0.985926)
			};

			% SLIC rec.min_min %%%%%%%%%%%%%%%%%%%%%%%%%%%%%%%%%%%%%%%%%%%%%%%%%%%%%%%%%%%%
			\addplot[SLIC] coordinates{
				(180.32,0.457059)
				(256.725,0.482344)
				(368.57,0.541715)
				(575.335,0.587124)
				(726.36,0.656673)
				(1002.87,0.702847)
				(1203.61,0.719943)
				(1203.61,0.719943)
				(1475.69,0.786775)
				(1814.8,0.817452)
				(1814.8,0.817452)
				(2334.56,0.847751)
				(3038.13,0.873217)
				(3038.13,0.873217)
				(3038.13,0.873217)
				(4188.97,0.935526)
				(4188.97,0.935526)
				(6139.41,0.967672)
			};

			% RW rec.min_min %%%%%%%%%%%%%%%%%%%%%%%%%%%%%%%%%%%%%%%%%%%%%%%%%%%%%%%%%%%%
			\addplot[RW] coordinates{
				(211.565,0.305516)
				(313.64,0.368431)
				(407.125,0.4311)
				(649.355,0.493047)
				(858.65,0.567275)
				(1032.29,0.622725)
				(1280.94,0.670125)
				(1505.72,0.705052)
				(1649.74,0.719343)
				(1940.98,0.752032)
				(2101.2,0.762868)
				(2570.18,0.803708)
				(2916.36,0.840229)
				(3413.13,0.869706)
				(3882.85,0.902163)
				(4457.4,0.917102)
				(5093.4,0.925495)
				(5663.8,0.968258)
			};

			% CW rec.min_min %%%%%%%%%%%%%%%%%%%%%%%%%%%%%%%%%%%%%%%%%%%%%%%%%%%%%%%%%%%%
			\addplot[CW] coordinates{
				(197.775,0.388397)
				(309.29,0.42927)
				(405.935,0.474996)
				(609.18,0.586243)
				(807.005,0.623304)
				(1045.56,0.676061)
				(1252.72,0.715415)
				(1492.83,0.755456)
				(1649.62,0.77494)
				(1816.43,0.778656)
				(2091.79,0.820244)
				(2565.8,0.841897)
				(2917.57,0.874146)
				(3319.52,0.905016)
				(4023.98,0.93413)
				(4023.98,0.93413)
				(4639.83,0.952023)
				(5846.73,0.974031)
			};

			% TP rec.min_min %%%%%%%%%%%%%%%%%%%%%%%%%%%%%%%%%%%%%%%%%%%%%%%%%%%%%%%%%%%%
			\addplot[TP] coordinates{
				(283.695,0.340479)
				(381.285,0.401246)
				(555.1,0.469388)
				(752.29,0.541087)
				(1007.04,0.610981)
				(1292.4,0.635904)
				(1495.14,0.671984)
				(1696.62,0.68512)
				(2017.73,0.728793)
				(2478.02,0.782389)
				(2831,0.818295)
				(3018.75,0.83447)
			};

			% POISE rec.min_min %%%%%%%%%%%%%%%%%%%%%%%%%%%%%%%%%%%%%%%%%%%%%%%%%%%%%%%%%%%%
			\addplot[POISE] coordinates{
				(204.855,0.513288)
				(306.68,0.550981)
				(408.385,0.568699)
				(611.79,0.618076)
				(814.975,0.649878)
				(1017.48,0.677761)
				(1216.26,0.687079)
				(1409.36,0.701132)
				(1587.37,0.710007)
				(1749.07,0.718364)
				(1887.06,0.718364)
				(2090.32,0.718364)
				(2221.84,0.718364)
				(2278.76,0.718364)
				(2287.5,0.718364)
				(2288.94,0.718364)
				(2288.94,0.718364)
				(2288.94,0.718364)
			};

			% FH rec.min_min %%%%%%%%%%%%%%%%%%%%%%%%%%%%%%%%%%%%%%%%%%%%%%%%%%%%%%%%%%%%
			\addplot[FH] coordinates{
				(628.745,0.475024)
				(782.42,0.46851)
				(799.09,0.465625)
				(963.36,0.574311)
				(1090.39,0.526923)
				(1187.04,0.620192)
				(1605.71,0.638202)
				(2533.01,0.73189)
				(3000.74,0.798247)
				(3219.63,0.823165)
				(3814.42,0.725834)
				(4746.96,0.768219)
			};

			% EAMS rec.min_min %%%%%%%%%%%%%%%%%%%%%%%%%%%%%%%%%%%%%%%%%%%%%%%%%%%%%%%%%%%%
			\addplot[EAMS] coordinates{
				(261.12,0.588369)
				(283.33,0.588671)
				(309.87,0.589879)
				(383.52,0.619335)
				(499.435,0.645317)
				(725.36,0.67281)
				(1417.35,0.71994)
				(1473.6,0.721903)
				(1534.33,0.722508)
				(1883.59,0.676737)
				(2063.65,0.679607)
				(2346.71,0.686103)
				(2642.78,0.693958)
				(3113.25,0.655136)
				(3640.44,0.67281)
				(4914.12,0.679456)
				(8189.58,0.688671)
			};

			% CRS rec.min_min %%%%%%%%%%%%%%%%%%%%%%%%%%%%%%%%%%%%%%%%%%%%%%%%%%%%%%%%%%%%
			\addplot[CRS] coordinates{
				(299.575,0.534846)
				(449.585,0.630111)
				(635.865,0.688357)
				(895.935,0.747081)
				(1180.31,0.791294)
				(1530.1,0.81408)
				(1758.22,0.845822)
				(2057.15,0.874051)
				(2308.66,0.882553)
				(2461.75,0.895781)
				(2755.4,0.911915)
				(3401.31,0.931216)
				(3879.51,0.943544)
				(4346.83,0.957478)
				(5208.23,0.970258)
				(5208.23,0.970258)
				(5821.7,0.978192)
				(7241.41,0.988419)
			};

			% SEAW rec.min_min %%%%%%%%%%%%%%%%%%%%%%%%%%%%%%%%%%%%%%%%%%%%%%%%%%%%%%%%%%%%
			\addplot[SEAW] coordinates{
				(165.505,0.19162)
				(356.805,0.33325)
				(923.49,0.53603)
				(2933.9,0.808019)
				(10727.5,0.996114)
			};

			% RESEEDS rec.min_min %%%%%%%%%%%%%%%%%%%%%%%%%%%%%%%%%%%%%%%%%%%%%%%%%%%%%%%%%%%%
			\addplot[RESEEDS] coordinates{
				(200.795,0.730664)
				(301.525,0.794586)
				(401.465,0.826678)
				(602.33,0.874571)
				(800.99,0.893805)
				(1020.12,0.901751)
				(1201.34,0.917465)
				(1378.1,0.932605)
				(1601.55,0.937754)
				(1802.11,0.94439)
				(2040.11,0.953778)
				(2402.13,0.961712)
				(2720.13,0.967599)
				(3200.2,0.973096)
				(3840.22,0.980061)
				(3840.22,0.980061)
				(4800.37,0.980661)
				(4800.37,0.980661)
			};

			% ERGC rec.min_min %%%%%%%%%%%%%%%%%%%%%%%%%%%%%%%%%%%%%%%%%%%%%%%%%%%%%%%%%%%%
			\addplot[ERGC] coordinates{
				(196,0.425301)
				(306,0.526104)
				(400,0.594546)
				(600,0.65894)
				(792.57,0.7232)
				(1024,0.780702)
				(1224,0.79317)
				(1453.2,0.826461)
				(1600,0.845223)
				(1760,0.856333)
				(2024,0.883608)
				(2472.98,0.900585)
				(2809,0.921246)
				(3180,0.935252)
				(3840,0.951737)
				(3840,0.951737)
				(4416,0.963013)
				(5520,0.975643)
			};

			% PF rec.min_min %%%%%%%%%%%%%%%%%%%%%%%%%%%%%%%%%%%%%%%%%%%%%%%%%%%%%%%%%%%%
			\addplot[PF] coordinates{
				(281.06,0.206705)
				(428.82,0.244695)
				(585.08,0.28371)
				(928.805,0.341547)
				(1172.76,0.398015)
				(1602.65,0.449008)
				(1861.72,0.510741)
				(2267.11,0.527378)
				(2697.08,0.557582)
				(3382.15,0.639357)
				(4207.66,0.650891)
				(6051.74,0.754072)
			};

			% TPS rec.min_min %%%%%%%%%%%%%%%%%%%%%%%%%%%%%%%%%%%%%%%%%%%%%%%%%%%%%%%%%%%%
			\addplot[TPS] coordinates{
				(224.26,0.353272)
				(903.595,0.552969)
				(1130.53,0.588296)
				(1332.49,0.603027)
				(1556.56,0.662218)
				(1736.84,0.678645)
				(1988.95,0.690623)
				(2167.04,0.706248)
				(2656.03,0.74538)
				(2987.25,0.806752)
				(3505.61,0.815677)
				(4030.29,0.865845)
				(4568.25,0.875086)
				(5242.04,0.894935)
				(5978.67,0.897673)
			};

			% NC rec.min_min %%%%%%%%%%%%%%%%%%%%%%%%%%%%%%%%%%%%%%%%%%%%%%%%%%%%%%%%%%%%
			\addplot[NC] coordinates{
				(223.505,0.437718)
				(321.69,0.472964)
				(416.03,0.532014)
				(594.185,0.563997)
				(763.865,0.596167)
				(922.645,0.646133)
				(1073.6,0.678303)
				(1213.8,0.714237)
				(1348.36,0.719306)
				(1473.43,0.741273)
				(1591.75,0.751198)
				(2000.27,0.8057)
			};

			% VC rec.min_min %%%%%%%%%%%%%%%%%%%%%%%%%%%%%%%%%%%%%%%%%%%%%%%%%%%%%%%%%%%%
			\addplot[VC] coordinates{
				(351.72,0.275838)
				(498.095,0.351814)
				(712.82,0.435181)
				(994.125,0.510608)
				(1201.27,0.548691)
				(1375.27,0.577086)
				(1685.08,0.627086)
				(1968.97,0.638794)
				(2224.75,0.679429)
				(2476.43,0.705978)
				(2712.29,0.740834)
				(2948.04,0.731082)
				(3169.69,0.741557)
				(3389.95,0.759256)
				(3804.08,0.775691)
				(4196.75,0.804407)
				(4566.66,0.787791)
				(4815.37,0.789055)
				(5157.48,0.812353)
				(5401.49,0.800253)
			};

			% PB rec.min_min %%%%%%%%%%%%%%%%%%%%%%%%%%%%%%%%%%%%%%%%%%%%%%%%%%%%%%%%%%%%
			\addplot[PB] coordinates{
				(324.39,0.390012)
				(398.425,0.430016)
				(494.17,0.485999)
				(692.845,0.572151)
				(853.905,0.609716)
				(1129.27,0.675095)
				(1323.88,0.725714)
				(1323.88,0.725714)
				(1601.37,0.723135)
				(1929.58,0.783999)
				(1929.58,0.783999)
				(2453.31,0.831136)
				(3126.91,0.87141)
				(3126.91,0.87141)
				(3126.91,0.87141)
				(4270.56,0.91909)
				(4270.56,0.91909)
				(6122.31,0.964963)
			};

			% PRESLIC rec.min_min %%%%%%%%%%%%%%%%%%%%%%%%%%%%%%%%%%%%%%%%%%%%%%%%%%%%%%%%%%%%
			\addplot[PRESLIC] coordinates{
				(369,0.538904)
				(581.54,0.546396)
				(734.575,0.648185)
				(1020.3,0.691168)
				(1229.22,0.703088)
				(1229.22,0.703088)
				(1511.3,0.73939)
				(1838.94,0.774968)
				(1838.94,0.774968)
				(2375.99,0.821925)
				(3059.43,0.878713)
				(3059.43,0.878713)
				(3059.43,0.878713)
				(4218.88,0.903197)
				(4218.88,0.903197)
				(6137.8,0.959184)
			};

			% W rec.min_min %%%%%%%%%%%%%%%%%%%%%%%%%%%%%%%%%%%%%%%%%%%%%%%%%%%%%%%%%%%%
			\addplot[W] coordinates{
				(188.285,0.323904)
				(296.48,0.411673)
				(387.92,0.454577)
				(608.055,0.540815)
				(794.47,0.623647)
				(1100.87,0.684654)
				(1302.83,0.697543)
				(1302.83,0.697543)
				(1572.26,0.755628)
				(1960.24,0.786733)
				(1960.24,0.786733)
				(2478.43,0.857651)
				(3292.41,0.897545)
				(3292.41,0.897545)
				(3292.41,0.897545)
				(4439.31,0.938943)
				(4439.31,0.938943)
				(6526.67,0.976517)
			};

			% LSC rec.min_min %%%%%%%%%%%%%%%%%%%%%%%%%%%%%%%%%%%%%%%%%%%%%%%%%%%%%%%%%%%%
			\addplot[LSC] coordinates{
				(548.755,0.587893)
				(756.92,0.62631)
				(1045.31,0.69383)
				(1360.95,0.759872)
				(1665.85,0.796425)
				(1919.65,0.812791)
				(2055.98,0.824278)
				(2239.5,0.839979)
				(2376.71,0.836095)
				(2441.63,0.849301)
				(2603.55,0.856033)
				(2909.2,0.867167)
				(3147.7,0.862765)
				(3373.12,0.874935)
				(3762.88,0.874158)
				(3762.88,0.874158)
				(4057.29,0.906198)
				(4401.67,0.899479)
			};

			% WP rec.min_min %%%%%%%%%%%%%%%%%%%%%%%%%%%%%%%%%%%%%%%%%%%%%%%%%%%%%%%%%%%%
			\addplot[WP] coordinates{
				(216,0.379356)
				(294,0.427653)
				(384,0.465868)
				(600,0.536585)
				(805,0.606082)
				(1080,0.658841)
				(1320,0.699837)
				(1536,0.710312)
				(1536,0.713022)
				(1944,0.758714)
				(1944,0.758714)
				(2400,0.818794)
				(3174,0.86571)
				(3174,0.867069)
				(4320,0.916019)
				(4320,0.916019)
				(4320,0.916019)
				(5836.85,0.870393)
			};

			% QS rec.min_min %%%%%%%%%%%%%%%%%%%%%%%%%%%%%%%%%%%%%%%%%%%%%%%%%%%%%%%%%%%%
			\addplot[QS] coordinates{
				(324.725,0.104395)
				(468.655,0.16087)
				(615,0.202985)
				(884.79,0.274156)
				(984.505,0.298898)
				(1275.62,0.378773)
				(1500.06,0.409451)
				(1646.89,0.422316)
				(1824.2,0.433944)
				(2040.32,0.435675)
				(2303.69,0.472044)
				(2626.4,0.488867)
				(3066.62,0.530925)
				(3646.03,0.569273)
				(4444.58,0.623949)
				(5554.87,0.679119)
			};

			% VLSLIC rec.min_min %%%%%%%%%%%%%%%%%%%%%%%%%%%%%%%%%%%%%%%%%%%%%%%%%%%%%%%%%%%%
			\addplot[VLSLIC] coordinates{
				(575.975,0.618161)
				(651.86,0.672625)
				(763.62,0.686181)
				(899,0.703161)
				(988.985,0.717115)
				(1193.67,0.746316)
				(1348.08,0.75149)
				(1349.01,0.746614)
				(1579.08,0.767022)
				(1849.65,0.790288)
				(1857.27,0.780683)
				(2307.73,0.817952)
				(2890.56,0.86898)
				(2924.52,0.865629)
				(2954.18,0.86174)
				(3858.98,0.911186)
				(3858.98,0.911186)
				(4809.57,0.915588)
			};

			% CIS rec.min_min %%%%%%%%%%%%%%%%%%%%%%%%%%%%%%%%%%%%%%%%%%%%%%%%%%%%%%%%%%%%
			\addplot[CIS] coordinates{
				(317.63,0.338067)
				(411.595,0.346202)
				(472.685,0.386436)
				(633.345,0.458028)
				(777.05,0.497374)
				(969.085,0.548664)
				(1150.5,0.556098)
				(1168.82,0.556098)
				(1371.27,0.601394)
				(1637.39,0.638792)
				(1667.02,0.65365)
				(2056.01,0.662043)
				(3126.04,0.7341)
				(3625.6,0.739175)
				(4558.92,0.803451)
				(4558.92,0.803451)
				(6448.65,0.860213)
			};

			% ERS rec.min_min %%%%%%%%%%%%%%%%%%%%%%%%%%%%%%%%%%%%%%%%%%%%%%%%%%%%%%%%%%%%
			\addplot[ERS] coordinates{
				(200,0.497829)
				(300,0.572497)
				(400,0.600506)
				(600,0.682335)
				(800,0.723157)
				(1000,0.762494)
				(1200,0.781049)
				(1400,0.815438)
				(1600,0.838446)
				(1800,0.851064)
				(2000,0.868382)
				(2400,0.895101)
				(2800,0.923553)
				(3200,0.932459)
				(3600,0.945324)
				(4000,0.955715)
				(4600,0.967231)
				(5200,0.977412)
			};

			% MSS rec.min_min %%%%%%%%%%%%%%%%%%%%%%%%%%%%%%%%%%%%%%%%%%%%%%%%%%%%%%%%%%%%
			\addplot[MSS] coordinates{
				(200.765,0.363508)
				(282.33,0.413583)
				(420.66,0.518326)
				(664.115,0.58133)
				(837.885,0.615629)
				(1179.3,0.698241)
				(1422.34,0.714261)
				(1423.04,0.719646)
				(1768,0.75997)
				(2156.96,0.798627)
				(2157.01,0.798376)
				(2826.35,0.845214)
				(3669.97,0.875109)
				(3670.34,0.874066)
				(3669.92,0.876151)
				(5213.45,0.925575)
				(5213.45,0.925575)
				(7846.35,0.970562)
			};

			% ETPS rec.min_min %%%%%%%%%%%%%%%%%%%%%%%%%%%%%%%%%%%%%%%%%%%%%%%%%%%%%%%%%%%%
			\addplot[ETPS] coordinates{
				(216,0.681047)
				(294,0.724894)
				(425,0.766693)
				(651,0.824634)
				(805,0.860485)
				(1107,0.876945)
				(1320,0.908514)
				(1320,0.908514)
				(1617,0.916952)
				(1944,0.936061)
				(1944,0.936061)
				(2501,0.950412)
				(3174,0.958459)
				(3174,0.958459)
				(3174,0.958459)
				(4374,0.976133)
				(4374,0.976133)
				(6305,0.985498)
			};

		\end{axis}
	\end{tikzpicture}
\end{subfigure}
\begin{subfigure}[b]{\fullthreeone\textwidth}\phantomsubcaption\label{subfig:experiments-quantitative-bsds500-ue_np.max_max}
	%%%%%%%%%%%%%%%%%%%%%%%%%%%%%%%%%%%%%%%%%%%%%%%%%%%%%%%%%%%%
	% ue_np.max_max
	%%%%%%%%%%%%%%%%%%%%%%%%%%%%%%%%%%%%%%%%%%%%%%%%%%%%%%%%%%%%
	\begin{tikzpicture}
		\begin{axis}[EQBSDS500UEMax,xmode=log]

			% CCS ue_np.max_max %%%%%%%%%%%%%%%%%%%%%%%%%%%%%%%%%%%%%%%%%%%%%%%%%%%%%%%%%%%%
			\addplot[CCS] coordinates{
				(218.865,0.430094)
				(303.21,0.375982)
				(453.82,0.300471)
				(706.35,0.242706)
				(871.12,0.222006)
				(1204.99,0.196754)
				(1438.62,0.18086)
				(1438.62,0.18086)
				(1762.49,0.170957)
				(2107.24,0.161476)
				(2107.24,0.161476)
				(2702.61,0.143075)
				(3406.7,0.128782)
				(3406.7,0.128782)
				(3406.7,0.128782)
				(4654.44,0.119805)
				(4654.44,0.119805)
				(6619.1,0.106359)
			};

			% SEEDS ue_np.max_max %%%%%%%%%%%%%%%%%%%%%%%%%%%%%%%%%%%%%%%%%%%%%%%%%%%%%%%%%%%%
			\addplot[SEEDS] coordinates{
				(261.62,0.580858)
				(365.675,0.426467)
				(468.81,0.404848)
				(670.57,0.349266)
				(870.75,0.324713)
				(1087.4,0.257421)
				(1270.11,0.251553)
				(1451.85,0.278314)
				(1669.18,0.241941)
				(1873.19,0.241061)
				(2104.62,0.233515)
				(2462.77,0.190679)
				(2793.43,0.215841)
				(3260.86,0.170439)
				(3895.78,0.151812)
				(3895.78,0.151812)
				(4846.12,0.133354)
				(4846.12,0.133354)
			};

			% SLIC ue_np.max_max %%%%%%%%%%%%%%%%%%%%%%%%%%%%%%%%%%%%%%%%%%%%%%%%%%%%%%%%%%%%
			\addplot[SLIC] coordinates{
				(180.32,0.444239)
				(256.725,0.366876)
				(368.57,0.360594)
				(575.335,0.288165)
				(726.36,0.268036)
				(1002.87,0.254331)
				(1203.61,0.231559)
				(1203.61,0.231559)
				(1475.69,0.217447)
				(1814.8,0.19086)
				(1814.8,0.19086)
				(2334.56,0.177512)
				(3038.13,0.153879)
				(3038.13,0.153879)
				(3038.13,0.153879)
				(4188.97,0.123004)
				(4188.97,0.123004)
				(6139.41,0.112441)
			};

			% RW ue_np.max_max %%%%%%%%%%%%%%%%%%%%%%%%%%%%%%%%%%%%%%%%%%%%%%%%%%%%%%%%%%%%
			\addplot[RW] coordinates{
				(211.565,0.484103)
				(313.64,0.437096)
				(407.125,0.348392)
				(649.355,0.311701)
				(858.65,0.251423)
				(1032.29,0.241339)
				(1280.94,0.233794)
				(1505.72,0.191346)
				(1649.74,0.193289)
				(1940.98,0.183308)
				(2101.2,0.170498)
				(2570.18,0.155647)
				(2916.36,0.146573)
				(3413.13,0.140226)
				(3882.85,0.133017)
				(4457.4,0.126366)
				(5093.4,0.124954)
				(5663.8,0.0930888)
			};

			% CW ue_np.max_max %%%%%%%%%%%%%%%%%%%%%%%%%%%%%%%%%%%%%%%%%%%%%%%%%%%%%%%%%%%%
			\addplot[CW] coordinates{
				(197.775,0.46466)
				(309.29,0.376267)
				(405.935,0.338366)
				(609.18,0.296501)
				(807.005,0.267932)
				(1045.56,0.235853)
				(1252.72,0.214072)
				(1492.83,0.203477)
				(1649.62,0.194623)
				(1816.43,0.180439)
				(2091.79,0.173606)
				(2565.8,0.161948)
				(2917.57,0.153315)
				(3319.52,0.148775)
				(4023.98,0.138736)
				(4023.98,0.138736)
				(4639.83,0.134293)
				(5846.73,0.128555)
			};

			% TP ue_np.max_max %%%%%%%%%%%%%%%%%%%%%%%%%%%%%%%%%%%%%%%%%%%%%%%%%%%%%%%%%%%%
			\addplot[TP] coordinates{
				(283.695,0.376047)
				(381.285,0.325801)
				(555.1,0.270905)
				(752.29,0.254357)
				(1007.04,0.217466)
				(1292.4,0.20336)
				(1495.14,0.190174)
				(2017.73,0.249338)
				(2478.02,0.178626)
				(2831,0.165252)
				(3018.75,0.173587)
			};

			% POISE ue_np.max_max %%%%%%%%%%%%%%%%%%%%%%%%%%%%%%%%%%%%%%%%%%%%%%%%%%%%%%%%%%%%
			\addplot[POISE] coordinates{
				(204.855,0.423741)
				(306.68,0.409563)
				(408.385,0.393074)
				(611.79,0.377018)
				(814.975,0.367912)
				(1017.48,0.362459)
				(1216.26,0.357239)
				(1409.36,0.353327)
				(1587.37,0.354195)
				(1749.07,0.345542)
				(1887.06,0.345542)
				(2090.32,0.345542)
				(2221.84,0.345542)
				(2278.76,0.345542)
				(2287.5,0.345542)
				(2288.94,0.345542)
				(2288.94,0.345542)
				(2288.94,0.345542)
			};

			% FH ue_np.max_max %%%%%%%%%%%%%%%%%%%%%%%%%%%%%%%%%%%%%%%%%%%%%%%%%%%%%%%%%%%%
			\addplot[FH] coordinates{
				(628.745,0.456454)
				(782.42,0.370891)
				(799.09,0.371727)
				(963.36,0.382161)
				(1090.39,0.318748)
				(1187.04,0.352699)
				(1605.71,0.268858)
				(2533.01,0.284752)
				(3000.74,0.187279)
				(3219.63,0.182071)
				(3814.42,0.328398)
				(4746.96,0.273943)
			};

			% EAMS ue_np.max_max %%%%%%%%%%%%%%%%%%%%%%%%%%%%%%%%%%%%%%%%%%%%%%%%%%%%%%%%%%%%
			\addplot[EAMS] coordinates{
				(261.12,0.451241)
				(283.33,0.451241)
				(309.87,0.451241)
				(383.52,0.442355)
				(499.435,0.433002)
				(725.36,0.424764)
				(1417.35,0.404609)
				(1473.6,0.403741)
				(1534.33,0.403132)
				(1883.59,0.426364)
				(2063.65,0.426053)
				(2346.71,0.423443)
				(2642.78,0.421908)
				(3113.25,0.558941)
				(3640.44,0.557095)
				(4914.12,0.55273)
				(8189.58,0.571479)
			};

			% CRS ue_np.max_max %%%%%%%%%%%%%%%%%%%%%%%%%%%%%%%%%%%%%%%%%%%%%%%%%%%%%%%%%%%%
			\addplot[CRS] coordinates{
				(299.575,0.376293)
				(449.585,0.328851)
				(635.865,0.282336)
				(895.935,0.254649)
				(1180.31,0.23369)
				(1530.1,0.20586)
				(1758.22,0.191599)
				(2057.15,0.18632)
				(2308.66,0.176074)
				(2461.75,0.168794)
				(2755.4,0.16338)
				(3401.31,0.152985)
				(3879.51,0.138775)
				(4346.83,0.134144)
				(5208.23,0.125453)
				(5208.23,0.125453)
				(5821.7,0.114008)
				(7241.41,0.10577)
			};

			% SEAW ue_np.max_max %%%%%%%%%%%%%%%%%%%%%%%%%%%%%%%%%%%%%%%%%%%%%%%%%%%%%%%%%%%%
			\addplot[SEAW] coordinates{
				(165.505,0.664238)
				(356.805,0.471577)
				(923.49,0.276753)
				(2933.9,0.172661)
				(10727.5,0.105783)
			};

			% RESEEDS ue_np.max_max %%%%%%%%%%%%%%%%%%%%%%%%%%%%%%%%%%%%%%%%%%%%%%%%%%%%%%%%%%%%
			\addplot[RESEEDS] coordinates{
				(200.795,0.463812)
				(301.525,0.423754)
				(401.465,0.323003)
				(602.33,0.278139)
				(800.99,0.253729)
				(1020.12,0.258994)
				(1201.34,0.224707)
				(1378.1,0.229124)
				(1601.55,0.204558)
				(1802.11,0.195886)
				(2040.11,0.214267)
				(2402.13,0.164403)
				(2720.13,0.176229)
				(3200.2,0.148561)
				(3840.22,0.136566)
				(3840.22,0.136566)
				(4800.37,0.123665)
				(4800.37,0.123665)
			};

			% ERGC ue_np.max_max %%%%%%%%%%%%%%%%%%%%%%%%%%%%%%%%%%%%%%%%%%%%%%%%%%%%%%%%%%%%
			\addplot[ERGC] coordinates{
				(196,0.407743)
				(306,0.318657)
				(400,0.289992)
				(600,0.25902)
				(792.57,0.22915)
				(1024,0.217518)
				(1224,0.201579)
				(1453.2,0.186391)
				(1600,0.187285)
				(1760,0.174824)
				(2024,0.162013)
				(2472.98,0.150498)
				(2809,0.144455)
				(3180,0.137214)
				(3840,0.128756)
				(3840,0.128756)
				(4416,0.122713)
				(5520,0.116761)
			};

			% PF ue_np.max_max %%%%%%%%%%%%%%%%%%%%%%%%%%%%%%%%%%%%%%%%%%%%%%%%%%%%%%%%%%%%
			\addplot[PF] coordinates{
				(281.06,0.634594)
				(428.82,0.563196)
				(585.08,0.49751)
				(928.805,0.437957)
				(1172.76,0.402439)
				(1602.65,0.396157)
				(1861.72,0.374026)
				(2267.11,0.332634)
				(2697.08,0.321442)
				(3382.15,0.336895)
				(4207.66,0.313521)
				(6051.74,0.291281)
			};

			% TPS ue_np.max_max %%%%%%%%%%%%%%%%%%%%%%%%%%%%%%%%%%%%%%%%%%%%%%%%%%%%%%%%%%%%
			\addplot[TPS] coordinates{
				(224.26,0.45773)
				(903.595,0.282284)
				(1130.53,0.260083)
				(1332.49,0.244351)
				(1556.56,0.233269)
				(1736.84,0.228561)
				(1988.95,0.215309)
				(2167.04,0.203613)
				(2656.03,0.190323)
				(2987.25,0.177389)
				(3505.61,0.172784)
				(4030.29,0.164189)
				(4568.25,0.156683)
				(5242.04,0.152143)
				(5978.67,0.144941)
			};

			% NC ue_np.max_max %%%%%%%%%%%%%%%%%%%%%%%%%%%%%%%%%%%%%%%%%%%%%%%%%%%%%%%%%%%%
			\addplot[NC] coordinates{
				(223.505,0.342452)
				(321.69,0.306099)
				(416.03,0.281578)
				(594.185,0.253515)
				(763.865,0.236669)
				(922.645,0.226281)
				(1073.6,0.220601)
				(1213.8,0.216819)
				(1348.36,0.209435)
				(1473.43,0.211903)
				(1591.75,0.203386)
				(2000.27,0.202259)
			};

			% VC ue_np.max_max %%%%%%%%%%%%%%%%%%%%%%%%%%%%%%%%%%%%%%%%%%%%%%%%%%%%%%%%%%%%
			\addplot[VC] coordinates{
				(351.72,0.738259)
				(498.095,0.569595)
				(712.82,0.464867)
				(994.125,0.358236)
				(1201.27,0.308923)
				(1375.27,0.279318)
				(1685.08,0.252155)
				(1968.97,0.223153)
				(2224.75,0.206663)
				(2476.43,0.189856)
				(2712.29,0.183833)
				(2948.04,0.171948)
				(3169.69,0.164831)
				(3389.95,0.159144)
				(3804.08,0.151294)
				(4196.75,0.14906)
				(4566.66,0.143406)
				(4815.37,0.135841)
				(5157.48,0.139241)
				(5401.49,0.132007)
			};

			% PB ue_np.max_max %%%%%%%%%%%%%%%%%%%%%%%%%%%%%%%%%%%%%%%%%%%%%%%%%%%%%%%%%%%%
			\addplot[PB] coordinates{
				(324.39,0.45966)
				(398.425,0.416824)
				(494.17,0.373113)
				(692.845,0.320004)
				(853.905,0.289914)
				(1129.27,0.251184)
				(1323.88,0.238146)
				(1323.88,0.238146)
				(1601.37,0.221423)
				(1929.58,0.200957)
				(1929.58,0.200957)
				(2453.31,0.183166)
				(3126.91,0.163496)
				(3126.91,0.163496)
				(3126.91,0.163496)
				(4270.56,0.142648)
				(4270.56,0.142648)
				(6122.31,0.12599)
			};

			% PRESLIC ue_np.max_max %%%%%%%%%%%%%%%%%%%%%%%%%%%%%%%%%%%%%%%%%%%%%%%%%%%%%%%%%%%%
			\addplot[PRESLIC] coordinates{
				(369,0.384246)
				(581.54,0.295471)
				(734.575,0.26856)
				(1020.3,0.219131)
				(1229.22,0.226844)
				(1229.22,0.226844)
				(1511.3,0.196851)
				(1838.94,0.173691)
				(1838.94,0.173691)
				(2375.99,0.155362)
				(3059.43,0.139623)
				(3059.43,0.139623)
				(3059.43,0.139623)
				(4218.88,0.129105)
				(4218.88,0.129105)
				(6137.8,0.114488)
			};

			% W ue_np.max_max %%%%%%%%%%%%%%%%%%%%%%%%%%%%%%%%%%%%%%%%%%%%%%%%%%%%%%%%%%%%
			\addplot[W] coordinates{
				(188.285,0.514284)
				(296.48,0.440217)
				(387.92,0.366474)
				(608.055,0.320445)
				(794.47,0.306021)
				(1100.87,0.257796)
				(1302.83,0.221909)
				(1302.83,0.221909)
				(1572.26,0.196236)
				(1960.24,0.17939)
				(1960.24,0.17939)
				(2478.43,0.163075)
				(3292.41,0.14711)
				(3292.41,0.14711)
				(3292.41,0.14711)
				(4439.31,0.136139)
				(4439.31,0.136139)
				(6526.67,0.124656)
			};

			% LSC ue_np.max_max %%%%%%%%%%%%%%%%%%%%%%%%%%%%%%%%%%%%%%%%%%%%%%%%%%%%%%%%%%%%
			\addplot[LSC] coordinates{
				(548.755,0.337142)
				(756.92,0.290134)
				(1045.31,0.254338)
				(1360.95,0.223936)
				(1665.85,0.212285)
				(1919.65,0.202317)
				(2055.98,0.193943)
				(2239.5,0.189383)
				(2376.71,0.186391)
				(2441.63,0.180925)
				(2603.55,0.178308)
				(2909.2,0.166793)
				(3147.7,0.171663)
				(3373.12,0.159539)
				(3762.88,0.160206)
				(3762.88,0.160206)
				(4057.29,0.168976)
				(4401.67,0.183211)
			};

			% WP ue_np.max_max %%%%%%%%%%%%%%%%%%%%%%%%%%%%%%%%%%%%%%%%%%%%%%%%%%%%%%%%%%%%
			\addplot[WP] coordinates{
				(216,0.413747)
				(294,0.367485)
				(384,0.327245)
				(600,0.285737)
				(805,0.249739)
				(1080,0.217648)
				(1320,0.200115)
				(1536,0.18972)
				(1536,0.189409)
				(1944,0.175368)
				(1944,0.17599)
				(2400,0.160265)
				(3174,0.14487)
				(3174,0.14566)
				(4320,0.131158)
				(4320,0.131158)
				(4320,0.131158)
				(5836.85,0.117298)
			};

			% QS ue_np.max_max %%%%%%%%%%%%%%%%%%%%%%%%%%%%%%%%%%%%%%%%%%%%%%%%%%%%%%%%%%%%
			\addplot[QS] coordinates{
				(324.725,0.938446)
				(468.655,0.783894)
				(615,0.664976)
				(884.79,0.542348)
				(984.505,0.506318)
				(1275.62,0.445302)
				(1500.06,0.407614)
				(1646.89,0.387485)
				(1824.2,0.368223)
				(2040.32,0.35457)
				(2303.69,0.316669)
				(2626.4,0.287064)
				(3066.62,0.272887)
				(3646.03,0.255413)
				(4444.58,0.233956)
				(5554.87,0.21893)
			};

			% VLSLIC ue_np.max_max %%%%%%%%%%%%%%%%%%%%%%%%%%%%%%%%%%%%%%%%%%%%%%%%%%%%%%%%%%%%
			\addplot[VLSLIC] coordinates{
				(575.975,0.375593)
				(651.86,0.314402)
				(763.62,0.310782)
				(899,0.285192)
				(988.985,0.265711)
				(1193.67,0.239448)
				(1348.08,0.234344)
				(1349.01,0.229726)
				(1579.08,0.215297)
				(1849.65,0.201896)
				(1857.27,0.201935)
				(2307.73,0.183146)
				(2890.56,0.169669)
				(2924.52,0.167868)
				(2954.18,0.162149)
				(3858.98,0.156288)
				(3858.98,0.156288)
				(4809.57,0.17509)
			};

			% CIS ue_np.max_max %%%%%%%%%%%%%%%%%%%%%%%%%%%%%%%%%%%%%%%%%%%%%%%%%%%%%%%%%%%%
			\addplot[CIS] coordinates{
				(317.63,0.427432)
				(411.595,0.405936)
				(472.685,0.364978)
				(633.345,0.321377)
				(777.05,0.299745)
				(969.085,0.257006)
				(1150.5,0.257634)
				(1168.82,0.257634)
				(1371.27,0.226132)
				(1637.39,0.211177)
				(1667.02,0.211093)
				(2056.01,0.189254)
				(3126.04,0.17689)
				(3625.6,0.175018)
				(4558.92,0.153322)
				(4558.92,0.153322)
				(6448.65,0.134701)
			};

			% ERS ue_np.max_max %%%%%%%%%%%%%%%%%%%%%%%%%%%%%%%%%%%%%%%%%%%%%%%%%%%%%%%%%%%%
			\addplot[ERS] coordinates{
				(200,0.39854)
				(300,0.340348)
				(400,0.304992)
				(600,0.272433)
				(800,0.248515)
				(1000,0.239416)
				(1200,0.219072)
				(1400,0.204733)
				(1600,0.195316)
				(1800,0.194584)
				(2000,0.184086)
				(2400,0.175659)
				(2800,0.168911)
				(3200,0.15713)
				(3600,0.155239)
				(4000,0.150174)
				(4600,0.142991)
				(5200,0.135265)
			};

			% MSS ue_np.max_max %%%%%%%%%%%%%%%%%%%%%%%%%%%%%%%%%%%%%%%%%%%%%%%%%%%%%%%%%%%%
			\addplot[MSS] coordinates{
				(200.765,0.450314)
				(282.33,0.38718)
				(420.66,0.321792)
				(664.115,0.26244)
				(837.885,0.239396)
				(1179.3,0.214254)
				(1422.34,0.198639)
				(1423.04,0.196644)
				(1768,0.190698)
				(2156.96,0.179746)
				(2157.01,0.178412)
				(2826.35,0.162084)
				(3669.97,0.151948)
				(3670.34,0.153134)
				(3669.92,0.151955)
				(5213.45,0.136061)
				(5213.45,0.136061)
				(7846.35,0.117506)
			};

			% ETPS ue_np.max_max %%%%%%%%%%%%%%%%%%%%%%%%%%%%%%%%%%%%%%%%%%%%%%%%%%%%%%%%%%%%
			\addplot[ETPS] coordinates{
				(216,0.424486)
				(294,0.370393)
				(425,0.332951)
				(651,0.292835)
				(805,0.243457)
				(1107,0.213049)
				(1320,0.197149)
				(1320,0.197149)
				(1617,0.183237)
				(1944,0.174779)
				(1944,0.174779)
				(2501,0.157117)
				(3174,0.140239)
				(3174,0.140239)
				(3174,0.140239)
				(4374,0.12316)
				(4374,0.12316)
				(6305,0.105919)
			};

		\end{axis}
	\end{tikzpicture}
\end{subfigure}
\begin{subfigure}[b]{\fullthreeone\textwidth}\phantomsubcaption\label{subfig:experiments-quantitative-bsds500-ev.min_min}
	%%%%%%%%%%%%%%%%%%%%%%%%%%%%%%%%%%%%%%%%%%%%%%%%%%%%%%%%%%%%
	% ev.min_min
	%%%%%%%%%%%%%%%%%%%%%%%%%%%%%%%%%%%%%%%%%%%%%%%%%%%%%%%%%%%%
	\begin{tikzpicture}
		\begin{axis}[EQBSDS500EVMin,xmode=log]

			% CCS ev.min_min %%%%%%%%%%%%%%%%%%%%%%%%%%%%%%%%%%%%%%%%%%%%%%%%%%%%%%%%%%%%
			\addplot[CCS] coordinates{
				(218.865,0.452912)
				(303.21,0.471777)
				(453.82,0.515281)
				(706.35,0.542573)
				(871.12,0.561686)
				(1204.99,0.588343)
				(1438.62,0.603342)
				(1438.62,0.603342)
				(1762.49,0.623572)
				(2107.24,0.646309)
				(2107.24,0.646309)
				(2702.61,0.670408)
				(3406.7,0.695042)
				(3406.7,0.695042)
				(3406.7,0.695042)
				(4654.44,0.713672)
				(4654.44,0.713672)
				(6619.1,0.750245)
			};

			% SEEDS ev.min_min %%%%%%%%%%%%%%%%%%%%%%%%%%%%%%%%%%%%%%%%%%%%%%%%%%%%%%%%%%%%
			\addplot[SEEDS] coordinates{
				(261.62,0.598554)
				(365.675,0.638214)
				(468.81,0.667503)
				(670.57,0.695205)
				(870.75,0.70126)
				(1087.4,0.6967)
				(1270.11,0.723837)
				(1451.85,0.720079)
				(1669.18,0.741579)
				(1873.19,0.733512)
				(2104.62,0.706937)
				(2462.77,0.75794)
				(2793.43,0.738016)
				(3260.86,0.768837)
				(3895.78,0.765452)
				(3895.78,0.765452)
				(4846.12,0.784821)
				(4846.12,0.784821)
			};

			% SLIC ev.min_min %%%%%%%%%%%%%%%%%%%%%%%%%%%%%%%%%%%%%%%%%%%%%%%%%%%%%%%%%%%%
			\addplot[SLIC] coordinates{
				(180.32,0.39049)
				(256.725,0.415656)
				(368.57,0.412946)
				(575.335,0.436217)
				(726.36,0.451471)
				(1002.87,0.464549)
				(1203.61,0.475299)
				(1203.61,0.475299)
				(1475.69,0.488741)
				(1814.8,0.515778)
				(1814.8,0.515778)
				(2334.56,0.540813)
				(3038.13,0.568949)
				(3038.13,0.568949)
				(3038.13,0.568949)
				(4188.97,0.604857)
				(4188.97,0.604857)
				(6139.41,0.657402)
			};

			% RW ev.min_min %%%%%%%%%%%%%%%%%%%%%%%%%%%%%%%%%%%%%%%%%%%%%%%%%%%%%%%%%%%%
			\addplot[RW] coordinates{
				(211.565,0.312954)
				(313.64,0.345218)
				(407.125,0.399106)
				(649.355,0.423547)
				(858.65,0.448591)
				(1032.29,0.462961)
				(1280.94,0.470456)
				(1505.72,0.492165)
				(1649.74,0.49092)
				(1940.98,0.503797)
				(2101.2,0.521669)
				(2570.18,0.540298)
				(2916.36,0.548279)
				(3413.13,0.569715)
				(3882.85,0.57656)
				(4457.4,0.596475)
				(5093.4,0.596144)
				(5663.8,0.811088)
			};

			% CW ev.min_min %%%%%%%%%%%%%%%%%%%%%%%%%%%%%%%%%%%%%%%%%%%%%%%%%%%%%%%%%%%%
			\addplot[CW] coordinates{
				(197.775,0.301098)
				(309.29,0.384505)
				(405.935,0.394712)
				(609.18,0.421549)
				(807.005,0.431932)
				(1045.56,0.450363)
				(1252.72,0.461871)
				(1492.83,0.476395)
				(1649.62,0.478621)
				(1816.43,0.483618)
				(2091.79,0.494789)
				(2565.8,0.501054)
				(2917.57,0.509265)
				(3319.52,0.518565)
				(4023.98,0.531737)
				(4023.98,0.531737)
				(4639.83,0.542061)
				(5846.73,0.559444)
			};

			% TP ev.min_min %%%%%%%%%%%%%%%%%%%%%%%%%%%%%%%%%%%%%%%%%%%%%%%%%%%%%%%%%%%%
			\addplot[TP] coordinates{
				(283.695,0.391725)
				(381.285,0.425007)
				(555.1,0.45416)
				(752.29,0.472187)
				(1007.04,0.489261)
				(1292.4,0.50067)
				(1495.14,0.50959)
				(2017.73,0.483324)
				(2478.02,0.510427)
				(2831,0.529309)
				(3018.75,0.531377)
			};

			% POISE ev.min_min %%%%%%%%%%%%%%%%%%%%%%%%%%%%%%%%%%%%%%%%%%%%%%%%%%%%%%%%%%%%
			\addplot[POISE] coordinates{
				(204.855,0.472069)
				(306.68,0.496871)
				(408.385,0.51209)
				(611.79,0.523295)
				(814.975,0.530197)
				(1017.48,0.536183)
				(1216.26,0.543384)
				(1409.36,0.546323)
				(1587.37,0.548406)
				(1749.07,0.552664)
				(1887.06,0.555279)
				(2090.32,0.558571)
				(2221.84,0.55984)
				(2278.76,0.560463)
				(2287.5,0.560463)
				(2288.94,0.560463)
				(2288.94,0.560463)
				(2288.94,0.560463)
			};

			% FH ev.min_min %%%%%%%%%%%%%%%%%%%%%%%%%%%%%%%%%%%%%%%%%%%%%%%%%%%%%%%%%%%%
			\addplot[FH] coordinates{
				(628.745,0.341115)
				(782.42,0.395031)
				(799.09,0.399095)
				(963.36,0.427085)
				(1090.39,0.441952)
				(1187.04,0.459929)
				(1605.71,0.501412)
				(2533.01,0.641916)
				(3000.74,0.634155)
				(3219.63,0.638238)
				(3814.42,0.740348)
				(4746.96,0.787016)
			};

			% EAMS ev.min_min %%%%%%%%%%%%%%%%%%%%%%%%%%%%%%%%%%%%%%%%%%%%%%%%%%%%%%%%%%%%
			\addplot[EAMS] coordinates{
				(261.12,0.517374)
				(283.33,0.520031)
				(309.87,0.523705)
				(383.52,0.534578)
				(499.435,0.553217)
				(725.36,0.593348)
				(1417.35,0.684187)
				(1473.6,0.689035)
				(1534.33,0.696624)
				(1883.59,0.71179)
				(2063.65,0.722349)
				(2346.71,0.744555)
				(2642.78,0.762576)
				(3113.25,0.780491)
				(3640.44,0.798214)
				(4914.12,0.831837)
				(8189.58,0.875964)
			};

			% CRS ev.min_min %%%%%%%%%%%%%%%%%%%%%%%%%%%%%%%%%%%%%%%%%%%%%%%%%%%%%%%%%%%%
			\addplot[CRS] coordinates{
				(299.575,0.473797)
				(449.585,0.510902)
				(635.865,0.518229)
				(895.935,0.538072)
				(1180.31,0.546695)
				(1530.1,0.576098)
				(1758.22,0.583935)
				(2057.15,0.589129)
				(2308.66,0.595382)
				(2461.75,0.602303)
				(2755.4,0.59984)
				(3401.31,0.617831)
				(3879.51,0.631509)
				(4346.83,0.640102)
				(5208.23,0.651649)
				(5208.23,0.651649)
				(5821.7,0.660871)
				(7241.41,0.681787)
			};

			% SEAW ev.min_min %%%%%%%%%%%%%%%%%%%%%%%%%%%%%%%%%%%%%%%%%%%%%%%%%%%%%%%%%%%%
			\addplot[SEAW] coordinates{
				(165.505,0.326806)
				(356.805,0.439878)
				(923.49,0.568903)
				(2933.9,0.677634)
				(10727.5,0.810541)
			};

			% RESEEDS ev.min_min %%%%%%%%%%%%%%%%%%%%%%%%%%%%%%%%%%%%%%%%%%%%%%%%%%%%%%%%%%%%
			\addplot[RESEEDS] coordinates{
				(200.795,0.617015)
				(301.525,0.646812)
				(401.465,0.666255)
				(602.33,0.684029)
				(800.99,0.700047)
				(1020.12,0.702515)
				(1201.34,0.706836)
				(1378.1,0.71917)
				(1601.55,0.724351)
				(1802.11,0.728548)
				(2040.11,0.720231)
				(2402.13,0.743112)
				(2720.13,0.742489)
				(3200.2,0.760467)
				(3840.22,0.767716)
				(3840.22,0.767716)
				(4800.37,0.785896)
				(4800.37,0.785896)
			};

			% ERGC ev.min_min %%%%%%%%%%%%%%%%%%%%%%%%%%%%%%%%%%%%%%%%%%%%%%%%%%%%%%%%%%%%
			\addplot[ERGC] coordinates{
				(196,0.416619)
				(306,0.480426)
				(400,0.511126)
				(600,0.538776)
				(792.57,0.551114)
				(1024,0.594017)
				(1224,0.614182)
				(1453.2,0.647998)
				(1600,0.657179)
				(1760,0.656869)
				(2024,0.681369)
				(2472.98,0.71241)
				(2809,0.733626)
				(3180,0.746621)
				(3840,0.776551)
				(3840,0.776551)
				(4416,0.797873)
				(5520,0.813343)
			};

			% PF ev.min_min %%%%%%%%%%%%%%%%%%%%%%%%%%%%%%%%%%%%%%%%%%%%%%%%%%%%%%%%%%%%
			\addplot[PF] coordinates{
				(281.06,0.164612)
				(428.82,0.236108)
				(585.08,0.240304)
				(928.805,0.328007)
				(1172.76,0.345525)
				(1602.65,0.365682)
				(1861.72,0.384632)
				(2267.11,0.411414)
				(2697.08,0.421399)
				(3382.15,0.449089)
				(4207.66,0.465496)
				(6051.74,0.496742)
			};

			% TPS ev.min_min %%%%%%%%%%%%%%%%%%%%%%%%%%%%%%%%%%%%%%%%%%%%%%%%%%%%%%%%%%%%
			\addplot[TPS] coordinates{
				(224.26,0.322123)
				(903.595,0.42042)
				(1130.53,0.43201)
				(1332.49,0.453213)
				(1556.56,0.474122)
				(1736.84,0.487459)
				(1988.95,0.491773)
				(2167.04,0.494253)
				(2656.03,0.504799)
				(2987.25,0.510124)
				(3505.61,0.516986)
				(4030.29,0.52298)
				(4568.25,0.528445)
				(5242.04,0.532524)
				(5978.67,0.539445)
			};

			% NC ev.min_min %%%%%%%%%%%%%%%%%%%%%%%%%%%%%%%%%%%%%%%%%%%%%%%%%%%%%%%%%%%%
			\addplot[NC] coordinates{
				(223.505,0.375374)
				(321.69,0.400293)
				(416.03,0.420843)
				(594.185,0.454552)
				(763.865,0.466268)
				(922.645,0.470596)
				(1073.6,0.475044)
				(1213.8,0.47659)
				(1348.36,0.481024)
				(1473.43,0.480932)
				(1591.75,0.484254)
				(2000.27,0.488179)
			};

			% VC ev.min_min %%%%%%%%%%%%%%%%%%%%%%%%%%%%%%%%%%%%%%%%%%%%%%%%%%%%%%%%%%%%
			\addplot[VC] coordinates{
				(351.72,0.170417)
				(498.095,0.335216)
				(712.82,0.465103)
				(994.125,0.513238)
				(1201.27,0.547647)
				(1375.27,0.551175)
				(1685.08,0.572442)
				(1968.97,0.585665)
				(2224.75,0.595157)
				(2476.43,0.610356)
				(2712.29,0.61348)
				(2948.04,0.626484)
				(3169.69,0.634391)
				(3389.95,0.643273)
				(3804.08,0.66112)
				(4196.75,0.672646)
				(4566.66,0.682852)
				(4815.37,0.693633)
				(5157.48,0.70203)
				(5401.49,0.711085)
			};

			% PB ev.min_min %%%%%%%%%%%%%%%%%%%%%%%%%%%%%%%%%%%%%%%%%%%%%%%%%%%%%%%%%%%%
			\addplot[PB] coordinates{
				(324.39,0.292087)
				(398.425,0.303925)
				(494.17,0.337288)
				(692.845,0.385016)
				(853.905,0.411437)
				(1129.27,0.441147)
				(1323.88,0.45707)
				(1323.88,0.45707)
				(1601.37,0.481622)
				(1929.58,0.50671)
				(1929.58,0.50671)
				(2453.31,0.516856)
				(3126.91,0.537641)
				(3126.91,0.537641)
				(3126.91,0.537641)
				(4270.56,0.566339)
				(4270.56,0.566339)
				(6122.31,0.608486)
			};

			% PRESLIC ev.min_min %%%%%%%%%%%%%%%%%%%%%%%%%%%%%%%%%%%%%%%%%%%%%%%%%%%%%%%%%%%%
			\addplot[PRESLIC] coordinates{
				(369,0.413094)
				(581.54,0.44619)
				(734.575,0.458591)
				(1020.3,0.482228)
				(1229.22,0.486507)
				(1229.22,0.486507)
				(1511.3,0.515634)
				(1838.94,0.537531)
				(1838.94,0.537531)
				(2375.99,0.565181)
				(3059.43,0.590003)
				(3059.43,0.590003)
				(3059.43,0.590003)
				(4218.88,0.618382)
				(4218.88,0.618382)
				(6137.8,0.679559)
			};

			% W ev.min_min %%%%%%%%%%%%%%%%%%%%%%%%%%%%%%%%%%%%%%%%%%%%%%%%%%%%%%%%%%%%
			\addplot[W] coordinates{
				(188.285,0.314651)
				(296.48,0.344545)
				(387.92,0.377135)
				(608.055,0.402403)
				(794.47,0.421086)
				(1100.87,0.444147)
				(1302.83,0.46759)
				(1302.83,0.46759)
				(1572.26,0.476365)
				(1960.24,0.481839)
				(1960.24,0.481839)
				(2478.43,0.501301)
				(3292.41,0.522781)
				(3292.41,0.522781)
				(3292.41,0.522781)
				(4439.31,0.541745)
				(4439.31,0.541745)
				(6526.67,0.564547)
			};

			% LSC ev.min_min %%%%%%%%%%%%%%%%%%%%%%%%%%%%%%%%%%%%%%%%%%%%%%%%%%%%%%%%%%%%
			\addplot[LSC] coordinates{
				(548.755,0.483659)
				(756.92,0.509577)
				(1045.31,0.507251)
				(1360.95,0.53368)
				(1665.85,0.543198)
				(1919.65,0.530862)
				(2055.98,0.52276)
				(2239.5,0.529299)
				(2376.71,0.528529)
				(2441.63,0.528497)
				(2603.55,0.531874)
				(2909.2,0.529434)
				(3147.7,0.524221)
				(3373.12,0.534282)
				(3762.88,0.531822)
				(3762.88,0.531822)
				(4057.29,0.529183)
				(4401.67,0.517701)
			};

			% WP ev.min_min %%%%%%%%%%%%%%%%%%%%%%%%%%%%%%%%%%%%%%%%%%%%%%%%%%%%%%%%%%%%
			\addplot[WP] coordinates{
				(216,0.351213)
				(294,0.379356)
				(384,0.400605)
				(600,0.418699)
				(805,0.441055)
				(1080,0.457875)
				(1320,0.466189)
				(1536,0.483483)
				(1536,0.484023)
				(1944,0.498719)
				(1944,0.498679)
				(2400,0.51948)
				(3174,0.538329)
				(3174,0.53811)
				(4320,0.571351)
				(4320,0.571351)
				(4320,0.571351)
				(5836.85,0.608472)
			};

			% QS ev.min_min %%%%%%%%%%%%%%%%%%%%%%%%%%%%%%%%%%%%%%%%%%%%%%%%%%%%%%%%%%%%
			\addplot[QS] coordinates{
				(324.725,0.100761)
				(468.655,0.177711)
				(615,0.222815)
				(884.79,0.317166)
				(984.505,0.356319)
				(1275.62,0.460537)
				(1500.06,0.499033)
				(1646.89,0.518511)
				(1824.2,0.532636)
				(2040.32,0.566375)
				(2303.69,0.594474)
				(2626.4,0.625056)
				(3066.62,0.657253)
				(3646.03,0.699247)
				(4444.58,0.739815)
				(5554.87,0.7778)
			};

			% VLSLIC ev.min_min %%%%%%%%%%%%%%%%%%%%%%%%%%%%%%%%%%%%%%%%%%%%%%%%%%%%%%%%%%%%
			\addplot[VLSLIC] coordinates{
				(575.975,0.479488)
				(651.86,0.487159)
				(763.62,0.484555)
				(899,0.478989)
				(988.985,0.480495)
				(1193.67,0.488625)
				(1348.08,0.491522)
				(1349.01,0.496849)
				(1579.08,0.505997)
				(1849.65,0.498075)
				(1857.27,0.500588)
				(2307.73,0.517214)
				(2890.56,0.511261)
				(2924.52,0.514225)
				(2954.18,0.522509)
				(3858.98,0.523677)
				(3858.98,0.523677)
				(4809.57,0.512264)
			};

			% CIS ev.min_min %%%%%%%%%%%%%%%%%%%%%%%%%%%%%%%%%%%%%%%%%%%%%%%%%%%%%%%%%%%%
			\addplot[CIS] coordinates{
				(317.63,0.470482)
				(411.595,0.479611)
				(472.685,0.477713)
				(633.345,0.486759)
				(777.05,0.521336)
				(969.085,0.534402)
				(1150.5,0.583749)
				(1168.82,0.590101)
				(1371.27,0.590242)
				(1637.39,0.619903)
				(1667.02,0.618997)
				(2056.01,0.623123)
				(3126.04,0.744533)
				(3625.6,0.755296)
				(4558.92,0.774592)
				(4558.92,0.774592)
				(6448.65,0.833621)
			};

			% ERS ev.min_min %%%%%%%%%%%%%%%%%%%%%%%%%%%%%%%%%%%%%%%%%%%%%%%%%%%%%%%%%%%%
			\addplot[ERS] coordinates{
				(200,0.349023)
				(300,0.399796)
				(400,0.414544)
				(600,0.453395)
				(800,0.467258)
				(1000,0.476213)
				(1200,0.487698)
				(1400,0.498404)
				(1600,0.503052)
				(1800,0.509764)
				(2000,0.519451)
				(2400,0.527971)
				(2800,0.536212)
				(3200,0.547363)
				(3600,0.552321)
				(4000,0.556708)
				(4600,0.566385)
				(5200,0.574615)
			};

			% MSS ev.min_min %%%%%%%%%%%%%%%%%%%%%%%%%%%%%%%%%%%%%%%%%%%%%%%%%%%%%%%%%%%%
			\addplot[MSS] coordinates{
				(200.765,0.384699)
				(282.33,0.404657)
				(420.66,0.439673)
				(664.115,0.476088)
				(837.885,0.485449)
				(1179.3,0.49878)
				(1422.34,0.510536)
				(1423.04,0.513246)
				(1768,0.520911)
				(2156.96,0.531085)
				(2157.01,0.529408)
				(2826.35,0.540112)
				(3669.97,0.54603)
				(3670.34,0.546814)
				(3669.92,0.54684)
				(5213.45,0.564738)
				(5213.45,0.564738)
				(7846.35,0.582763)
			};

			% ETPS ev.min_min %%%%%%%%%%%%%%%%%%%%%%%%%%%%%%%%%%%%%%%%%%%%%%%%%%%%%%%%%%%%
			\addplot[ETPS] coordinates{
				(216,0.679907)
				(294,0.68622)
				(425,0.715978)
				(651,0.719659)
				(805,0.728443)
				(1107,0.749357)
				(1320,0.743314)
				(1320,0.743314)
				(1617,0.755771)
				(1944,0.768058)
				(1944,0.768058)
				(2501,0.785723)
				(3174,0.792274)
				(3174,0.792274)
				(3174,0.792274)
				(4374,0.823175)
				(4374,0.823175)
				(6305,0.841907)
			};

		\end{axis}
	\end{tikzpicture}
\end{subfigure}
\\[-4px]
   	\begin{subfigure}[b]{\fullthreeone\textwidth}\phantomsubcaption\label{subfig:appendix-experiments-bsds500-rec.std[0]}
	%%%%%%%%%%%%%%%%%%%%%%%%%%%%%%%%%%%%%%%%%%%%%%%%%%%%%%%%%%%%
	% rec.std[0]
	%%%%%%%%%%%%%%%%%%%%%%%%%%%%%%%%%%%%%%%%%%%%%%%%%%%%%%%%%%%%
	\begin{tikzpicture}
		\begin{axis}[AEBSDS500RecStd,xmode=log]

			% CCS rec.std[0] %%%%%%%%%%%%%%%%%%%%%%%%%%%%%%%%%%%%%%%%%%%%%%%%%%%%%%%%%%%%
			\addplot[CCS] coordinates{
				(218.865,0.104428)
				(303.21,0.0965316)
				(453.82,0.0876016)
				(706.35,0.0721017)
				(871.12,0.065018)
				(1204.99,0.0516212)
				(1438.62,0.0447604)
				(1438.62,0.0447604)
				(1762.49,0.0384721)
				(2107.24,0.0315613)
				(2107.24,0.0315613)
				(2702.61,0.0238609)
				(3406.7,0.0166285)
				(3406.7,0.0166285)
				(3406.7,0.0166285)
				(4654.44,0.0103781)
				(4654.44,0.0103781)
				(6619.1,0.0040042)
			};

			% SEEDS rec.std[0] %%%%%%%%%%%%%%%%%%%%%%%%%%%%%%%%%%%%%%%%%%%%%%%%%%%%%%%%%%%%
			\addplot[SEEDS] coordinates{
				(261.62,0.0434701)
				(365.675,0.0352299)
				(468.81,0.03131)
				(670.57,0.0220214)
				(870.75,0.0162954)
				(1087.4,0.0136173)
				(1270.11,0.012314)
				(1451.85,0.00880262)
				(1669.18,0.0099889)
				(1873.19,0.0068794)
				(2104.62,0.00544822)
				(2462.77,0.00587461)
				(2793.43,0.00356312)
				(3260.86,0.00339171)
				(3895.78,0.00195312)
				(3895.78,0.00195312)
				(4846.12,0.00174351)
				(4846.12,0.00174351)
			};

			% SLIC rec.std[0] %%%%%%%%%%%%%%%%%%%%%%%%%%%%%%%%%%%%%%%%%%%%%%%%%%%%%%%%%%%%
			\addplot[SLIC] coordinates{
				(180.32,0.0982844)
				(256.725,0.0910176)
				(368.57,0.0832541)
				(575.335,0.0702315)
				(726.36,0.0621337)
				(1002.87,0.0512858)
				(1203.61,0.0430685)
				(1203.61,0.0430685)
				(1475.69,0.0379464)
				(1814.8,0.0309395)
				(1814.8,0.0309395)
				(2334.56,0.0245455)
				(3038.13,0.0181832)
				(3038.13,0.0181832)
				(3038.13,0.0181832)
				(4188.97,0.0103263)
				(4188.97,0.0103263)
				(6139.41,0.00452813)
			};

			% RW rec.std[0] %%%%%%%%%%%%%%%%%%%%%%%%%%%%%%%%%%%%%%%%%%%%%%%%%%%%%%%%%%%%
			\addplot[RW] coordinates{
				(211.565,0.122283)
				(313.64,0.112486)
				(407.125,0.103059)
				(649.355,0.0876781)
				(858.65,0.0796044)
				(1032.29,0.0704687)
				(1280.94,0.0629764)
				(1505.72,0.0542915)
				(1649.74,0.0516108)
				(1940.98,0.0454305)
				(2101.2,0.0417894)
				(2570.18,0.0321442)
				(2916.36,0.0287216)
				(3413.13,0.0226025)
				(3882.85,0.0185193)
				(4457.4,0.0153654)
				(5093.4,0.0104125)
				(5663.8,0.00472776)
			};

			% CW rec.std[0] %%%%%%%%%%%%%%%%%%%%%%%%%%%%%%%%%%%%%%%%%%%%%%%%%%%%%%%%%%%%
			\addplot[CW] coordinates{
				(197.775,0.111632)
				(309.29,0.100043)
				(405.935,0.0894725)
				(609.18,0.0762735)
				(807.005,0.0651265)
				(1045.56,0.0562637)
				(1252.72,0.0499714)
				(1492.83,0.0447704)
				(1649.62,0.0404758)
				(1816.43,0.0375207)
				(2091.79,0.0328258)
				(2565.8,0.0256812)
				(2917.57,0.021016)
				(3319.52,0.0174897)
				(4023.98,0.0112887)
				(4023.98,0.0112887)
				(4639.83,0.00874487)
				(5846.73,0.00377432)
			};

			% TP rec.std[0] %%%%%%%%%%%%%%%%%%%%%%%%%%%%%%%%%%%%%%%%%%%%%%%%%%%%%%%%%%%%
			\addplot[TP] coordinates{
				(283.695,0.11054)
				(381.285,0.0997296)
				(555.1,0.0852687)
				(752.29,0.0761394)
				(1007.04,0.0632988)
				(1292.4,0.0572633)
				(1495.14,0.0522598)
				(1696.62,0.0549577)
				(2017.73,0.0430152)
				(2478.02,0.0359111)
				(2831,0.0284735)
				(3018.75,0.0278791)
			};

			% POISE rec.std[0] %%%%%%%%%%%%%%%%%%%%%%%%%%%%%%%%%%%%%%%%%%%%%%%%%%%%%%%%%%%%
			\addplot[POISE] coordinates{
				(204.855,0.0796145)
				(306.68,0.0730927)
				(408.385,0.0681572)
				(611.79,0.0598966)
				(814.975,0.0529813)
				(1017.48,0.0480703)
				(1216.26,0.044757)
				(1409.36,0.0418813)
				(1587.37,0.0402151)
				(1749.07,0.0387607)
				(1887.06,0.037804)
				(2090.32,0.0370129)
				(2221.84,0.036773)
				(2278.76,0.036674)
				(2287.5,0.0366756)
				(2288.94,0.0366902)
				(2288.94,0.0366902)
				(2288.94,0.0366902)
			};

			% FH rec.std[0] %%%%%%%%%%%%%%%%%%%%%%%%%%%%%%%%%%%%%%%%%%%%%%%%%%%%%%%%%%%%
			\addplot[FH] coordinates{
				(628.745,0.0922458)
				(782.42,0.0905133)
				(799.09,0.0949964)
				(963.36,0.0766777)
				(1090.39,0.0822844)
				(1187.04,0.0689415)
				(1605.71,0.063983)
				(2533.01,0.0401112)
				(3000.74,0.0369782)
				(3219.63,0.0321126)
				(3814.42,0.0388191)
				(4746.96,0.0307074)
			};

			% EAMS rec.std[0] %%%%%%%%%%%%%%%%%%%%%%%%%%%%%%%%%%%%%%%%%%%%%%%%%%%%%%%%%%%%
			\addplot[EAMS] coordinates{
				(261.12,0.0696242)
				(283.33,0.0681056)
				(309.87,0.0644776)
				(383.52,0.0590003)
				(499.435,0.0517088)
				(725.36,0.04136)
				(1417.35,0.0263491)
				(1473.6,0.0254914)
				(1534.33,0.0245601)
				(1883.59,0.029062)
				(2063.65,0.0265822)
				(2346.71,0.0241156)
				(2642.78,0.0223132)
				(3113.25,0.0263875)
				(3640.44,0.0250551)
				(4914.12,0.0234121)
				(8189.58,0.0245662)
			};

			% CRS rec.std[0] %%%%%%%%%%%%%%%%%%%%%%%%%%%%%%%%%%%%%%%%%%%%%%%%%%%%%%%%%%%%
			\addplot[CRS] coordinates{
				(299.575,0.0778491)
				(449.585,0.0675577)
				(635.865,0.0582453)
				(895.935,0.046289)
				(1180.31,0.0390213)
				(1530.1,0.0314146)
				(1758.22,0.0281197)
				(2057.15,0.023347)
				(2308.66,0.0200983)
				(2461.75,0.0186188)
				(2755.4,0.0155849)
				(3401.31,0.0113755)
				(3879.51,0.00918047)
				(4346.83,0.00770108)
				(5208.23,0.00497951)
				(5208.23,0.00497951)
				(5821.7,0.00351257)
				(7241.41,0.00196832)
			};

			% SEAW rec.std[0] %%%%%%%%%%%%%%%%%%%%%%%%%%%%%%%%%%%%%%%%%%%%%%%%%%%%%%%%%%%%
			\addplot[SEAW] coordinates{
				(165.505,0.114757)
				(356.805,0.103055)
				(923.49,0.0777759)
				(2933.9,0.0336896)
				(10727.5,0.000732422)
			};

			% RESEEDS rec.std[0] %%%%%%%%%%%%%%%%%%%%%%%%%%%%%%%%%%%%%%%%%%%%%%%%%%%%%%%%%%%%
			\addplot[RESEEDS] coordinates{
				(200.795,0.044785)
				(301.525,0.0330628)
				(401.465,0.0294065)
				(602.33,0.0216667)
				(800.99,0.0190602)
				(1020.12,0.0162935)
				(1201.34,0.0131519)
				(1378.1,0.0114017)
				(1601.55,0.0103061)
				(1802.11,0.00930941)
				(2040.11,0.00816686)
				(2402.13,0.00663236)
				(2720.13,0.00532652)
				(3200.2,0.00476543)
				(3840.22,0.00294996)
				(3840.22,0.00294996)
				(4800.37,0.00227719)
				(4800.37,0.00227719)
			};

			% ERGC rec.std[0] %%%%%%%%%%%%%%%%%%%%%%%%%%%%%%%%%%%%%%%%%%%%%%%%%%%%%%%%%%%%
			\addplot[ERGC] coordinates{
				(196,0.10377)
				(306,0.0874334)
				(400,0.0777541)
				(600,0.0643184)
				(792.57,0.0541519)
				(1024,0.0434131)
				(1224,0.0373288)
				(1453.2,0.0323687)
				(1600,0.0281726)
				(1760,0.0268466)
				(2024,0.0221415)
				(2472.98,0.0172028)
				(2809,0.0135405)
				(3180,0.0112172)
				(3840,0.00773583)
				(3840,0.00773583)
				(4416,0.00601498)
				(5520,0.00296005)
			};

			% PF rec.std[0] %%%%%%%%%%%%%%%%%%%%%%%%%%%%%%%%%%%%%%%%%%%%%%%%%%%%%%%%%%%%
			\addplot[PF] coordinates{
				(281.06,0.105609)
				(428.82,0.103671)
				(585.08,0.101884)
				(928.805,0.0915889)
				(1172.76,0.0890037)
				(1602.65,0.0824218)
				(1861.72,0.0771181)
				(2267.11,0.0727437)
				(2697.08,0.0666499)
				(3382.15,0.0612097)
				(4207.66,0.0540935)
				(6051.74,0.0424405)
			};

			% TPS rec.std[0] %%%%%%%%%%%%%%%%%%%%%%%%%%%%%%%%%%%%%%%%%%%%%%%%%%%%%%%%%%%%
			\addplot[TPS] coordinates{
				(224.26,0.100941)
				(903.595,0.0727363)
				(1130.53,0.0679558)
				(1332.49,0.0602211)
				(1556.56,0.05652)
				(1736.84,0.0527615)
				(1988.95,0.049716)
				(2167.04,0.0455)
				(2656.03,0.0391722)
				(2987.25,0.0336834)
				(3505.61,0.0278117)
				(4030.29,0.0223225)
				(4568.25,0.0204816)
				(5242.04,0.0181536)
				(5978.67,0.0181503)
			};

			% NC rec.std[0] %%%%%%%%%%%%%%%%%%%%%%%%%%%%%%%%%%%%%%%%%%%%%%%%%%%%%%%%%%%%
			\addplot[NC] coordinates{
				(223.505,0.0974716)
				(321.69,0.0893965)
				(416.03,0.082275)
				(594.185,0.0741247)
				(763.865,0.0662076)
				(922.645,0.0609144)
				(1073.6,0.0561391)
				(1213.8,0.0524346)
				(1348.36,0.0491736)
				(1473.43,0.0458971)
				(1591.75,0.0437318)
				(2000.27,0.0358155)
			};

			% VC rec.std[0] %%%%%%%%%%%%%%%%%%%%%%%%%%%%%%%%%%%%%%%%%%%%%%%%%%%%%%%%%%%%
			\addplot[VC] coordinates{
				(351.72,0.116234)
				(498.095,0.107885)
				(712.82,0.100095)
				(994.125,0.0846101)
				(1201.27,0.0780685)
				(1375.27,0.0737741)
				(1685.08,0.0657305)
				(1968.97,0.0612126)
				(2224.75,0.0534629)
				(2476.43,0.0491305)
				(2712.29,0.0450769)
				(2948.04,0.0424018)
				(3169.69,0.0399638)
				(3389.95,0.037127)
				(3804.08,0.0345638)
				(4196.75,0.0308246)
				(4566.66,0.0287993)
				(4815.37,0.0287817)
				(5157.48,0.0260945)
				(5401.49,0.0273154)
			};

			% PB rec.std[0] %%%%%%%%%%%%%%%%%%%%%%%%%%%%%%%%%%%%%%%%%%%%%%%%%%%%%%%%%%%%
			\addplot[PB] coordinates{
				(324.39,0.0729539)
				(398.425,0.0726198)
				(494.17,0.0690655)
				(692.845,0.0605095)
				(853.905,0.0555698)
				(1129.27,0.0502153)
				(1323.88,0.0457234)
				(1323.88,0.0457234)
				(1601.37,0.0408423)
				(1929.58,0.0347598)
				(1929.58,0.0347598)
				(2453.31,0.0279879)
				(3126.91,0.0203868)
				(3126.91,0.0203868)
				(3126.91,0.0203868)
				(4270.56,0.0129371)
				(4270.56,0.0129371)
				(6122.31,0.0060001)
			};

			% PRESLIC rec.std[0] %%%%%%%%%%%%%%%%%%%%%%%%%%%%%%%%%%%%%%%%%%%%%%%%%%%%%%%%%%%%
			\addplot[PRESLIC] coordinates{
				(369,0.07808)
				(581.54,0.0672633)
				(734.575,0.0617473)
				(1020.3,0.051963)
				(1229.22,0.0481137)
				(1229.22,0.0481137)
				(1511.3,0.0414622)
				(1838.94,0.035257)
				(1838.94,0.035257)
				(2375.99,0.0282961)
				(3059.43,0.0205818)
				(3059.43,0.0205818)
				(3059.43,0.0205818)
				(4218.88,0.0129602)
				(4218.88,0.0129602)
				(6137.8,0.00534328)
			};

			% W rec.std[0] %%%%%%%%%%%%%%%%%%%%%%%%%%%%%%%%%%%%%%%%%%%%%%%%%%%%%%%%%%%%
			\addplot[W] coordinates{
				(188.285,0.115268)
				(296.48,0.104674)
				(387.92,0.0966368)
				(608.055,0.0774303)
				(794.47,0.0691172)
				(1100.87,0.056233)
				(1302.83,0.0497304)
				(1302.83,0.0497304)
				(1572.26,0.0420475)
				(1960.24,0.0341099)
				(1960.24,0.0341099)
				(2478.43,0.0266225)
				(3292.41,0.0176357)
				(3292.41,0.0176357)
				(3292.41,0.0176357)
				(4439.31,0.00980826)
				(4439.31,0.00980826)
				(6526.67,0.00334749)
			};

			% LSC rec.std[0] %%%%%%%%%%%%%%%%%%%%%%%%%%%%%%%%%%%%%%%%%%%%%%%%%%%%%%%%%%%%
			\addplot[LSC] coordinates{
				(548.755,0.065969)
				(756.92,0.0572847)
				(1045.31,0.0474739)
				(1360.95,0.0376571)
				(1665.85,0.0348797)
				(1919.65,0.030069)
				(2055.98,0.0270104)
				(2239.5,0.0244775)
				(2376.71,0.0240141)
				(2441.63,0.0231959)
				(2603.55,0.0220106)
				(2909.2,0.0200017)
				(3147.7,0.0192159)
				(3373.12,0.0166625)
				(3762.88,0.0162899)
				(3762.88,0.0162899)
				(4057.29,0.0141032)
				(4401.67,0.0145094)
			};

			% WP rec.std[0] %%%%%%%%%%%%%%%%%%%%%%%%%%%%%%%%%%%%%%%%%%%%%%%%%%%%%%%%%%%%
			\addplot[WP] coordinates{
				(216,0.112737)
				(294,0.101799)
				(384,0.0936533)
				(600,0.0811609)
				(805,0.0715128)
				(1080,0.0609105)
				(1320,0.0543321)
				(1536,0.0495666)
				(1536,0.0496122)
				(1944,0.0427023)
				(1944,0.0429681)
				(2400,0.0335656)
				(3174,0.0244908)
				(3174,0.0247174)
				(4320,0.0148184)
				(4320,0.0148184)
				(4320,0.0148184)
				(5836.85,0.00757623)
			};

			% QS rec.std[0] %%%%%%%%%%%%%%%%%%%%%%%%%%%%%%%%%%%%%%%%%%%%%%%%%%%%%%%%%%%%
			\addplot[QS] coordinates{
				(324.725,0.115725)
				(468.655,0.133299)
				(615,0.139092)
				(884.79,0.132898)
				(984.505,0.129267)
				(1275.62,0.11497)
				(1500.06,0.108377)
				(1646.89,0.104462)
				(1824.2,0.0998854)
				(2040.32,0.0963985)
				(2303.69,0.0898238)
				(2626.4,0.083945)
				(3066.62,0.0770189)
				(3646.03,0.0697298)
				(4444.58,0.0616217)
				(5554.87,0.0527022)
			};

			% VLSLIC rec.std[0] %%%%%%%%%%%%%%%%%%%%%%%%%%%%%%%%%%%%%%%%%%%%%%%%%%%%%%%%%%%%
			\addplot[VLSLIC] coordinates{
				(575.975,0.0601791)
				(651.86,0.0561868)
				(763.62,0.0510814)
				(899,0.0493623)
				(988.985,0.0463128)
				(1193.67,0.0439005)
				(1348.08,0.040495)
				(1349.01,0.0411563)
				(1579.08,0.0394542)
				(1849.65,0.0354769)
				(1857.27,0.0362333)
				(2307.73,0.0303904)
				(2890.56,0.0231676)
				(2924.52,0.0228151)
				(2954.18,0.0233509)
				(3858.98,0.0154408)
				(3858.98,0.0154408)
				(4809.57,0.0101016)
			};

			% CIS rec.std[0] %%%%%%%%%%%%%%%%%%%%%%%%%%%%%%%%%%%%%%%%%%%%%%%%%%%%%%%%%%%%
			\addplot[CIS] coordinates{
				(317.63,0.113543)
				(411.595,0.109012)
				(472.685,0.104195)
				(633.345,0.0945805)
				(777.05,0.0865708)
				(969.085,0.078534)
				(1150.5,0.0751878)
				(1168.82,0.0750013)
				(1371.27,0.0680207)
				(1637.39,0.0613575)
				(1667.02,0.0605971)
				(2056.01,0.0523538)
				(3126.04,0.045378)
				(3625.6,0.0435345)
				(4558.92,0.0322552)
				(4558.92,0.0322552)
				(6448.65,0.0227614)
			};

			% ERS rec.std[0] %%%%%%%%%%%%%%%%%%%%%%%%%%%%%%%%%%%%%%%%%%%%%%%%%%%%%%%%%%%%
			\addplot[ERS] coordinates{
				(200,0.0875587)
				(300,0.0784619)
				(400,0.0703833)
				(600,0.0577391)
				(800,0.0494962)
				(1000,0.0425954)
				(1200,0.0373479)
				(1400,0.032754)
				(1600,0.0285718)
				(1800,0.0249705)
				(2000,0.0225549)
				(2400,0.0177921)
				(2800,0.0133387)
				(3200,0.0111799)
				(3600,0.00843964)
				(4000,0.00636173)
				(4600,0.00455438)
				(5200,0.00322044)
			};

			% MSS rec.std[0] %%%%%%%%%%%%%%%%%%%%%%%%%%%%%%%%%%%%%%%%%%%%%%%%%%%%%%%%%%%%
			\addplot[MSS] coordinates{
				(200.765,0.102978)
				(282.33,0.0949629)
				(420.66,0.079937)
				(664.115,0.068117)
				(837.885,0.0627921)
				(1179.3,0.0517267)
				(1422.34,0.0467655)
				(1423.04,0.0468712)
				(1768,0.0408657)
				(2156.96,0.0348908)
				(2157.01,0.0347461)
				(2826.35,0.0273198)
				(3669.97,0.0202945)
				(3670.34,0.0203077)
				(3669.92,0.0201309)
				(5213.45,0.0122558)
				(5213.45,0.0122558)
				(7846.35,0.00543179)
			};

			% ETPS rec.std[0] %%%%%%%%%%%%%%%%%%%%%%%%%%%%%%%%%%%%%%%%%%%%%%%%%%%%%%%%%%%%
			\addplot[ETPS] coordinates{
				(216,0.0524135)
				(294,0.0430699)
				(425,0.0371062)
				(651,0.028091)
				(805,0.0234959)
				(1107,0.0188527)
				(1320,0.0152524)
				(1320,0.0152524)
				(1617,0.0141201)
				(1944,0.0109046)
				(1944,0.0109046)
				(2501,0.00817415)
				(3174,0.00539324)
				(3174,0.00539324)
				(3174,0.00539324)
				(4374,0.0035798)
				(4374,0.0035798)
				(6305,0.00119604)
			};

			\end{axis}
	\end{tikzpicture}
\end{subfigure}%
\begin{subfigure}[b]{\fullthreeone\textwidth}\phantomsubcaption\label{subfig:appendix-experiments-bsds500-ue_np.std[0]}
	%%%%%%%%%%%%%%%%%%%%%%%%%%%%%%%%%%%%%%%%%%%%%%%%%%%%%%%%%%%%
	% ue_np.std[0]
	%%%%%%%%%%%%%%%%%%%%%%%%%%%%%%%%%%%%%%%%%%%%%%%%%%%%%%%%%%%%
	\begin{tikzpicture}
		\begin{axis}[AEBSDS500UEStd,xmode=log]

			% CCS ue_np.std[0] %%%%%%%%%%%%%%%%%%%%%%%%%%%%%%%%%%%%%%%%%%%%%%%%%%%%%%%%%%%%
			\addplot[CCS] coordinates{
				(218.865,0.0573979)
				(303.21,0.0507333)
				(453.82,0.0428808)
				(706.35,0.0357631)
				(871.12,0.0333757)
				(1204.99,0.0292992)
				(1438.62,0.0277122)
				(1438.62,0.0277122)
				(1762.49,0.0257009)
				(2107.24,0.0242115)
				(2107.24,0.0242115)
				(2702.61,0.0222749)
				(3406.7,0.0205896)
				(3406.7,0.0205896)
				(3406.7,0.0205896)
				(4654.44,0.0185554)
				(4654.44,0.0185554)
				(6619.1,0.0164139)
			};

			% SEEDS ue_np.std[0] %%%%%%%%%%%%%%%%%%%%%%%%%%%%%%%%%%%%%%%%%%%%%%%%%%%%%%%%%%%%
			\addplot[SEEDS] coordinates{
				(261.62,0.0786418)
				(365.675,0.0659283)
				(468.81,0.0591277)
				(670.57,0.0508161)
				(870.75,0.0443438)
				(1087.4,0.0403172)
				(1270.11,0.0380875)
				(1451.85,0.0362885)
				(1669.18,0.0332317)
				(1873.19,0.0321516)
				(2104.62,0.0351687)
				(2462.77,0.0287896)
				(2793.43,0.0290278)
				(3260.86,0.025396)
				(3895.78,0.0240398)
				(3895.78,0.0240398)
				(4846.12,0.0215983)
				(4846.12,0.0215983)
			};

			% SLIC ue_np.std[0] %%%%%%%%%%%%%%%%%%%%%%%%%%%%%%%%%%%%%%%%%%%%%%%%%%%%%%%%%%%%
			\addplot[SLIC] coordinates{
				(180.32,0.0582102)
				(256.725,0.0498522)
				(368.57,0.0452457)
				(575.335,0.0386477)
				(726.36,0.0351438)
				(1002.87,0.0322361)
				(1203.61,0.0307743)
				(1203.61,0.0307743)
				(1475.69,0.0289912)
				(1814.8,0.0263025)
				(1814.8,0.0263025)
				(2334.56,0.0242399)
				(3038.13,0.0221133)
				(3038.13,0.0221133)
				(3038.13,0.0221133)
				(4188.97,0.0197188)
				(4188.97,0.0197188)
				(6139.41,0.017199)
			};

			% RW ue_np.std[0] %%%%%%%%%%%%%%%%%%%%%%%%%%%%%%%%%%%%%%%%%%%%%%%%%%%%%%%%%%%%
			\addplot[RW] coordinates{
				(211.565,0.0750155)
				(313.64,0.0627322)
				(407.125,0.0559593)
				(649.355,0.0441522)
				(858.65,0.0400708)
				(1032.29,0.0375646)
				(1280.94,0.033949)
				(1505.72,0.0318381)
				(1649.74,0.0306342)
				(1940.98,0.0290988)
				(2101.2,0.0277584)
				(2570.18,0.0256713)
				(2916.36,0.0246111)
				(3413.13,0.0235912)
				(3882.85,0.0223593)
				(4457.4,0.0209933)
				(5093.4,0.020496)
				(5663.8,0.0142829)
			};

			% CW ue_np.std[0] %%%%%%%%%%%%%%%%%%%%%%%%%%%%%%%%%%%%%%%%%%%%%%%%%%%%%%%%%%%%
			\addplot[CW] coordinates{
				(197.775,0.0640942)
				(309.29,0.0549975)
				(405.935,0.0493221)
				(609.18,0.0425949)
				(807.005,0.0384396)
				(1045.56,0.0348167)
				(1252.72,0.0330449)
				(1492.83,0.0307064)
				(1649.62,0.0298564)
				(1816.43,0.0288791)
				(2091.79,0.0278232)
				(2565.8,0.0255589)
				(2917.57,0.0246846)
				(3319.52,0.0233364)
				(4023.98,0.0217765)
				(4023.98,0.0217765)
				(4639.83,0.0206358)
				(5846.73,0.0193244)
			};

			% TP ue_np.std[0] %%%%%%%%%%%%%%%%%%%%%%%%%%%%%%%%%%%%%%%%%%%%%%%%%%%%%%%%%%%%
			\addplot[TP] coordinates{
				(283.695,0.0551568)
				(381.285,0.0473309)
				(555.1,0.04001)
				(752.29,0.0360284)
				(1007.04,0.0324786)
				(1292.4,0.0294384)
				(1495.14,0.0282358)
				(1696.62,0.0557787)
				(2017.73,0.036745)
				(2478.02,0.0277704)
				(2831,0.0250031)
				(3018.75,0.0259149)
			};

			% POISE ue_np.std[0] %%%%%%%%%%%%%%%%%%%%%%%%%%%%%%%%%%%%%%%%%%%%%%%%%%%%%%%%%%%%
			\addplot[POISE] coordinates{
				(204.855,0.059946)
				(306.68,0.0506179)
				(408.385,0.0461524)
				(611.79,0.0419868)
				(814.975,0.0373747)
				(1017.48,0.0352989)
				(1216.26,0.0338858)
				(1409.36,0.0325358)
				(1587.37,0.0315744)
				(1749.07,0.0309883)
				(1887.06,0.0305959)
				(2090.32,0.0299685)
				(2221.84,0.0296882)
				(2278.76,0.0296267)
				(2287.5,0.0296188)
				(2288.94,0.0296186)
				(2288.94,0.0296186)
				(2288.94,0.0296186)
			};

			% FH ue_np.std[0] %%%%%%%%%%%%%%%%%%%%%%%%%%%%%%%%%%%%%%%%%%%%%%%%%%%%%%%%%%%%
			\addplot[FH] coordinates{
				(628.745,0.0625047)
				(782.42,0.0566026)
				(799.09,0.0541148)
				(963.36,0.0496079)
				(1090.39,0.04738)
				(1187.04,0.0443132)
				(1605.71,0.0393973)
				(2533.01,0.0308612)
				(3000.74,0.0295201)
				(3219.63,0.0286032)
				(3814.42,0.0305052)
				(4746.96,0.0260483)
			};

			% EAMS ue_np.std[0] %%%%%%%%%%%%%%%%%%%%%%%%%%%%%%%%%%%%%%%%%%%%%%%%%%%%%%%%%%%%
			\addplot[EAMS] coordinates{
				(261.12,0.0570883)
				(283.33,0.0551414)
				(309.87,0.0531531)
				(383.52,0.0486786)
				(499.435,0.0442447)
				(725.36,0.0391673)
				(1417.35,0.0317302)
				(1473.6,0.0314314)
				(1534.33,0.0310098)
				(1883.59,0.0296475)
				(2063.65,0.0288875)
				(2346.71,0.0278151)
				(2642.78,0.0269548)
				(3113.25,0.0441089)
				(3640.44,0.0434444)
				(4914.12,0.0423792)
				(8189.58,0.0429032)
			};

			% CRS ue_np.std[0] %%%%%%%%%%%%%%%%%%%%%%%%%%%%%%%%%%%%%%%%%%%%%%%%%%%%%%%%%%%%
			\addplot[CRS] coordinates{
				(299.575,0.0483747)
				(449.585,0.0409608)
				(635.865,0.0367769)
				(895.935,0.0325031)
				(1180.31,0.0296947)
				(1530.1,0.0274322)
				(1758.22,0.0263886)
				(2057.15,0.024715)
				(2308.66,0.0237685)
				(2461.75,0.0233474)
				(2755.4,0.0224843)
				(3401.31,0.0206808)
				(3879.51,0.0200573)
				(4346.83,0.0193858)
				(5208.23,0.0182965)
				(5208.23,0.0182965)
				(5821.7,0.0174823)
				(7241.41,0.0160087)
			};

			% SEAW ue_np.std[0] %%%%%%%%%%%%%%%%%%%%%%%%%%%%%%%%%%%%%%%%%%%%%%%%%%%%%%%%%%%%
			\addplot[SEAW] coordinates{
				(165.505,0.113138)
				(356.805,0.0696652)
				(923.49,0.0418806)
				(2933.9,0.0251984)
				(10727.5,0.0147954)
			};

			% RESEEDS ue_np.std[0] %%%%%%%%%%%%%%%%%%%%%%%%%%%%%%%%%%%%%%%%%%%%%%%%%%%%%%%%%%%%
			\addplot[RESEEDS] coordinates{
				(200.795,0.0597004)
				(301.525,0.0534303)
				(401.465,0.0448365)
				(602.33,0.0379767)
				(800.99,0.0354624)
				(1020.12,0.0342144)
				(1201.34,0.0309877)
				(1378.1,0.0302996)
				(1601.55,0.0273695)
				(1802.11,0.027186)
				(2040.11,0.0287296)
				(2402.13,0.0240812)
				(2720.13,0.024206)
				(3200.2,0.0221167)
				(3840.22,0.021223)
				(3840.22,0.021223)
				(4800.37,0.0194078)
				(4800.37,0.0194078)
			};

			% ERGC ue_np.std[0] %%%%%%%%%%%%%%%%%%%%%%%%%%%%%%%%%%%%%%%%%%%%%%%%%%%%%%%%%%%%
			\addplot[ERGC] coordinates{
				(196,0.0545114)
				(306,0.0455632)
				(400,0.0413559)
				(600,0.035617)
				(792.57,0.0328048)
				(1024,0.0302129)
				(1224,0.0286113)
				(1453.2,0.0270771)
				(1600,0.0260915)
				(1760,0.0253235)
				(2024,0.0242403)
				(2472.98,0.0228327)
				(2809,0.0221469)
				(3180,0.0210941)
				(3840,0.0199626)
				(3840,0.0199626)
				(4416,0.0190125)
				(5520,0.0179008)
			};

			% PF ue_np.std[0] %%%%%%%%%%%%%%%%%%%%%%%%%%%%%%%%%%%%%%%%%%%%%%%%%%%%%%%%%%%%
			\addplot[PF] coordinates{
				(281.06,0.103684)
				(428.82,0.0932107)
				(585.08,0.0843692)
				(928.805,0.0712006)
				(1172.76,0.0668073)
				(1602.65,0.0601947)
				(1861.72,0.0571693)
				(2267.11,0.052679)
				(2697.08,0.0503843)
				(3382.15,0.0484765)
				(4207.66,0.0448751)
				(6051.74,0.0396746)
			};

			% TPS ue_np.std[0] %%%%%%%%%%%%%%%%%%%%%%%%%%%%%%%%%%%%%%%%%%%%%%%%%%%%%%%%%%%%
			\addplot[TPS] coordinates{
				(224.26,0.0658458)
				(903.595,0.0389243)
				(1130.53,0.0363763)
				(1332.49,0.033961)
				(1556.56,0.0324287)
				(1736.84,0.0308534)
				(1988.95,0.0293387)
				(2167.04,0.0282131)
				(2656.03,0.0267515)
				(2987.25,0.0253983)
				(3505.61,0.023968)
				(4030.29,0.0222943)
				(4568.25,0.0216836)
				(5242.04,0.0205659)
				(5978.67,0.0195821)
			};

			% NC ue_np.std[0] %%%%%%%%%%%%%%%%%%%%%%%%%%%%%%%%%%%%%%%%%%%%%%%%%%%%%%%%%%%%
			\addplot[NC] coordinates{
				(223.505,0.0517745)
				(321.69,0.0455704)
				(416.03,0.0419068)
				(594.185,0.0381477)
				(763.865,0.0357005)
				(922.645,0.0342714)
				(1073.6,0.0331687)
				(1213.8,0.032324)
				(1348.36,0.0315903)
				(1473.43,0.0308718)
				(1591.75,0.0304236)
				(2000.27,0.0291325)
			};

			% VC ue_np.std[0] %%%%%%%%%%%%%%%%%%%%%%%%%%%%%%%%%%%%%%%%%%%%%%%%%%%%%%%%%%%%
			\addplot[VC] coordinates{
				(351.72,0.133761)
				(498.095,0.0951477)
				(712.82,0.0704989)
				(994.125,0.052287)
				(1201.27,0.0448528)
				(1375.27,0.0399932)
				(1685.08,0.0347299)
				(1968.97,0.0313477)
				(2224.75,0.0289444)
				(2476.43,0.0273979)
				(2712.29,0.0264313)
				(2948.04,0.0250424)
				(3169.69,0.024115)
				(3389.95,0.0232462)
				(3804.08,0.0221652)
				(4196.75,0.0212288)
				(4566.66,0.0206571)
				(4815.37,0.0203753)
				(5157.48,0.0196214)
				(5401.49,0.0189977)
			};

			% PB ue_np.std[0] %%%%%%%%%%%%%%%%%%%%%%%%%%%%%%%%%%%%%%%%%%%%%%%%%%%%%%%%%%%%
			\addplot[PB] coordinates{
				(324.39,0.0709193)
				(398.425,0.0623595)
				(494.17,0.0549955)
				(692.845,0.0467013)
				(853.905,0.04128)
				(1129.27,0.0370284)
				(1323.88,0.0344589)
				(1323.88,0.0344589)
				(1601.37,0.0316696)
				(1929.58,0.0296654)
				(1929.58,0.0296654)
				(2453.31,0.0267422)
				(3126.91,0.0244123)
				(3126.91,0.0244123)
				(3126.91,0.0244123)
				(4270.56,0.0214788)
				(4270.56,0.0214788)
				(6122.31,0.0188372)
			};

			% PRESLIC ue_np.std[0] %%%%%%%%%%%%%%%%%%%%%%%%%%%%%%%%%%%%%%%%%%%%%%%%%%%%%%%%%%%%
			\addplot[PRESLIC] coordinates{
				(369,0.0503489)
				(581.54,0.0413338)
				(734.575,0.0389771)
				(1020.3,0.0339636)
				(1229.22,0.0315266)
				(1229.22,0.0315266)
				(1511.3,0.0289835)
				(1838.94,0.0272065)
				(1838.94,0.0272065)
				(2375.99,0.0249788)
				(3059.43,0.0229036)
				(3059.43,0.0229036)
				(3059.43,0.0229036)
				(4218.88,0.0202563)
				(4218.88,0.0202563)
				(6137.8,0.0176205)
			};

			% W ue_np.std[0] %%%%%%%%%%%%%%%%%%%%%%%%%%%%%%%%%%%%%%%%%%%%%%%%%%%%%%%%%%%%
			\addplot[W] coordinates{
				(188.285,0.0806662)
				(296.48,0.0640348)
				(387.92,0.0562621)
				(608.055,0.0457299)
				(794.47,0.0414263)
				(1100.87,0.0356777)
				(1302.83,0.0333972)
				(1302.83,0.0333972)
				(1572.26,0.0314562)
				(1960.24,0.0289976)
				(1960.24,0.0289976)
				(2478.43,0.0263062)
				(3292.41,0.0237052)
				(3292.41,0.0237052)
				(3292.41,0.0237052)
				(4439.31,0.0215917)
				(4439.31,0.0215917)
				(6526.67,0.0186598)
			};

			% LSC ue_np.std[0] %%%%%%%%%%%%%%%%%%%%%%%%%%%%%%%%%%%%%%%%%%%%%%%%%%%%%%%%%%%%
			\addplot[LSC] coordinates{
				(548.755,0.0476709)
				(756.92,0.0394901)
				(1045.31,0.034449)
				(1360.95,0.0311868)
				(1665.85,0.0294457)
				(1919.65,0.0280409)
				(2055.98,0.0272788)
				(2239.5,0.0264446)
				(2376.71,0.0262862)
				(2441.63,0.0257111)
				(2603.55,0.0252999)
				(2909.2,0.0248916)
				(3147.7,0.0246666)
				(3373.12,0.0240973)
				(3762.88,0.0240207)
				(3762.88,0.0240207)
				(4057.29,0.0240433)
				(4401.67,0.0259204)
			};

			% WP ue_np.std[0] %%%%%%%%%%%%%%%%%%%%%%%%%%%%%%%%%%%%%%%%%%%%%%%%%%%%%%%%%%%%
			\addplot[WP] coordinates{
				(216,0.0643212)
				(294,0.0549375)
				(384,0.0494517)
				(600,0.042305)
				(805,0.0381088)
				(1080,0.0338887)
				(1320,0.0312889)
				(1536,0.0301621)
				(1536,0.0300343)
				(1944,0.0275528)
				(1944,0.0274787)
				(2400,0.0254554)
				(3174,0.0232561)
				(3174,0.0231169)
				(4320,0.021109)
				(4320,0.021109)
				(4320,0.021109)
				(5836.85,0.0186523)
			};

			% QS ue_np.std[0] %%%%%%%%%%%%%%%%%%%%%%%%%%%%%%%%%%%%%%%%%%%%%%%%%%%%%%%%%%%%
			\addplot[QS] coordinates{
				(324.725,0.172189)
				(468.655,0.138231)
				(615,0.119204)
				(884.79,0.0908425)
				(984.505,0.0832242)
				(1275.62,0.0658419)
				(1500.06,0.0581744)
				(1646.89,0.0546641)
				(1824.2,0.0505733)
				(2040.32,0.0469403)
				(2303.69,0.043629)
				(2626.4,0.0400423)
				(3066.62,0.0363715)
				(3646.03,0.0328123)
				(4444.58,0.0294767)
				(5554.87,0.0263526)
			};

			% VLSLIC ue_np.std[0] %%%%%%%%%%%%%%%%%%%%%%%%%%%%%%%%%%%%%%%%%%%%%%%%%%%%%%%%%%%%
			\addplot[VLSLIC] coordinates{
				(575.975,0.0479674)
				(651.86,0.0429637)
				(763.62,0.040284)
				(899,0.0366771)
				(988.985,0.0350354)
				(1193.67,0.0326467)
				(1348.08,0.0310631)
				(1349.01,0.0307935)
				(1579.08,0.0297305)
				(1849.65,0.0281545)
				(1857.27,0.027929)
				(2307.73,0.0264578)
				(2890.56,0.0247041)
				(2924.52,0.0242181)
				(2954.18,0.0238397)
				(3858.98,0.0231625)
				(3858.98,0.0231625)
				(4809.57,0.0248127)
			};

			% CIS ue_np.std[0] %%%%%%%%%%%%%%%%%%%%%%%%%%%%%%%%%%%%%%%%%%%%%%%%%%%%%%%%%%%%
			\addplot[CIS] coordinates{
				(317.63,0.0536811)
				(411.595,0.0465458)
				(472.685,0.0434679)
				(633.345,0.0388311)
				(777.05,0.0352492)
				(969.085,0.0323703)
				(1150.5,0.0307622)
				(1168.82,0.0307927)
				(1371.27,0.0290148)
				(1637.39,0.0272229)
				(1667.02,0.027178)
				(2056.01,0.0251695)
				(3126.04,0.0229196)
				(3625.6,0.0222322)
				(4558.92,0.0202851)
				(4558.92,0.0202851)
				(6448.65,0.0179344)
			};

			% ERS ue_np.std[0] %%%%%%%%%%%%%%%%%%%%%%%%%%%%%%%%%%%%%%%%%%%%%%%%%%%%%%%%%%%%
			\addplot[ERS] coordinates{
				(200,0.0602279)
				(300,0.0518758)
				(400,0.0462857)
				(600,0.0401517)
				(800,0.0367174)
				(1000,0.0342942)
				(1200,0.0320226)
				(1400,0.0305639)
				(1600,0.028815)
				(1800,0.0278926)
				(2000,0.0270545)
				(2400,0.0254745)
				(2800,0.024111)
				(3200,0.0232118)
				(3600,0.0220919)
				(4000,0.0211965)
				(4600,0.0200966)
				(5200,0.0192066)
			};

			% MSS ue_np.std[0] %%%%%%%%%%%%%%%%%%%%%%%%%%%%%%%%%%%%%%%%%%%%%%%%%%%%%%%%%%%%
			\addplot[MSS] coordinates{
				(200.765,0.0687556)
				(282.33,0.0605501)
				(420.66,0.0466281)
				(664.115,0.0389518)
				(837.885,0.0367794)
				(1179.3,0.0316532)
				(1422.34,0.0303124)
				(1423.04,0.030266)
				(1768,0.0279107)
				(2156.96,0.0264377)
				(2157.01,0.0265663)
				(2826.35,0.0240326)
				(3669.97,0.0225641)
				(3670.34,0.0226317)
				(3669.92,0.0226196)
				(5213.45,0.0202434)
				(5213.45,0.0202434)
				(7846.35,0.0178862)
			};

			% ETPS ue_np.std[0] %%%%%%%%%%%%%%%%%%%%%%%%%%%%%%%%%%%%%%%%%%%%%%%%%%%%%%%%%%%%
			\addplot[ETPS] coordinates{
				(216,0.0539346)
				(294,0.047493)
				(425,0.0416631)
				(651,0.035156)
				(805,0.0323982)
				(1107,0.0294626)
				(1320,0.028082)
				(1320,0.028082)
				(1617,0.0260102)
				(1944,0.0245532)
				(1944,0.0245532)
				(2501,0.0226704)
				(3174,0.0209835)
				(3174,0.0209835)
				(3174,0.0209835)
				(4374,0.0188582)
				(4374,0.0188582)
				(6305,0.0168011)
			};

			\end{axis}
	\end{tikzpicture}
\end{subfigure}%
\begin{subfigure}[b]{\fullthreeone\textwidth}\phantomsubcaption\label{subfig:appendix-experiments-bsds500-ev.std[0]}
	%%%%%%%%%%%%%%%%%%%%%%%%%%%%%%%%%%%%%%%%%%%%%%%%%%%%%%%%%%%%
	% ev.std[0]
	%%%%%%%%%%%%%%%%%%%%%%%%%%%%%%%%%%%%%%%%%%%%%%%%%%%%%%%%%%%%
	\begin{tikzpicture}
		\begin{axis}[AEBSDS500EVStd,xmode=log]

			% CCS ev.std[0] %%%%%%%%%%%%%%%%%%%%%%%%%%%%%%%%%%%%%%%%%%%%%%%%%%%%%%%%%%%%
			\addplot[CCS] coordinates{
				(218.865,0.112122)
				(303.21,0.103546)
				(453.82,0.09489)
				(706.35,0.0851011)
				(871.12,0.079558)
				(1204.99,0.0733991)
				(1438.62,0.069246)
				(1438.62,0.069246)
				(1762.49,0.0654315)
				(2107.24,0.0623525)
				(2107.24,0.0623525)
				(2702.61,0.0575245)
				(3406.7,0.0537886)
				(3406.7,0.0537886)
				(3406.7,0.0537886)
				(4654.44,0.048533)
				(4654.44,0.048533)
				(6619.1,0.0423779)
			};

			% SEEDS ev.std[0] %%%%%%%%%%%%%%%%%%%%%%%%%%%%%%%%%%%%%%%%%%%%%%%%%%%%%%%%%%%%
			\addplot[SEEDS] coordinates{
				(261.62,0.0761222)
				(365.675,0.0708231)
				(468.81,0.0659464)
				(670.57,0.0605109)
				(870.75,0.0585113)
				(1087.4,0.0597467)
				(1270.11,0.0539064)
				(1451.85,0.0561943)
				(1669.18,0.0498866)
				(1873.19,0.0516967)
				(2104.62,0.0556833)
				(2462.77,0.0443524)
				(2793.43,0.0488245)
				(3260.86,0.0424334)
				(3895.78,0.0421579)
				(3895.78,0.0421579)
				(4846.12,0.0379684)
				(4846.12,0.0379684)
			};

			% SLIC ev.std[0] %%%%%%%%%%%%%%%%%%%%%%%%%%%%%%%%%%%%%%%%%%%%%%%%%%%%%%%%%%%%
			\addplot[SLIC] coordinates{
				(180.32,0.129535)
				(256.725,0.117302)
				(368.57,0.113529)
				(575.335,0.103693)
				(726.36,0.0986443)
				(1002.87,0.0931581)
				(1203.61,0.0913119)
				(1203.61,0.0913119)
				(1475.69,0.0879188)
				(1814.8,0.081483)
				(1814.8,0.081483)
				(2334.56,0.0771168)
				(3038.13,0.0694875)
				(3038.13,0.0694875)
				(3038.13,0.0694875)
				(4188.97,0.0631461)
				(4188.97,0.0631461)
				(6139.41,0.0539612)
			};

			% RW ev.std[0] %%%%%%%%%%%%%%%%%%%%%%%%%%%%%%%%%%%%%%%%%%%%%%%%%%%%%%%%%%%%
			\addplot[RW] coordinates{
				(211.565,0.151031)
				(313.64,0.141405)
				(407.125,0.132337)
				(649.355,0.117554)
				(858.65,0.108431)
				(1032.29,0.102838)
				(1280.94,0.0974289)
				(1505.72,0.0915133)
				(1649.74,0.089247)
				(1940.98,0.0861657)
				(2101.2,0.0836501)
				(2570.18,0.0768558)
				(2916.36,0.0736196)
				(3413.13,0.0693411)
				(3882.85,0.0664377)
				(4457.4,0.0620473)
				(5093.4,0.0610146)
				(5663.8,0.0516673)
			};

			% CW ev.std[0] %%%%%%%%%%%%%%%%%%%%%%%%%%%%%%%%%%%%%%%%%%%%%%%%%%%%%%%%%%%%
			\addplot[CW] coordinates{
				(197.775,0.144986)
				(309.29,0.132476)
				(405.935,0.127511)
				(609.18,0.116252)
				(807.005,0.109564)
				(1045.56,0.103281)
				(1252.72,0.0999451)
				(1492.83,0.0958301)
				(1649.62,0.0941981)
				(1816.43,0.0923068)
				(2091.79,0.0898056)
				(2565.8,0.0860332)
				(2917.57,0.0836712)
				(3319.52,0.0813509)
				(4023.98,0.0780662)
				(4023.98,0.0780662)
				(4639.83,0.0757725)
				(5846.73,0.0723345)
			};

			% TP ev.std[0] %%%%%%%%%%%%%%%%%%%%%%%%%%%%%%%%%%%%%%%%%%%%%%%%%%%%%%%%%%%%
			\addplot[TP] coordinates{
				(283.695,0.131103)
				(381.285,0.120175)
				(555.1,0.107619)
				(752.29,0.101677)
				(1007.04,0.0964946)
				(1292.4,0.0892794)
				(1495.14,0.0870549)
				(1696.62,0.117647)
				(2017.73,0.10202)
				(2478.02,0.0905689)
				(2831,0.0841942)
				(3018.75,0.0866348)
			};

			% POISE ev.std[0] %%%%%%%%%%%%%%%%%%%%%%%%%%%%%%%%%%%%%%%%%%%%%%%%%%%%%%%%%%%%
			\addplot[POISE] coordinates{
				(204.855,0.10511)
				(306.68,0.0968257)
				(408.385,0.0921795)
				(611.79,0.0879005)
				(814.975,0.0852404)
				(1017.48,0.0836719)
				(1216.26,0.0826011)
				(1409.36,0.0817663)
				(1587.37,0.0812581)
				(1749.07,0.0807559)
				(1887.06,0.0803823)
				(2090.32,0.0799112)
				(2221.84,0.0796374)
				(2278.76,0.079525)
				(2287.5,0.0794999)
				(2288.94,0.0794891)
				(2288.94,0.0794891)
				(2288.94,0.0794891)
			};

			% FH ev.std[0] %%%%%%%%%%%%%%%%%%%%%%%%%%%%%%%%%%%%%%%%%%%%%%%%%%%%%%%%%%%%
			\addplot[FH] coordinates{
				(628.745,0.137992)
				(782.42,0.130095)
				(799.09,0.119457)
				(963.36,0.116719)
				(1090.39,0.108953)
				(1187.04,0.107385)
				(1605.71,0.0945612)
				(2533.01,0.0713041)
				(3000.74,0.0703286)
				(3219.63,0.0697085)
				(3814.42,0.0541811)
				(4746.96,0.0455098)
			};

			% EAMS ev.std[0] %%%%%%%%%%%%%%%%%%%%%%%%%%%%%%%%%%%%%%%%%%%%%%%%%%%%%%%%%%%%
			\addplot[EAMS] coordinates{
				(261.12,0.0917868)
				(283.33,0.0898391)
				(309.87,0.087675)
				(383.52,0.0825502)
				(499.435,0.0771998)
				(725.36,0.0686669)
				(1417.35,0.0542712)
				(1473.6,0.0534378)
				(1534.33,0.0525532)
				(1883.59,0.051936)
				(2063.65,0.0502776)
				(2346.71,0.0477649)
				(2642.78,0.0455922)
				(3113.25,0.0428562)
				(3640.44,0.0400115)
				(4914.12,0.0348035)
				(8189.58,0.0255217)
			};

			% CRS ev.std[0] %%%%%%%%%%%%%%%%%%%%%%%%%%%%%%%%%%%%%%%%%%%%%%%%%%%%%%%%%%%%
			\addplot[CRS] coordinates{
				(299.575,0.109474)
				(449.585,0.102128)
				(635.865,0.0940239)
				(895.935,0.0878024)
				(1180.31,0.0836256)
				(1530.1,0.0778598)
				(1758.22,0.0756474)
				(2057.15,0.0734604)
				(2308.66,0.0713024)
				(2461.75,0.070365)
				(2755.4,0.0691616)
				(3401.31,0.0658465)
				(3879.51,0.0633106)
				(4346.83,0.0615481)
				(5208.23,0.0584716)
				(5208.23,0.0584716)
				(5821.7,0.0572883)
				(7241.41,0.0537902)
			};

			% SEAW ev.std[0] %%%%%%%%%%%%%%%%%%%%%%%%%%%%%%%%%%%%%%%%%%%%%%%%%%%%%%%%%%%%
			\addplot[SEAW] coordinates{
				(165.505,0.136397)
				(356.805,0.113113)
				(923.49,0.0861526)
				(2933.9,0.0591163)
				(10727.5,0.0336773)
			};

			% RESEEDS ev.std[0] %%%%%%%%%%%%%%%%%%%%%%%%%%%%%%%%%%%%%%%%%%%%%%%%%%%%%%%%%%%%
			\addplot[RESEEDS] coordinates{
				(200.795,0.077126)
				(301.525,0.0733731)
				(401.465,0.0674102)
				(602.33,0.0622396)
				(800.99,0.0583777)
				(1020.12,0.0572888)
				(1201.34,0.05458)
				(1378.1,0.0536343)
				(1601.55,0.0505471)
				(1802.11,0.0495118)
				(2040.11,0.05203)
				(2402.13,0.0454705)
				(2720.13,0.0458406)
				(3200.2,0.0415577)
				(3840.22,0.0405582)
				(3840.22,0.0405582)
				(4800.37,0.0369315)
				(4800.37,0.0369315)
			};

			% ERGC ev.std[0] %%%%%%%%%%%%%%%%%%%%%%%%%%%%%%%%%%%%%%%%%%%%%%%%%%%%%%%%%%%%
			\addplot[ERGC] coordinates{
				(196,0.11738)
				(306,0.105497)
				(400,0.0984562)
				(600,0.0884001)
				(792.57,0.0815272)
				(1024,0.0750477)
				(1224,0.0706284)
				(1453.2,0.0661864)
				(1600,0.0638595)
				(1760,0.06198)
				(2024,0.0584379)
				(2472.98,0.0543623)
				(2809,0.0514083)
				(3180,0.0489287)
				(3840,0.0448502)
				(3840,0.0448502)
				(4416,0.0421091)
				(5520,0.0381906)
			};

			% PF ev.std[0] %%%%%%%%%%%%%%%%%%%%%%%%%%%%%%%%%%%%%%%%%%%%%%%%%%%%%%%%%%%%
			\addplot[PF] coordinates{
				(281.06,0.164132)
				(428.82,0.157079)
				(585.08,0.152177)
				(928.805,0.14299)
				(1172.76,0.140249)
				(1602.65,0.133477)
				(1861.72,0.131141)
				(2267.11,0.126898)
				(2697.08,0.122903)
				(3382.15,0.115289)
				(4207.66,0.109599)
				(6051.74,0.100664)
			};

			% TPS ev.std[0] %%%%%%%%%%%%%%%%%%%%%%%%%%%%%%%%%%%%%%%%%%%%%%%%%%%%%%%%%%%%
			\addplot[TPS] coordinates{
				(224.26,0.146875)
				(903.595,0.119288)
				(1130.53,0.115108)
				(1332.49,0.11216)
				(1556.56,0.108486)
				(1736.84,0.105864)
				(1988.95,0.104085)
				(2167.04,0.101588)
				(2656.03,0.0967176)
				(2987.25,0.0942035)
				(3505.61,0.0905512)
				(4030.29,0.0878489)
				(4568.25,0.0862487)
				(5242.04,0.0836986)
				(5978.67,0.0820418)
			};

			% NC ev.std[0] %%%%%%%%%%%%%%%%%%%%%%%%%%%%%%%%%%%%%%%%%%%%%%%%%%%%%%%%%%%%
			\addplot[NC] coordinates{
				(223.505,0.131663)
				(321.69,0.124082)
				(416.03,0.119147)
				(594.185,0.11293)
				(763.865,0.109388)
				(922.645,0.107261)
				(1073.6,0.105227)
				(1213.8,0.103887)
				(1348.36,0.102701)
				(1473.43,0.101854)
				(1591.75,0.101154)
				(2000.27,0.0993541)
			};

			% VC ev.std[0] %%%%%%%%%%%%%%%%%%%%%%%%%%%%%%%%%%%%%%%%%%%%%%%%%%%%%%%%%%%%
			\addplot[VC] coordinates{
				(351.72,0.156596)
				(498.095,0.140105)
				(712.82,0.12108)
				(994.125,0.105463)
				(1201.27,0.0961059)
				(1375.27,0.0897352)
				(1685.08,0.0815627)
				(1968.97,0.076559)
				(2224.75,0.0721786)
				(2476.43,0.0686929)
				(2712.29,0.0665385)
				(2948.04,0.0635835)
				(3169.69,0.0618905)
				(3389.95,0.059911)
				(3804.08,0.0569575)
				(4196.75,0.0542404)
				(4566.66,0.0517797)
				(4815.37,0.0503125)
				(5157.48,0.047783)
				(5401.49,0.0464342)
			};

			% PB ev.std[0] %%%%%%%%%%%%%%%%%%%%%%%%%%%%%%%%%%%%%%%%%%%%%%%%%%%%%%%%%%%%
			\addplot[PB] coordinates{
				(324.39,0.145757)
				(398.425,0.142426)
				(494.17,0.137383)
				(692.845,0.127772)
				(853.905,0.121202)
				(1129.27,0.114891)
				(1323.88,0.11017)
				(1323.88,0.11017)
				(1601.37,0.105394)
				(1929.58,0.10013)
				(1929.58,0.10013)
				(2453.31,0.093574)
				(3126.91,0.0861688)
				(3126.91,0.0861688)
				(3126.91,0.0861688)
				(4270.56,0.0773136)
				(4270.56,0.0773136)
				(6122.31,0.0664175)
			};

			% PRESLIC ev.std[0] %%%%%%%%%%%%%%%%%%%%%%%%%%%%%%%%%%%%%%%%%%%%%%%%%%%%%%%%%%%%
			\addplot[PRESLIC] coordinates{
				(369,0.119157)
				(581.54,0.106385)
				(734.575,0.100642)
				(1020.3,0.0933464)
				(1229.22,0.0889785)
				(1229.22,0.0889785)
				(1511.3,0.084168)
				(1838.94,0.0794309)
				(1838.94,0.0794309)
				(2375.99,0.073745)
				(3059.43,0.0676538)
				(3059.43,0.0676538)
				(3059.43,0.0676538)
				(4218.88,0.0608145)
				(4218.88,0.0608145)
				(6137.8,0.0517463)
			};

			% W ev.std[0] %%%%%%%%%%%%%%%%%%%%%%%%%%%%%%%%%%%%%%%%%%%%%%%%%%%%%%%%%%%%
			\addplot[W] coordinates{
				(188.285,0.153729)
				(296.48,0.140106)
				(387.92,0.130909)
				(608.055,0.119289)
				(794.47,0.112549)
				(1100.87,0.103828)
				(1302.83,0.100809)
				(1302.83,0.100809)
				(1572.26,0.0960869)
				(1960.24,0.0920109)
				(1960.24,0.0920109)
				(2478.43,0.0872696)
				(3292.41,0.0816945)
				(3292.41,0.0816945)
				(3292.41,0.0816945)
				(4439.31,0.0766419)
				(4439.31,0.0766419)
				(6526.67,0.0706225)
			};

			% LSC ev.std[0] %%%%%%%%%%%%%%%%%%%%%%%%%%%%%%%%%%%%%%%%%%%%%%%%%%%%%%%%%%%%
			\addplot[LSC] coordinates{
				(548.755,0.091383)
				(756.92,0.0860093)
				(1045.31,0.0826595)
				(1360.95,0.0785458)
				(1665.85,0.0768306)
				(1919.65,0.0777422)
				(2055.98,0.0774372)
				(2239.5,0.0778717)
				(2376.71,0.0778621)
				(2441.63,0.0778629)
				(2603.55,0.0768896)
				(2909.2,0.0778625)
				(3147.7,0.0784786)
				(3373.12,0.0779616)
				(3762.88,0.0796871)
				(3762.88,0.0796871)
				(4057.29,0.0798873)
				(4401.67,0.0854813)
			};

			% WP ev.std[0] %%%%%%%%%%%%%%%%%%%%%%%%%%%%%%%%%%%%%%%%%%%%%%%%%%%%%%%%%%%%
			\addplot[WP] coordinates{
				(216,0.142423)
				(294,0.133292)
				(384,0.125233)
				(600,0.113178)
				(805,0.106083)
				(1080,0.0982438)
				(1320,0.0933891)
				(1536,0.0896315)
				(1536,0.0896814)
				(1944,0.0841237)
				(1944,0.0841283)
				(2400,0.0796561)
				(3174,0.073741)
				(3174,0.0739706)
				(4320,0.0679654)
				(4320,0.0679654)
				(4320,0.0679654)
				(5836.85,0.0600581)
			};

			% QS ev.std[0] %%%%%%%%%%%%%%%%%%%%%%%%%%%%%%%%%%%%%%%%%%%%%%%%%%%%%%%%%%%%
			\addplot[QS] coordinates{
				(324.725,0.164788)
				(468.655,0.161233)
				(615,0.151095)
				(884.79,0.125871)
				(984.505,0.116717)
				(1275.62,0.0994881)
				(1500.06,0.090452)
				(1646.89,0.0845781)
				(1824.2,0.07954)
				(2040.32,0.0743135)
				(2303.69,0.0684974)
				(2626.4,0.0626519)
				(3066.62,0.0565226)
				(3646.03,0.0504025)
				(4444.58,0.0434755)
				(5554.87,0.0367454)
			};

			% VLSLIC ev.std[0] %%%%%%%%%%%%%%%%%%%%%%%%%%%%%%%%%%%%%%%%%%%%%%%%%%%%%%%%%%%%
			\addplot[VLSLIC] coordinates{
				(575.975,0.090231)
				(651.86,0.0902393)
				(763.62,0.0910802)
				(899,0.0914541)
				(988.985,0.0896009)
				(1193.67,0.088699)
				(1348.08,0.0870292)
				(1349.01,0.0862736)
				(1579.08,0.0848226)
				(1849.65,0.0844158)
				(1857.27,0.0836754)
				(2307.73,0.0815141)
				(2890.56,0.0798459)
				(2924.52,0.0780876)
				(2954.18,0.0763434)
				(3858.98,0.0772936)
				(3858.98,0.0772936)
				(4809.57,0.0836441)
			};

			% CIS ev.std[0] %%%%%%%%%%%%%%%%%%%%%%%%%%%%%%%%%%%%%%%%%%%%%%%%%%%%%%%%%%%%
			\addplot[CIS] coordinates{
				(317.63,0.104738)
				(411.595,0.0969238)
				(472.685,0.0937395)
				(633.345,0.0879174)
				(777.05,0.0816193)
				(969.085,0.0772064)
				(1150.5,0.0707026)
				(1168.82,0.0700118)
				(1371.27,0.0677784)
				(1637.39,0.0630394)
				(1667.02,0.0624284)
				(2056.01,0.0599016)
				(3126.04,0.046431)
				(3625.6,0.042652)
				(4558.92,0.0398511)
				(4558.92,0.0398511)
				(6448.65,0.0310664)
			};

			% ERS ev.std[0] %%%%%%%%%%%%%%%%%%%%%%%%%%%%%%%%%%%%%%%%%%%%%%%%%%%%%%%%%%%%
			\addplot[ERS] coordinates{
				(200,0.141398)
				(300,0.132681)
				(400,0.127015)
				(600,0.118495)
				(800,0.112997)
				(1000,0.108326)
				(1200,0.1045)
				(1400,0.101874)
				(1600,0.0991824)
				(1800,0.0969109)
				(2000,0.094993)
				(2400,0.0916458)
				(2800,0.0890692)
				(3200,0.086286)
				(3600,0.0840993)
				(4000,0.0820287)
				(4600,0.0790186)
				(5200,0.076739)
			};

			% MSS ev.std[0] %%%%%%%%%%%%%%%%%%%%%%%%%%%%%%%%%%%%%%%%%%%%%%%%%%%%%%%%%%%%
			\addplot[MSS] coordinates{
				(200.765,0.1249)
				(282.33,0.117345)
				(420.66,0.106233)
				(664.115,0.0961521)
				(837.885,0.0936463)
				(1179.3,0.0877294)
				(1422.34,0.0849237)
				(1423.04,0.0850829)
				(1768,0.0824252)
				(2156.96,0.0806156)
				(2157.01,0.0805827)
				(2826.35,0.0775107)
				(3669.97,0.0753086)
				(3670.34,0.0753066)
				(3669.92,0.0754138)
				(5213.45,0.0716431)
				(5213.45,0.0716431)
				(7846.35,0.0674959)
			};

			% ETPS ev.std[0] %%%%%%%%%%%%%%%%%%%%%%%%%%%%%%%%%%%%%%%%%%%%%%%%%%%%%%%%%%%%
			\addplot[ETPS] coordinates{
				(216,0.0657559)
				(294,0.0619559)
				(425,0.0570213)
				(651,0.0527496)
				(805,0.0505153)
				(1107,0.0459568)
				(1320,0.0452155)
				(1320,0.0452155)
				(1617,0.0425744)
				(1944,0.0408759)
				(1944,0.0408759)
				(2501,0.0375485)
				(3174,0.0356554)
				(3174,0.0356554)
				(3174,0.0356554)
				(4374,0.0311574)
				(4374,0.0311574)
				(6305,0.0277505)
			};

			\end{axis}
	\end{tikzpicture}
\end{subfigure}%
\\[-8px]
   	\begin{subfigure}[t]{\fullthreeone\textwidth}\phantomsubcaption\label{subfig:experiments-quantitative-bsds500-sp.max[0]}
	%%%%%%%%%%%%%%%%%%%%%%%%%%%%%%%%%%%%%%%%%%%%%%%%%%%%%%%%%%%%
	% sp.max[0]
	%%%%%%%%%%%%%%%%%%%%%%%%%%%%%%%%%%%%%%%%%%%%%%%%%%%%%%%%%%%%
	\begin{tikzpicture}
		\begin{axis}[EQBSDS500KMax]

			% CCS sp.max[0] %%%%%%%%%%%%%%%%%%%%%%%%%%%%%%%%%%%%%%%%%%%%%%%%%%%%%%%%%%%%
			\addplot[CCS] coordinates{
				(218.865,465)
				(303.21,666)
				(453.82,976)
				(706.35,1381)
				(871.12,1665)
				(1204.99,2118)
				(1438.62,2455)
				(1438.62,2455)
				(1762.49,2970)
				(2107.24,3402)
				(2107.24,3402)
				(2702.61,4185)
				(3406.7,5011)
				(3406.7,5011)
				(3406.7,5011)
				(4654.44,6449)
				(4654.44,6449)
				(6619.1,8647)
			};

			% SEEDS sp.max[0] %%%%%%%%%%%%%%%%%%%%%%%%%%%%%%%%%%%%%%%%%%%%%%%%%%%%%%%%%%%%
			\addplot[SEEDS] coordinates{
				(261.62,339)
				(365.675,440)
				(468.81,546)
				(670.57,743)
				(870.75,940)
				(1087.4,1163)
				(1270.11,1345)
				(1451.85,1523)
				(1669.18,1747)
				(1873.19,1951)
				(2104.62,2171)
				(2462.77,2543)
				(2793.43,2870)
				(3260.86,3328)
				(3895.78,3961)
				(3895.78,3961)
				(4846.12,4907)
				(4846.12,4907)
			};

			% SLIC sp.max[0] %%%%%%%%%%%%%%%%%%%%%%%%%%%%%%%%%%%%%%%%%%%%%%%%%%%%%%%%%%%%
			\addplot[SLIC] coordinates{
				(180.32,224)
				(256.725,316)
				(368.57,430)
				(575.335,644)
				(726.36,829)
				(1002.87,1125)
				(1203.61,1348)
				(1203.61,1348)
				(1475.69,1644)
				(1814.8,2018)
				(1814.8,2018)
				(2334.56,2520)
				(3038.13,3227)
				(3038.13,3227)
				(3038.13,3227)
				(4188.97,4443)
				(4188.97,4443)
				(6139.41,6464)
			};

			% RW sp.max[0] %%%%%%%%%%%%%%%%%%%%%%%%%%%%%%%%%%%%%%%%%%%%%%%%%%%%%%%%%%%%
			\addplot[RW] coordinates{
				(211.565,233)
				(313.64,330)
				(407.125,429)
				(649.355,671)
				(858.65,887)
				(1032.29,1054)
				(1280.94,1376)
				(1505.72,1557)
				(1649.74,1681)
				(1940.98,2049)
				(2101.2,2264)
				(2570.18,2628)
				(2916.36,2954)
				(3413.13,3542)
				(3882.85,3925)
				(4457.4,4505)
				(5093.4,5618)
				(5663.8,6438)
			};

			% CW sp.max[0] %%%%%%%%%%%%%%%%%%%%%%%%%%%%%%%%%%%%%%%%%%%%%%%%%%%%%%%%%%%%
			\addplot[CW] coordinates{
				(197.775,227)
				(309.29,363)
				(405.935,463)
				(609.18,708)
				(807.005,967)
				(1045.56,1221)
				(1252.72,1478)
				(1492.83,1793)
				(1649.62,1947)
				(1816.43,2147)
				(2091.79,2463)
				(2565.8,3135)
				(2917.57,3463)
				(3319.52,3955)
				(4023.98,4801)
				(4023.98,4801)
				(4639.83,5509)
				(5846.73,6980)
			};

			% TP sp.max[0] %%%%%%%%%%%%%%%%%%%%%%%%%%%%%%%%%%%%%%%%%%%%%%%%%%%%%%%%%%%%
			\addplot[TP] coordinates{
				(283.695,353)
				(381.285,454)
				(555.1,697)
				(752.29,901)
				(1007.04,1185)
				(1292.4,1510)
				(1495.14,1703)
				(1696.62,1986)
				(2017.73,2311)
				(2478.02,2674)
				(2831,3076)
				(3018.75,3774)
			};

			% POISE sp.max[0] %%%%%%%%%%%%%%%%%%%%%%%%%%%%%%%%%%%%%%%%%%%%%%%%%%%%%%%%%%%%
			\addplot[POISE] coordinates{
				(204.855,205)
				(306.68,307)
				(408.385,409)
				(611.79,613)
				(814.975,817)
				(1017.48,1021)
				(1216.26,1225)
				(1409.36,1429)
				(1587.37,1633)
				(1749.07,1837)
				(1887.06,2041)
				(2090.32,2449)
				(2221.84,2852)
				(2278.76,3248)
				(2287.5,3641)
				(2288.94,3883)
				(2288.94,3883)
				(2288.94,3883)
			};

			% FH sp.max[0] %%%%%%%%%%%%%%%%%%%%%%%%%%%%%%%%%%%%%%%%%%%%%%%%%%%%%%%%%%%%
			\addplot[FH] coordinates{
				(628.745,1997)
				(963.36,2886)
				(782.42,2052)
				(799.09,2197)
				(1090.39,2802)
				(1187.04,3251)
				(1605.71,3741)
				(2533.01,6608)
				(3000.74,6632)
				(3219.63,6840)
				(3814.42,9541)
				(4746.96,11435)
			};

			% EAMS sp.max[0] %%%%%%%%%%%%%%%%%%%%%%%%%%%%%%%%%%%%%%%%%%%%%%%%%%%%%%%%%%%%
			\addplot[EAMS] coordinates{
				(261.12,510)
				(283.33,552)
				(309.87,595)
				(383.52,698)
				(499.435,844)
				(725.36,1091)
				(1417.35,1963)
				(1473.6,2026)
				(1534.33,2080)
				(1883.59,2901)
				(2063.65,3111)
				(2346.71,3465)
				(2642.78,3783)
				(3113.25,4598)
				(3640.44,5310)
				(4914.12,7106)
				(8189.58,12091)
			};

			% CRS sp.max[0] %%%%%%%%%%%%%%%%%%%%%%%%%%%%%%%%%%%%%%%%%%%%%%%%%%%%%%%%%%%%
			\addplot[CRS] coordinates{
				(299.575,500)
				(449.585,715)
				(635.865,936)
				(895.935,1279)
				(1180.31,1738)
				(1530.1,2073)
				(1758.22,2445)
				(2057.15,2852)
				(2308.66,3227)
				(2461.75,3436)
				(2755.4,3709)
				(3401.31,4612)
				(3879.51,4964)
				(4346.83,5531)
				(5208.23,6573)
				(5208.23,6573)
				(5821.7,7094)
				(7241.41,8855)
			};

			% SEAW sp.max[0] %%%%%%%%%%%%%%%%%%%%%%%%%%%%%%%%%%%%%%%%%%%%%%%%%%%%%%%%%%%%
			\addplot[SEAW] coordinates{
				(165.505,491)
				(356.805,831)
				(923.49,1625)
				(2933.9,3965)
				(10727.5,12095)
			};

			% RESEEDS sp.max[0] %%%%%%%%%%%%%%%%%%%%%%%%%%%%%%%%%%%%%%%%%%%%%%%%%%%%%%%%%%%%
			\addplot[RESEEDS] coordinates{
				(200.795,204)
				(301.525,306)
				(401.465,407)
				(602.33,609)
				(800.99,806)
				(1020.12,1022)
				(1201.34,1205)
				(1378.1,1379)
				(1601.55,1607)
				(1802.11,1804)
				(2040.11,2042)
				(2402.13,2408)
				(2720.13,2723)
				(3200.2,3202)
				(3840.22,3843)
				(3840.22,3843)
				(4800.37,4804)
				(4800.37,4804)
			};

			% ERGC sp.max[0] %%%%%%%%%%%%%%%%%%%%%%%%%%%%%%%%%%%%%%%%%%%%%%%%%%%%%%%%%%%%
			\addplot[ERGC] coordinates{
				(196,196)
				(306,306)
				(400,400)
				(600,600)
				(792.57,812)
				(1024,1024)
				(1224,1224)
				(1453.2,1480)
				(1600,1600)
				(1760,1760)
				(2024,2024)
				(2472.98,2544)
				(2809,2809)
				(3180,3180)
				(3840,3840)
				(3840,3840)
				(4416,4416)
				(5520,5520)
			};

			% PF sp.max[0] %%%%%%%%%%%%%%%%%%%%%%%%%%%%%%%%%%%%%%%%%%%%%%%%%%%%%%%%%%%%
			\addplot[PF] coordinates{
				(281.06,543)
				(428.82,720)
				(585.08,1025)
				(928.805,1427)
				(1172.76,1818)
				(1602.65,2736)
				(1861.72,2808)
				(2267.11,3890)
				(2697.08,4024)
				(3382.15,5389)
				(4207.66,6258)
				(6051.74,8243)
			};

			% TPS sp.max[0] %%%%%%%%%%%%%%%%%%%%%%%%%%%%%%%%%%%%%%%%%%%%%%%%%%%%%%%%%%%%
			\addplot[TPS] coordinates{
				(224.26,272)
				(903.595,1003)
				(1130.53,1249)
				(1332.49,1507)
				(1556.56,1657)
				(1736.84,1885)
				(1988.95,2217)
				(2167.04,2516)
				(2656.03,2818)
				(2987.25,3148)
				(3505.61,3749)
				(4030.29,4171)
				(4568.25,4736)
				(5242.04,5949)
				(5978.67,6776)
			};

			% NC sp.max[0] %%%%%%%%%%%%%%%%%%%%%%%%%%%%%%%%%%%%%%%%%%%%%%%%%%%%%%%%%%%%
			\addplot[NC] coordinates{
				(223.505,247)
				(321.69,346)
				(416.03,442)
				(594.185,620)
				(763.865,802)
				(922.645,977)
				(1073.6,1155)
				(1213.8,1323)
				(1348.36,1495)
				(1473.43,1684)
				(1591.75,1851)
				(2000.27,2436)
			};

			% VC sp.max[0] %%%%%%%%%%%%%%%%%%%%%%%%%%%%%%%%%%%%%%%%%%%%%%%%%%%%%%%%%%%%
			\addplot[VC] coordinates{
				(351.72,4897)
				(498.095,5974)
				(712.82,5864)
				(994.125,7034)
				(1201.27,7083)
				(1375.27,7651)
				(1685.08,7815)
				(1968.97,8318)
				(2224.75,8315)
				(2476.43,8384)
				(2712.29,8674)
				(2948.04,8974)
				(3169.69,9374)
				(3389.95,9572)
				(3804.08,10346)
				(4196.75,10838)
				(4566.66,11647)
				(4815.37,12103)
				(5157.48,12944)
				(5401.49,13566)
			};

			% PB sp.max[0] %%%%%%%%%%%%%%%%%%%%%%%%%%%%%%%%%%%%%%%%%%%%%%%%%%%%%%%%%%%%
			\addplot[PB] coordinates{
				(324.39,765)
				(398.425,846)
				(494.17,964)
				(692.845,1144)
				(853.905,1302)
				(1129.27,1550)
				(1323.88,1761)
				(1323.88,1761)
				(1601.37,2042)
				(1929.58,2318)
				(1929.58,2318)
				(2453.31,2772)
				(3126.91,3444)
				(3126.91,3444)
				(3126.91,3444)
				(4270.56,4527)
				(4270.56,4527)
				(6122.31,6353)
			};

			% PRESLIC sp.max[0] %%%%%%%%%%%%%%%%%%%%%%%%%%%%%%%%%%%%%%%%%%%%%%%%%%%%%%%%%%%%
			\addplot[PRESLIC] coordinates{
				(369,413)
				(581.54,635)
				(734.575,794)
				(1020.3,1100)
				(1229.22,1322)
				(1229.22,1322)
				(1511.3,1629)
				(1838.94,1979)
				(1838.94,1979)
				(2375.99,2557)
				(3059.43,3248)
				(3059.43,3248)
				(3059.43,3248)
				(4218.88,4361)
				(4218.88,4361)
				(6137.8,6339)
			};

			% W sp.max[0] %%%%%%%%%%%%%%%%%%%%%%%%%%%%%%%%%%%%%%%%%%%%%%%%%%%%%%%%%%%%
			\addplot[W] coordinates{
				(188.285,205)
				(296.48,335)
				(387.92,440)
				(608.055,702)
				(794.47,930)
				(1100.87,1288)
				(1302.83,1529)
				(1302.83,1529)
				(1572.26,1865)
				(1960.24,2305)
				(1960.24,2305)
				(2478.43,2929)
				(3292.41,3898)
				(3292.41,3898)
				(3292.41,3898)
				(4439.31,5265)
				(4439.31,5265)
				(6526.67,7749)
			};

			% LSC sp.max[0] %%%%%%%%%%%%%%%%%%%%%%%%%%%%%%%%%%%%%%%%%%%%%%%%%%%%%%%%%%%%
			\addplot[LSC] coordinates{
				(548.755,1352)
				(756.92,1583)
				(1045.31,2056)
				(1360.95,2378)
				(1665.85,2774)
				(1919.65,3041)
				(2055.98,3118)
				(2239.5,3360)
				(2376.71,3423)
				(2441.63,3666)
				(2603.55,3673)
				(2909.2,3994)
				(3147.7,4140)
				(3373.12,4315)
				(3762.88,4728)
				(3762.88,4728)
				(4057.29,4843)
				(4401.67,5070)
			};

			% WP sp.max[0] %%%%%%%%%%%%%%%%%%%%%%%%%%%%%%%%%%%%%%%%%%%%%%%%%%%%%%%%%%%%
			\addplot[WP] coordinates{
				(216,216)
				(294,294)
				(384,384)
				(600,600)
				(805,805)
				(1080,1080)
				(1320,1320)
				(1536,1536)
				(1536,1536)
				(1944,1944)
				(1944,1944)
				(2400,2400)
				(3174,3174)
				(3174,3174)
				(4320,4320)
				(4320,4320)
				(4320,4320)
				(5836.85,6247)
			};

			% QS sp.max[0] %%%%%%%%%%%%%%%%%%%%%%%%%%%%%%%%%%%%%%%%%%%%%%%%%%%%%%%%%%%%
			\addplot[QS] coordinates{
				(324.725,1591)
				(468.655,2339)
				(615,3096)
				(884.79,4180)
				(984.505,4338)
				(1275.62,5486)
				(1500.06,6175)
				(1646.89,6724)
				(1824.2,7234)
				(2040.32,7867)
				(2303.69,8703)
				(2626.4,9623)
				(3066.62,11008)
				(3646.03,13184)
				(4444.58,15817)
				(5554.87,19026)
			};

			% VLSLIC sp.max[0] %%%%%%%%%%%%%%%%%%%%%%%%%%%%%%%%%%%%%%%%%%%%%%%%%%%%%%%%%%%%
			\addplot[VLSLIC] coordinates{
				(575.975,950)
				(651.86,1003)
				(763.62,1115)
				(899,1254)
				(988.985,1285)
				(1193.67,1476)
				(1348.08,1562)
				(1349.01,1553)
				(1579.08,1733)
				(1849.65,1986)
				(1857.27,1980)
				(2307.73,2493)
				(2890.56,3158)
				(2924.52,3165)
				(2954.18,3166)
				(3858.98,4308)
				(3858.98,4308)
				(4809.57,5738)
			};

			% CIS sp.max[0] %%%%%%%%%%%%%%%%%%%%%%%%%%%%%%%%%%%%%%%%%%%%%%%%%%%%%%%%%%%%
			\addplot[CIS] coordinates{
				(317.63,749)
				(411.595,958)
				(472.685,984)
				(633.345,1251)
				(777.05,1455)
				(969.085,1718)
				(1150.5,2163)
				(1168.82,2245)
				(1371.27,2472)
				(1637.39,3114)
				(1667.02,3234)
				(2056.01,3588)
				(3126.04,6853)
				(3625.6,8506)
				(4558.92,9110)
				(4558.92,9110)
				(6448.65,12621)
			};

			% ERS sp.max[0] %%%%%%%%%%%%%%%%%%%%%%%%%%%%%%%%%%%%%%%%%%%%%%%%%%%%%%%%%%%%
			\addplot[ERS] coordinates{
				(200,200)
				(300,300)
				(400,400)
				(600,600)
				(800,800)
				(1000,1000)
				(1200,1200)
				(1400,1400)
				(1600,1600)
				(1800,1800)
				(2000,2000)
				(2400,2400)
				(2800,2800)
				(3200,3200)
				(3600,3600)
				(4000,4000)
				(4600,4600)
				(5200,5200)
			};

			% MSS sp.max[0] %%%%%%%%%%%%%%%%%%%%%%%%%%%%%%%%%%%%%%%%%%%%%%%%%%%%%%%%%%%%
			\addplot[MSS] coordinates{
				(200.765,274)
				(282.33,383)
				(420.66,577)
				(664.115,901)
				(837.885,1142)
				(1179.3,1624)
				(1422.34,1941)
				(1423.04,1955)
				(1768,2420)
				(2156.96,2944)
				(2157.01,2958)
				(2826.35,3806)
				(3669.97,4965)
				(3670.34,4918)
				(3669.92,4931)
				(5213.45,6909)
				(5213.45,6909)
				(7846.35,10196)
			};

			% ETPS sp.max[0] %%%%%%%%%%%%%%%%%%%%%%%%%%%%%%%%%%%%%%%%%%%%%%%%%%%%%%%%%%%%
			\addplot[ETPS] coordinates{
				(216,216)
				(294,294)
				(425,425)
				(651,651)
				(805,805)
				(1107,1107)
				(1320,1320)
				(1320,1320)
				(1617,1617)
				(1944,1944)
				(1944,1944)
				(2501,2501)
				(3174,3174)
				(3174,3174)
				(3174,3174)
				(4374,4374)
				(4374,4374)
				(6305,6305)
			};

		\end{axis}
	\end{tikzpicture}
\end{subfigure}
\begin{subfigure}[t]{\fullthreetwo\textwidth}\phantomsubcaption\label{subfig:experiments-quantitative-bsds500-sp.std[0]}
	\vspace{5px}
	\begin{tikzpicture}
		\begin{axis}[EQBSDS500K,%
				symbolic x coords = {
					W,
					EAMS,
					NC,
					FH,
					RW,
					QS,
					PF,
					TP,
					CIS,
					SLIC,
					%vlSLIC,
					%SLIC3D,
					CRS,
					ERS,
					PB,
					%DASP,
					SEEDS,
					%reSEEDS,
					%reSEEDS3D,
					TPS,
					VC,
					CCS,
					%VCCS,
					CW,
					ERGC,
					MSS,
					preSLIC,
					WP,
					%LRW,
					ETPS,
					LSC,
					POISE,
					SEAW,
					%reFH,
				},
			]

			%%%%%%%%%%%%%%%%%%%%%%%%%%%%%%%%%%%%%%%%%%%%%%%%%%%%%%%%%%%%
			% sp.std[0]
			%%%%%%%%%%%%%%%%%%%%%%%%%%%%%%%%%%%%%%%%%%%%%%%%%%%%%%%%%%%%
			\addplot[fill=blue] coordinates {
				(CCS,38.0481)
				(SEEDS,23.5218)
				(SLIC,12.4562)
				(RW,5.58982)
				(CW,3.10745)
				(TP,21.4895)
				(POISE,0.419263)
				(FH,177.584)
				(EAMS,53.5679)
				(CRS,42.9576)
				(SEAW,81.1302)
				%(reSEEDS,0.956066)
				(ERGC,0)
				(PF,68.3049)
				(TPS,16.8364)
				(NC,8.02486)
				(VC,561.252)
				(PB,91.8723)
				(preSLIC,22.6492)
				(W,2.19196)
				(LSC,179.603)
				(WP,0)
				(QS,243.366)
				%(vlSLIC,114.057)
				(CIS,106.887)
				(ERS,0)
				(MSS,19.1875)
				(ETPS,0)
			};

		\end{axis}
	\end{tikzpicture}
	\vspace{-18px}
	\begin{center}{\scriptsize(k)}\end{center}
\end{subfigure}
\\[-4px]
    \caption{Quantitative experiments on the \BSDS dataset; remember that \K denotes the number of generated superpixels.
    \Rec (higher is better) and \UE (lower is better) give a concise overview of the performance with respect to ground
    truth. In contrast, \EV (higher is better) gives a ground truth independent view on performance.
    While top-performers as well as poorly performing algorithms are easily
    identified, we provide more find-grained experimental results by considering
    $\min\Rec$, $\max\UE$ and $\min\EV$. These statistics additionally can be used
    to quantity the stability of superpixel algorithms. In particular, stable
	algorithms are expected to exhibit monotonically improving $\min\Rec$, $\max\UE$ and $\min\EV$.
	The corresponding $\text{std }\Rec$, $\text{std }\UE$ and $\text{std }\EV$ as
	well as $\max\K$ and $\text{std }\K$ help to identify stable algorithms.
	\textbf{Best viewed in color.}}
    \label{fig:experiments-quantitative-bsds500}
	\vskip 12px
	% Total: 26 (+2 depth)
%\begin{mdframed}
	{\scriptsize
		\begin{tabularx}{\textwidth}{X X X X X X X X l}
			\ref{plot:w} \W &
			\ref{plot:eams} \EAMS &
			\ref{plot:nc} \NC &
			\ref{plot:fh} \FH &
			\ref{plot:rw} \RW &
			\ref{plot:qs} \QS &
			\ref{plot:pf} \PF &
			\ref{plot:tp} \TP &
			\ref{plot:cis} \CIS \\
			\ref{plot:slic} \SLIC &
			\ref{plot:crs} \CRS &
			\ref{plot:ers} \ERS &
			\ref{plot:pb} \PB &
			\ref{plot:seeds} \SEEDS &
			\ref{plot:tps} \TPS &
			\ref{plot:vc} \VC &
			\ref{plot:ccs} \CCS &
			\ref{plot:cw} \CW \\
			\ref{plot:ergc} \ERGC &
			\ref{plot:mss} \MSS &
			\ref{plot:preslic} \preSLIC &
			\ref{plot:wp} \WP &
			\ref{plot:etps} \ETPS &
			\ref{plot:lsc} \LSC &
			\ref{plot:poise} \POISE &
			\ref{plot:seaw} \SEAW &
		\end{tabularx}
	}
%\end{mdframed}

\end{figure*}
\begin{figure*}
	\centering
	\vspace{-12px}
	\begin{subfigure}[b]{0.325\textwidth}\phantomsubcaption\label{subfig:experiments-quantitative-nyuv2-rec.mean[0]}
	%%%%%%%%%%%%%%%%%%%%%%%%%%%%%%%%%%%%%%%%%%%%%%%%%%%%%%%%%%%%
	% rec.mean[0]
	%%%%%%%%%%%%%%%%%%%%%%%%%%%%%%%%%%%%%%%%%%%%%%%%%%%%%%%%%%%%
	\begin{tikzpicture}
		\begin{axis}[EQNYUV2Rec,xmode=log]		
	
			% CCS rec.mean[0] %%%%%%%%%%%%%%%%%%%%%%%%%%%%%%%%%%%%%%%%%%%%%%%%%%%%%%%%%%%%
			\addplot[CCS] coordinates{
				(194.01,0.581264)
				(282.682,0.646661)
				(397.098,0.703874)
				(597.153,0.767783)
				(803.471,0.810919)
				(925.103,0.831917)
				(1177.26,0.865176)
				(1397.83,0.885457)
				(1588.43,0.901197)
				(1878.46,0.920473)
				(1878.46,0.920473)
				(2231.81,0.938706)
				(2677.06,0.955592)
				(3328.01,0.97313)
				(3328.01,0.97313)
				(4313.98,0.987289)
				(4313.98,0.987289)
				(5573.99,0.995412)
			};
	
			% SEEDS rec.mean[0] %%%%%%%%%%%%%%%%%%%%%%%%%%%%%%%%%%%%%%%%%%%%%%%%%%%%%%%%%%%%
			\addplot[SEEDS] coordinates{
				(246.043,0.859625)
				(347.734,0.897607)
				(450.607,0.919503)
				(654.489,0.942479)
				(857.654,0.956077)
				(1057.59,0.96742)
				(1258.91,0.974696)
				(1462.49,0.975208)
				(1661.42,0.981597)
				(2066.14,0.987725)
				(2473.07,0.989817)
				(2864.58,0.990187)
				(3245.7,0.99384)
				(3711.17,0.996269)
				(4314.51,0.99532)
				(4499.92,0.996514)
				(4922.37,0.997944)
			};
	
			% SLIC rec.mean[0] %%%%%%%%%%%%%%%%%%%%%%%%%%%%%%%%%%%%%%%%%%%%%%%%%%%%%%%%%%%%
			\addplot[SLIC] coordinates{
				(184.484,0.71949)
				(279.316,0.770781)
				(385.211,0.810203)
				(596.942,0.857103)
				(819.424,0.888232)
				(939.241,0.902059)
				(1211.08,0.922846)
				(1425.96,0.935971)
				(1642.44,0.9451)
				(1940.44,0.956653)
				(1940.44,0.956653)
				(2316.71,0.96627)
				(2774.73,0.975028)
				(3441.91,0.983614)
				(3441.91,0.983614)
				(4395.74,0.991109)
				(4395.74,0.991109)
				(5687.73,0.995625)
			};
	
			% RW rec.mean[0] %%%%%%%%%%%%%%%%%%%%%%%%%%%%%%%%%%%%%%%%%%%%%%%%%%%%%%%%%%%%
			\addplot[RW] coordinates{
				(215.912,0.609953)
				(306.055,0.660616)
				(440.393,0.711692)
				(635.679,0.761482)
				(846.511,0.801678)
				(1060.11,0.827532)
				(1306.99,0.855574)
				(1464.51,0.86959)
				(1685.01,0.885379)
				(1868.17,0.897956)
				(2109.97,0.911217)
				(3967.51,0.968421)
				(4401.89,0.974547)
				(5110.46,0.983485)
			};
	
			% CW rec.mean[0] %%%%%%%%%%%%%%%%%%%%%%%%%%%%%%%%%%%%%%%%%%%%%%%%%%%%%%%%%%%%
			\addplot[CW] coordinates{
				(198.614,0.661197)
				(293.173,0.712431)
				(407.073,0.754982)
				(613.509,0.80392)
				(834.84,0.840698)
				(1056.32,0.869139)
				(1266,0.888169)
				(1460.01,0.903051)
				(1700.24,0.918796)
				(1828.56,0.92615)
				(2009.84,0.935008)
				(2449.61,0.952593)
				(2991.17,0.966938)
				(3233.9,0.971639)
				(3707.85,0.980006)
				(4124.98,0.984679)
				(5174.47,0.991962)
				(5281.06,0.992156)
			};
	
			% TP rec.mean[0] %%%%%%%%%%%%%%%%%%%%%%%%%%%%%%%%%%%%%%%%%%%%%%%%%%%%%%%%%%%%
			\addplot[TP] coordinates{
				(295.99,0.667566)
				(395.489,0.700297)
				(551.496,0.759489)
				(775.682,0.797945)
				(1000.76,0.827955)
				(1130.49,0.842449)
				(1401.47,0.868435)
				(1572.01,0.880839)
				(1815.33,0.896479)
				(2110.15,0.913661)
			};
	
			% POISE rec.mean[0] %%%%%%%%%%%%%%%%%%%%%%%%%%%%%%%%%%%%%%%%%%%%%%%%%%%%%%%%%%%%
			\addplot[POISE] coordinates{
				(204.942,0.771876)
				(306.83,0.808035)
				(408.614,0.83298)
				(612.1,0.865512)
				(815.501,0.888081)
				(1019.01,0.904427)
				(1222.5,0.916425)
				(1425.91,0.925461)
				(1628.65,0.932735)
				(1830.61,0.937971)
				(2031.68,0.942223)
				(2427.44,0.948137)
				(2807.4,0.951991)
				(3158.64,0.954436)
				(3466.96,0.955909)
				(3736.5,0.956846)
				(4042.51,0.957624)
				(4219.03,0.957918)
			};
	
			% FH rec.mean[0] %%%%%%%%%%%%%%%%%%%%%%%%%%%%%%%%%%%%%%%%%%%%%%%%%%%%%%%%%%%%
			\addplot[FH] coordinates{
				(689.802,0.80558)
				(763.792,0.823931)
				(813.815,0.833886)
				(988.712,0.861569)
				(1077.66,0.870823)
				(1359.76,0.896399)
				(1408.18,0.870769)
				(1559.12,0.883173)
				(1923.5,0.929427)
				(2206.32,0.920596)
				(2448.19,0.933465)
				(2699.96,0.952097)
				(3024.27,0.947871)
				(3568.29,0.951718)
				(4199.72,0.968104)
				(4594.6,0.979201)
			};
	
			% EAMS rec.mean[0] %%%%%%%%%%%%%%%%%%%%%%%%%%%%%%%%%%%%%%%%%%%%%%%%%%%%%%%%%%%%
			\addplot[EAMS] coordinates{
				(419.068,0.863241)
				(455.657,0.871071)
				(499.416,0.879448)
				(641.188,0.889652)
				(842.328,0.911347)
				(1268.71,0.933633)
				(2584.6,0.963033)
				(2683.21,0.964335)
				(2779.09,0.960963)
				(2997.7,0.963007)
				(3180.29,0.960032)
				(3421.45,0.95791)
				(3646.16,0.955574)
				(3943.04,0.953608)
				(4448.4,0.955153)
				(5663.91,0.957472)
				(9098.39,0.960598)
			};
	
			% CRS rec.mean[0] %%%%%%%%%%%%%%%%%%%%%%%%%%%%%%%%%%%%%%%%%%%%%%%%%%%%%%%%%%%%
			\addplot[CRS] coordinates{
				(254.263,0.745212)
				(396.694,0.811338)
				(514.895,0.848042)
				(764.997,0.892482)
				(988.589,0.916642)
				(1232.03,0.936506)
				(1498.25,0.950993)
				(1707.65,0.958301)
				(1968.19,0.967739)
				(2099.81,0.970839)
				(2299.23,0.975038)
				(2706.1,0.982142)
				(3259.07,0.988068)
				(3558.75,0.990332)
				(4055.02,0.99349)
				(4394.15,0.9949)
				(5418.11,0.997378)
				(5695.52,0.997989)
			};
	
			% SEAW rec.mean[0] %%%%%%%%%%%%%%%%%%%%%%%%%%%%%%%%%%%%%%%%%%%%%%%%%%%%%%%%%%%%
			\addplot[SEAW] coordinates{
				(124.667,0.451606)
				(349.414,0.607782)
				(1192.36,0.793405)
				(4497.65,0.972556)
			};
	
			% RESEEDS rec.mean[0] %%%%%%%%%%%%%%%%%%%%%%%%%%%%%%%%%%%%%%%%%%%%%%%%%%%%%%%%%%%%
			\addplot[RESEEDS] coordinates{
				(200.441,0.878008)
				(300.263,0.904614)
				(400.323,0.922132)
				(600.361,0.944076)
				(800.526,0.957377)
				(1000.06,0.965854)
				(1200.06,0.971607)
				(1400.96,0.976291)
				(1600.21,0.979592)
				(1792.08,0.982461)
				(1998.13,0.985222)
				(2408.17,0.988617)
				(2801.44,0.991392)
				(3182.16,0.993061)
				(3648.12,0.994942)
				(4257.44,0.996129)
				(4444.16,0.996428)
				(4864.15,0.99737)
			};
	
			% ERGC rec.mean[0] %%%%%%%%%%%%%%%%%%%%%%%%%%%%%%%%%%%%%%%%%%%%%%%%%%%%%%%%%%%%
			\addplot[ERGC] coordinates{
				(196,0.678596)
				(289,0.737797)
				(400,0.78247)
				(600,0.83455)
				(812,0.870965)
				(1024,0.896188)
				(1224,0.914002)
				(1406,0.926804)
				(1681,0.942539)
				(1763,0.945164)
				(1935,0.953166)
				(2350,0.967323)
				(2856,0.977957)
				(3080,0.98066)
				(3520,0.986592)
				(3904,0.990086)
				(4864,0.994933)
				(5100,0.996313)
			};
	
			% PF rec.mean[0] %%%%%%%%%%%%%%%%%%%%%%%%%%%%%%%%%%%%%%%%%%%%%%%%%%%%%%%%%%%%
			\addplot[PF] coordinates{
				(281.19,0.426466)
				(430.376,0.482665)
				(586.371,0.524826)
				(907.19,0.58579)
				(1201.57,0.631964)
				(1354.07,0.651023)
				(1718.72,0.686135)
				(1939.43,0.706879)
				(2251.38,0.727853)
				(2592.38,0.750845)
				(3186.03,0.798589)
				(4180.57,0.838)
				(6182.51,0.893933)
			};
	
			% TPS rec.mean[0] %%%%%%%%%%%%%%%%%%%%%%%%%%%%%%%%%%%%%%%%%%%%%%%%%%%%%%%%%%%%
			\addplot[TPS] coordinates{
				(230.419,0.560123)
				(313.739,0.604396)
				(450.436,0.655309)
				(638.008,0.706407)
				(857.559,0.75027)
				(1043.09,0.779122)
				(1317.64,0.812505)
				(1503.8,0.830713)
				(1703.97,0.85032)
				(1873.98,0.861634)
				(2144.93,0.879491)
				(2580.57,0.904672)
				(2894.48,0.915267)
				(3314.47,0.927702)
				(3924.13,0.954876)
				(4380.02,0.965444)
				(5132.34,0.974631)
				(5738.91,0.981302)
			};
	
			% NC rec.mean[0] %%%%%%%%%%%%%%%%%%%%%%%%%%%%%%%%%%%%%%%%%%%%%%%%%%%%%%%%%%%%
			\addplot[NC] coordinates{
				(439.952,0.738955)
				(1154.47,0.858669)
				(2631.49,0.922245)
				(3874.82,0.972529)
			};
	
			% VC rec.mean[0] %%%%%%%%%%%%%%%%%%%%%%%%%%%%%%%%%%%%%%%%%%%%%%%%%%%%%%%%%%%%
			\addplot[VC] coordinates{
				(243.744,0.670814)
				(400.291,0.746209)
				(531.672,0.782367)
				(660.84,0.808127)
				(895.915,0.840624)
				(1125.96,0.862498)
				(1348.09,0.878558)
				(1564.34,0.891458)
				(1781.1,0.90271)
				(1995.21,0.911891)
				(2205.86,0.919603)
				(2416.36,0.926906)
				(2820.76,0.93869)
				(3224.25,0.948162)
				(3610.4,0.955564)
				(3988.04,0.961363)
				(4351.59,0.966494)
				(4847.23,0.971486)
				(5262.69,0.975304)
			};
	
			% PB rec.mean[0] %%%%%%%%%%%%%%%%%%%%%%%%%%%%%%%%%%%%%%%%%%%%%%%%%%%%%%%%%%%%
			\addplot[PB] coordinates{
				(273.248,0.610781)
				(360.188,0.658459)
				(463.193,0.698741)
				(656.962,0.753802)
				(857.526,0.795019)
				(970.915,0.814327)
				(1217.27,0.845613)
				(1387.49,0.864576)
				(1616.5,0.883185)
				(1896.95,0.900814)
				(1896.95,0.900814)
				(2243.04,0.92244)
				(2681.85,0.940357)
				(3316.29,0.960856)
				(3316.29,0.960856)
				(4151.92,0.9768)
				(4151.92,0.9768)
				(5425.24,0.989619)
			};
	
			% VCCS rec.mean[0] %%%%%%%%%%%%%%%%%%%%%%%%%%%%%%%%%%%%%%%%%%%%%%%%%%%%%%%%%%%%
			\addplot[VCCS] coordinates{
				(564.123,0.745359)
				(600.283,0.756124)
				(648.629,0.769111)
				(755.632,0.791207)
				(836.05,0.805641)
				(1051.64,0.822051)
				(1109.07,0.828853)
				(1179.74,0.836199)
				(1237.23,0.84116)
				(1322.17,0.850298)
				(1400.72,0.854865)
				(1526.22,0.865262)
				(1630.95,0.872456)
				(1756.59,0.880461)
				(1914.09,0.888678)
				(2024.07,0.892443)
				(2258.58,0.903227)
				(2558.8,0.916269)
				(2780.39,0.921516)
				(3044.22,0.92786)
				(3687.7,0.948228)
				(4171.76,0.956663)
				(4255.91,0.950409)
			};
	
			% PRESLIC rec.mean[0] %%%%%%%%%%%%%%%%%%%%%%%%%%%%%%%%%%%%%%%%%%%%%%%%%%%%%%%%%%%%
			\addplot[PRESLIC] coordinates{
				(189.89,0.713975)
				(388.539,0.805891)
				(589.772,0.850454)
				(798.767,0.88363)
				(920.201,0.899726)
				(1188.65,0.921606)
				(1401.21,0.937812)
				(1612.25,0.946871)
				(1904.93,0.95792)
				(1904.93,0.95792)
				(2282.12,0.968689)
				(2738.74,0.976302)
				(3422.72,0.985804)
				(3422.72,0.985804)
				(4393.15,0.993343)
				(4393.15,0.993343)
				(5724.22,0.996835)
			};
	
			% W rec.mean[0] %%%%%%%%%%%%%%%%%%%%%%%%%%%%%%%%%%%%%%%%%%%%%%%%%%%%%%%%%%%%
			\addplot[W] coordinates{
				(193.87,0.655199)
				(303.997,0.712842)
				(397.103,0.748076)
				(621.108,0.805128)
				(870.236,0.845307)
				(959.915,0.8563)
				(1234.38,0.88514)
				(1418.74,0.901457)
				(1652.83,0.917047)
				(1957.01,0.933264)
				(1957.01,0.933264)
				(2345.3,0.949297)
				(2867.42,0.964637)
				(3518.42,0.976316)
				(3518.42,0.976316)
				(4497.46,0.987714)
				(4497.46,0.987714)
				(5940.33,0.995325)
			};
	
			% LSC rec.mean[0] %%%%%%%%%%%%%%%%%%%%%%%%%%%%%%%%%%%%%%%%%%%%%%%%%%%%%%%%%%%%
			\addplot[LSC] coordinates{
				(383.095,0.815293)
				(552.409,0.849918)
				(718.163,0.874173)
				(1059.33,0.906777)
				(1414.95,0.930453)
				(1816.61,0.948134)
				(2009.32,0.953645)
				(2263.42,0.961211)
				(2398.35,0.965381)
				(2446.8,0.966665)
				(2588.34,0.969625)
				(2830.36,0.974024)
				(3264.3,0.980519)
				(3438.52,0.981946)
				(3802.19,0.986264)
				(4059.6,0.988512)
				(4900.83,0.993639)
				(5001.09,0.994539)
			};
	
			% WP rec.mean[0] %%%%%%%%%%%%%%%%%%%%%%%%%%%%%%%%%%%%%%%%%%%%%%%%%%%%%%%%%%%%
			\addplot[WP] coordinates{
				(204,0.682968)
				(315,0.732281)
				(432,0.765368)
				(638,0.810027)
				(850,0.841184)
				(1064,0.866288)
				(1230,0.881086)
				(1408,0.89487)
				(1645,0.908727)
				(1938,0.922919)
				(1938,0.922969)
				(2296,0.938633)
				(2745,0.953131)
				(3400,0.968103)
				(3400,0.968103)
				(4256,0.980664)
				(4256,0.980664)
				(5568,0.991208)
			};
	
			% QS rec.mean[0] %%%%%%%%%%%%%%%%%%%%%%%%%%%%%%%%%%%%%%%%%%%%%%%%%%%%%%%%%%%%
			\addplot[QS] coordinates{
				(223.521,0.476669)
				(325.303,0.562415)
				(414.609,0.612542)
				(663.504,0.697269)
				(828.486,0.731644)
				(1077.18,0.768805)
				(1252.44,0.788625)
				(1475.56,0.807987)
				(1767.37,0.828718)
				(2165.34,0.850695)
				(2703.49,0.873626)
				(3460.91,0.897484)
				(4543.63,0.920371)
				(6144.37,0.942389)
			};
	
			% CIS rec.mean[0] %%%%%%%%%%%%%%%%%%%%%%%%%%%%%%%%%%%%%%%%%%%%%%%%%%%%%%%%%%%%
			\addplot[CIS] coordinates{
				(292.103,0.610493)
				(366.787,0.645196)
				(436.398,0.668583)
				(575.972,0.708601)
				(721.491,0.737979)
				(783.672,0.748744)
				(963.875,0.774646)
				(1083.42,0.789764)
				(1227.51,0.807127)
				(1408.35,0.823753)
				(1424.97,0.825153)
				(1751.64,0.847142)
				(2118.16,0.872296)
				(2717.81,0.898401)
				(3240.8,0.92119)
				(3240.8,0.92119)
				(4824.68,0.944772)
			};
	
			% RESEEDS3D rec.mean[0] %%%%%%%%%%%%%%%%%%%%%%%%%%%%%%%%%%%%%%%%%%%%%%%%%%%%%%%%%%%%
			\addplot[RESEEDS3D] coordinates{
				(200.035,0.659841)
				(300.105,0.828449)
				(400.128,0.861224)
				(600.233,0.897873)
				(800.351,0.919173)
				(1000.04,0.937113)
				(1200.03,0.948741)
				(1400.65,0.95732)
				(1600.08,0.964406)
				(1792.08,0.971717)
				(1998.07,0.976362)
				(2408.16,0.982185)
				(2801.12,0.987118)
				(3182.14,0.990265)
				(3648.05,0.992362)
				(4257.09,0.995168)
				(4444.11,0.995488)
				(4864.04,0.996636)
			};
	
			% ERS rec.mean[0] %%%%%%%%%%%%%%%%%%%%%%%%%%%%%%%%%%%%%%%%%%%%%%%%%%%%%%%%%%%%
			\addplot[ERS] coordinates{
				(200,0.777384)
				(300,0.819032)
				(400,0.847678)
				(600,0.884792)
				(800,0.90986)
				(1000,0.928467)
				(1200,0.942134)
				(1400,0.952925)
				(1600,0.961016)
				(1800,0.96755)
				(2000,0.973544)
				(2400,0.981608)
				(2800,0.987248)
				(3200,0.990947)
				(3600,0.993611)
				(4000,0.995306)
				(4600,0.997182)
				(5200,0.998226)
			};
	
			% DASP rec.mean[0] %%%%%%%%%%%%%%%%%%%%%%%%%%%%%%%%%%%%%%%%%%%%%%%%%%%%%%%%%%%%
			\addplot[DASP] coordinates{
				(417.987,0.753406)
				(521.05,0.788945)
				(620.704,0.81366)
				(812.654,0.846548)
				(1006.72,0.869387)
				(1197.12,0.885835)
				(1388.06,0.899261)
				(1576.93,0.910491)
				(1767.2,0.919433)
				(1947.7,0.927872)
				(2131.65,0.933644)
				(2494.37,0.945005)
				(2860.93,0.953898)
				(3211.97,0.960711)
				(3564.56,0.966089)
				(3909.24,0.970412)
				(4418.09,0.975655)
				(4919.67,0.979571)
			};
	
			% MSS rec.mean[0] %%%%%%%%%%%%%%%%%%%%%%%%%%%%%%%%%%%%%%%%%%%%%%%%%%%%%%%%%%%%
			\addplot[MSS] coordinates{
				(207.576,0.683642)
				(308.962,0.725936)
				(436.368,0.767152)
				(668.784,0.810237)
				(912.997,0.840666)
				(1055.23,0.856045)
				(1363.6,0.879998)
				(1575.7,0.893037)
				(1862.23,0.908087)
				(2222.47,0.923461)
				(2222.47,0.923461)
				(2666.45,0.937986)
				(3236.21,0.951357)
				(4079.47,0.967442)
				(4078.85,0.96749)
				(5209.16,0.978985)
				(5209.16,0.978985)
				(6994.34,0.990446)
			};
	
			% ETPS rec.mean[0] %%%%%%%%%%%%%%%%%%%%%%%%%%%%%%%%%%%%%%%%%%%%%%%%%%%%%%%%%%%%
			\addplot[ETPS] coordinates{
				(221,0.847186)
				(315,0.877062)
				(432,0.901643)
				(638,0.930948)
				(850,0.945119)
				(972,0.951682)
				(1230,0.962653)
				(1408,0.967574)
				(1645,0.974413)
				(1938,0.979011)
				(1938,0.979011)
				(2296,0.986204)
				(2745,0.988495)
				(3400,0.992648)
				(3400,0.992648)
				(4256,0.995195)
				(4256,0.995195)
				(5568,0.99785)
			};

			\end{axis}
	\end{tikzpicture}
\end{subfigure}
\begin{subfigure}[b]{0.325\textwidth}\phantomsubcaption\label{subfig:experiments-quantitative-nyuv2-ue_np.mean[0]}
	%%%%%%%%%%%%%%%%%%%%%%%%%%%%%%%%%%%%%%%%%%%%%%%%%%%%%%%%%%%%
	% ue_np.mean[0]
	%%%%%%%%%%%%%%%%%%%%%%%%%%%%%%%%%%%%%%%%%%%%%%%%%%%%%%%%%%%%
	\begin{tikzpicture}
		\begin{axis}[EQNYUV2UE,xmode=log]		
	
			% CCS ue_np.mean[0] %%%%%%%%%%%%%%%%%%%%%%%%%%%%%%%%%%%%%%%%%%%%%%%%%%%%%%%%%%%%
			\addplot[CCS] coordinates{
				(194.01,0.216993)
				(282.682,0.181444)
				(397.098,0.154589)
				(597.153,0.129512)
				(803.471,0.115026)
				(925.103,0.108389)
				(1177.26,0.0988794)
				(1397.83,0.0931286)
				(1588.43,0.0888312)
				(1878.46,0.0834146)
				(1878.46,0.0834146)
				(2231.81,0.0781559)
				(2677.06,0.0730322)
				(3328.01,0.066989)
				(3328.01,0.066989)
				(4313.98,0.0606271)
				(4313.98,0.0606271)
				(5573.99,0.0544813)
			};
	
			% SEEDS ue_np.mean[0] %%%%%%%%%%%%%%%%%%%%%%%%%%%%%%%%%%%%%%%%%%%%%%%%%%%%%%%%%%%%
			\addplot[SEEDS] coordinates{
				(246.043,0.255681)
				(347.734,0.238093)
				(450.607,0.205785)
				(654.489,0.169993)
				(857.654,0.149621)
				(1057.59,0.133604)
				(1258.91,0.125471)
				(1462.49,0.119259)
				(1661.42,0.1105)
				(2066.14,0.11113)
				(2473.07,0.0964617)
				(2864.58,0.0923406)
				(3245.7,0.0878599)
				(3711.17,0.08469)
				(4314.51,0.076744)
				(4499.92,0.0736838)
				(4922.37,0.0731233)
			};
	
			% SLIC ue_np.mean[0] %%%%%%%%%%%%%%%%%%%%%%%%%%%%%%%%%%%%%%%%%%%%%%%%%%%%%%%%%%%%
			\addplot[SLIC] coordinates{
				(184.484,0.20446)
				(279.316,0.173027)
				(385.211,0.15327)
				(596.942,0.1304)
				(819.424,0.116049)
				(939.241,0.110653)
				(1211.08,0.100956)
				(1425.96,0.0956362)
				(1642.44,0.0906285)
				(1940.44,0.0854002)
				(1940.44,0.0854002)
				(2316.71,0.0798519)
				(2774.73,0.0745244)
				(3441.91,0.0685611)
				(3441.91,0.0685611)
				(4395.74,0.0624498)
				(4395.74,0.0624498)
				(5687.73,0.05599)
			};
	
			% RW ue_np.mean[0] %%%%%%%%%%%%%%%%%%%%%%%%%%%%%%%%%%%%%%%%%%%%%%%%%%%%%%%%%%%%
			\addplot[RW] coordinates{
				(215.912,0.224623)
				(306.055,0.19181)
				(440.393,0.162739)
				(635.679,0.140094)
				(846.511,0.12565)
				(1060.11,0.115735)
				(1306.99,0.106652)
				(1464.51,0.102514)
				(1685.01,0.0972265)
				(1868.17,0.0939696)
				(2109.97,0.0916813)
				(3967.51,0.0720991)
				(4401.89,0.0693895)
				(5110.46,0.0660025)
			};
	
			% CW ue_np.mean[0] %%%%%%%%%%%%%%%%%%%%%%%%%%%%%%%%%%%%%%%%%%%%%%%%%%%%%%%%%%%%
			\addplot[CW] coordinates{
				(198.614,0.213808)
				(293.173,0.182339)
				(407.073,0.160195)
				(613.509,0.138853)
				(834.84,0.12352)
				(1056.32,0.113553)
				(1266,0.106607)
				(1460.01,0.101244)
				(1700.24,0.0967138)
				(1828.56,0.094263)
				(2009.84,0.0913538)
				(2449.61,0.0852219)
				(2991.17,0.0795497)
				(3233.9,0.077063)
				(3707.85,0.0739416)
				(4124.98,0.0708589)
				(5174.47,0.0651113)
				(5281.06,0.0653677)
			};
	
			% TP ue_np.mean[0] %%%%%%%%%%%%%%%%%%%%%%%%%%%%%%%%%%%%%%%%%%%%%%%%%%%%%%%%%%%%
			\addplot[TP] coordinates{
				(295.99,0.173268)
				(395.489,0.152659)
				(551.496,0.143646)
				(775.682,0.124888)
				(1000.76,0.113298)
				(1130.49,0.107902)
				(1401.47,0.098847)
				(1572.01,0.0956769)
				(1815.33,0.0908693)
				(2110.15,0.0862166)
			};
	
			% POISE ue_np.mean[0] %%%%%%%%%%%%%%%%%%%%%%%%%%%%%%%%%%%%%%%%%%%%%%%%%%%%%%%%%%%%
			\addplot[POISE] coordinates{
				(204.942,0.199433)
				(306.83,0.167724)
				(408.614,0.150395)
				(612.1,0.131616)
				(815.501,0.120815)
				(1019.01,0.114183)
				(1222.5,0.108918)
				(1425.91,0.104877)
				(1628.65,0.101643)
				(1830.61,0.0989652)
				(2031.68,0.0967829)
				(2427.44,0.0934294)
				(2807.4,0.0908592)
				(3158.64,0.0892863)
				(3466.96,0.0879902)
				(3736.5,0.0871307)
				(4042.51,0.0864045)
				(4219.03,0.0860905)
			};
	
			% FH ue_np.mean[0] %%%%%%%%%%%%%%%%%%%%%%%%%%%%%%%%%%%%%%%%%%%%%%%%%%%%%%%%%%%%
			\addplot[FH] coordinates{
				(689.802,0.158879)
				(763.792,0.151115)
				(813.815,0.147271)
				(988.712,0.131736)
				(1077.66,0.125833)
				(1359.76,0.113046)
				(1408.18,0.117883)
				(1559.12,0.115032)
				(1923.5,0.0978674)
				(2206.32,0.100858)
				(2448.19,0.0971935)
				(2699.96,0.0850741)
				(3024.27,0.0872945)
				(3568.29,0.0847671)
				(4199.72,0.079623)
				(4594.6,0.0736738)
			};
	
			% EAMS ue_np.mean[0] %%%%%%%%%%%%%%%%%%%%%%%%%%%%%%%%%%%%%%%%%%%%%%%%%%%%%%%%%%%%
			\addplot[EAMS] coordinates{
				(419.068,0.144734)
				(455.657,0.140388)
				(499.416,0.135681)
				(641.188,0.125544)
				(842.328,0.114359)
				(1268.71,0.101801)
				(2584.6,0.0832607)
				(2683.21,0.0822668)
				(2779.09,0.0827376)
				(2997.7,0.0808278)
				(3180.29,0.080521)
				(3421.45,0.0793529)
				(3646.16,0.0787786)
				(3943.04,0.0783151)
				(4448.4,0.0755675)
				(5663.91,0.0704212)
				(9098.39,0.0619494)
			};
	
			% CRS ue_np.mean[0] %%%%%%%%%%%%%%%%%%%%%%%%%%%%%%%%%%%%%%%%%%%%%%%%%%%%%%%%%%%%
			\addplot[CRS] coordinates{
				(254.263,0.174765)
				(396.694,0.145756)
				(514.895,0.130322)
				(764.997,0.111728)
				(988.589,0.101933)
				(1232.03,0.0933611)
				(1498.25,0.0871983)
				(1707.65,0.0835093)
				(1968.19,0.0790225)
				(2099.81,0.0768389)
				(2299.23,0.0743375)
				(2706.1,0.0701011)
				(3259.07,0.0654525)
				(3558.75,0.0633238)
				(4055.02,0.0602152)
				(4394.15,0.0583992)
				(5418.11,0.0537133)
				(5695.52,0.0527932)
			};
	
			% SEAW ue_np.mean[0] %%%%%%%%%%%%%%%%%%%%%%%%%%%%%%%%%%%%%%%%%%%%%%%%%%%%%%%%%%%%
			\addplot[SEAW] coordinates{
				(124.667,0.35668)
				(349.414,0.211234)
				(1192.36,0.119281)
				(4497.65,0.0695662)
			};
	
			% RESEEDS ue_np.mean[0] %%%%%%%%%%%%%%%%%%%%%%%%%%%%%%%%%%%%%%%%%%%%%%%%%%%%%%%%%%%%
			\addplot[RESEEDS] coordinates{
				(200.441,0.214926)
				(300.263,0.207865)
				(400.323,0.17559)
				(600.361,0.140873)
				(800.526,0.122851)
				(1000.06,0.108806)
				(1200.06,0.102369)
				(1400.96,0.096608)
				(1600.21,0.0903118)
				(1792.08,0.101512)
				(1998.13,0.0869894)
				(2408.17,0.0776285)
				(2801.44,0.0743663)
				(3182.16,0.0703948)
				(3648.12,0.0689841)
				(4257.44,0.0624645)
				(4444.16,0.0609398)
				(4864.15,0.0601597)
			};
	
			% ERGC ue_np.mean[0] %%%%%%%%%%%%%%%%%%%%%%%%%%%%%%%%%%%%%%%%%%%%%%%%%%%%%%%%%%%%
			\addplot[ERGC] coordinates{
				(196,0.194213)
				(289,0.165664)
				(400,0.145713)
				(600,0.126109)
				(812,0.112044)
				(1024,0.103719)
				(1224,0.0976167)
				(1406,0.0927796)
				(1681,0.0878355)
				(1763,0.0865362)
				(1935,0.0839173)
				(2350,0.0784336)
				(2856,0.0734824)
				(3080,0.0714722)
				(3520,0.0685269)
				(3904,0.0656898)
				(4864,0.0602308)
				(5100,0.0600402)
			};
	
			% PF ue_np.mean[0] %%%%%%%%%%%%%%%%%%%%%%%%%%%%%%%%%%%%%%%%%%%%%%%%%%%%%%%%%%%%
			\addplot[PF] coordinates{
				(281.19,0.36559)
				(430.376,0.320023)
				(586.371,0.291422)
				(907.19,0.255785)
				(1201.57,0.232596)
				(1354.07,0.22401)
				(1718.72,0.208817)
				(1939.43,0.200007)
				(2251.38,0.191596)
				(2592.38,0.182522)
				(3186.03,0.174335)
				(4180.57,0.160473)
				(6182.51,0.14268)
			};
	
			% TPS ue_np.mean[0] %%%%%%%%%%%%%%%%%%%%%%%%%%%%%%%%%%%%%%%%%%%%%%%%%%%%%%%%%%%%
			\addplot[TPS] coordinates{
				(230.419,0.215002)
				(313.739,0.188408)
				(450.436,0.163971)
				(638.008,0.142489)
				(857.559,0.126421)
				(1043.09,0.117862)
				(1317.64,0.107356)
				(1503.8,0.101886)
				(1703.97,0.0975202)
				(1873.98,0.0938476)
				(2144.93,0.0909679)
				(2580.57,0.0843347)
				(2894.48,0.0808023)
				(3314.47,0.0773751)
				(3924.13,0.0708422)
				(4380.02,0.0687939)
				(5132.34,0.0642085)
				(5738.91,0.0623916)
			};
	
			% NC ue_np.mean[0] %%%%%%%%%%%%%%%%%%%%%%%%%%%%%%%%%%%%%%%%%%%%%%%%%%%%%%%%%%%%
			\addplot[NC] coordinates{
				(439.952,0.142744)
				(1154.47,0.114141)
				(2631.49,0.0836629)
				(3874.82,0.072947)
			};
	
			% VC ue_np.mean[0] %%%%%%%%%%%%%%%%%%%%%%%%%%%%%%%%%%%%%%%%%%%%%%%%%%%%%%%%%%%%
			\addplot[VC] coordinates{
				(243.744,0.239429)
				(400.291,0.177749)
				(531.672,0.150768)
				(660.84,0.134662)
				(895.915,0.115686)
				(1125.96,0.104743)
				(1348.09,0.0975676)
				(1564.34,0.09205)
				(1781.1,0.0876126)
				(1995.21,0.0842898)
				(2205.86,0.0813086)
				(2416.36,0.0785084)
				(2820.76,0.0743309)
				(3224.25,0.0708998)
				(3610.4,0.0680529)
				(3988.04,0.0657843)
				(4351.59,0.0636596)
				(4847.23,0.0611841)
				(5262.69,0.0592493)
			};
	
			% PB ue_np.mean[0] %%%%%%%%%%%%%%%%%%%%%%%%%%%%%%%%%%%%%%%%%%%%%%%%%%%%%%%%%%%%
			\addplot[PB] coordinates{
				(273.248,0.250511)
				(360.188,0.211851)
				(463.193,0.185017)
				(656.962,0.155462)
				(857.526,0.137242)
				(970.915,0.129242)
				(1217.27,0.117229)
				(1387.49,0.111035)
				(1616.5,0.104393)
				(1896.95,0.0984057)
				(1896.95,0.0984057)
				(2243.04,0.0917164)
				(2681.85,0.0860795)
				(3316.29,0.0792165)
				(3316.29,0.0792165)
				(4151.92,0.0726646)
				(4151.92,0.0726646)
				(5425.24,0.0661418)
			};
	
			% VCCS ue_np.mean[0] %%%%%%%%%%%%%%%%%%%%%%%%%%%%%%%%%%%%%%%%%%%%%%%%%%%%%%%%%%%%
			\addplot[VCCS] coordinates{
				(564.123,0.199063)
				(600.283,0.191901)
				(648.629,0.181498)
				(755.632,0.167167)
				(836.05,0.158543)
				(1051.64,0.148687)
				(1109.07,0.144169)
				(1179.74,0.140147)
				(1237.23,0.137731)
				(1322.17,0.133494)
				(1400.72,0.130995)
				(1526.22,0.124518)
				(1630.95,0.121193)
				(1756.59,0.116931)
				(1914.09,0.112234)
				(2024.07,0.111118)
				(2258.58,0.104267)
				(2558.8,0.096803)
				(2780.39,0.0943227)
				(3044.22,0.090837)
				(3687.7,0.078021)
				(4171.76,0.0730024)
				(4255.91,0.0774884)
			};
	
			% PRESLIC ue_np.mean[0] %%%%%%%%%%%%%%%%%%%%%%%%%%%%%%%%%%%%%%%%%%%%%%%%%%%%%%%%%%%%
			\addplot[PRESLIC] coordinates{
				(189.89,0.217645)
				(388.539,0.161778)
				(589.772,0.140151)
				(798.767,0.125066)
				(920.201,0.118191)
				(1188.65,0.107943)
				(1401.21,0.101064)
				(1612.25,0.096223)
				(1904.93,0.0904268)
				(1904.93,0.0904268)
				(2282.12,0.0843557)
				(2738.74,0.0794571)
				(3422.72,0.0727227)
				(3422.72,0.0727227)
				(4393.15,0.0658819)
				(4393.15,0.0658819)
				(5724.22,0.0596778)
			};
	
			% W ue_np.mean[0] %%%%%%%%%%%%%%%%%%%%%%%%%%%%%%%%%%%%%%%%%%%%%%%%%%%%%%%%%%%%
			\addplot[W] coordinates{
				(193.87,0.230706)
				(303.997,0.192795)
				(397.103,0.17269)
				(621.108,0.144759)
				(870.236,0.127816)
				(959.915,0.122907)
				(1234.38,0.112302)
				(1418.74,0.106601)
				(1652.83,0.100844)
				(1957.01,0.0947564)
				(1957.01,0.0947564)
				(2345.3,0.0889117)
				(2867.42,0.0822838)
				(3518.42,0.076682)
				(3518.42,0.076682)
				(4497.46,0.0696664)
				(4497.46,0.0696664)
				(5940.33,0.0629077)
			};
	
			% LSC ue_np.mean[0] %%%%%%%%%%%%%%%%%%%%%%%%%%%%%%%%%%%%%%%%%%%%%%%%%%%%%%%%%%%%
			\addplot[LSC] coordinates{
				(383.095,0.161783)
				(552.409,0.140294)
				(718.163,0.126949)
				(1059.33,0.110189)
				(1414.95,0.0988616)
				(1816.61,0.0898628)
				(2009.32,0.086118)
				(2263.42,0.0819968)
				(2398.35,0.0792781)
				(2446.8,0.0779983)
				(2588.34,0.0758377)
				(2830.36,0.0724911)
				(3264.3,0.0682116)
				(3438.52,0.0668361)
				(3802.19,0.0641309)
				(4059.6,0.0629008)
				(4900.83,0.0597229)
				(5001.09,0.058997)
			};
	
			% WP ue_np.mean[0] %%%%%%%%%%%%%%%%%%%%%%%%%%%%%%%%%%%%%%%%%%%%%%%%%%%%%%%%%%%%
			\addplot[WP] coordinates{
				(204,0.190887)
				(315,0.162058)
				(432,0.145964)
				(638,0.127101)
				(850,0.11559)
				(1064,0.106748)
				(1230,0.102756)
				(1408,0.0984739)
				(1645,0.0941458)
				(1938,0.0898955)
				(1938,0.0897323)
				(2296,0.0848078)
				(2745,0.0799514)
				(3400,0.0751955)
				(3400,0.0751955)
				(4256,0.0701715)
				(4256,0.0701715)
				(5568,0.064637)
			};
	
			% QS ue_np.mean[0] %%%%%%%%%%%%%%%%%%%%%%%%%%%%%%%%%%%%%%%%%%%%%%%%%%%%%%%%%%%%
			\addplot[QS] coordinates{
				(223.521,0.407091)
				(325.303,0.309692)
				(414.609,0.262196)
				(663.504,0.195078)
				(828.486,0.172383)
				(1077.18,0.150497)
				(1252.44,0.139966)
				(1475.56,0.130129)
				(1767.37,0.120231)
				(2165.34,0.110591)
				(2703.49,0.101247)
				(3460.91,0.0915782)
				(4543.63,0.082224)
				(6144.37,0.0728167)
			};
	
			% CIS ue_np.mean[0] %%%%%%%%%%%%%%%%%%%%%%%%%%%%%%%%%%%%%%%%%%%%%%%%%%%%%%%%%%%%
			\addplot[CIS] coordinates{
				(292.103,0.18675)
				(366.787,0.164434)
				(436.398,0.151046)
				(575.972,0.133225)
				(721.491,0.122199)
				(783.672,0.118371)
				(963.875,0.110277)
				(1083.42,0.106211)
				(1227.51,0.101446)
				(1408.35,0.097497)
				(1424.97,0.0971798)
				(1751.64,0.0914291)
				(2118.16,0.0856974)
				(2717.81,0.0794659)
				(3240.8,0.0747416)
				(3240.8,0.0747416)
				(4824.68,0.0667908)
			};
	
			% RESEEDS3D ue_np.mean[0] %%%%%%%%%%%%%%%%%%%%%%%%%%%%%%%%%%%%%%%%%%%%%%%%%%%%%%%%%%%%
			\addplot[RESEEDS3D] coordinates{
				(200.035,0.300818)
				(300.105,0.172161)
				(400.128,0.149737)
				(600.233,0.125157)
				(800.351,0.111215)
				(1000.04,0.100143)
				(1200.03,0.0939219)
				(1400.65,0.0895527)
				(1600.08,0.0841377)
				(1792.08,0.0907046)
				(1998.07,0.0806915)
				(2408.16,0.0731884)
				(2801.12,0.0698269)
				(3182.14,0.0663)
				(3648.05,0.0636726)
				(4257.09,0.0593419)
				(4444.11,0.0578795)
				(4864.04,0.0566276)
			};
	
			% ERS ue_np.mean[0] %%%%%%%%%%%%%%%%%%%%%%%%%%%%%%%%%%%%%%%%%%%%%%%%%%%%%%%%%%%%
			\addplot[ERS] coordinates{
				(200,0.164743)
				(300,0.142188)
				(400,0.129084)
				(600,0.113761)
				(800,0.103738)
				(1000,0.0970346)
				(1200,0.0915101)
				(1400,0.0872521)
				(1600,0.0834982)
				(1800,0.0804217)
				(2000,0.0777025)
				(2400,0.0728562)
				(2800,0.0690113)
				(3200,0.065847)
				(3600,0.0629311)
				(4000,0.0605456)
				(4600,0.057255)
				(5200,0.0545465)
			};
	
			% DASP ue_np.mean[0] %%%%%%%%%%%%%%%%%%%%%%%%%%%%%%%%%%%%%%%%%%%%%%%%%%%%%%%%%%%%
			\addplot[DASP] coordinates{
				(417.987,0.173612)
				(521.05,0.14796)
				(620.704,0.132322)
				(812.654,0.115135)
				(1006.72,0.104708)
				(1197.12,0.0974706)
				(1388.06,0.0921737)
				(1576.93,0.0877052)
				(1767.2,0.0841756)
				(1947.7,0.0811193)
				(2131.65,0.0788187)
				(2494.37,0.0743063)
				(2860.93,0.07091)
				(3211.97,0.0678386)
				(3564.56,0.06541)
				(3909.24,0.0632915)
				(4418.09,0.0605536)
				(4919.67,0.0582789)
			};
	
			% MSS ue_np.mean[0] %%%%%%%%%%%%%%%%%%%%%%%%%%%%%%%%%%%%%%%%%%%%%%%%%%%%%%%%%%%%
			\addplot[MSS] coordinates{
				(207.576,0.211847)
				(308.962,0.184244)
				(436.368,0.157)
				(668.784,0.136278)
				(912.997,0.122561)
				(1055.23,0.115873)
				(1363.6,0.106897)
				(1575.7,0.102165)
				(1862.23,0.0969694)
				(2222.47,0.0918802)
				(2222.47,0.0918802)
				(2666.45,0.086867)
				(3236.21,0.08235)
				(4079.47,0.0762429)
				(4078.85,0.0763146)
				(5209.16,0.0709364)
				(5209.16,0.0709364)
				(6994.34,0.0642183)
			};
	
			% ETPS ue_np.mean[0] %%%%%%%%%%%%%%%%%%%%%%%%%%%%%%%%%%%%%%%%%%%%%%%%%%%%%%%%%%%%
			\addplot[ETPS] coordinates{
				(221,0.185724)
				(315,0.157975)
				(432,0.141052)
				(638,0.120615)
				(850,0.107936)
				(972,0.103176)
				(1230,0.0942883)
				(1408,0.0897313)
				(1645,0.0850996)
				(1938,0.079887)
				(1938,0.079887)
				(2296,0.0752435)
				(2745,0.0701251)
				(3400,0.0650547)
				(3400,0.0650547)
				(4256,0.0592612)
				(4256,0.0592612)
				(5568,0.053455)
			};

			\end{axis}
	\end{tikzpicture}
\end{subfigure}
\begin{subfigure}[b]{0.325\textwidth}\phantomsubcaption\label{subfig:experiments-quantitative-nyuv2-ev.mean[0]}
	%%%%%%%%%%%%%%%%%%%%%%%%%%%%%%%%%%%%%%%%%%%%%%%%%%%%%%%%%%%%
	% ev.mean[0]
	%%%%%%%%%%%%%%%%%%%%%%%%%%%%%%%%%%%%%%%%%%%%%%%%%%%%%%%%%%%%
	\begin{tikzpicture}
		\begin{axis}[EQNYUV2EV,xmode=log]		
	
			% CCS ev.mean[0] %%%%%%%%%%%%%%%%%%%%%%%%%%%%%%%%%%%%%%%%%%%%%%%%%%%%%%%%%%%%
			\addplot[CCS] coordinates{
				(194.01,0.893177)
				(282.682,0.911908)
				(397.098,0.92873)
				(597.153,0.942614)
				(803.471,0.950433)
				(925.103,0.954406)
				(1177.26,0.959688)
				(1397.83,0.962867)
				(1588.43,0.965142)
				(1878.46,0.967916)
				(1878.46,0.967916)
				(2231.81,0.97053)
				(2677.06,0.973016)
				(3328.01,0.975848)
				(3328.01,0.975848)
				(4313.98,0.978679)
				(4313.98,0.978679)
				(5573.99,0.981363)
			};
	
			% SEEDS ev.mean[0] %%%%%%%%%%%%%%%%%%%%%%%%%%%%%%%%%%%%%%%%%%%%%%%%%%%%%%%%%%%%
			\addplot[SEEDS] coordinates{
				(246.043,0.934881)
				(347.734,0.93724)
				(450.607,0.946175)
				(654.489,0.955868)
				(857.654,0.961278)
				(1057.59,0.964005)
				(1258.91,0.966181)
				(1462.49,0.969441)
				(1661.42,0.970682)
				(1870.25,0.96334)
				(2066.14,0.970979)
				(2473.07,0.974838)
				(2864.58,0.976619)
				(3245.7,0.977109)
				(3711.17,0.975841)
				(4314.51,0.980465)
				(4499.92,0.980441)
				(4922.37,0.979746)
			};
	
			% SLIC ev.mean[0] %%%%%%%%%%%%%%%%%%%%%%%%%%%%%%%%%%%%%%%%%%%%%%%%%%%%%%%%%%%%
			\addplot[SLIC] coordinates{
				(184.484,0.878556)
				(279.316,0.899742)
				(385.211,0.909271)
				(596.942,0.925306)
				(819.424,0.935728)
				(939.241,0.938751)
				(1211.08,0.946337)
				(1425.96,0.949045)
				(1642.44,0.953171)
				(1940.44,0.956773)
				(1940.44,0.956773)
				(2316.71,0.960537)
				(2774.73,0.964185)
				(3441.91,0.967887)
				(3441.91,0.967887)
				(4395.74,0.971133)
				(4395.74,0.971133)
				(5687.73,0.975497)
			};
	
			% RW ev.mean[0] %%%%%%%%%%%%%%%%%%%%%%%%%%%%%%%%%%%%%%%%%%%%%%%%%%%%%%%%%%%%
			\addplot[RW] coordinates{
				(215.912,0.84156)
				(306.055,0.865414)
				(440.393,0.887146)
				(635.679,0.903829)
				(846.511,0.915441)
				(1060.11,0.923573)
				(1306.99,0.93091)
				(1464.51,0.934017)
				(1685.01,0.938829)
				(1868.17,0.941459)
				(2109.97,0.940449)
				(3967.51,0.959975)
				(4401.89,0.962427)
				(5110.46,0.964837)
			};
	
			% CW ev.mean[0] %%%%%%%%%%%%%%%%%%%%%%%%%%%%%%%%%%%%%%%%%%%%%%%%%%%%%%%%%%%%
			\addplot[CW] coordinates{
				(198.614,0.861677)
				(293.173,0.882706)
				(407.073,0.898629)
				(613.509,0.913959)
				(834.84,0.924798)
				(1056.32,0.931478)
				(1266,0.936329)
				(1460.01,0.939954)
				(1700.24,0.942547)
				(1828.56,0.944683)
				(2009.84,0.946716)
				(2449.61,0.95061)
				(2991.17,0.954065)
				(3233.9,0.955356)
				(3707.85,0.95733)
				(4124.98,0.959121)
				(5174.47,0.962357)
				(5281.06,0.962256)
			};
	
			% TP ev.mean[0] %%%%%%%%%%%%%%%%%%%%%%%%%%%%%%%%%%%%%%%%%%%%%%%%%%%%%%%%%%%%
			\addplot[TP] coordinates{
				(295.99,0.899608)
				(395.489,0.912407)
				(551.496,0.908226)
				(775.682,0.920697)
				(1000.76,0.928974)
				(1130.49,0.93244)
				(1401.47,0.939073)
				(1572.01,0.940609)
				(1815.33,0.944452)
				(2110.15,0.947718)
			};
	
			% POISE ev.mean[0] %%%%%%%%%%%%%%%%%%%%%%%%%%%%%%%%%%%%%%%%%%%%%%%%%%%%%%%%%%%%
			\addplot[POISE] coordinates{
				(204.942,0.895482)
				(306.83,0.914532)
				(408.614,0.924498)
				(612.1,0.933996)
				(815.501,0.938936)
				(1019.01,0.941664)
				(1222.5,0.943604)
				(1425.91,0.944939)
				(1628.65,0.946043)
				(1830.61,0.946824)
				(2031.68,0.94742)
				(2427.44,0.948434)
				(2807.4,0.949079)
				(3158.64,0.949422)
				(3466.96,0.949698)
				(3736.5,0.949975)
				(4042.51,0.950109)
				(4219.03,0.950159)
			};
	
			% FH ev.mean[0] %%%%%%%%%%%%%%%%%%%%%%%%%%%%%%%%%%%%%%%%%%%%%%%%%%%%%%%%%%%%
			\addplot[FH] coordinates{
				(689.802,0.893898)
				(763.792,0.902944)
				(813.815,0.907919)
				(988.712,0.919994)
				(1077.66,0.923711)
				(1359.76,0.934975)
				(1408.18,0.939764)
				(1559.12,0.940477)
				(1923.5,0.948613)
				(2206.32,0.951963)
				(2448.19,0.95291)
				(2699.96,0.961446)
				(3024.27,0.963632)
				(3568.29,0.966536)
				(4199.72,0.967728)
				(4594.6,0.968987)
			};
	
			% EAMS ev.mean[0] %%%%%%%%%%%%%%%%%%%%%%%%%%%%%%%%%%%%%%%%%%%%%%%%%%%%%%%%%%%%
			\addplot[EAMS] coordinates{
				(419.068,0.947248)
				(455.657,0.949342)
				(499.416,0.951533)
				(641.188,0.954835)
				(842.328,0.960246)
				(1268.71,0.967075)
				(2584.6,0.976054)
				(2683.21,0.976489)
				(2779.09,0.976984)
				(2997.7,0.977911)
				(3180.29,0.978676)
				(3421.45,0.979844)
				(3646.16,0.98083)
				(3943.04,0.981875)
				(4448.4,0.983239)
				(5663.91,0.985732)
				(9098.39,0.989579)
			};
	
			% CRS ev.mean[0] %%%%%%%%%%%%%%%%%%%%%%%%%%%%%%%%%%%%%%%%%%%%%%%%%%%%%%%%%%%%
			\addplot[CRS] coordinates{
				(254.263,0.891014)
				(396.694,0.911181)
				(514.895,0.921485)
				(764.997,0.933545)
				(988.589,0.939859)
				(1232.03,0.945659)
				(1498.25,0.949881)
				(1707.65,0.952232)
				(1968.19,0.95531)
				(2099.81,0.956616)
				(2299.23,0.958232)
				(2706.1,0.96105)
				(3259.07,0.964042)
				(3558.75,0.96549)
				(4055.02,0.967512)
				(4394.15,0.968571)
				(5418.11,0.971326)
				(5695.52,0.97209)
			};
	
			% SEAW ev.mean[0] %%%%%%%%%%%%%%%%%%%%%%%%%%%%%%%%%%%%%%%%%%%%%%%%%%%%%%%%%%%%
			\addplot[SEAW] coordinates{
				(124.667,0.811606)
				(349.414,0.885082)
				(1192.36,0.934554)
				(4497.65,0.964793)
			};
	
			% RESEEDS ev.mean[0] %%%%%%%%%%%%%%%%%%%%%%%%%%%%%%%%%%%%%%%%%%%%%%%%%%%%%%%%%%%%
			\addplot[RESEEDS] coordinates{
				(200.441,0.945281)
				(300.263,0.946744)
				(400.323,0.955077)
				(600.361,0.964077)
				(800.526,0.96891)
				(1000.06,0.973217)
				(1200.06,0.974894)
				(1400.96,0.976275)
				(1600.21,0.978391)
				(1792.08,0.973253)
				(1998.13,0.978927)
				(2408.17,0.981768)
				(2801.44,0.982456)
				(3182.16,0.983714)
				(3648.12,0.983135)
				(4257.44,0.98574)
				(4444.16,0.986248)
				(4864.15,0.985933)
			};
	
			% ERGC ev.mean[0] %%%%%%%%%%%%%%%%%%%%%%%%%%%%%%%%%%%%%%%%%%%%%%%%%%%%%%%%%%%%
			\addplot[ERGC] coordinates{
				(196,0.89148)
				(289,0.908694)
				(400,0.921646)
				(600,0.934897)
				(812,0.94432)
				(1024,0.949815)
				(1224,0.953986)
				(1406,0.957061)
				(1681,0.960442)
				(1763,0.961247)
				(1935,0.962988)
				(2350,0.966545)
				(2856,0.969554)
				(3080,0.97076)
				(3520,0.97255)
				(3904,0.974148)
				(4864,0.976977)
				(5100,0.977423)
			};
	
			% PF ev.mean[0] %%%%%%%%%%%%%%%%%%%%%%%%%%%%%%%%%%%%%%%%%%%%%%%%%%%%%%%%%%%%
			\addplot[PF] coordinates{
				(281.19,0.726914)
				(430.376,0.76164)
				(586.371,0.78353)
				(907.19,0.81125)
				(1201.57,0.828991)
				(1354.07,0.835696)
				(1718.72,0.847546)
				(1939.43,0.854598)
				(2251.38,0.861227)
				(2592.38,0.868339)
				(3186.03,0.878114)
				(4180.57,0.888378)
				(6182.51,0.902822)
			};
	
			% TPS ev.mean[0] %%%%%%%%%%%%%%%%%%%%%%%%%%%%%%%%%%%%%%%%%%%%%%%%%%%%%%%%%%%%
			\addplot[TPS] coordinates{
				(230.419,0.829957)
				(313.739,0.849634)
				(450.436,0.870214)
				(638.008,0.887409)
				(857.559,0.900927)
				(1043.09,0.908195)
				(1317.64,0.917528)
				(1503.8,0.922142)
				(1703.97,0.925493)
				(1873.98,0.928906)
				(2144.93,0.932283)
				(2580.57,0.938222)
				(2894.48,0.941779)
				(3314.47,0.945189)
				(3924.13,0.947752)
				(4380.02,0.950053)
				(5132.34,0.95494)
				(5738.91,0.957128)
			};
	
			% NC ev.mean[0] %%%%%%%%%%%%%%%%%%%%%%%%%%%%%%%%%%%%%%%%%%%%%%%%%%%%%%%%%%%%
			\addplot[NC] coordinates{
				(439.952,0.899222)
				(1154.47,0.918983)
				(2631.49,0.931375)
				(3874.82,0.938979)
			};
	
			% VC ev.mean[0] %%%%%%%%%%%%%%%%%%%%%%%%%%%%%%%%%%%%%%%%%%%%%%%%%%%%%%%%%%%%
			\addplot[VC] coordinates{
				(243.744,0.893413)
				(400.291,0.924994)
				(531.672,0.938571)
				(660.84,0.946357)
				(895.915,0.955693)
				(1125.96,0.961044)
				(1348.09,0.96465)
				(1564.34,0.967439)
				(1781.1,0.969465)
				(1995.21,0.971118)
				(2205.86,0.972453)
				(2416.36,0.97368)
				(2820.76,0.975629)
				(3224.25,0.977244)
				(3610.4,0.978548)
				(3988.04,0.979661)
				(4351.59,0.98072)
				(4847.23,0.981948)
				(5262.69,0.982725)
			};
	
			% PB ev.mean[0] %%%%%%%%%%%%%%%%%%%%%%%%%%%%%%%%%%%%%%%%%%%%%%%%%%%%%%%%%%%%
			\addplot[PB] coordinates{
				(273.248,0.800373)
				(360.188,0.831171)
				(463.193,0.850733)
				(656.962,0.875129)
				(857.526,0.890435)
				(970.915,0.897103)
				(1217.27,0.908504)
				(1387.49,0.914525)
				(1616.5,0.920503)
				(1896.95,0.926975)
				(1896.95,0.926975)
				(2243.04,0.933346)
				(2681.85,0.938959)
				(3316.29,0.945978)
				(3316.29,0.945978)
				(4151.92,0.95249)
				(4151.92,0.95249)
				(5425.24,0.959358)
			};
	
			% VCCS ev.mean[0] %%%%%%%%%%%%%%%%%%%%%%%%%%%%%%%%%%%%%%%%%%%%%%%%%%%%%%%%%%%%
			\addplot[VCCS] coordinates{
				(564.123,0.884018)
				(600.283,0.888045)
				(648.629,0.89207)
				(755.632,0.899679)
				(836.05,0.90477)
				(1051.64,0.911654)
				(1109.07,0.912702)
				(1179.74,0.915936)
				(1237.23,0.917232)
				(1322.17,0.919472)
				(1400.72,0.920536)
				(1526.22,0.92429)
				(1630.95,0.925928)
				(1756.59,0.927818)
				(1914.09,0.931222)
				(2024.07,0.930662)
				(2258.58,0.934804)
				(2558.8,0.940185)
				(2780.39,0.941096)
				(3044.22,0.94373)
				(3687.7,0.953704)
				(4171.76,0.957181)
				(4255.91,0.953049)
			};
	
			% PRESLIC ev.mean[0] %%%%%%%%%%%%%%%%%%%%%%%%%%%%%%%%%%%%%%%%%%%%%%%%%%%%%%%%%%%%
			\addplot[PRESLIC] coordinates{
				(189.89,0.872234)
				(388.539,0.904722)
				(589.772,0.91762)
				(798.767,0.926864)
				(920.201,0.932088)
				(1188.65,0.938682)
				(1401.21,0.94316)
				(1612.25,0.946497)
				(1904.93,0.950356)
				(1904.93,0.950356)
				(2282.12,0.954621)
				(2738.74,0.957512)
				(3422.72,0.962674)
				(3422.72,0.962674)
				(4393.15,0.967232)
				(4393.15,0.967232)
				(5724.22,0.970976)
			};
	
			% W ev.mean[0] %%%%%%%%%%%%%%%%%%%%%%%%%%%%%%%%%%%%%%%%%%%%%%%%%%%%%%%%%%%%
			\addplot[W] coordinates{
				(193.87,0.845949)
				(303.997,0.874358)
				(397.103,0.888758)
				(621.108,0.908318)
				(870.236,0.921058)
				(959.915,0.924688)
				(1234.38,0.931749)
				(1418.74,0.935813)
				(1652.83,0.940085)
				(1957.01,0.944165)
				(1957.01,0.944165)
				(2345.3,0.947945)
				(2867.42,0.952014)
				(3518.42,0.955648)
				(3518.42,0.955648)
				(4497.46,0.959697)
				(4497.46,0.959697)
				(5940.33,0.963546)
			};
	
			% LSC ev.mean[0] %%%%%%%%%%%%%%%%%%%%%%%%%%%%%%%%%%%%%%%%%%%%%%%%%%%%%%%%%%%%
			\addplot[LSC] coordinates{
				(383.095,0.939892)
				(552.409,0.94743)
				(718.163,0.952285)
				(1059.33,0.958395)
				(1414.95,0.962502)
				(1816.61,0.965588)
				(2009.32,0.967046)
				(2263.42,0.968251)
				(2398.35,0.969033)
				(2446.8,0.969556)
				(2588.34,0.969993)
				(2830.36,0.970869)
				(3264.3,0.97189)
				(3438.52,0.972409)
				(3802.19,0.972692)
				(4059.6,0.972977)
				(4900.83,0.973259)
				(5001.09,0.972352)
			};
	
			% WP ev.mean[0] %%%%%%%%%%%%%%%%%%%%%%%%%%%%%%%%%%%%%%%%%%%%%%%%%%%%%%%%%%%%
			\addplot[WP] coordinates{
				(204,0.881827)
				(315,0.902738)
				(432,0.915094)
				(638,0.929208)
				(850,0.937979)
				(1064,0.943928)
				(1230,0.947073)
				(1408,0.950075)
				(1645,0.952911)
				(1938,0.955936)
				(1938,0.956014)
				(2296,0.959586)
				(2745,0.962479)
				(3400,0.96548)
				(3400,0.96548)
				(4256,0.96814)
				(4256,0.96814)
				(5568,0.970962)
			};
	
			% QS ev.mean[0] %%%%%%%%%%%%%%%%%%%%%%%%%%%%%%%%%%%%%%%%%%%%%%%%%%%%%%%%%%%%
			\addplot[QS] coordinates{
				(223.521,0.76477)
				(325.303,0.836662)
				(414.609,0.86879)
				(663.504,0.910244)
				(828.486,0.923349)
				(1077.18,0.936121)
				(1252.44,0.942407)
				(1475.56,0.948327)
				(1767.37,0.954279)
				(2165.34,0.960079)
				(2703.49,0.965729)
				(3460.91,0.971249)
				(4543.63,0.976466)
				(6144.37,0.98139)
			};
	
			% CIS ev.mean[0] %%%%%%%%%%%%%%%%%%%%%%%%%%%%%%%%%%%%%%%%%%%%%%%%%%%%%%%%%%%%
			\addplot[CIS] coordinates{
				(292.103,0.886449)
				(366.787,0.900029)
				(436.398,0.908295)
				(575.972,0.919747)
				(721.491,0.92719)
				(783.672,0.930085)
				(963.875,0.93696)
				(1083.42,0.938743)
				(1227.51,0.94247)
				(1408.35,0.944388)
				(1424.97,0.94462)
				(1751.64,0.951087)
				(2118.16,0.953808)
				(2717.81,0.958313)
				(3240.8,0.960297)
				(3240.8,0.960297)
				(4824.68,0.969746)
			};
	
			% RESEEDS3D ev.mean[0] %%%%%%%%%%%%%%%%%%%%%%%%%%%%%%%%%%%%%%%%%%%%%%%%%%%%%%%%%%%%
			\addplot[RESEEDS3D] coordinates{
				(200.035,0.808713)
				(300.105,0.939082)
				(400.128,0.948884)
				(600.233,0.959507)
				(800.351,0.964908)
				(1000.04,0.969724)
				(1200.03,0.972174)
				(1400.65,0.973888)
				(1600.08,0.976021)
				(1792.08,0.972568)
				(1998.07,0.977385)
				(2408.16,0.980214)
				(2801.12,0.981219)
				(3182.14,0.982568)
				(3648.05,0.98268)
				(4257.09,0.984899)
				(4444.11,0.985424)
				(4864.04,0.98541)
			};
	
			% ERS ev.mean[0] %%%%%%%%%%%%%%%%%%%%%%%%%%%%%%%%%%%%%%%%%%%%%%%%%%%%%%%%%%%%
			\addplot[ERS] coordinates{
				(200,0.874065)
				(300,0.889788)
				(400,0.89875)
				(600,0.910053)
				(800,0.916934)
				(1000,0.921725)
				(1200,0.925844)
				(1400,0.92891)
				(1600,0.931564)
				(1800,0.933983)
				(2000,0.936166)
				(2400,0.939914)
				(2800,0.943145)
				(3200,0.945885)
				(3600,0.948298)
				(4000,0.950534)
				(4600,0.953726)
				(5200,0.956364)
			};
	
			% DASP ev.mean[0] %%%%%%%%%%%%%%%%%%%%%%%%%%%%%%%%%%%%%%%%%%%%%%%%%%%%%%%%%%%%
			\addplot[DASP] coordinates{
				(417.987,0.917908)
				(521.05,0.931824)
				(620.704,0.940121)
				(812.654,0.95017)
				(1006.72,0.955821)
				(1197.12,0.959762)
				(1388.06,0.962588)
				(1576.93,0.964917)
				(1767.2,0.966767)
				(1947.7,0.968253)
				(2131.65,0.969447)
				(2494.37,0.97152)
				(2860.93,0.973169)
				(3211.97,0.974559)
				(3564.56,0.975731)
				(3909.24,0.976675)
				(4418.09,0.977796)
				(4919.67,0.978768)
			};
	
			% MSS ev.mean[0] %%%%%%%%%%%%%%%%%%%%%%%%%%%%%%%%%%%%%%%%%%%%%%%%%%%%%%%%%%%%
			\addplot[MSS] coordinates{
				(207.576,0.865354)
				(308.962,0.886994)
				(436.368,0.906804)
				(668.784,0.922271)
				(912.997,0.931961)
				(1055.23,0.936858)
				(1363.6,0.942139)
				(1575.7,0.945125)
				(1862.23,0.948524)
				(2222.47,0.951585)
				(2222.47,0.951585)
				(2666.45,0.954117)
				(3236.21,0.95617)
				(4079.47,0.958878)
				(4078.85,0.958801)
				(5209.16,0.961026)
				(5209.16,0.961026)
				(6994.34,0.963708)
			};
	
			% ETPS ev.mean[0] %%%%%%%%%%%%%%%%%%%%%%%%%%%%%%%%%%%%%%%%%%%%%%%%%%%%%%%%%%%%
			\addplot[ETPS] coordinates{
				(221,0.954043)
				(315,0.960964)
				(432,0.965378)
				(638,0.970609)
				(850,0.973829)
				(972,0.975111)
				(1230,0.977359)
				(1408,0.978551)
				(1645,0.979854)
				(1938,0.981314)
				(1938,0.981314)
				(2296,0.98257)
				(2745,0.983922)
				(3400,0.98527)
				(3400,0.98527)
				(4256,0.986766)
				(4256,0.986766)
				(5568,0.988251)
			};

		\end{axis}
	\end{tikzpicture}
\end{subfigure}\\
\begin{subfigure}[b]{0.325\textwidth}\phantomsubcaption\label{subfig:appendix-experiments-nyuv2-rec.min[0]}
	%%%%%%%%%%%%%%%%%%%%%%%%%%%%%%%%%%%%%%%%%%%%%%%%%%%%%%%%%%%%
	% rec.min[0]
	%%%%%%%%%%%%%%%%%%%%%%%%%%%%%%%%%%%%%%%%%%%%%%%%%%%%%%%%%%%%
	\begin{tikzpicture}
		\begin{axis}[EQNYUV2RecMin,xmode=log]		
	
			% CCS rec.min[0] %%%%%%%%%%%%%%%%%%%%%%%%%%%%%%%%%%%%%%%%%%%%%%%%%%%%%%%%%%%%
			\addplot[CCS] coordinates{
				(194.01,0.376068)
				(282.682,0.432253)
				(397.098,0.490271)
				(597.153,0.553919)
				(803.471,0.618749)
				(925.103,0.635025)
				(1177.26,0.709856)
				(1397.83,0.73777)
				(1588.43,0.767049)
				(1878.46,0.803419)
				(1878.46,0.803419)
				(2231.81,0.855499)
				(2677.06,0.884979)
				(3328.01,0.933624)
				(3328.01,0.933624)
				(4313.98,0.96658)
				(4313.98,0.96658)
				(5573.99,0.987244)
			};
	
			% SEEDS rec.min[0] %%%%%%%%%%%%%%%%%%%%%%%%%%%%%%%%%%%%%%%%%%%%%%%%%%%%%%%%%%%%
			\addplot[SEEDS] coordinates{
				(246.043,0.67747)
				(347.734,0.785156)
				(450.607,0.806713)
				(654.489,0.832899)
				(857.654,0.836364)
				(1057.59,0.854545)
				(1258.91,0.88303)
				(1462.49,0.889495)
				(1661.42,0.921212)
				(1870.25,0.931713)
				(2066.14,0.929293)
				(2473.07,0.92)
				(2864.58,0.930303)
				(3245.7,0.949363)
				(3711.17,0.969697)
				(4314.51,0.958788)
				(4499.92,0.972323)
				(4922.37,0.98101)
			};
	
			% SLIC rec.min[0] %%%%%%%%%%%%%%%%%%%%%%%%%%%%%%%%%%%%%%%%%%%%%%%%%%%%%%%%%%%%
			\addplot[SLIC] coordinates{
				(184.484,0.533737)
				(279.316,0.553939)
				(385.211,0.604849)
				(596.942,0.650909)
				(819.424,0.713333)
				(939.241,0.74)
				(1211.08,0.802222)
				(1425.96,0.850059)
				(1642.44,0.830303)
				(1940.44,0.867475)
				(1940.44,0.867475)
				(2316.71,0.890909)
				(2774.73,0.91899)
				(3441.91,0.945455)
				(3441.91,0.945455)
				(4395.74,0.968081)
				(4395.74,0.968081)
				(5687.73,0.983636)
			};
	
			% RW rec.min[0] %%%%%%%%%%%%%%%%%%%%%%%%%%%%%%%%%%%%%%%%%%%%%%%%%%%%%%%%%%%%
			\addplot[RW] coordinates{
				(215.912,0.352015)
				(306.055,0.416566)
				(440.393,0.485671)
				(635.679,0.565993)
				(846.511,0.645701)
				(1060.11,0.651515)
				(1306.99,0.717778)
				(1464.51,0.745948)
				(1685.01,0.759257)
				(1868.17,0.803964)
				(2109.97,0.814747)
				(3967.51,0.935585)
				(4401.89,0.942626)
				(5110.46,0.962728)
			};
	
			% CW rec.min[0] %%%%%%%%%%%%%%%%%%%%%%%%%%%%%%%%%%%%%%%%%%%%%%%%%%%%%%%%%%%%
			\addplot[CW] coordinates{
				(198.614,0.448642)
				(293.173,0.516797)
				(407.073,0.597199)
				(613.509,0.658688)
				(834.84,0.669899)
				(1056.32,0.737172)
				(1266,0.78287)
				(1460.01,0.808687)
				(1700.24,0.833939)
				(1828.56,0.839596)
				(2009.84,0.873737)
				(2449.61,0.878788)
				(2991.17,0.913131)
				(3233.9,0.929591)
				(3707.85,0.951594)
				(4124.98,0.960932)
				(5174.47,0.976387)
				(5281.06,0.972781)
			};
	
			% TP rec.min[0] %%%%%%%%%%%%%%%%%%%%%%%%%%%%%%%%%%%%%%%%%%%%%%%%%%%%%%%%%%%%
			\addplot[TP] coordinates{
				(295.99,0.455833)
				(395.489,0.450101)
				(551.496,0.517576)
				(775.682,0.602222)
				(1000.76,0.656392)
				(1130.49,0.670909)
				(1401.47,0.746136)
				(1572.01,0.750909)
				(1815.33,0.790101)
				(2110.15,0.830303)
			};
	
			% POISE rec.min[0] %%%%%%%%%%%%%%%%%%%%%%%%%%%%%%%%%%%%%%%%%%%%%%%%%%%%%%%%%%%%
			\addplot[POISE] coordinates{
				(204.942,0.573939)
				(306.83,0.617576)
				(408.614,0.643232)
				(612.1,0.66404)
				(815.501,0.688889)
				(1019.01,0.72101)
				(1222.5,0.760202)
				(1425.91,0.779596)
				(1628.65,0.789822)
				(1830.61,0.794127)
				(2031.68,0.796664)
				(2427.44,0.799431)
				(2807.4,0.807811)
				(3158.64,0.811116)
				(3466.96,0.812961)
				(3736.5,0.812961)
				(4042.51,0.812961)
				(4219.03,0.812961)
			};
	
			% FH rec.min[0] %%%%%%%%%%%%%%%%%%%%%%%%%%%%%%%%%%%%%%%%%%%%%%%%%%%%%%%%%%%%
			\addplot[FH] coordinates{
				(689.802,0.646117)
				(763.792,0.661848)
				(813.815,0.66403)
				(988.712,0.695126)
				(1077.66,0.719767)
				(1359.76,0.747071)
				(1923.5,0.792323)
				(2699.96,0.834698)
				(1559.12,0.756076)
				(1408.18,0.726061)
				(3024.27,0.82024)
				(2206.32,0.80101)
				(2448.19,0.834747)
				(4594.6,0.87495)
				(3568.29,0.826241)
				(4199.72,0.842222)
			};
	
			% EAMS rec.min[0] %%%%%%%%%%%%%%%%%%%%%%%%%%%%%%%%%%%%%%%%%%%%%%%%%%%%%%%%%%%%
			\addplot[EAMS] coordinates{
				(419.068,0.615758)
				(455.657,0.618788)
				(499.416,0.628485)
				(641.188,0.632727)
				(842.328,0.676566)
				(1268.71,0.648485)
				(2584.6,0.719596)
				(2683.21,0.726465)
				(2779.09,0.716566)
				(2997.7,0.720606)
				(3180.29,0.691111)
				(3421.45,0.672323)
				(3646.16,0.654545)
				(3943.04,0.648081)
				(4448.4,0.655151)
				(5663.91,0.659192)
				(9098.39,0.666263)
			};
	
			% CRS rec.min[0] %%%%%%%%%%%%%%%%%%%%%%%%%%%%%%%%%%%%%%%%%%%%%%%%%%%%%%%%%%%%
			\addplot[CRS] coordinates{
				(254.263,0.572121)
				(396.694,0.646208)
				(514.895,0.698491)
				(764.997,0.735354)
				(988.589,0.785253)
				(1232.03,0.831515)
				(1498.25,0.86505)
				(1707.65,0.88505)
				(1968.19,0.884646)
				(2099.81,0.914747)
				(2299.23,0.925657)
				(2706.1,0.939798)
				(3259.07,0.931111)
				(3558.75,0.954343)
				(4055.02,0.967677)
				(4394.15,0.973535)
				(5418.11,0.984848)
				(5695.52,0.989091)
			};
	
			% SEAW rec.min[0] %%%%%%%%%%%%%%%%%%%%%%%%%%%%%%%%%%%%%%%%%%%%%%%%%%%%%%%%%%%%
			\addplot[SEAW] coordinates{
				(124.667,0.216539)
				(349.414,0.411802)
				(1192.36,0.646299)
				(4497.65,0.949354)
			};
	
			% RESEEDS rec.min[0] %%%%%%%%%%%%%%%%%%%%%%%%%%%%%%%%%%%%%%%%%%%%%%%%%%%%%%%%%%%%
			\addplot[RESEEDS] coordinates{
				(200.441,0.695152)
				(300.263,0.710101)
				(400.323,0.752929)
				(600.361,0.786061)
				(800.526,0.786061)
				(1000.06,0.83596)
				(1200.06,0.869495)
				(1400.96,0.845051)
				(1600.21,0.872727)
				(1792.08,0.897374)
				(1998.13,0.895354)
				(2408.17,0.905657)
				(2801.44,0.927273)
				(3182.16,0.958182)
				(3648.12,0.948485)
				(4257.44,0.950909)
				(4444.16,0.962828)
				(4864.15,0.966869)
			};
	
			% ERGC rec.min[0] %%%%%%%%%%%%%%%%%%%%%%%%%%%%%%%%%%%%%%%%%%%%%%%%%%%%%%%%%%%%
			\addplot[ERGC] coordinates{
				(196,0.470303)
				(289,0.552121)
				(400,0.618788)
				(600,0.701818)
				(812,0.709091)
				(1024,0.775151)
				(1224,0.823838)
				(1406,0.838614)
				(1681,0.871522)
				(1763,0.87525)
				(1935,0.899618)
				(2350,0.91596)
				(2856,0.932525)
				(3080,0.953939)
				(3520,0.966263)
				(3904,0.969907)
				(4864,0.985859)
				(5100,0.988687)
			};
	
			% PF rec.min[0] %%%%%%%%%%%%%%%%%%%%%%%%%%%%%%%%%%%%%%%%%%%%%%%%%%%%%%%%%%%%
			\addplot[PF] coordinates{
				(281.19,0.281803)
				(430.376,0.345055)
				(586.371,0.357328)
				(907.19,0.425773)
				(1201.57,0.501298)
				(1354.07,0.509323)
				(1718.72,0.537645)
				(1939.43,0.548265)
				(2251.38,0.58532)
				(2592.38,0.597357)
				(3186.03,0.672274)
				(4180.57,0.724242)
				(6182.51,0.773102)
			};
	
			% TPS rec.min[0] %%%%%%%%%%%%%%%%%%%%%%%%%%%%%%%%%%%%%%%%%%%%%%%%%%%%%%%%%%%%
			\addplot[TPS] coordinates{
				(230.419,0.379192)
				(313.739,0.429645)
				(450.436,0.475585)
				(638.008,0.542329)
				(857.559,0.612727)
				(1043.09,0.660503)
				(1317.64,0.680404)
				(1503.8,0.730909)
				(1703.97,0.741616)
				(1873.98,0.750909)
				(2144.93,0.757007)
				(2580.57,0.834654)
				(2894.48,0.847899)
				(3314.47,0.866061)
				(3924.13,0.91596)
				(4380.02,0.925657)
				(5132.34,0.940634)
				(5738.91,0.952994)
			};
	
			% NC rec.min[0] %%%%%%%%%%%%%%%%%%%%%%%%%%%%%%%%%%%%%%%%%%%%%%%%%%%%%%%%%%%%
			\addplot[NC] coordinates{
				(439.952,0.547273)
				(1154.47,0.741862)
				(2631.49,0.86527)
				(3874.82,0.952361)
			};
	
			% VC rec.min[0] %%%%%%%%%%%%%%%%%%%%%%%%%%%%%%%%%%%%%%%%%%%%%%%%%%%%%%%%%%%%
			\addplot[VC] coordinates{
				(243.744,0.37625)
				(400.291,0.481633)
				(531.672,0.517912)
				(660.84,0.563375)
				(895.915,0.610101)
				(1125.96,0.644444)
				(1348.09,0.691313)
				(1564.34,0.714949)
				(1781.1,0.729293)
				(1995.21,0.765685)
				(2205.86,0.786061)
				(2416.36,0.80651)
				(2820.76,0.835879)
				(3224.25,0.867475)
				(3610.4,0.885525)
				(3988.04,0.892253)
				(4351.59,0.879798)
				(4847.23,0.914141)
				(5262.69,0.925943)
			};
	
			% PB rec.min[0] %%%%%%%%%%%%%%%%%%%%%%%%%%%%%%%%%%%%%%%%%%%%%%%%%%%%%%%%%%%%
			\addplot[PB] coordinates{
				(273.248,0.415477)
				(360.188,0.435711)
				(463.193,0.542506)
				(656.962,0.59473)
				(857.526,0.635076)
				(970.915,0.687355)
				(1217.27,0.735202)
				(1387.49,0.760411)
				(1616.5,0.789325)
				(1896.95,0.807879)
				(1896.95,0.807879)
				(2243.04,0.842424)
				(2681.85,0.877778)
				(3316.29,0.919987)
				(3316.29,0.919987)
				(4151.92,0.953208)
				(4151.92,0.953208)
				(5425.24,0.972727)
			};
	
			% VCCS rec.min[0] %%%%%%%%%%%%%%%%%%%%%%%%%%%%%%%%%%%%%%%%%%%%%%%%%%%%%%%%%%%%
			\addplot[VCCS] coordinates{
				(564.123,0.484724)
				(600.283,0.481542)
				(648.629,0.493453)
				(755.632,0.506036)
				(836.05,0.537007)
				(1051.64,0.530839)
				(1109.07,0.523883)
				(1179.74,0.492844)
				(1237.23,0.476921)
				(1322.17,0.494913)
				(1400.72,0.504972)
				(1526.22,0.532333)
				(1630.95,0.519687)
				(1756.59,0.519342)
				(1914.09,0.512272)
				(2024.07,0.504225)
				(2258.58,0.515836)
				(2558.8,0.552222)
				(2780.39,0.537909)
				(3044.22,0.569083)
				(3687.7,0.687652)
				(4171.76,0.732899)
				(4255.91,0.642542)
			};
	
			% PRESLIC rec.min[0] %%%%%%%%%%%%%%%%%%%%%%%%%%%%%%%%%%%%%%%%%%%%%%%%%%%%%%%%%%%%
			\addplot[PRESLIC] coordinates{
				(189.89,0.538371)
				(388.539,0.627071)
				(589.772,0.681616)
				(798.767,0.782142)
				(920.201,0.81495)
				(1188.65,0.808081)
				(1401.21,0.871919)
				(1612.25,0.859394)
				(1904.93,0.888081)
				(1904.93,0.888081)
				(2282.12,0.920404)
				(2738.74,0.945084)
				(3422.72,0.954949)
				(3422.72,0.954949)
				(4393.15,0.973333)
				(4393.15,0.973333)
				(5724.22,0.983361)
			};
	
			% W rec.min[0] %%%%%%%%%%%%%%%%%%%%%%%%%%%%%%%%%%%%%%%%%%%%%%%%%%%%%%%%%%%%
			\addplot[W] coordinates{
				(193.87,0.440202)
				(303.997,0.514545)
				(397.103,0.584444)
				(621.108,0.62404)
				(870.236,0.723636)
				(959.915,0.739554)
				(1234.38,0.764646)
				(1418.74,0.818586)
				(1652.83,0.808485)
				(1957.01,0.841212)
				(1957.01,0.841212)
				(2345.3,0.894847)
				(2867.42,0.912525)
				(3518.42,0.944646)
				(3518.42,0.944646)
				(4497.46,0.96)
				(4497.46,0.96)
				(5940.33,0.988446)
			};
	
			% LSC rec.min[0] %%%%%%%%%%%%%%%%%%%%%%%%%%%%%%%%%%%%%%%%%%%%%%%%%%%%%%%%%%%%
			\addplot[LSC] coordinates{
				(383.095,0.588485)
				(552.409,0.635354)
				(718.163,0.667677)
				(1059.33,0.747071)
				(1414.95,0.792525)
				(1816.61,0.857374)
				(2009.32,0.828687)
				(2263.42,0.854141)
				(2398.35,0.887879)
				(2446.8,0.868889)
				(2588.34,0.879798)
				(2830.36,0.905253)
				(3264.3,0.933326)
				(3438.52,0.93596)
				(3802.19,0.958788)
				(4059.6,0.962222)
				(4900.83,0.977851)
				(5001.09,0.981145)
			};
	
			% WP rec.min[0] %%%%%%%%%%%%%%%%%%%%%%%%%%%%%%%%%%%%%%%%%%%%%%%%%%%%%%%%%%%%
			\addplot[WP] coordinates{
				(204,0.423103)
				(315,0.507879)
				(432,0.551515)
				(638,0.61495)
				(850,0.673939)
				(1064,0.715758)
				(1230,0.752727)
				(1408,0.779394)
				(1645,0.807677)
				(1938,0.836566)
				(1938,0.834949)
				(2296,0.867273)
				(2745,0.899798)
				(3400,0.935709)
				(3400,0.935709)
				(4256,0.955151)
				(4256,0.955151)
				(5568,0.982771)
			};
	
			% QS rec.min[0] %%%%%%%%%%%%%%%%%%%%%%%%%%%%%%%%%%%%%%%%%%%%%%%%%%%%%%%%%%%%
			\addplot[QS] coordinates{
				(223.521,0.231874)
				(325.303,0.308254)
				(414.609,0.365025)
				(663.504,0.451991)
				(828.486,0.473995)
				(1077.18,0.523186)
				(1252.44,0.546281)
				(1475.56,0.572831)
				(1767.37,0.598563)
				(2165.34,0.637298)
				(2703.49,0.680124)
				(3460.91,0.728678)
				(4543.63,0.765788)
				(6144.37,0.809632)
			};
	
			% CIS rec.min[0] %%%%%%%%%%%%%%%%%%%%%%%%%%%%%%%%%%%%%%%%%%%%%%%%%%%%%%%%%%%%
			\addplot[CIS] coordinates{
				(292.103,0.318694)
				(366.787,0.369697)
				(436.398,0.378182)
				(575.972,0.417172)
				(721.491,0.466263)
				(783.672,0.514545)
				(963.875,0.568283)
				(1083.42,0.576559)
				(1227.51,0.613475)
				(1408.35,0.631919)
				(1424.97,0.641844)
				(1751.64,0.668283)
				(2118.16,0.712675)
				(2717.81,0.753131)
				(3240.8,0.814545)
				(3240.8,0.814545)
				(4824.68,0.84404)
			};
	
			% RESEEDS3D rec.min[0] %%%%%%%%%%%%%%%%%%%%%%%%%%%%%%%%%%%%%%%%%%%%%%%%%%%%%%%%%%%%
			\addplot[RESEEDS3D] coordinates{
				(200.035,0.453849)
				(300.105,0.631717)
				(400.128,0.690909)
				(600.233,0.769293)
				(800.351,0.786061)
				(1000.04,0.805657)
				(1200.03,0.82303)
				(1400.65,0.854545)
				(1600.08,0.85798)
				(1792.08,0.905455)
				(1998.07,0.903838)
				(2408.16,0.916768)
				(2801.12,0.929495)
				(3182.14,0.956566)
				(3648.05,0.962626)
				(4257.09,0.976566)
				(4444.11,0.980808)
				(4864.04,0.98505)
			};
	
			% ERS rec.min[0] %%%%%%%%%%%%%%%%%%%%%%%%%%%%%%%%%%%%%%%%%%%%%%%%%%%%%%%%%%%%
			\addplot[ERS] coordinates{
				(200,0.539394)
				(300,0.615748)
				(400,0.6503)
				(600,0.716364)
				(800,0.786416)
				(1000,0.841414)
				(1200,0.856736)
				(1400,0.888314)
				(1600,0.899961)
				(1800,0.91274)
				(2000,0.928972)
				(2400,0.95256)
				(2800,0.960824)
				(3200,0.972121)
				(3600,0.979715)
				(4000,0.984477)
				(4600,0.986742)
				(5200,0.991606)
			};
	
			% DASP rec.min[0] %%%%%%%%%%%%%%%%%%%%%%%%%%%%%%%%%%%%%%%%%%%%%%%%%%%%%%%%%%%%
			\addplot[DASP] coordinates{
				(417.987,0.468722)
				(521.05,0.505365)
				(620.704,0.530187)
				(812.654,0.587561)
				(1006.72,0.643753)
				(1197.12,0.665576)
				(1388.06,0.687943)
				(1576.93,0.703855)
				(1767.2,0.741044)
				(1947.7,0.750591)
				(2131.65,0.764684)
				(2494.37,0.78487)
				(2860.93,0.82615)
				(3211.97,0.824604)
				(3564.56,0.855849)
				(3909.24,0.860156)
				(4418.09,0.877793)
				(4919.67,0.895981)
			};
	
			% MSS rec.min[0] %%%%%%%%%%%%%%%%%%%%%%%%%%%%%%%%%%%%%%%%%%%%%%%%%%%%%%%%%%%%
			\addplot[MSS] coordinates{
				(207.576,0.452173)
				(308.962,0.48918)
				(436.368,0.542099)
				(668.784,0.608111)
				(912.997,0.672213)
				(1055.23,0.708583)
				(1363.6,0.741818)
				(1575.7,0.767503)
				(1862.23,0.799506)
				(2222.47,0.826059)
				(2222.47,0.826059)
				(2666.45,0.868158)
				(3236.21,0.892981)
				(4079.47,0.931201)
				(4078.85,0.929637)
				(5209.16,0.940828)
				(5209.16,0.940828)
				(6994.34,0.963652)
			};
	
			% ETPS rec.min[0] %%%%%%%%%%%%%%%%%%%%%%%%%%%%%%%%%%%%%%%%%%%%%%%%%%%%%%%%%%%%
			\addplot[ETPS] coordinates{
				(221,0.613131)
				(315,0.673535)
				(432,0.711111)
				(638,0.762626)
				(850,0.807879)
				(972,0.833131)
				(1230,0.868485)
				(1408,0.872323)
				(1645,0.89697)
				(1938,0.920606)
				(1938,0.920606)
				(2296,0.943232)
				(2745,0.948485)
				(3400,0.958788)
				(3400,0.958788)
				(4256,0.971919)
				(4256,0.971919)
				(5568,0.989437)
			};

			\end{axis}
	\end{tikzpicture}
\end{subfigure}
\begin{subfigure}[b]{0.325\textwidth}\phantomsubcaption\label{subfig:appendix-experiments-nyuv2-ue_np.max[0]}
	%%%%%%%%%%%%%%%%%%%%%%%%%%%%%%%%%%%%%%%%%%%%%%%%%%%%%%%%%%%%
	% ue_np.max[0]
	%%%%%%%%%%%%%%%%%%%%%%%%%%%%%%%%%%%%%%%%%%%%%%%%%%%%%%%%%%%%
	\begin{tikzpicture}
		\begin{axis}[EQNYUV2UEMax,xmode=log]		
	
			% CCS ue_np.max[0] %%%%%%%%%%%%%%%%%%%%%%%%%%%%%%%%%%%%%%%%%%%%%%%%%%%%%%%%%%%%
			\addplot[CCS] coordinates{
				(194.01,0.41226)
				(282.682,0.370627)
				(397.098,0.302716)
				(597.153,0.252603)
				(803.471,0.230807)
				(925.103,0.225817)
				(1177.26,0.206337)
				(1397.83,0.190048)
				(1588.43,0.180088)
				(1878.46,0.171857)
				(1878.46,0.171857)
				(2231.81,0.159444)
				(2677.06,0.15484)
				(3328.01,0.138852)
				(3328.01,0.138852)
				(4313.98,0.128029)
				(4313.98,0.128029)
				(5573.99,0.114504)
			};
	
			% SEEDS ue_np.max[0] %%%%%%%%%%%%%%%%%%%%%%%%%%%%%%%%%%%%%%%%%%%%%%%%%%%%%%%%%%%%
			\addplot[SEEDS] coordinates{
				(246.043,0.534675)
				(347.734,0.497705)
				(450.607,0.442493)
				(654.489,0.381017)
				(857.654,0.318143)
				(1057.59,0.265878)
				(1258.91,0.248825)
				(1462.49,0.249108)
				(1661.42,0.219058)
				(1870.25,0.263088)
				(2066.14,0.222752)
				(2473.07,0.190228)
				(2864.58,0.186744)
				(3245.7,0.175796)
				(3711.17,0.17316)
				(4314.51,0.155428)
				(4499.92,0.150736)
				(4922.37,0.152773)
			};
	
			% SLIC ue_np.max[0] %%%%%%%%%%%%%%%%%%%%%%%%%%%%%%%%%%%%%%%%%%%%%%%%%%%%%%%%%%%%
			\addplot[SLIC] coordinates{
				(184.484,0.399183)
				(279.316,0.339726)
				(385.211,0.318396)
				(596.942,0.273412)
				(819.424,0.233978)
				(939.241,0.231563)
				(1211.08,0.205757)
				(1425.96,0.204035)
				(1642.44,0.196932)
				(1940.44,0.185048)
				(1940.44,0.185048)
				(2316.71,0.16744)
				(2774.73,0.158144)
				(3441.91,0.140151)
				(3441.91,0.140151)
				(4395.74,0.128767)
				(4395.74,0.128767)
				(5687.73,0.119185)
			};
	
			% RW ue_np.max[0] %%%%%%%%%%%%%%%%%%%%%%%%%%%%%%%%%%%%%%%%%%%%%%%%%%%%%%%%%%%%
			\addplot[RW] coordinates{
				(215.912,0.459865)
				(306.055,0.439483)
				(440.393,0.349253)
				(635.679,0.32876)
				(846.511,0.28299)
				(1060.11,0.257111)
				(1306.99,0.229489)
				(1464.51,0.23363)
				(1685.01,0.218522)
				(1868.17,0.214631)
				(2109.97,0.208547)
				(3967.51,0.161691)
				(4401.89,0.158956)
				(5110.46,0.154598)
			};
	
			% CW ue_np.max[0] %%%%%%%%%%%%%%%%%%%%%%%%%%%%%%%%%%%%%%%%%%%%%%%%%%%%%%%%%%%%
			\addplot[CW] coordinates{
				(198.614,0.444736)
				(293.173,0.370716)
				(407.073,0.338067)
				(613.509,0.313998)
				(834.84,0.283126)
				(1056.32,0.237932)
				(1266,0.234305)
				(1460.01,0.226445)
				(1700.24,0.214436)
				(1828.56,0.212098)
				(2009.84,0.203966)
				(2449.61,0.194226)
				(2991.17,0.175796)
				(3233.9,0.175678)
				(3707.85,0.171009)
				(4124.98,0.154183)
				(5174.47,0.146572)
				(5281.06,0.143474)
			};
	
			% TP ue_np.max[0] %%%%%%%%%%%%%%%%%%%%%%%%%%%%%%%%%%%%%%%%%%%%%%%%%%%%%%%%%%%%
			\addplot[TP] coordinates{
				(295.99,0.383727)
				(395.489,0.324068)
				(551.496,0.294878)
				(775.682,0.270288)
				(1000.76,0.230957)
				(1130.49,0.232583)
				(1401.47,0.211881)
				(1572.01,0.207674)
				(1815.33,0.194189)
				(2110.15,0.18645)
			};
	
			% POISE ue_np.max[0] %%%%%%%%%%%%%%%%%%%%%%%%%%%%%%%%%%%%%%%%%%%%%%%%%%%%%%%%%%%%
			\addplot[POISE] coordinates{
				(204.942,0.520904)
				(306.83,0.501696)
				(408.614,0.489746)
				(612.1,0.47766)
				(815.501,0.465934)
				(1019.01,0.464557)
				(1222.5,0.45818)
				(1425.91,0.458819)
				(1628.65,0.451396)
				(1830.61,0.450375)
				(2031.68,0.447442)
				(2427.44,0.447442)
				(2807.4,0.447442)
				(3158.64,0.447442)
				(3466.96,0.447442)
				(3736.5,0.447442)
				(4042.51,0.447442)
				(4219.03,0.447442)
			};
	
			% FH ue_np.max[0] %%%%%%%%%%%%%%%%%%%%%%%%%%%%%%%%%%%%%%%%%%%%%%%%%%%%%%%%%%%%
			\addplot[FH] coordinates{
				(689.802,0.343669)
				(763.792,0.341239)
				(813.815,0.336734)
				(988.712,0.313227)
				(1077.66,0.296251)
				(1359.76,0.272329)
				(1923.5,0.253884)
				(2699.96,0.243212)
				(1559.12,0.298193)
				(1408.18,0.298924)
				(3024.27,0.253972)
				(2206.32,0.269289)
				(2448.19,0.262886)
				(4594.6,0.220586)
				(3568.29,0.25105)
				(4199.72,0.2378)
			};
	
			% EAMS ue_np.max[0] %%%%%%%%%%%%%%%%%%%%%%%%%%%%%%%%%%%%%%%%%%%%%%%%%%%%%%%%%%%%
			\addplot[EAMS] coordinates{
				(419.068,0.334014)
				(455.657,0.330497)
				(499.416,0.32662)
				(641.188,0.312533)
				(842.328,0.3029)
				(1268.71,0.283199)
				(2584.6,0.262581)
				(2683.21,0.26228)
				(2779.09,0.265908)
				(2997.7,0.263767)
				(3180.29,0.26818)
				(3421.45,0.268988)
				(3646.16,0.27198)
				(3943.04,0.278676)
				(4448.4,0.275857)
				(5663.91,0.270805)
				(9098.39,0.262894)
			};
	
			% CRS ue_np.max[0] %%%%%%%%%%%%%%%%%%%%%%%%%%%%%%%%%%%%%%%%%%%%%%%%%%%%%%%%%%%%
			\addplot[CRS] coordinates{
				(254.263,0.36736)
				(396.694,0.306035)
				(514.895,0.263892)
				(764.997,0.229933)
				(988.589,0.204876)
				(1232.03,0.191527)
				(1498.25,0.181887)
				(1707.65,0.173325)
				(1968.19,0.167201)
				(2099.81,0.161779)
				(2299.23,0.15426)
				(2706.1,0.146837)
				(3259.07,0.136139)
				(3558.75,0.134861)
				(4055.02,0.125323)
				(4394.15,0.122048)
				(5418.11,0.112385)
				(5695.52,0.109955)
			};
	
			% SEAW ue_np.max[0] %%%%%%%%%%%%%%%%%%%%%%%%%%%%%%%%%%%%%%%%%%%%%%%%%%%%%%%%%%%%
			\addplot[SEAW] coordinates{
				(124.667,0.631487)
				(349.414,0.408188)
				(1192.36,0.249203)
				(4497.65,0.145915)
			};
	
			% RESEEDS ue_np.max[0] %%%%%%%%%%%%%%%%%%%%%%%%%%%%%%%%%%%%%%%%%%%%%%%%%%%%%%%%%%%%
			\addplot[RESEEDS] coordinates{
				(200.441,0.416375)
				(300.263,0.436601)
				(400.323,0.372305)
				(600.361,0.278845)
				(800.526,0.255698)
				(1000.06,0.231677)
				(1200.06,0.223159)
				(1400.96,0.199582)
				(1600.21,0.199013)
				(1792.08,0.20868)
				(1998.13,0.180389)
				(2408.17,0.164525)
				(2801.44,0.155457)
				(3182.16,0.153243)
				(3648.12,0.15288)
				(4257.44,0.134659)
				(4444.16,0.132449)
				(4864.15,0.134534)
			};
	
			% ERGC ue_np.max[0] %%%%%%%%%%%%%%%%%%%%%%%%%%%%%%%%%%%%%%%%%%%%%%%%%%%%%%%%%%%%
			\addplot[ERGC] coordinates{
				(196,0.411074)
				(289,0.358527)
				(400,0.297984)
				(600,0.256249)
				(812,0.236603)
				(1024,0.211683)
				(1224,0.208397)
				(1406,0.193095)
				(1681,0.187199)
				(1763,0.179364)
				(1935,0.177988)
				(2350,0.166082)
				(2856,0.15882)
				(3080,0.150901)
				(3520,0.143514)
				(3904,0.137302)
				(4864,0.129325)
				(5100,0.122988)
			};
	
			% PF ue_np.max[0] %%%%%%%%%%%%%%%%%%%%%%%%%%%%%%%%%%%%%%%%%%%%%%%%%%%%%%%%%%%%
			\addplot[PF] coordinates{
				(281.19,0.668956)
				(430.376,0.582424)
				(586.371,0.550958)
				(907.19,0.477712)
				(1201.57,0.432808)
				(1354.07,0.417407)
				(1718.72,0.413673)
				(1939.43,0.399172)
				(2251.38,0.370884)
				(2592.38,0.351078)
				(3186.03,0.345861)
				(4180.57,0.330027)
				(6182.51,0.292165)
			};
	
			% TPS ue_np.max[0] %%%%%%%%%%%%%%%%%%%%%%%%%%%%%%%%%%%%%%%%%%%%%%%%%%%%%%%%%%%%
			\addplot[TPS] coordinates{
				(230.419,0.428002)
				(313.739,0.386084)
				(450.436,0.317985)
				(638.008,0.29823)
				(857.559,0.255419)
				(1043.09,0.224639)
				(1317.64,0.215978)
				(1503.8,0.204035)
				(1703.97,0.194923)
				(1873.98,0.186277)
				(2144.93,0.178182)
				(2580.57,0.174963)
				(2894.48,0.162726)
				(3314.47,0.160608)
				(3924.13,0.144307)
				(4380.02,0.139245)
				(5132.34,0.131858)
				(5738.91,0.130511)
			};
	
			% NC ue_np.max[0] %%%%%%%%%%%%%%%%%%%%%%%%%%%%%%%%%%%%%%%%%%%%%%%%%%%%%%%%%%%%
			\addplot[NC] coordinates{
				(439.952,0.301057)
				(1154.47,0.236211)
				(2631.49,0.172951)
				(3874.82,0.147905)
			};
	
			% VC ue_np.max[0] %%%%%%%%%%%%%%%%%%%%%%%%%%%%%%%%%%%%%%%%%%%%%%%%%%%%%%%%%%%%
			\addplot[VC] coordinates{
				(243.744,0.479375)
				(400.291,0.374739)
				(531.672,0.310352)
				(660.84,0.275225)
				(895.915,0.237217)
				(1125.96,0.214598)
				(1348.09,0.20372)
				(1564.34,0.187647)
				(1781.1,0.177958)
				(1995.21,0.172701)
				(2205.86,0.164907)
				(2416.36,0.164345)
				(2820.76,0.151811)
				(3224.25,0.149957)
				(3610.4,0.141319)
				(3988.04,0.137442)
				(4351.59,0.131149)
				(4847.23,0.127698)
				(5262.69,0.124203)
			};
	
			% PB ue_np.max[0] %%%%%%%%%%%%%%%%%%%%%%%%%%%%%%%%%%%%%%%%%%%%%%%%%%%%%%%%%%%%
			\addplot[PB] coordinates{
				(273.248,0.487202)
				(360.188,0.417892)
				(463.193,0.35327)
				(656.962,0.31341)
				(857.526,0.273904)
				(970.915,0.258536)
				(1217.27,0.23208)
				(1387.49,0.22577)
				(1616.5,0.206202)
				(1896.95,0.199259)
				(1896.95,0.199259)
				(2243.04,0.185569)
				(2681.85,0.178579)
				(3316.29,0.162638)
				(3316.29,0.162638)
				(4151.92,0.149021)
				(4151.92,0.149021)
				(5425.24,0.137314)
			};
	
			% VCCS ue_np.max[0] %%%%%%%%%%%%%%%%%%%%%%%%%%%%%%%%%%%%%%%%%%%%%%%%%%%%%%%%%%%%
			\addplot[VCCS] coordinates{
				(564.123,0.521815)
				(600.283,0.511219)
				(648.629,0.515559)
				(755.632,0.511208)
				(836.05,0.491754)
				(1051.64,0.493065)
				(1109.07,0.525332)
				(1179.74,0.471008)
				(1237.23,0.527057)
				(1322.17,0.498458)
				(1400.72,0.506289)
				(1526.22,0.486053)
				(1630.95,0.496567)
				(1756.59,0.486156)
				(1914.09,0.504868)
				(2024.07,0.498906)
				(2258.58,0.478086)
				(2558.8,0.445078)
				(2780.39,0.482242)
				(3044.22,0.474521)
				(3687.7,0.336015)
				(4171.76,0.289158)
				(4255.91,0.400211)
			};
	
			% PRESLIC ue_np.max[0] %%%%%%%%%%%%%%%%%%%%%%%%%%%%%%%%%%%%%%%%%%%%%%%%%%%%%%%%%%%%
			\addplot[PRESLIC] coordinates{
				(189.89,0.399994)
				(388.539,0.318734)
				(589.772,0.271536)
				(798.767,0.256832)
				(920.201,0.231636)
				(1188.65,0.210717)
				(1401.21,0.204795)
				(1612.25,0.190492)
				(1904.93,0.183451)
				(1904.93,0.183451)
				(2282.12,0.173476)
				(2738.74,0.15806)
				(3422.72,0.149822)
				(3422.72,0.149822)
				(4393.15,0.135199)
				(4393.15,0.135199)
				(5724.22,0.123653)
			};
	
			% W ue_np.max[0] %%%%%%%%%%%%%%%%%%%%%%%%%%%%%%%%%%%%%%%%%%%%%%%%%%%%%%%%%%%%
			\addplot[W] coordinates{
				(193.87,0.451102)
				(303.997,0.385342)
				(397.103,0.385228)
				(621.108,0.306758)
				(870.236,0.291996)
				(959.915,0.273907)
				(1234.38,0.250786)
				(1418.74,0.233476)
				(1652.83,0.227469)
				(1957.01,0.20876)
				(1957.01,0.20876)
				(2345.3,0.196619)
				(2867.42,0.186571)
				(3518.42,0.173509)
				(3518.42,0.173509)
				(4497.46,0.158273)
				(4497.46,0.158273)
				(5940.33,0.140687)
			};
	
			% LSC ue_np.max[0] %%%%%%%%%%%%%%%%%%%%%%%%%%%%%%%%%%%%%%%%%%%%%%%%%%%%%%%%%%%%
			\addplot[LSC] coordinates{
				(383.095,0.333918)
				(552.409,0.299177)
				(718.163,0.254898)
				(1059.33,0.230634)
				(1414.95,0.203613)
				(1816.61,0.183847)
				(2009.32,0.17928)
				(2263.42,0.170175)
				(2398.35,0.171199)
				(2446.8,0.163299)
				(2588.34,0.15955)
				(2830.36,0.149565)
				(3264.3,0.142464)
				(3438.52,0.13735)
				(3802.19,0.129828)
				(4059.6,0.132115)
				(4900.83,0.122944)
				(5001.09,0.122063)
			};
	
			% WP ue_np.max[0] %%%%%%%%%%%%%%%%%%%%%%%%%%%%%%%%%%%%%%%%%%%%%%%%%%%%%%%%%%%%
			\addplot[WP] coordinates{
				(204,0.394204)
				(315,0.373319)
				(432,0.325886)
				(638,0.284921)
				(850,0.271304)
				(1064,0.247037)
				(1230,0.235921)
				(1408,0.226467)
				(1645,0.216657)
				(1938,0.204315)
				(1938,0.204469)
				(2296,0.192427)
				(2745,0.179122)
				(3400,0.168035)
				(3400,0.168035)
				(4256,0.155898)
				(4256,0.155898)
				(5568,0.140599)
			};
	
			% QS ue_np.max[0] %%%%%%%%%%%%%%%%%%%%%%%%%%%%%%%%%%%%%%%%%%%%%%%%%%%%%%%%%%%%
			\addplot[QS] coordinates{
				(223.521,0.731544)
				(325.303,0.617426)
				(414.609,0.560848)
				(663.504,0.424702)
				(828.486,0.383158)
				(1077.18,0.355109)
				(1252.44,0.317302)
				(1475.56,0.297191)
				(1767.37,0.281547)
				(2165.34,0.261678)
				(2703.49,0.244831)
				(3460.91,0.22353)
				(4543.63,0.199924)
				(6144.37,0.177015)
			};
	
			% CIS ue_np.max[0] %%%%%%%%%%%%%%%%%%%%%%%%%%%%%%%%%%%%%%%%%%%%%%%%%%%%%%%%%%%%
			\addplot[CIS] coordinates{
				(292.103,0.389586)
				(366.787,0.328988)
				(436.398,0.314758)
				(575.972,0.273386)
				(721.491,0.260327)
				(783.672,0.248216)
				(963.875,0.235109)
				(1083.42,0.22241)
				(1227.51,0.21679)
				(1408.35,0.210064)
				(1424.97,0.208724)
				(1751.64,0.198987)
				(2118.16,0.189916)
				(2717.81,0.18174)
				(3240.8,0.166419)
				(3240.8,0.166419)
				(4824.68,0.155931)
			};
	
			% RESEEDS3D ue_np.max[0] %%%%%%%%%%%%%%%%%%%%%%%%%%%%%%%%%%%%%%%%%%%%%%%%%%%%%%%%%%%%
			\addplot[RESEEDS3D] coordinates{
				(200.035,0.556336)
				(300.105,0.360311)
				(400.128,0.3159)
				(600.233,0.257717)
				(800.351,0.236574)
				(1000.04,0.220887)
				(1200.03,0.201587)
				(1400.65,0.191994)
				(1600.08,0.185705)
				(1792.08,0.193752)
				(1998.07,0.176563)
				(2408.16,0.16476)
				(2801.12,0.153177)
				(3182.14,0.1489)
				(3648.05,0.14263)
				(4257.09,0.135012)
				(4444.11,0.130702)
				(4864.04,0.129027)
			};
	
			% ERS ue_np.max[0] %%%%%%%%%%%%%%%%%%%%%%%%%%%%%%%%%%%%%%%%%%%%%%%%%%%%%%%%%%%%
			\addplot[ERS] coordinates{
				(200,0.355726)
				(300,0.297543)
				(400,0.281151)
				(600,0.243773)
				(800,0.223064)
				(1000,0.208401)
				(1200,0.19539)
				(1400,0.186094)
				(1600,0.18258)
				(1800,0.173835)
				(2000,0.168174)
				(2400,0.156547)
				(2800,0.149403)
				(3200,0.139891)
				(3600,0.137115)
				(4000,0.131234)
				(4600,0.123851)
				(5200,0.119625)
			};
	
			% DASP ue_np.max[0] %%%%%%%%%%%%%%%%%%%%%%%%%%%%%%%%%%%%%%%%%%%%%%%%%%%%%%%%%%%%
			\addplot[DASP] coordinates{
				(417.987,0.383018)
				(521.05,0.342226)
				(620.704,0.271011)
				(812.654,0.234933)
				(1006.72,0.225024)
				(1197.12,0.216804)
				(1388.06,0.207369)
				(1576.93,0.188917)
				(1767.2,0.176079)
				(1947.7,0.174357)
				(2131.65,0.16614)
				(2494.37,0.1581)
				(2860.93,0.151507)
				(3211.97,0.146973)
				(3564.56,0.14256)
				(3909.24,0.140339)
				(4418.09,0.131759)
				(4919.67,0.127115)
			};
	
			% MSS ue_np.max[0] %%%%%%%%%%%%%%%%%%%%%%%%%%%%%%%%%%%%%%%%%%%%%%%%%%%%%%%%%%%%
			\addplot[MSS] coordinates{
				(207.576,0.468673)
				(308.962,0.387589)
				(436.368,0.355182)
				(668.784,0.317526)
				(912.997,0.272791)
				(1055.23,0.25782)
				(1363.6,0.231908)
				(1575.7,0.218717)
				(1862.23,0.211011)
				(2222.47,0.20351)
				(2222.47,0.20351)
				(2666.45,0.193018)
				(3236.21,0.185396)
				(4079.47,0.170047)
				(4078.85,0.169397)
				(5209.16,0.153173)
				(5209.16,0.153173)
				(6994.34,0.136241)
			};
	
			% ETPS ue_np.max[0] %%%%%%%%%%%%%%%%%%%%%%%%%%%%%%%%%%%%%%%%%%%%%%%%%%%%%%%%%%%%
			\addplot[ETPS] coordinates{
				(221,0.395662)
				(315,0.350601)
				(432,0.305921)
				(638,0.259083)
				(850,0.236629)
				(972,0.225825)
				(1230,0.209693)
				(1408,0.203096)
				(1645,0.187933)
				(1938,0.176328)
				(1938,0.176328)
				(2296,0.167495)
				(2745,0.154642)
				(3400,0.144102)
				(3400,0.144102)
				(4256,0.131102)
				(4256,0.131102)
				(5568,0.118017)
			};

			\end{axis}
	\end{tikzpicture}
\end{subfigure}
\begin{subfigure}[b]{0.325\textwidth}\phantomsubcaption\label{subfig:appendix-experiments-nyuv2-ue_np.max[0]}
	%%%%%%%%%%%%%%%%%%%%%%%%%%%%%%%%%%%%%%%%%%%%%%%%%%%%%%%%%%%%
	% ev.min[0]
	%%%%%%%%%%%%%%%%%%%%%%%%%%%%%%%%%%%%%%%%%%%%%%%%%%%%%%%%%%%%
	\begin{tikzpicture}
		\begin{axis}[EQNYUV2EVMin,xmode=log]		
	
			% CCS ev.min[0] %%%%%%%%%%%%%%%%%%%%%%%%%%%%%%%%%%%%%%%%%%%%%%%%%%%%%%%%%%%%
			\addplot[CCS] coordinates{
				(194.01,0.64395)
				(282.682,0.726036)
				(397.098,0.756791)
				(597.153,0.784264)
				(803.471,0.807378)
				(925.103,0.813997)
				(1177.26,0.837619)
				(1397.83,0.839951)
				(1588.43,0.858953)
				(1878.46,0.872405)
				(1878.46,0.872405)
				(2231.81,0.880403)
				(2677.06,0.892475)
				(3328.01,0.905678)
				(3328.01,0.905678)
				(4313.98,0.921286)
				(4313.98,0.921286)
				(5573.99,0.932579)
			};
	
			% SEEDS ev.min[0] %%%%%%%%%%%%%%%%%%%%%%%%%%%%%%%%%%%%%%%%%%%%%%%%%%%%%%%%%%%%
			\addplot[SEEDS] coordinates{
				(246.043,0.637492)
				(347.734,0.651874)
				(450.607,0.689669)
				(654.489,0.732287)
				(857.654,0.756214)
				(1057.59,0.768704)
				(1258.91,0.785395)
				(1462.49,0.789703)
				(1661.42,0.805021)
				(1870.25,0.743421)
				(2066.14,0.794148)
				(2473.07,0.826257)
				(2864.58,0.835109)
				(3245.7,0.841456)
				(3711.17,0.858947)
				(4314.51,0.869249)
				(4499.92,0.882185)
				(4922.37,0.881443)
			};
	
			% SLIC ev.min[0] %%%%%%%%%%%%%%%%%%%%%%%%%%%%%%%%%%%%%%%%%%%%%%%%%%%%%%%%%%%%
			\addplot[SLIC] coordinates{
				(184.484,0.558146)
				(279.316,0.585645)
				(385.211,0.629177)
				(596.942,0.665338)
				(819.424,0.702426)
				(939.241,0.737292)
				(1211.08,0.754148)
				(1425.96,0.781185)
				(1642.44,0.805665)
				(1940.44,0.822878)
				(1940.44,0.822878)
				(2316.71,0.833024)
				(2774.73,0.858259)
				(3441.91,0.878847)
				(3441.91,0.878847)
				(4395.74,0.889564)
				(4395.74,0.889564)
				(5687.73,0.906389)
			};
	
			% RW ev.min[0] %%%%%%%%%%%%%%%%%%%%%%%%%%%%%%%%%%%%%%%%%%%%%%%%%%%%%%%%%%%%
			\addplot[RW] coordinates{
				(215.912,0.425844)
				(306.055,0.45454)
				(440.393,0.505747)
				(635.679,0.5255)
				(846.511,0.622222)
				(1060.11,0.608624)
				(1306.99,0.687057)
				(1464.51,0.674183)
				(1685.01,0.705741)
				(1868.17,0.707332)
				(2109.97,0.730607)
				(3967.51,0.801116)
				(4401.89,0.818306)
				(5110.46,0.826941)
			};
	
			% CW ev.min[0] %%%%%%%%%%%%%%%%%%%%%%%%%%%%%%%%%%%%%%%%%%%%%%%%%%%%%%%%%%%%
			\addplot[CW] coordinates{
				(198.614,0.483345)
				(293.173,0.505208)
				(407.073,0.610405)
				(613.509,0.651203)
				(834.84,0.646337)
				(1056.32,0.677838)
				(1266,0.713249)
				(1460.01,0.733192)
				(1700.24,0.738723)
				(1828.56,0.731214)
				(2009.84,0.775203)
				(2449.61,0.785118)
				(2991.17,0.799732)
				(3233.9,0.803622)
				(3707.85,0.804886)
				(4124.98,0.823163)
				(5174.47,0.820131)
				(5281.06,0.838877)
			};
	
			% TP ev.min[0] %%%%%%%%%%%%%%%%%%%%%%%%%%%%%%%%%%%%%%%%%%%%%%%%%%%%%%%%%%%%
			\addplot[TP] coordinates{
				(295.99,0.661537)
				(395.489,0.706141)
				(551.496,0.733457)
				(775.682,0.770995)
				(1000.76,0.777413)
				(1130.49,0.774995)
				(1401.47,0.810351)
				(1572.01,0.791293)
				(1815.33,0.800545)
				(2110.15,0.809864)
			};
	
			% POISE ev.min[0] %%%%%%%%%%%%%%%%%%%%%%%%%%%%%%%%%%%%%%%%%%%%%%%%%%%%%%%%%%%%
			\addplot[POISE] coordinates{
				(204.942,0.64021)
				(306.83,0.705397)
				(408.614,0.739619)
				(612.1,0.745433)
				(815.501,0.746541)
				(1019.01,0.751043)
				(1222.5,0.751982)
				(1425.91,0.753963)
				(1628.65,0.754268)
				(1830.61,0.754819)
				(2031.68,0.755246)
				(2427.44,0.754696)
				(2807.4,0.754037)
				(3158.64,0.754437)
				(3466.96,0.754651)
				(3736.5,0.754306)
				(4042.51,0.754306)
				(4219.03,0.754306)
			};
	
			% FH ev.min[0] %%%%%%%%%%%%%%%%%%%%%%%%%%%%%%%%%%%%%%%%%%%%%%%%%%%%%%%%%%%%
			\addplot[FH] coordinates{
				(689.802,0.560104)
				(763.792,0.605468)
				(813.815,0.640812)
				(988.712,0.683112)
				(1077.66,0.662788)
				(1359.76,0.718871)
				(1923.5,0.778069)
				(2699.96,0.861601)
				(1559.12,0.705488)
				(1408.18,0.701493)
				(3024.27,0.855386)
				(2206.32,0.766911)
				(2448.19,0.772793)
				(4594.6,0.865621)
				(3568.29,0.860286)
				(4199.72,0.856864)
			};
	
			% EAMS ev.min[0] %%%%%%%%%%%%%%%%%%%%%%%%%%%%%%%%%%%%%%%%%%%%%%%%%%%%%%%%%%%%
			\addplot[EAMS] coordinates{
				(419.068,0.693534)
				(455.657,0.721681)
				(499.416,0.724872)
				(641.188,0.728628)
				(842.328,0.787512)
				(1268.71,0.831887)
				(2584.6,0.891934)
				(2683.21,0.892861)
				(2779.09,0.896982)
				(2997.7,0.900961)
				(3180.29,0.907452)
				(3421.45,0.913851)
				(3646.16,0.922165)
				(3943.04,0.921553)
				(4448.4,0.925094)
				(5663.91,0.939516)
				(9098.39,0.951383)
			};
	
			% CRS ev.min[0] %%%%%%%%%%%%%%%%%%%%%%%%%%%%%%%%%%%%%%%%%%%%%%%%%%%%%%%%%%%%
			\addplot[CRS] coordinates{
				(254.263,0.658033)
				(396.694,0.728437)
				(514.895,0.766253)
				(764.997,0.775657)
				(988.589,0.782307)
				(1232.03,0.811984)
				(1498.25,0.823872)
				(1707.65,0.838962)
				(1968.19,0.841978)
				(2099.81,0.845662)
				(2299.23,0.866271)
				(2706.1,0.86319)
				(3259.07,0.87461)
				(3558.75,0.882667)
				(4055.02,0.887659)
				(4394.15,0.890638)
				(5418.11,0.894452)
				(5695.52,0.896679)
			};
	
			% SEAW ev.min[0] %%%%%%%%%%%%%%%%%%%%%%%%%%%%%%%%%%%%%%%%%%%%%%%%%%%%%%%%%%%%
			\addplot[SEAW] coordinates{
				(124.667,0.451795)
				(349.414,0.593748)
				(1192.36,0.740353)
				(4497.65,0.83656)
			};
	
			% RESEEDS ev.min[0] %%%%%%%%%%%%%%%%%%%%%%%%%%%%%%%%%%%%%%%%%%%%%%%%%%%%%%%%%%%%
			\addplot[RESEEDS] coordinates{
				(200.441,0.807781)
				(300.263,0.7971)
				(400.323,0.820457)
				(600.361,0.862097)
				(800.526,0.869257)
				(1000.06,0.881735)
				(1200.06,0.889218)
				(1400.96,0.895326)
				(1600.21,0.907844)
				(1792.08,0.890043)
				(1998.13,0.924478)
				(2408.17,0.927206)
				(2801.44,0.930262)
				(3182.16,0.938596)
				(3648.12,0.915066)
				(4257.44,0.942184)
				(4444.16,0.946434)
				(4864.15,0.933543)
			};
	
			% ERGC ev.min[0] %%%%%%%%%%%%%%%%%%%%%%%%%%%%%%%%%%%%%%%%%%%%%%%%%%%%%%%%%%%%
			\addplot[ERGC] coordinates{
				(196,0.600677)
				(289,0.655585)
				(400,0.660882)
				(600,0.710352)
				(812,0.753679)
				(1024,0.776007)
				(1224,0.789787)
				(1406,0.80394)
				(1681,0.811227)
				(1763,0.818607)
				(1935,0.840394)
				(2350,0.846539)
				(2856,0.860317)
				(3080,0.865215)
				(3520,0.877895)
				(3904,0.879947)
				(4864,0.89135)
				(5100,0.8987)
			};
	
			% PF ev.min[0] %%%%%%%%%%%%%%%%%%%%%%%%%%%%%%%%%%%%%%%%%%%%%%%%%%%%%%%%%%%%
			\addplot[PF] coordinates{
				(281.19,0.323655)
				(430.376,0.393591)
				(586.371,0.399422)
				(907.19,0.433775)
				(1201.57,0.47676)
				(1354.07,0.466403)
				(1718.72,0.484953)
				(1939.43,0.507532)
				(2251.38,0.511785)
				(2592.38,0.506763)
				(3186.03,0.563748)
				(4180.57,0.563198)
				(6182.51,0.617164)
			};
	
			% TPS ev.min[0] %%%%%%%%%%%%%%%%%%%%%%%%%%%%%%%%%%%%%%%%%%%%%%%%%%%%%%%%%%%%
			\addplot[TPS] coordinates{
				(230.419,0.452844)
				(313.739,0.488495)
				(450.436,0.517629)
				(638.008,0.561957)
				(857.559,0.629027)
				(1043.09,0.637376)
				(1317.64,0.673553)
				(1503.8,0.673533)
				(1703.97,0.68949)
				(1873.98,0.686707)
				(2144.93,0.706509)
				(2580.57,0.728563)
				(2894.48,0.732256)
				(3314.47,0.74136)
				(3924.13,0.763615)
				(4380.02,0.77278)
				(5132.34,0.784144)
				(5738.91,0.798122)
			};
	
			% NC ev.min[0] %%%%%%%%%%%%%%%%%%%%%%%%%%%%%%%%%%%%%%%%%%%%%%%%%%%%%%%%%%%%
			\addplot[NC] coordinates{
				(439.952,0.621406)
				(1154.47,0.661599)
				(2631.49,0.701778)
				(3874.82,0.725167)
			};
	
			% VC ev.min[0] %%%%%%%%%%%%%%%%%%%%%%%%%%%%%%%%%%%%%%%%%%%%%%%%%%%%%%%%%%%%
			\addplot[VC] coordinates{
				(243.744,0.560848)
				(400.291,0.698392)
				(531.672,0.751075)
				(660.84,0.772007)
				(895.915,0.806875)
				(1125.96,0.840263)
				(1348.09,0.859593)
				(1564.34,0.873085)
				(1781.1,0.877549)
				(1995.21,0.879878)
				(2205.86,0.893992)
				(2416.36,0.903668)
				(2820.76,0.906138)
				(3224.25,0.918235)
				(3610.4,0.924723)
				(3988.04,0.929133)
				(4351.59,0.933089)
				(4847.23,0.936636)
				(5262.69,0.9387)
			};
	
			% PB ev.min[0] %%%%%%%%%%%%%%%%%%%%%%%%%%%%%%%%%%%%%%%%%%%%%%%%%%%%%%%%%%%%
			\addplot[PB] coordinates{
				(273.248,0.372685)
				(360.188,0.489059)
				(463.193,0.482483)
				(656.962,0.57536)
				(857.526,0.582432)
				(970.915,0.607424)
				(1217.27,0.642082)
				(1387.49,0.655374)
				(1616.5,0.672367)
				(1896.95,0.69637)
				(1896.95,0.69637)
				(2243.04,0.691623)
				(2681.85,0.714839)
				(3316.29,0.739186)
				(3316.29,0.739186)
				(4151.92,0.764849)
				(4151.92,0.764849)
				(5425.24,0.795098)
			};
	
			% VCCS ev.min[0] %%%%%%%%%%%%%%%%%%%%%%%%%%%%%%%%%%%%%%%%%%%%%%%%%%%%%%%%%%%%
			\addplot[VCCS] coordinates{
				(564.123,0.553389)
				(600.283,0.564347)
				(648.629,0.563403)
				(755.632,0.557979)
				(836.05,0.557383)
				(1051.64,0.556897)
				(1109.07,0.554482)
				(1179.74,0.560547)
				(1237.23,0.546893)
				(1322.17,0.548687)
				(1400.72,0.547016)
				(1526.22,0.557148)
				(1630.95,0.565593)
				(1756.59,0.498849)
				(1914.09,0.55507)
				(2024.07,0.505818)
				(2258.58,0.554705)
				(2558.8,0.611646)
				(2780.39,0.575985)
				(3044.22,0.611947)
				(3687.7,0.747279)
				(4171.76,0.794722)
				(4255.91,0.6354)
			};
	
			% PRESLIC ev.min[0] %%%%%%%%%%%%%%%%%%%%%%%%%%%%%%%%%%%%%%%%%%%%%%%%%%%%%%%%%%%%
			\addplot[PRESLIC] coordinates{
				(189.89,0.560581)
				(388.539,0.616609)
				(589.772,0.672496)
				(798.767,0.689736)
				(920.201,0.73855)
				(1188.65,0.741026)
				(1401.21,0.744539)
				(1612.25,0.770514)
				(1904.93,0.789028)
				(1904.93,0.789028)
				(2282.12,0.793508)
				(2738.74,0.822918)
				(3422.72,0.833363)
				(3422.72,0.833363)
				(4393.15,0.867024)
				(4393.15,0.867024)
				(5724.22,0.888337)
			};
	
			% W ev.min[0] %%%%%%%%%%%%%%%%%%%%%%%%%%%%%%%%%%%%%%%%%%%%%%%%%%%%%%%%%%%%
			\addplot[W] coordinates{
				(193.87,0.443995)
				(303.997,0.509078)
				(397.103,0.547269)
				(621.108,0.594347)
				(870.236,0.667461)
				(959.915,0.661634)
				(1234.38,0.681863)
				(1418.74,0.688804)
				(1652.83,0.711177)
				(1957.01,0.743683)
				(1957.01,0.743683)
				(2345.3,0.749368)
				(2867.42,0.776021)
				(3518.42,0.803132)
				(3518.42,0.803132)
				(4497.46,0.818363)
				(4497.46,0.818363)
				(5940.33,0.834023)
			};
	
			% LSC ev.min[0] %%%%%%%%%%%%%%%%%%%%%%%%%%%%%%%%%%%%%%%%%%%%%%%%%%%%%%%%%%%%
			\addplot[LSC] coordinates{
				(383.095,0.771388)
				(552.409,0.815143)
				(718.163,0.817448)
				(1059.33,0.853517)
				(1414.95,0.873169)
				(1816.61,0.879009)
				(2009.32,0.877446)
				(2263.42,0.887326)
				(2398.35,0.886258)
				(2446.8,0.887502)
				(2588.34,0.893462)
				(2830.36,0.880232)
				(3264.3,0.899221)
				(3438.52,0.88991)
				(3802.19,0.889493)
				(4059.6,0.895023)
				(4900.83,0.890804)
				(5001.09,0.883573)
			};
	
			% WP ev.min[0] %%%%%%%%%%%%%%%%%%%%%%%%%%%%%%%%%%%%%%%%%%%%%%%%%%%%%%%%%%%%
			\addplot[WP] coordinates{
				(204,0.490516)
				(315,0.557599)
				(432,0.595939)
				(638,0.642593)
				(850,0.700708)
				(1064,0.710052)
				(1230,0.720966)
				(1408,0.728133)
				(1645,0.741192)
				(1938,0.788858)
				(1938,0.787582)
				(2296,0.796986)
				(2745,0.818033)
				(3400,0.848654)
				(3400,0.848654)
				(4256,0.854409)
				(4256,0.854409)
				(5568,0.873518)
			};
	
			% QS ev.min[0] %%%%%%%%%%%%%%%%%%%%%%%%%%%%%%%%%%%%%%%%%%%%%%%%%%%%%%%%%%%%
			\addplot[QS] coordinates{
				(223.521,0.266369)
				(325.303,0.353714)
				(414.609,0.472188)
				(663.504,0.664675)
				(828.486,0.720318)
				(1077.18,0.762032)
				(1252.44,0.784716)
				(1475.56,0.7996)
				(1767.37,0.822576)
				(2165.34,0.836203)
				(2703.49,0.857548)
				(3460.91,0.874813)
				(4543.63,0.892488)
				(6144.37,0.906971)
			};
	
			% CIS ev.min[0] %%%%%%%%%%%%%%%%%%%%%%%%%%%%%%%%%%%%%%%%%%%%%%%%%%%%%%%%%%%%
			\addplot[CIS] coordinates{
				(292.103,0.490193)
				(366.787,0.534967)
				(436.398,0.601035)
				(575.972,0.619594)
				(721.491,0.656192)
				(783.672,0.659791)
				(963.875,0.695449)
				(1083.42,0.707835)
				(1227.51,0.730922)
				(1408.35,0.735077)
				(1424.97,0.735701)
				(1751.64,0.771602)
				(2118.16,0.780456)
				(2717.81,0.811569)
				(3240.8,0.822334)
				(3240.8,0.822334)
				(4824.68,0.86182)
			};
	
			% RESEEDS3D ev.min[0] %%%%%%%%%%%%%%%%%%%%%%%%%%%%%%%%%%%%%%%%%%%%%%%%%%%%%%%%%%%%
			\addplot[RESEEDS3D] coordinates{
				(200.035,0.479536)
				(300.105,0.782724)
				(400.128,0.806317)
				(600.233,0.848056)
				(800.351,0.861588)
				(1000.04,0.876345)
				(1200.03,0.885543)
				(1400.65,0.888579)
				(1600.08,0.900593)
				(1792.08,0.890209)
				(1998.07,0.916708)
				(2408.16,0.92555)
				(2801.12,0.923456)
				(3182.14,0.934066)
				(3648.05,0.91663)
				(4257.09,0.939888)
				(4444.11,0.940153)
				(4864.04,0.932516)
			};
	
			% ERS ev.min[0] %%%%%%%%%%%%%%%%%%%%%%%%%%%%%%%%%%%%%%%%%%%%%%%%%%%%%%%%%%%%
			\addplot[ERS] coordinates{
				(200,0.616425)
				(300,0.646446)
				(400,0.665512)
				(600,0.700991)
				(800,0.726208)
				(1000,0.741076)
				(1200,0.747773)
				(1400,0.746134)
				(1600,0.760528)
				(1800,0.769481)
				(2000,0.771537)
				(2400,0.783816)
				(2800,0.795235)
				(3200,0.803484)
				(3600,0.810587)
				(4000,0.82167)
				(4600,0.834727)
				(5200,0.837364)
			};
	
			% DASP ev.min[0] %%%%%%%%%%%%%%%%%%%%%%%%%%%%%%%%%%%%%%%%%%%%%%%%%%%%%%%%%%%%
			\addplot[DASP] coordinates{
				(417.987,0.709381)
				(521.05,0.74895)
				(620.704,0.760206)
				(812.654,0.804714)
				(1006.72,0.817375)
				(1197.12,0.833863)
				(1388.06,0.843665)
				(1576.93,0.866867)
				(1767.2,0.862377)
				(1947.7,0.878286)
				(2131.65,0.878775)
				(2494.37,0.890294)
				(2860.93,0.898079)
				(3211.97,0.898842)
				(3564.56,0.904694)
				(3909.24,0.909368)
				(4418.09,0.914456)
				(4919.67,0.917256)
			};
	
			% MSS ev.min[0] %%%%%%%%%%%%%%%%%%%%%%%%%%%%%%%%%%%%%%%%%%%%%%%%%%%%%%%%%%%%
			\addplot[MSS] coordinates{
				(207.576,0.479709)
				(308.962,0.562182)
				(436.368,0.564717)
				(668.784,0.676271)
				(912.997,0.702136)
				(1055.23,0.758656)
				(1363.6,0.749648)
				(1575.7,0.768309)
				(1862.23,0.776627)
				(2222.47,0.804618)
				(2222.47,0.804618)
				(2666.45,0.810184)
				(3236.21,0.817244)
				(4079.47,0.827063)
				(4078.85,0.825624)
				(5209.16,0.826434)
				(5209.16,0.826434)
				(6994.34,0.838454)
			};
	
			% ETPS ev.min[0] %%%%%%%%%%%%%%%%%%%%%%%%%%%%%%%%%%%%%%%%%%%%%%%%%%%%%%%%%%%%
			\addplot[ETPS] coordinates{
				(221,0.83379)
				(315,0.865536)
				(432,0.876072)
				(638,0.885292)
				(850,0.902027)
				(972,0.906229)
				(1230,0.910204)
				(1408,0.919809)
				(1645,0.926343)
				(1938,0.931049)
				(1938,0.931049)
				(2296,0.936038)
				(2745,0.941665)
				(3400,0.946831)
				(3400,0.946831)
				(4256,0.951508)
				(4256,0.951508)
				(5568,0.956564)
			};

			\end{axis}
	\end{tikzpicture}
\end{subfigure}\\[-4px]
	\begin{subfigure}[b]{0.325\textwidth}\phantomsubcaption\label{subfig:experiments-quantitative-nyuv2-rec.std[0]}
	%%%%%%%%%%%%%%%%%%%%%%%%%%%%%%%%%%%%%%%%%%%%%%%%%%%%%%%%%%%%
	% rec.std[0]
	%%%%%%%%%%%%%%%%%%%%%%%%%%%%%%%%%%%%%%%%%%%%%%%%%%%%%%%%%%%%
	\begin{tikzpicture}
		\begin{axis}[AENYUV2RecStd,xmode=log]

			% CCS rec.std[0] %%%%%%%%%%%%%%%%%%%%%%%%%%%%%%%%%%%%%%%%%%%%%%%%%%%%%%%%%%%%
			\addplot[CCS] coordinates{
				(194.01,0.0621346)
				(282.682,0.0627852)
				(397.098,0.0602058)
				(597.153,0.0539926)
				(803.471,0.0470476)
				(925.103,0.0441287)
				(1177.26,0.0377614)
				(1397.83,0.0341308)
				(1588.43,0.0300005)
				(1878.46,0.0257808)
				(1878.46,0.0257808)
				(2231.81,0.0209776)
				(2677.06,0.0161631)
				(3328.01,0.00986884)
				(3328.01,0.00986884)
				(4313.98,0.00509781)
				(4313.98,0.00509781)
				(5573.99,0.00225087)
			};

			% SEEDS rec.std[0] %%%%%%%%%%%%%%%%%%%%%%%%%%%%%%%%%%%%%%%%%%%%%%%%%%%%%%%%%%%%
			\addplot[SEEDS] coordinates{
				(246.043,0.0418849)
				(347.734,0.0322367)
				(450.607,0.0278759)
				(654.489,0.0223092)
				(857.654,0.0188906)
				(1057.59,0.0145197)
				(1258.91,0.0118301)
				(1462.49,0.0125916)
				(1661.42,0.00896033)
				(1870.25,0.00777043)
				(2066.14,0.00752096)
				(2473.07,0.00696122)
				(2864.58,0.00691397)
				(3245.7,0.00472776)
				(3711.17,0.00313604)
				(4314.51,0.00394421)
				(4499.92,0.00289901)
				(4922.37,0.00211432)
			};

			% SLIC rec.std[0] %%%%%%%%%%%%%%%%%%%%%%%%%%%%%%%%%%%%%%%%%%%%%%%%%%%%%%%%%%%%
			\addplot[SLIC] coordinates{
				(184.484,0.0529577)
				(279.316,0.0487866)
				(385.211,0.0434227)
				(596.942,0.0370989)
				(819.424,0.0311401)
				(939.241,0.0281536)
				(1211.08,0.0233547)
				(1425.96,0.0198281)
				(1642.44,0.0190101)
				(1940.44,0.0154273)
				(1940.44,0.0154273)
				(2316.71,0.0126859)
				(2774.73,0.0101311)
				(3441.91,0.00756836)
				(3441.91,0.00756836)
				(4395.74,0.00423568)
				(4395.74,0.00423568)
				(5687.73,0.0024657)
			};

			% RW rec.std[0] %%%%%%%%%%%%%%%%%%%%%%%%%%%%%%%%%%%%%%%%%%%%%%%%%%%%%%%%%%%%
			\addplot[RW] coordinates{
				(215.912,0.06953)
				(306.055,0.0649407)
				(440.393,0.0604405)
				(635.679,0.0522894)
				(846.511,0.0466066)
				(1060.11,0.0434447)
				(1306.99,0.0375247)
				(1464.51,0.0342781)
				(1685.01,0.0316885)
				(1868.17,0.0279592)
				(2109.97,0.0279975)
				(3967.51,0.00982647)
				(4401.89,0.0082936)
				(5110.46,0.00550265)
			};

			% CW rec.std[0] %%%%%%%%%%%%%%%%%%%%%%%%%%%%%%%%%%%%%%%%%%%%%%%%%%%%%%%%%%%%
			\addplot[CW] coordinates{
				(198.614,0.0598702)
				(293.173,0.056003)
				(407.073,0.0491457)
				(613.509,0.0436022)
				(834.84,0.0380076)
				(1056.32,0.0318292)
				(1266,0.0289438)
				(1460.01,0.0255964)
				(1700.24,0.0227234)
				(1828.56,0.0202062)
				(2009.84,0.0180268)
				(2449.61,0.0142921)
				(2991.17,0.0106138)
				(3233.9,0.00910877)
				(3707.85,0.00687506)
				(4124.98,0.00538771)
				(5174.47,0.00324808)
				(5281.06,0.00350408)
			};

			% TP rec.std[0] %%%%%%%%%%%%%%%%%%%%%%%%%%%%%%%%%%%%%%%%%%%%%%%%%%%%%%%%%%%%
			\addplot[TP] coordinates{
				(295.99,0.0647244)
				(395.489,0.0624232)
				(551.496,0.054693)
				(775.682,0.0481595)
				(1000.76,0.0427532)
				(1130.49,0.0396832)
				(1401.47,0.034272)
				(1572.01,0.0320029)
				(1815.33,0.0282539)
				(2110.15,0.023407)
			};

			% POISE rec.std[0] %%%%%%%%%%%%%%%%%%%%%%%%%%%%%%%%%%%%%%%%%%%%%%%%%%%%%%%%%%%%
			\addplot[POISE] coordinates{
				(204.942,0.051928)
				(306.83,0.0489098)
				(408.614,0.0453103)
				(612.1,0.0397874)
				(815.501,0.0353507)
				(1019.01,0.0323069)
				(1222.5,0.0299528)
				(1425.91,0.0284054)
				(1628.65,0.0271863)
				(1830.61,0.026627)
				(2031.68,0.026139)
				(2427.44,0.0254574)
				(2807.4,0.025118)
				(3158.64,0.0251891)
				(3466.96,0.0252895)
				(3736.5,0.025373)
				(4042.51,0.0255882)
				(4219.03,0.0256882)
			};

			% FH rec.std[0] %%%%%%%%%%%%%%%%%%%%%%%%%%%%%%%%%%%%%%%%%%%%%%%%%%%%%%%%%%%%
			\addplot[FH] coordinates{
				(689.802,0.0438999)
				(763.792,0.0421685)
				(813.815,0.0410817)
				(988.712,0.0374873)
				(1077.66,0.0348967)
				(1359.76,0.0313842)
				(1408.18,0.0373096)
				(1559.12,0.0326693)
				(1923.5,0.0247463)
				(2206.32,0.0255089)
				(2448.19,0.0216199)
				(2699.96,0.0213578)
				(3024.27,0.0247679)
				(3568.29,0.0239469)
				(4199.72,0.0150221)
				(4594.6,0.0117365)
			};

			% EAMS rec.std[0] %%%%%%%%%%%%%%%%%%%%%%%%%%%%%%%%%%%%%%%%%%%%%%%%%%%%%%%%%%%%
			\addplot[EAMS] coordinates{
				(419.068,0.0424531)
				(455.657,0.0419012)
				(499.416,0.0411368)
				(641.188,0.0407247)
				(842.328,0.0363737)
				(1268.71,0.0368232)
				(2584.6,0.0330403)
				(2683.21,0.0327267)
				(2779.09,0.0361147)
				(2997.7,0.0359127)
				(3180.29,0.039048)
				(3421.45,0.0411469)
				(3646.16,0.0432763)
				(3943.04,0.0451766)
				(4448.4,0.0449219)
				(5663.91,0.0443214)
				(9098.39,0.0433492)
			};

			% CRS rec.std[0] %%%%%%%%%%%%%%%%%%%%%%%%%%%%%%%%%%%%%%%%%%%%%%%%%%%%%%%%%%%%
			\addplot[CRS] coordinates{
				(254.263,0.0517382)
				(396.694,0.0420511)
				(514.895,0.0372401)
				(764.997,0.0288923)
				(988.589,0.0252883)
				(1232.03,0.0204219)
				(1498.25,0.0162734)
				(1707.65,0.0146911)
				(1968.19,0.0123936)
				(2099.81,0.0111959)
				(2299.23,0.00969825)
				(2706.1,0.00752096)
				(3259.07,0.00601498)
				(3558.75,0.00474035)
				(4055.02,0.00347847)
				(4394.15,0.00284714)
				(5418.11,0.00172633)
				(5695.52,0.0012207)
			};

			% RESEEDS rec.std[0] %%%%%%%%%%%%%%%%%%%%%%%%%%%%%%%%%%%%%%%%%%%%%%%%%%%%%%%%%%%%
			\addplot[RESEEDS] coordinates{
				(200.441,0.037562)
				(300.263,0.0315008)
				(400.323,0.0277075)
				(600.361,0.0228438)
				(800.526,0.0187178)
				(1000.06,0.0155063)
				(1200.06,0.0137306)
				(1400.96,0.012974)
				(1600.21,0.0115161)
				(1792.08,0.0103753)
				(1998.13,0.0092064)
				(2408.17,0.00799724)
				(2801.44,0.00654188)
				(3182.16,0.005179)
				(3648.12,0.00433307)
				(4257.44,0.00401906)
				(4444.16,0.00311698)
				(4864.15,0.00272958)
			};

			% ERGC rec.std[0] %%%%%%%%%%%%%%%%%%%%%%%%%%%%%%%%%%%%%%%%%%%%%%%%%%%%%%%%%%%%
			\addplot[ERGC] coordinates{
				(196,0.0575921)
				(289,0.0533094)
				(400,0.0484587)
				(600,0.0415491)
				(812,0.0338159)
				(1024,0.0299955)
				(1224,0.0249943)
				(1406,0.0227575)
				(1681,0.0184128)
				(1763,0.0173906)
				(1935,0.0147093)
				(2350,0.0109347)
				(2856,0.00800841)
				(3080,0.00687506)
				(3520,0.00511531)
				(3904,0.0038292)
				(4864,0.00239208)
				(5100,0.00182698)
			};

			% PF rec.std[0] %%%%%%%%%%%%%%%%%%%%%%%%%%%%%%%%%%%%%%%%%%%%%%%%%%%%%%%%%%%%
			\addplot[PF] coordinates{
				(281.19,0.0594524)
				(430.376,0.0565133)
				(586.371,0.0544428)
				(907.19,0.0509438)
				(1201.57,0.0467078)
				(1354.07,0.0460527)
				(1718.72,0.043385)
				(1939.43,0.0406643)
				(2251.38,0.0401313)
				(2592.38,0.0379252)
				(3186.03,0.0395184)
				(4180.57,0.0354929)
				(6182.51,0.0305985)
			};

			% TPS rec.std[0] %%%%%%%%%%%%%%%%%%%%%%%%%%%%%%%%%%%%%%%%%%%%%%%%%%%%%%%%%%%%
			\addplot[TPS] coordinates{
				(230.419,0.0616413)
				(313.739,0.059701)
				(450.436,0.054693)
				(638.008,0.0505784)
				(857.559,0.0456079)
				(1043.09,0.0430145)
				(1317.64,0.0383751)
				(1503.8,0.0350535)
				(1703.97,0.0310951)
				(1873.98,0.0309404)
				(2144.93,0.0284547)
				(2580.57,0.0225589)
				(2894.48,0.0209036)
				(3314.47,0.0187798)
				(3924.13,0.0116627)
				(4380.02,0.00994405)
				(5132.34,0.00738904)
				(5738.91,0.00601994)
			};

			% NC rec.std[0] %%%%%%%%%%%%%%%%%%%%%%%%%%%%%%%%%%%%%%%%%%%%%%%%%%%%%%%%%%%%
			\addplot[NC] coordinates{
				(439.952,0.0575763)
				(1154.47,0.0338758)
				(2631.49,0.015987)
				(3874.82,0.00596523)
			};

			% VC rec.std[0] %%%%%%%%%%%%%%%%%%%%%%%%%%%%%%%%%%%%%%%%%%%%%%%%%%%%%%%%%%%%
			\addplot[VC] coordinates{
				(243.744,0.0716329)
				(400.291,0.0664982)
				(531.672,0.0642113)
				(660.84,0.0598428)
				(895.915,0.0559226)
				(1125.96,0.0499481)
				(1348.09,0.0462638)
				(1564.34,0.0419652)
				(1781.1,0.0380531)
				(1995.21,0.03455)
				(2205.86,0.031832)
				(2416.36,0.0291643)
				(2820.76,0.0252683)
				(3224.25,0.0208851)
				(3610.4,0.0181125)
				(3988.04,0.0160484)
				(4351.59,0.0144559)
				(4847.23,0.0127515)
				(5262.69,0.0115858)
			};

			% PB rec.std[0] %%%%%%%%%%%%%%%%%%%%%%%%%%%%%%%%%%%%%%%%%%%%%%%%%%%%%%%%%%%%
			\addplot[PB] coordinates{
				(273.248,0.0598996)
				(360.188,0.0575499)
				(463.193,0.0522649)
				(656.962,0.0473091)
				(857.526,0.0412518)
				(970.915,0.0389571)
				(1217.27,0.0345965)
				(1387.49,0.0304746)
				(1616.5,0.02763)
				(1896.95,0.0252635)
				(1896.95,0.0252635)
				(2243.04,0.0197814)
				(2681.85,0.0162312)
				(3316.29,0.0108663)
				(3316.29,0.0108663)
				(4151.92,0.00674376)
				(4151.92,0.00674376)
				(5425.24,0.00333857)
			};

			% VCCS rec.std[0] %%%%%%%%%%%%%%%%%%%%%%%%%%%%%%%%%%%%%%%%%%%%%%%%%%%%%%%%%%%%
			\addplot[VCCS] coordinates{
				(564.123,0.072902)
				(600.283,0.0730556)
				(648.629,0.0734206)
				(755.632,0.0719619)
				(836.05,0.0714632)
				(1051.64,0.070924)
				(1109.07,0.0708096)
				(1179.74,0.0706786)
				(1237.23,0.072062)
				(1322.17,0.0709694)
				(1400.72,0.0716019)
				(1526.22,0.0689532)
				(1630.95,0.0701364)
				(1756.59,0.069345)
				(1914.09,0.0680666)
				(2024.07,0.070826)
				(2258.58,0.0666513)
				(2558.8,0.0622166)
				(2780.39,0.0632272)
				(3044.22,0.0617203)
				(3687.7,0.0430159)
				(4171.76,0.0362686)
				(4255.91,0.0501006)
			};

			% PRESLIC rec.std[0] %%%%%%%%%%%%%%%%%%%%%%%%%%%%%%%%%%%%%%%%%%%%%%%%%%%%%%%%%%%%
			\addplot[PRESLIC] coordinates{
				(189.89,0.0507959)
				(388.539,0.0388337)
				(589.772,0.0334838)
				(798.767,0.0265092)
				(920.201,0.0241872)
				(1188.65,0.0194824)
				(1401.21,0.0161225)
				(1612.25,0.0148886)
				(1904.93,0.0126718)
				(1904.93,0.0126718)
				(2282.12,0.00977173)
				(2738.74,0.00742123)
				(3422.72,0.00533211)
				(3422.72,0.00533211)
				(4393.15,0.0029901)
				(4393.15,0.0029901)
				(5724.22,0.00174351)
			};

			% W rec.std[0] %%%%%%%%%%%%%%%%%%%%%%%%%%%%%%%%%%%%%%%%%%%%%%%%%%%%%%%%%%%%
			\addplot[W] coordinates{
				(193.87,0.0617611)
				(303.997,0.0566422)
				(397.103,0.0518619)
				(621.108,0.0440138)
				(870.236,0.0367576)
				(959.915,0.0358205)
				(1234.38,0.0301501)
				(1418.74,0.0258743)
				(1652.83,0.0229972)
				(1957.01,0.0195343)
				(1957.01,0.0195343)
				(2345.3,0.015229)
				(2867.42,0.0115445)
				(3518.42,0.00841843)
				(3518.42,0.00841843)
				(4497.46,0.00458048)
				(4497.46,0.00458048)
				(5940.33,0.00215619)
			};

			% LSC rec.std[0] %%%%%%%%%%%%%%%%%%%%%%%%%%%%%%%%%%%%%%%%%%%%%%%%%%%%%%%%%%%%
			\addplot[LSC] coordinates{
				(383.095,0.045851)
				(552.409,0.0407459)
				(718.163,0.0368175)
				(1059.33,0.03006)
				(1414.95,0.0246558)
				(1816.61,0.0187385)
				(2009.32,0.0184694)
				(2263.42,0.0157712)
				(2398.35,0.0146525)
				(2446.8,0.0149365)
				(2588.34,0.0136129)
				(2830.36,0.0118276)
				(3264.3,0.0091512)
				(3438.52,0.00863858)
				(3802.19,0.00643624)
				(4059.6,0.0055833)
				(4900.83,0.00325724)
				(5001.09,0.0030688)
			};

			% WP rec.std[0] %%%%%%%%%%%%%%%%%%%%%%%%%%%%%%%%%%%%%%%%%%%%%%%%%%%%%%%%%%%%
			\addplot[WP] coordinates{
				(204,0.0667485)
				(315,0.0608508)
				(432,0.0550947)
				(638,0.0464445)
				(850,0.0405825)
				(1064,0.0352054)
				(1230,0.0309626)
				(1408,0.0285551)
				(1645,0.0249239)
				(1938,0.0206728)
				(1938,0.0206281)
				(2296,0.0169937)
				(2745,0.0131655)
				(3400,0.00911858)
				(3400,0.00911858)
				(4256,0.00582878)
				(4256,0.00582878)
				(5568,0.0029901)
			};

			% QS rec.std[0] %%%%%%%%%%%%%%%%%%%%%%%%%%%%%%%%%%%%%%%%%%%%%%%%%%%%%%%%%%%%
			\addplot[QS] coordinates{
				(223.521,0.072243)
				(325.303,0.07436)
				(414.609,0.0730847)
				(663.504,0.0686604)
				(828.486,0.0654333)
				(1077.18,0.0617019)
				(1252.44,0.0590326)
				(1475.56,0.0564593)
				(1767.37,0.0529402)
				(2165.34,0.0493889)
				(2703.49,0.0446991)
				(3460.91,0.039547)
				(4543.63,0.0335638)
				(6144.37,0.0269452)
			};

			% CIS rec.std[0] %%%%%%%%%%%%%%%%%%%%%%%%%%%%%%%%%%%%%%%%%%%%%%%%%%%%%%%%%%%%
			\addplot[CIS] coordinates{
				(292.103,0.0807886)
				(366.787,0.0753104)
				(436.398,0.0726953)
				(575.972,0.0687259)
				(721.491,0.0620079)
				(783.672,0.0609467)
				(963.875,0.0564894)
				(1083.42,0.0526161)
				(1227.51,0.0498238)
				(1408.35,0.0465803)
				(1424.97,0.0463841)
				(1751.64,0.0431363)
				(2118.16,0.0369032)
				(2717.81,0.0323696)
				(3240.8,0.0248484)
				(3240.8,0.0248484)
				(4824.68,0.0189174)
			};

			% RESEEDS3D rec.std[0] %%%%%%%%%%%%%%%%%%%%%%%%%%%%%%%%%%%%%%%%%%%%%%%%%%%%%%%%%%%%
			\addplot[RESEEDS3D] coordinates{
				(200.035,0.0690338)
				(300.105,0.0503244)
				(400.128,0.0447197)
				(600.233,0.0359799)
				(800.351,0.0297741)
				(1000.04,0.0253754)
				(1200.03,0.0212907)
				(1400.65,0.0182714)
				(1600.08,0.0159796)
				(1792.08,0.0135162)
				(1998.07,0.0113702)
				(2408.16,0.00840425)
				(2801.12,0.00670832)
				(3182.14,0.00510365)
				(3648.05,0.00442831)
				(4257.09,0.00283666)
				(4444.11,0.00245358)
				(4864.04,0.00221079)
			};

			% ERS rec.std[0] %%%%%%%%%%%%%%%%%%%%%%%%%%%%%%%%%%%%%%%%%%%%%%%%%%%%%%%%%%%%
			\addplot[ERS] coordinates{
				(200,0.0597147)
				(300,0.0528529)
				(400,0.0474236)
				(600,0.0391441)
				(800,0.0325697)
				(1000,0.0276009)
				(1200,0.0233777)
				(1400,0.0191646)
				(1600,0.0163063)
				(1800,0.014367)
				(2000,0.0116576)
				(2400,0.00849595)
				(2800,0.00625305)
				(3200,0.0046129)
				(3600,0.00328458)
				(4000,0.00245358)
				(4600,0.00150498)
				(5200,0.00150498)
			};

			% DASP rec.std[0] %%%%%%%%%%%%%%%%%%%%%%%%%%%%%%%%%%%%%%%%%%%%%%%%%%%%%%%%%%%%
			\addplot[DASP] coordinates{
				(417.987,0.0657183)
				(521.05,0.0627717)
				(620.704,0.0599006)
				(812.654,0.0537747)
				(1006.72,0.0477749)
				(1197.12,0.0440774)
				(1388.06,0.040295)
				(1576.93,0.0374817)
				(1767.2,0.0338036)
				(1947.7,0.0315801)
				(2131.65,0.0299876)
				(2494.37,0.0266449)
				(2860.93,0.0228933)
				(3211.97,0.0210386)
				(3564.56,0.0187607)
				(3909.24,0.017829)
				(4418.09,0.0158522)
				(4919.67,0.0143794)
			};

			% MSS rec.std[0] %%%%%%%%%%%%%%%%%%%%%%%%%%%%%%%%%%%%%%%%%%%%%%%%%%%%%%%%%%%%
			\addplot[MSS] coordinates{
				(207.576,0.0614219)
				(308.962,0.0574732)
				(436.368,0.053124)
				(668.784,0.0468451)
				(912.997,0.0397994)
				(1055.23,0.0391136)
				(1363.6,0.0329951)
				(1575.7,0.0305966)
				(1862.23,0.0269706)
				(2222.47,0.0228477)
				(2222.47,0.0228477)
				(2666.45,0.0182877)
				(3236.21,0.0148565)
				(4079.47,0.0103032)
				(4078.85,0.0102452)
				(5209.16,0.00749317)
				(5209.16,0.00749317)
				(6994.34,0.00410708)
			};

			% ETPS rec.std[0] %%%%%%%%%%%%%%%%%%%%%%%%%%%%%%%%%%%%%%%%%%%%%%%%%%%%%%%%%%%%
			\addplot[ETPS] coordinates{
				(221,0.0491063)
				(315,0.0415771)
				(432,0.0361336)
				(638,0.0278502)
				(850,0.023269)
				(972,0.0206483)
				(1230,0.0167463)
				(1408,0.0148063)
				(1645,0.0123019)
				(1938,0.0106726)
				(1938,0.0106726)
				(2296,0.00774738)
				(2745,0.00658275)
				(3400,0.00472776)
				(3400,0.00472776)
				(4256,0.00327549)
				(4256,0.00327549)
				(5568,0.00152466)
			};

			% SEAW rec.std[0] %%%%%%%%%%%%%%%%%%%%%%%%%%%%%%%%%%%%%%%%%%%%%%%%%%%%%%%%%%%%
			\addplot[SEAW] coordinates{
				(124.667,0.067148)
				(349.414,0.0622393)
				(1192.36,0.0438591)
				(4497.65,0.00705055)
			};

			\end{axis}
	\end{tikzpicture}
\end{subfigure}
\begin{subfigure}[b]{0.325\textwidth}\phantomsubcaption\label{subfig:experiments-quantitative-nyuv2-ue_np.std[0]}
	%%%%%%%%%%%%%%%%%%%%%%%%%%%%%%%%%%%%%%%%%%%%%%%%%%%%%%%%%%%%
	% ue_np.std[0]
	%%%%%%%%%%%%%%%%%%%%%%%%%%%%%%%%%%%%%%%%%%%%%%%%%%%%%%%%%%%%
	\begin{tikzpicture}
		\begin{axis}[AENYUV2UEStd,xmode=log]

			% CCS ue_np.std[0] %%%%%%%%%%%%%%%%%%%%%%%%%%%%%%%%%%%%%%%%%%%%%%%%%%%%%%%%%%%%
			\addplot[CCS] coordinates{
				(194.01,0.0663401)
				(282.682,0.0564082)
				(397.098,0.0486903)
				(597.153,0.04019)
				(803.471,0.0352308)
				(925.103,0.033177)
				(1177.26,0.0299379)
				(1397.83,0.0281335)
				(1588.43,0.0266408)
				(1878.46,0.0248899)
				(1878.46,0.0248899)
				(2231.81,0.0232667)
				(2677.06,0.021684)
				(3328.01,0.0198448)
				(3328.01,0.0198448)
				(4313.98,0.0182467)
				(4313.98,0.0182467)
				(5573.99,0.016328)
			};

			% SEEDS ue_np.std[0] %%%%%%%%%%%%%%%%%%%%%%%%%%%%%%%%%%%%%%%%%%%%%%%%%%%%%%%%%%%%
			\addplot[SEEDS] coordinates{
				(246.043,0.0862657)
				(347.734,0.0785964)
				(450.607,0.0697585)
				(654.489,0.0570863)
				(857.654,0.0504837)
				(1057.59,0.0438329)
				(1258.91,0.0412873)
				(1462.49,0.0390518)
				(1661.42,0.0355636)
				(1870.25,0.0442319)
				(2066.14,0.0357405)
				(2473.07,0.0304752)
				(2864.58,0.0295699)
				(3245.7,0.0279317)
				(3711.17,0.027496)
				(4314.51,0.0243308)
				(4499.92,0.0232058)
				(4922.37,0.0233507)
			};

			% SLIC ue_np.std[0] %%%%%%%%%%%%%%%%%%%%%%%%%%%%%%%%%%%%%%%%%%%%%%%%%%%%%%%%%%%%
			\addplot[SLIC] coordinates{
				(184.484,0.0663777)
				(279.316,0.0562526)
				(385.211,0.048525)
				(596.942,0.0412155)
				(819.424,0.0363886)
				(939.241,0.034178)
				(1211.08,0.0310126)
				(1425.96,0.0294476)
				(1642.44,0.0278608)
				(1940.44,0.0262881)
				(1940.44,0.0262881)
				(2316.71,0.0244388)
				(2774.73,0.0228458)
				(3441.91,0.0207826)
				(3441.91,0.0207826)
				(4395.74,0.0189854)
				(4395.74,0.0189854)
				(5687.73,0.0169267)
			};

			% RW ue_np.std[0] %%%%%%%%%%%%%%%%%%%%%%%%%%%%%%%%%%%%%%%%%%%%%%%%%%%%%%%%%%%%
			\addplot[RW] coordinates{
				(215.912,0.0725338)
				(306.055,0.0620529)
				(440.393,0.054306)
				(635.679,0.0454838)
				(846.511,0.0406096)
				(1060.11,0.0368403)
				(1306.99,0.0336386)
				(1464.51,0.0324257)
				(1685.01,0.030456)
				(1868.17,0.0291013)
				(2109.97,0.0327013)
				(3967.51,0.0217361)
				(4401.89,0.0209139)
				(5110.46,0.0199536)
			};

			% CW ue_np.std[0] %%%%%%%%%%%%%%%%%%%%%%%%%%%%%%%%%%%%%%%%%%%%%%%%%%%%%%%%%%%%
			\addplot[CW] coordinates{
				(198.614,0.0678924)
				(293.173,0.0582853)
				(407.073,0.0518894)
				(613.509,0.0450539)
				(834.84,0.0394662)
				(1056.32,0.0357857)
				(1266,0.0334145)
				(1460.01,0.031481)
				(1700.24,0.0302784)
				(1828.56,0.0289848)
				(2009.84,0.027897)
				(2449.61,0.026031)
				(2991.17,0.0240895)
				(3233.9,0.0235582)
				(3707.85,0.0224239)
				(4124.98,0.0213144)
				(5174.47,0.0194721)
				(5281.06,0.0193536)
			};

			% TP ue_np.std[0] %%%%%%%%%%%%%%%%%%%%%%%%%%%%%%%%%%%%%%%%%%%%%%%%%%%%%%%%%%%%
			\addplot[TP] coordinates{
				(295.99,0.0559224)
				(395.489,0.0488477)
				(551.496,0.0486559)
				(775.682,0.0407764)
				(1000.76,0.0361206)
				(1130.49,0.0338677)
				(1401.47,0.030131)
				(1572.01,0.0289239)
				(1815.33,0.0270591)
				(2110.15,0.0256594)
			};

			% POISE ue_np.std[0] %%%%%%%%%%%%%%%%%%%%%%%%%%%%%%%%%%%%%%%%%%%%%%%%%%%%%%%%%%%%
			\addplot[POISE] coordinates{
				(204.942,0.0736554)
				(306.83,0.0653864)
				(408.614,0.0609276)
				(612.1,0.0565369)
				(815.501,0.053752)
				(1019.01,0.0525399)
				(1222.5,0.0504834)
				(1425.91,0.0502877)
				(1628.65,0.0490718)
				(1830.61,0.0484738)
				(2031.68,0.0480577)
				(2427.44,0.0476286)
				(2807.4,0.0471305)
				(3158.64,0.0470347)
				(3466.96,0.0468004)
				(3736.5,0.0467362)
				(4042.51,0.0467519)
				(4219.03,0.0467861)
			};

			% FH ue_np.std[0] %%%%%%%%%%%%%%%%%%%%%%%%%%%%%%%%%%%%%%%%%%%%%%%%%%%%%%%%%%%%
			\addplot[FH] coordinates{
				(689.802,0.0521826)
				(763.792,0.0494177)
				(813.815,0.0479145)
				(988.712,0.0432927)
				(1077.66,0.0408635)
				(1359.76,0.0367136)
				(1408.18,0.038665)
				(1559.12,0.0380448)
				(1923.5,0.031825)
				(2206.32,0.0329695)
				(2448.19,0.0319833)
				(2699.96,0.0280579)
				(3024.27,0.0296993)
				(3568.29,0.0291149)
				(4199.72,0.0265795)
				(4594.6,0.0247169)
			};

			% EAMS ue_np.std[0] %%%%%%%%%%%%%%%%%%%%%%%%%%%%%%%%%%%%%%%%%%%%%%%%%%%%%%%%%%%%
			\addplot[EAMS] coordinates{
				(419.068,0.0482863)
				(455.657,0.0467096)
				(499.416,0.0452877)
				(641.188,0.0436126)
				(842.328,0.0399434)
				(1268.71,0.0366725)
				(2584.6,0.0317206)
				(2683.21,0.0315435)
				(2779.09,0.0328843)
				(2997.7,0.0324179)
				(3180.29,0.0336743)
				(3421.45,0.0336881)
				(3646.16,0.0349893)
				(3943.04,0.0364858)
				(4448.4,0.0361117)
				(5663.91,0.0354843)
				(9098.39,0.0346729)
			};

			% CRS ue_np.std[0] %%%%%%%%%%%%%%%%%%%%%%%%%%%%%%%%%%%%%%%%%%%%%%%%%%%%%%%%%%%%
			\addplot[CRS] coordinates{
				(254.263,0.0585971)
				(396.694,0.048625)
				(514.895,0.0430782)
				(764.997,0.0364083)
				(988.589,0.03255)
				(1232.03,0.0297399)
				(1498.25,0.0279198)
				(1707.65,0.0265501)
				(1968.19,0.0249084)
				(2099.81,0.0243027)
				(2299.23,0.0233585)
				(2706.1,0.0218617)
				(3259.07,0.0204063)
				(3558.75,0.0196972)
				(4055.02,0.0187729)
				(4394.15,0.0181036)
				(5418.11,0.0165171)
				(5695.52,0.01637)
			};

			% RESEEDS ue_np.std[0] %%%%%%%%%%%%%%%%%%%%%%%%%%%%%%%%%%%%%%%%%%%%%%%%%%%%%%%%%%%%
			\addplot[RESEEDS] coordinates{
				(200.441,0.0736126)
				(300.263,0.0720045)
				(400.323,0.0613786)
				(600.361,0.0486628)
				(800.526,0.0418804)
				(1000.06,0.036891)
				(1200.06,0.0347433)
				(1400.96,0.0319719)
				(1600.21,0.0301123)
				(1792.08,0.0340067)
				(1998.13,0.0285027)
				(2408.17,0.0251846)
				(2801.44,0.0241284)
				(3182.16,0.022923)
				(3648.12,0.0228125)
				(4257.44,0.0199676)
				(4444.16,0.0196898)
				(4864.15,0.0197405)
			};

			% ERGC ue_np.std[0] %%%%%%%%%%%%%%%%%%%%%%%%%%%%%%%%%%%%%%%%%%%%%%%%%%%%%%%%%%%%
			\addplot[ERGC] coordinates{
				(196,0.0645976)
				(289,0.0543326)
				(400,0.0479789)
				(600,0.0409592)
				(812,0.0360889)
				(1024,0.0327736)
				(1224,0.0310184)
				(1406,0.0293212)
				(1681,0.0278262)
				(1763,0.0269542)
				(1935,0.0260604)
				(2350,0.024304)
				(2856,0.0226732)
				(3080,0.022063)
				(3520,0.0210303)
				(3904,0.0201054)
				(4864,0.0183313)
				(5100,0.0181264)
			};

			% PF ue_np.std[0] %%%%%%%%%%%%%%%%%%%%%%%%%%%%%%%%%%%%%%%%%%%%%%%%%%%%%%%%%%%%
			\addplot[PF] coordinates{
				(281.19,0.105288)
				(430.376,0.0956619)
				(586.371,0.0879299)
				(907.19,0.0782266)
				(1201.57,0.0719523)
				(1354.07,0.0688184)
				(1718.72,0.065114)
				(1939.43,0.0621775)
				(2251.38,0.059527)
				(2592.38,0.0574569)
				(3186.03,0.0558896)
				(4180.57,0.0522571)
				(6182.51,0.0464151)
			};

			% TPS ue_np.std[0] %%%%%%%%%%%%%%%%%%%%%%%%%%%%%%%%%%%%%%%%%%%%%%%%%%%%%%%%%%%%
			\addplot[TPS] coordinates{
				(230.419,0.0691592)
				(313.739,0.0604698)
				(450.436,0.0521982)
				(638.008,0.0452923)
				(857.559,0.0398816)
				(1043.09,0.0369097)
				(1317.64,0.0337861)
				(1503.8,0.0315512)
				(1703.97,0.0299095)
				(1873.98,0.0286608)
				(2144.93,0.0277401)
				(2580.57,0.0256032)
				(2894.48,0.0245132)
				(3314.47,0.0234863)
				(3924.13,0.0209154)
				(4380.02,0.0201772)
				(5132.34,0.0189196)
				(5738.91,0.0184475)
			};

			% NC ue_np.std[0] %%%%%%%%%%%%%%%%%%%%%%%%%%%%%%%%%%%%%%%%%%%%%%%%%%%%%%%%%%%%
			\addplot[NC] coordinates{
				(439.952,0.0477806)
				(1154.47,0.0363545)
				(2631.49,0.0246752)
				(3874.82,0.0212149)
			};

			% VC ue_np.std[0] %%%%%%%%%%%%%%%%%%%%%%%%%%%%%%%%%%%%%%%%%%%%%%%%%%%%%%%%%%%%
			\addplot[VC] coordinates{
				(243.744,0.0791542)
				(400.291,0.0597544)
				(531.672,0.0499796)
				(660.84,0.0441567)
				(895.915,0.036946)
				(1125.96,0.0331762)
				(1348.09,0.0305034)
				(1564.34,0.0285742)
				(1781.1,0.0271541)
				(1995.21,0.0259488)
				(2205.86,0.0249281)
				(2416.36,0.0237928)
				(2820.76,0.0225577)
				(3224.25,0.02123)
				(3610.4,0.0203669)
				(3988.04,0.0197552)
				(4351.59,0.0189854)
				(4847.23,0.0181206)
				(5262.69,0.0175143)
			};

			% PB ue_np.std[0] %%%%%%%%%%%%%%%%%%%%%%%%%%%%%%%%%%%%%%%%%%%%%%%%%%%%%%%%%%%%
			\addplot[PB] coordinates{
				(273.248,0.0751414)
				(360.188,0.0625057)
				(463.193,0.0559858)
				(656.962,0.0472731)
				(857.526,0.0413562)
				(970.915,0.0390741)
				(1217.27,0.03514)
				(1387.49,0.0332528)
				(1616.5,0.031012)
				(1896.95,0.0293938)
				(1896.95,0.0293938)
				(2243.04,0.0270968)
				(2681.85,0.0251268)
				(3316.29,0.0232758)
				(3316.29,0.0232758)
				(4151.92,0.0212576)
				(4151.92,0.0212576)
				(5425.24,0.0191697)
			};

			% VCCS ue_np.std[0] %%%%%%%%%%%%%%%%%%%%%%%%%%%%%%%%%%%%%%%%%%%%%%%%%%%%%%%%%%%%
			\addplot[VCCS] coordinates{
				(564.123,0.0766575)
				(600.283,0.0747961)
				(648.629,0.0711627)
				(755.632,0.0691726)
				(836.05,0.0667875)
				(1051.64,0.0649486)
				(1109.07,0.0649633)
				(1179.74,0.0629823)
				(1237.23,0.0645331)
				(1322.17,0.0637682)
				(1400.72,0.064028)
				(1526.22,0.0604107)
				(1630.95,0.0601628)
				(1756.59,0.0603836)
				(1914.09,0.0579956)
				(2024.07,0.0623882)
				(2258.58,0.0569511)
				(2558.8,0.0514164)
				(2780.39,0.0526516)
				(3044.22,0.0519587)
				(3687.7,0.0347163)
				(4171.76,0.0302225)
				(4255.91,0.0422433)
			};

			% PRESLIC ue_np.std[0] %%%%%%%%%%%%%%%%%%%%%%%%%%%%%%%%%%%%%%%%%%%%%%%%%%%%%%%%%%%%
			\addplot[PRESLIC] coordinates{
				(189.89,0.0686827)
				(388.539,0.0508399)
				(589.772,0.043028)
				(798.767,0.0382586)
				(920.201,0.0362205)
				(1188.65,0.0329266)
				(1401.21,0.0308718)
				(1612.25,0.0292333)
				(1904.93,0.0272894)
				(1904.93,0.0272894)
				(2282.12,0.0252519)
				(2738.74,0.0237244)
				(3422.72,0.0219647)
				(3422.72,0.0219647)
				(4393.15,0.0196892)
				(4393.15,0.0196892)
				(5724.22,0.0179049)
			};

			% W ue_np.std[0] %%%%%%%%%%%%%%%%%%%%%%%%%%%%%%%%%%%%%%%%%%%%%%%%%%%%%%%%%%%%
			\addplot[W] coordinates{
				(193.87,0.0739633)
				(303.997,0.0638199)
				(397.103,0.0569615)
				(621.108,0.0475453)
				(870.236,0.0414378)
				(959.915,0.039398)
				(1234.38,0.035719)
				(1418.74,0.0337972)
				(1652.83,0.0315449)
				(1957.01,0.0293408)
				(1957.01,0.0293408)
				(2345.3,0.027473)
				(2867.42,0.0251476)
				(3518.42,0.0233104)
				(3518.42,0.0233104)
				(4497.46,0.0209263)
				(4497.46,0.0209263)
				(5940.33,0.0189054)
			};

			% LSC ue_np.std[0] %%%%%%%%%%%%%%%%%%%%%%%%%%%%%%%%%%%%%%%%%%%%%%%%%%%%%%%%%%%%
			\addplot[LSC] coordinates{
				(383.095,0.0540105)
				(552.409,0.0460946)
				(718.163,0.0413873)
				(1059.33,0.0354658)
				(1414.95,0.0316205)
				(1816.61,0.0283496)
				(2009.32,0.0271771)
				(2263.42,0.0257133)
				(2398.35,0.02481)
				(2446.8,0.0243357)
				(2588.34,0.0234415)
				(2830.36,0.0223161)
				(3264.3,0.0211345)
				(3438.52,0.0204566)
				(3802.19,0.0195612)
				(4059.6,0.0190596)
				(4900.83,0.0179512)
				(5001.09,0.0178394)
			};

			% WP ue_np.std[0] %%%%%%%%%%%%%%%%%%%%%%%%%%%%%%%%%%%%%%%%%%%%%%%%%%%%%%%%%%%%
			\addplot[WP] coordinates{
				(204,0.0621437)
				(315,0.05364)
				(432,0.0473566)
				(638,0.0402761)
				(850,0.0363828)
				(1064,0.0329319)
				(1230,0.0315545)
				(1408,0.0302638)
				(1645,0.0286445)
				(1938,0.0271247)
				(1938,0.0270973)
				(2296,0.0254942)
				(2745,0.0237006)
				(3400,0.0221859)
				(3400,0.0221859)
				(4256,0.020651)
				(4256,0.020651)
				(5568,0.0188781)
			};

			% QS ue_np.std[0] %%%%%%%%%%%%%%%%%%%%%%%%%%%%%%%%%%%%%%%%%%%%%%%%%%%%%%%%%%%%
			\addplot[QS] coordinates{
				(223.521,0.105361)
				(325.303,0.0882609)
				(414.609,0.0772985)
				(663.504,0.0601235)
				(828.486,0.0532602)
				(1077.18,0.0468168)
				(1252.44,0.0429994)
				(1475.56,0.0400266)
				(1767.37,0.0371347)
				(2165.34,0.0342277)
				(2703.49,0.0312812)
				(3460.91,0.0280732)
				(4543.63,0.0251729)
				(6144.37,0.0222695)
			};

			% CIS ue_np.std[0] %%%%%%%%%%%%%%%%%%%%%%%%%%%%%%%%%%%%%%%%%%%%%%%%%%%%%%%%%%%%
			\addplot[CIS] coordinates{
				(292.103,0.057249)
				(366.787,0.0500497)
				(436.398,0.0460481)
				(575.972,0.0403262)
				(721.491,0.0365333)
				(783.672,0.0354451)
				(963.875,0.0326184)
				(1083.42,0.0314623)
				(1227.51,0.030069)
				(1408.35,0.0286819)
				(1424.97,0.0285807)
				(1751.64,0.02691)
				(2118.16,0.025009)
				(2717.81,0.0231547)
				(3240.8,0.021524)
				(3240.8,0.021524)
				(4824.68,0.0195218)
			};

			% RESEEDS3D ue_np.std[0] %%%%%%%%%%%%%%%%%%%%%%%%%%%%%%%%%%%%%%%%%%%%%%%%%%%%%%%%%%%%
			\addplot[RESEEDS3D] coordinates{
				(200.035,0.0766573)
				(300.105,0.05821)
				(400.128,0.0508988)
				(600.233,0.0415325)
				(800.351,0.0366267)
				(1000.04,0.032931)
				(1200.03,0.0309303)
				(1400.65,0.0292417)
				(1600.08,0.0274171)
				(1792.08,0.0301106)
				(1998.07,0.0264029)
				(2408.16,0.0237706)
				(2801.12,0.022618)
				(3182.14,0.0215346)
				(3648.05,0.0206513)
				(4257.09,0.0190915)
				(4444.11,0.0185455)
				(4864.04,0.0183239)
			};

			% ERS ue_np.std[0] %%%%%%%%%%%%%%%%%%%%%%%%%%%%%%%%%%%%%%%%%%%%%%%%%%%%%%%%%%%%
			\addplot[ERS] coordinates{
				(200,0.058431)
				(300,0.0495212)
				(400,0.0445341)
				(600,0.038306)
				(800,0.0344305)
				(1000,0.0317499)
				(1200,0.0297893)
				(1400,0.0283736)
				(1600,0.0270206)
				(1800,0.0259692)
				(2000,0.0250977)
				(2400,0.0232697)
				(2800,0.0220895)
				(3200,0.0209899)
				(3600,0.0200995)
				(4000,0.0193753)
				(4600,0.0181826)
				(5200,0.0173878)
			};

			% DASP ue_np.std[0] %%%%%%%%%%%%%%%%%%%%%%%%%%%%%%%%%%%%%%%%%%%%%%%%%%%%%%%%%%%%
			\addplot[DASP] coordinates{
				(417.987,0.062147)
				(521.05,0.0531971)
				(620.704,0.0461059)
				(812.654,0.0399471)
				(1006.72,0.0359331)
				(1197.12,0.0331179)
				(1388.06,0.031283)
				(1576.93,0.0292706)
				(1767.2,0.0278134)
				(1947.7,0.0269282)
				(2131.65,0.0260386)
				(2494.37,0.0245765)
				(2860.93,0.0232514)
				(3211.97,0.0221538)
				(3564.56,0.0213581)
				(3909.24,0.0206665)
				(4418.09,0.019669)
				(4919.67,0.0190352)
			};

			% MSS ue_np.std[0] %%%%%%%%%%%%%%%%%%%%%%%%%%%%%%%%%%%%%%%%%%%%%%%%%%%%%%%%%%%%
			\addplot[MSS] coordinates{
				(207.576,0.0697112)
				(308.962,0.0602066)
				(436.368,0.0512937)
				(668.784,0.0438381)
				(912.997,0.0382497)
				(1055.23,0.0362757)
				(1363.6,0.0328124)
				(1575.7,0.030957)
				(1862.23,0.0292564)
				(2222.47,0.0275473)
				(2222.47,0.0275473)
				(2666.45,0.0259527)
				(3236.21,0.0242959)
				(4079.47,0.0223252)
				(4078.85,0.0223408)
				(5209.16,0.0206312)
				(5209.16,0.0206312)
				(6994.34,0.0188004)
			};

			% ETPS ue_np.std[0] %%%%%%%%%%%%%%%%%%%%%%%%%%%%%%%%%%%%%%%%%%%%%%%%%%%%%%%%%%%%
			\addplot[ETPS] coordinates{
				(221,0.06568)
				(315,0.0553424)
				(432,0.049364)
				(638,0.0416504)
				(850,0.036992)
				(972,0.0351024)
				(1230,0.031986)
				(1408,0.0302513)
				(1645,0.0283219)
				(1938,0.0266803)
				(1938,0.0266803)
				(2296,0.0251556)
				(2745,0.0231709)
				(3400,0.0213513)
				(3400,0.0213513)
				(4256,0.0193084)
				(4256,0.0193084)
				(5568,0.0172117)
			};

			% SEAW ue_np.std[0] %%%%%%%%%%%%%%%%%%%%%%%%%%%%%%%%%%%%%%%%%%%%%%%%%%%%%%%%%%%%
			\addplot[SEAW] coordinates{
				(124.667,0.0982134)
				(349.414,0.0628387)
				(1192.36,0.0355588)
				(4497.65,0.020243)
			};

			\end{axis}
	\end{tikzpicture}
\end{subfigure}
\begin{subfigure}[b]{0.325\textwidth}\phantomsubcaption\label{subfig:experiments-quantitative-nyuv2-ev.std[0]}
	%%%%%%%%%%%%%%%%%%%%%%%%%%%%%%%%%%%%%%%%%%%%%%%%%%%%%%%%%%%%
	% ev.std[0]
	%%%%%%%%%%%%%%%%%%%%%%%%%%%%%%%%%%%%%%%%%%%%%%%%%%%%%%%%%%%%
	\begin{tikzpicture}
		\begin{axis}[AENYUV2EVStd,xmode=log]

			% CCS ev.std[0] %%%%%%%%%%%%%%%%%%%%%%%%%%%%%%%%%%%%%%%%%%%%%%%%%%%%%%%%%%%%
			\addplot[CCS] coordinates{
				(194.01,0.0489123)
				(282.682,0.0401298)
				(397.098,0.0336843)
				(597.153,0.0278887)
				(803.471,0.0241773)
				(925.103,0.0225563)
				(1177.26,0.0197739)
				(1397.83,0.0184758)
				(1588.43,0.0171542)
				(1878.46,0.0157239)
				(1878.46,0.0157239)
				(2231.81,0.014532)
				(2677.06,0.0133431)
				(3328.01,0.0117315)
				(3328.01,0.0117315)
				(4313.98,0.0102131)
				(4313.98,0.0102131)
				(5573.99,0.00898358)
			};

			% SEEDS ev.std[0] %%%%%%%%%%%%%%%%%%%%%%%%%%%%%%%%%%%%%%%%%%%%%%%%%%%%%%%%%%%%
			\addplot[SEEDS] coordinates{
				(246.043,0.0394671)
				(347.734,0.0385526)
				(450.607,0.0341413)
				(654.489,0.0285258)
				(857.654,0.0252009)
				(1057.59,0.0231715)
				(1258.91,0.0218216)
				(1462.49,0.0203239)
				(1661.42,0.018952)
				(1870.25,0.0232857)
				(2066.14,0.0188115)
				(2473.07,0.0163538)
				(2864.58,0.0154389)
				(3245.7,0.0147861)
				(3711.17,0.0157731)
				(4314.51,0.0127515)
				(4499.92,0.0124631)
				(4922.37,0.0130084)
			};

			% SLIC ev.std[0] %%%%%%%%%%%%%%%%%%%%%%%%%%%%%%%%%%%%%%%%%%%%%%%%%%%%%%%%%%%%
			\addplot[SLIC] coordinates{
				(184.484,0.0553731)
				(279.316,0.0469887)
				(385.211,0.0424889)
				(596.942,0.0353574)
				(819.424,0.0307801)
				(939.241,0.0288386)
				(1211.08,0.0257357)
				(1425.96,0.0241736)
				(1642.44,0.0221079)
				(1940.44,0.020221)
				(1940.44,0.020221)
				(2316.71,0.018471)
				(2774.73,0.0166249)
				(3441.91,0.0147761)
				(3441.91,0.0147761)
				(4395.74,0.0133521)
				(4395.74,0.0133521)
				(5687.73,0.0114304)
			};

			% RW ev.std[0] %%%%%%%%%%%%%%%%%%%%%%%%%%%%%%%%%%%%%%%%%%%%%%%%%%%%%%%%%%%%
			\addplot[RW] coordinates{
				(215.912,0.0758276)
				(306.055,0.06525)
				(440.393,0.0569324)
				(635.679,0.0499051)
				(846.511,0.0429785)
				(1060.11,0.0396088)
				(1306.99,0.0352079)
				(1464.51,0.0340521)
				(1685.01,0.0311162)
				(1868.17,0.0300957)
				(2109.97,0.0353271)
				(3967.51,0.0202916)
				(4401.89,0.0187576)
				(5110.46,0.017551)
			};

			% CW ev.std[0] %%%%%%%%%%%%%%%%%%%%%%%%%%%%%%%%%%%%%%%%%%%%%%%%%%%%%%%%%%%%
			\addplot[CW] coordinates{
				(198.614,0.0652669)
				(293.173,0.0564862)
				(407.073,0.0490461)
				(613.509,0.0419432)
				(834.84,0.0369468)
				(1056.32,0.034011)
				(1266,0.0313205)
				(1460.01,0.0292164)
				(1700.24,0.0280082)
				(1828.56,0.027228)
				(2009.84,0.0254785)
				(2449.61,0.0237984)
				(2991.17,0.0220255)
				(3233.9,0.0214399)
				(3707.85,0.0204831)
				(4124.98,0.0192825)
				(5174.47,0.0180598)
				(5281.06,0.0177301)
			};

			% TP ev.std[0] %%%%%%%%%%%%%%%%%%%%%%%%%%%%%%%%%%%%%%%%%%%%%%%%%%%%%%%%%%%%
			\addplot[TP] coordinates{
				(295.99,0.0464079)
				(395.489,0.0408204)
				(551.496,0.0445809)
				(775.682,0.0380898)
				(1000.76,0.0342093)
				(1130.49,0.0325001)
				(1401.47,0.0283666)
				(1572.01,0.028161)
				(1815.33,0.0259605)
				(2110.15,0.0243444)
			};

			% POISE ev.std[0] %%%%%%%%%%%%%%%%%%%%%%%%%%%%%%%%%%%%%%%%%%%%%%%%%%%%%%%%%%%%
			\addplot[POISE] coordinates{
				(204.942,0.0461548)
				(306.83,0.0396606)
				(408.614,0.0349837)
				(612.1,0.0313414)
				(815.501,0.0295511)
				(1019.01,0.0284966)
				(1222.5,0.0278406)
				(1425.91,0.0273438)
				(1628.65,0.0266885)
				(1830.61,0.0266985)
				(2031.68,0.026553)
				(2427.44,0.0259904)
				(2807.4,0.0257855)
				(3158.64,0.0258028)
				(3466.96,0.0257427)
				(3736.5,0.0255731)
				(4042.51,0.0255591)
				(4219.03,0.0255696)
			};

			% FH ev.std[0] %%%%%%%%%%%%%%%%%%%%%%%%%%%%%%%%%%%%%%%%%%%%%%%%%%%%%%%%%%%%
			\addplot[FH] coordinates{
				(689.802,0.0479828)
				(763.792,0.0433671)
				(813.815,0.0409735)
				(988.712,0.0360957)
				(1077.66,0.0356963)
				(1359.76,0.0301936)
				(1408.18,0.0303491)
				(1559.12,0.030061)
				(1923.5,0.0240525)
				(2206.32,0.0238309)
				(2448.19,0.023487)
				(2699.96,0.0175561)
				(3024.27,0.0172046)
				(3568.29,0.016341)
				(4199.72,0.0157428)
				(4594.6,0.0149445)
			};

			% EAMS ev.std[0] %%%%%%%%%%%%%%%%%%%%%%%%%%%%%%%%%%%%%%%%%%%%%%%%%%%%%%%%%%%%
			\addplot[EAMS] coordinates{
				(419.068,0.0287112)
				(455.657,0.0271249)
				(499.416,0.0263276)
				(641.188,0.0242659)
				(842.328,0.0208007)
				(1268.71,0.0169445)
				(2584.6,0.0120572)
				(2683.21,0.0118678)
				(2779.09,0.0115832)
				(2997.7,0.0110161)
				(3180.29,0.0106502)
				(3421.45,0.00995004)
				(3646.16,0.00945868)
				(3943.04,0.00898358)
				(4448.4,0.00831871)
				(5663.91,0.00697832)
				(9098.39,0.00534885)
			};

			% CRS ev.std[0] %%%%%%%%%%%%%%%%%%%%%%%%%%%%%%%%%%%%%%%%%%%%%%%%%%%%%%%%%%%%
			\addplot[CRS] coordinates{
				(254.263,0.0496507)
				(396.694,0.0409262)
				(514.895,0.0365975)
				(764.997,0.0314335)
				(988.589,0.0283939)
				(1232.03,0.0255463)
				(1498.25,0.0237106)
				(1707.65,0.022547)
				(1968.19,0.0211897)
				(2099.81,0.0207131)
				(2299.23,0.0196514)
				(2706.1,0.0184952)
				(3259.07,0.0170776)
				(3558.75,0.0163465)
				(4055.02,0.0154813)
				(4394.15,0.0149863)
				(5418.11,0.0137848)
				(5695.52,0.0134676)
			};

			% RESEEDS ev.std[0] %%%%%%%%%%%%%%%%%%%%%%%%%%%%%%%%%%%%%%%%%%%%%%%%%%%%%%%%%%%%
			\addplot[RESEEDS] coordinates{
				(200.441,0.028161)
				(300.263,0.0280931)
				(400.323,0.0240103)
				(600.361,0.0190977)
				(800.526,0.0167908)
				(1000.06,0.0144146)
				(1200.06,0.013525)
				(1400.96,0.0127609)
				(1600.21,0.011678)
				(1792.08,0.0145074)
				(1998.13,0.0111933)
				(2408.17,0.00962422)
				(2801.44,0.0092998)
				(3182.16,0.00863513)
				(3648.12,0.00935412)
				(4257.44,0.0075447)
				(4444.16,0.00724238)
				(4864.15,0.00782774)
			};

			% ERGC ev.std[0] %%%%%%%%%%%%%%%%%%%%%%%%%%%%%%%%%%%%%%%%%%%%%%%%%%%%%%%%%%%%
			\addplot[ERGC] coordinates{
				(196,0.0522004)
				(289,0.0454777)
				(400,0.0407517)
				(600,0.03377)
				(812,0.0284892)
				(1024,0.0262096)
				(1224,0.0240723)
				(1406,0.0224609)
				(1681,0.0209378)
				(1763,0.0205049)
				(1935,0.0193873)
				(2350,0.0174949)
				(2856,0.0160168)
				(3080,0.0153207)
				(3520,0.0141307)
				(3904,0.0134588)
				(4864,0.0119454)
				(5100,0.0116063)
			};

			% PF ev.std[0] %%%%%%%%%%%%%%%%%%%%%%%%%%%%%%%%%%%%%%%%%%%%%%%%%%%%%%%%%%%%
			\addplot[PF] coordinates{
				(281.19,0.108179)
				(430.376,0.0962549)
				(586.371,0.0890069)
				(907.19,0.0793153)
				(1201.57,0.0724263)
				(1354.07,0.0696802)
				(1718.72,0.0654306)
				(1939.43,0.0626196)
				(2251.38,0.0599657)
				(2592.38,0.0578263)
				(3186.03,0.0528901)
				(4180.57,0.0490218)
				(6182.51,0.0428235)
			};

			% TPS ev.std[0] %%%%%%%%%%%%%%%%%%%%%%%%%%%%%%%%%%%%%%%%%%%%%%%%%%%%%%%%%%%%
			\addplot[TPS] coordinates{
				(230.419,0.0747657)
				(313.739,0.0674858)
				(450.436,0.0601949)
				(638.008,0.0523868)
				(857.559,0.0466142)
				(1043.09,0.0436398)
				(1317.64,0.0394731)
				(1503.8,0.0376824)
				(1703.97,0.0361699)
				(1873.98,0.0348685)
				(2144.93,0.0330754)
				(2580.57,0.0302636)
				(2894.48,0.0287216)
				(3314.47,0.0272859)
				(3924.13,0.0258443)
				(4380.02,0.0247836)
				(5132.34,0.0225351)
				(5738.91,0.0213843)
			};

			% NC ev.std[0] %%%%%%%%%%%%%%%%%%%%%%%%%%%%%%%%%%%%%%%%%%%%%%%%%%%%%%%%%%%%
			\addplot[NC] coordinates{
				(439.952,0.0471912)
				(1154.47,0.0384721)
				(2631.49,0.0337656)
				(3874.82,0.0302587)
			};

			% VC ev.std[0] %%%%%%%%%%%%%%%%%%%%%%%%%%%%%%%%%%%%%%%%%%%%%%%%%%%%%%%%%%%%
			\addplot[VC] coordinates{
				(243.744,0.0544116)
				(400.291,0.0380429)
				(531.672,0.0317167)
				(660.84,0.0275674)
				(895.915,0.0228281)
				(1125.96,0.0198115)
				(1348.09,0.0178256)
				(1564.34,0.0162184)
				(1781.1,0.0150854)
				(1995.21,0.0142713)
				(2205.86,0.0134277)
				(2416.36,0.0127982)
				(2820.76,0.0118678)
				(3224.25,0.0109619)
				(3610.4,0.010309)
				(3988.04,0.00972587)
				(4351.59,0.00922257)
				(4847.23,0.00867988)
				(5262.69,0.00832229)
			};

			% PB ev.std[0] %%%%%%%%%%%%%%%%%%%%%%%%%%%%%%%%%%%%%%%%%%%%%%%%%%%%%%%%%%%%
			\addplot[PB] coordinates{
				(273.248,0.0827982)
				(360.188,0.0711304)
				(463.193,0.0658773)
				(656.962,0.0565859)
				(857.526,0.0504639)
				(970.915,0.0479958)
				(1217.27,0.0429514)
				(1387.49,0.0403468)
				(1616.5,0.0380523)
				(1896.95,0.0349948)
				(1896.95,0.0349948)
				(2243.04,0.0326583)
				(2681.85,0.0298701)
				(3316.29,0.0265732)
				(3316.29,0.0265732)
				(4151.92,0.0234985)
				(4151.92,0.0234985)
				(5425.24,0.0200523)
			};

			% VCCS ev.std[0] %%%%%%%%%%%%%%%%%%%%%%%%%%%%%%%%%%%%%%%%%%%%%%%%%%%%%%%%%%%%
			\addplot[VCCS] coordinates{
				(564.123,0.0679356)
				(600.283,0.0665793)
				(648.629,0.0660638)
				(755.632,0.0644688)
				(836.05,0.0629528)
				(1051.64,0.0624241)
				(1109.07,0.0642758)
				(1179.74,0.0611313)
				(1237.23,0.061714)
				(1322.17,0.0614391)
				(1400.72,0.0620996)
				(1526.22,0.0593899)
				(1630.95,0.0594942)
				(1756.59,0.0613935)
				(1914.09,0.0576446)
				(2024.07,0.0619665)
				(2258.58,0.0590008)
				(2558.8,0.0520878)
				(2780.39,0.0536399)
				(3044.22,0.0511357)
				(3687.7,0.0338476)
				(4171.76,0.0292775)
				(4255.91,0.0410469)
			};

			% PRESLIC ev.std[0] %%%%%%%%%%%%%%%%%%%%%%%%%%%%%%%%%%%%%%%%%%%%%%%%%%%%%%%%%%%%
			\addplot[PRESLIC] coordinates{
				(189.89,0.0590078)
				(388.539,0.0455452)
				(589.772,0.0388997)
				(798.767,0.0347032)
				(920.201,0.032081)
				(1188.65,0.0292399)
				(1401.21,0.0276354)
				(1612.25,0.0257218)
				(1904.93,0.0238222)
				(1904.93,0.0238222)
				(2282.12,0.0217655)
				(2738.74,0.0199748)
				(3422.72,0.0180317)
				(3422.72,0.0180317)
				(4393.15,0.0155792)
				(4393.15,0.0155792)
				(5724.22,0.0136282)
			};

			% W ev.std[0] %%%%%%%%%%%%%%%%%%%%%%%%%%%%%%%%%%%%%%%%%%%%%%%%%%%%%%%%%%%%
			\addplot[W] coordinates{
				(193.87,0.0718043)
				(303.997,0.0608219)
				(397.103,0.053454)
				(621.108,0.0455595)
				(870.236,0.0390778)
				(959.915,0.0372024)
				(1234.38,0.0334348)
				(1418.74,0.0321033)
				(1652.83,0.0297621)
				(1957.01,0.0271534)
				(1957.01,0.0271534)
				(2345.3,0.0255124)
				(2867.42,0.023138)
				(3518.42,0.0212079)
				(3518.42,0.0212079)
				(4497.46,0.0192856)
				(4497.46,0.0192856)
				(5940.33,0.0172996)
			};

			% LSC ev.std[0] %%%%%%%%%%%%%%%%%%%%%%%%%%%%%%%%%%%%%%%%%%%%%%%%%%%%%%%%%%%%
			\addplot[LSC] coordinates{
				(383.095,0.0288923)
				(552.409,0.0254738)
				(718.163,0.023049)
				(1059.33,0.0198971)
				(1414.95,0.0177955)
				(1816.61,0.0164501)
				(2009.32,0.0158504)
				(2263.42,0.0151643)
				(2398.35,0.0148665)
				(2446.8,0.014575)
				(2588.34,0.0143981)
				(2830.36,0.0141243)
				(3264.3,0.01341)
				(3438.52,0.0133208)
				(3802.19,0.0132287)
				(4059.6,0.0129808)
				(4900.83,0.0126718)
				(5001.09,0.0131745)
			};

			% WP ev.std[0] %%%%%%%%%%%%%%%%%%%%%%%%%%%%%%%%%%%%%%%%%%%%%%%%%%%%%%%%%%%%
			\addplot[WP] coordinates{
				(204,0.0594722)
				(315,0.0498107)
				(432,0.0438489)
				(638,0.0371968)
				(850,0.0320271)
				(1064,0.0293457)
				(1230,0.0277194)
				(1408,0.0260934)
				(1645,0.0247607)
				(1938,0.0224994)
				(1938,0.0224384)
				(2296,0.0205673)
				(2745,0.0188558)
				(3400,0.0168191)
				(3400,0.0168191)
				(4256,0.015514)
				(4256,0.015514)
				(5568,0.0137934)
			};

			% QS ev.std[0] %%%%%%%%%%%%%%%%%%%%%%%%%%%%%%%%%%%%%%%%%%%%%%%%%%%%%%%%%%%%
			\addplot[QS] coordinates{
				(223.521,0.0939754)
				(325.303,0.0707321)
				(414.609,0.0567951)
				(663.504,0.0400815)
				(828.486,0.0343354)
				(1077.18,0.0285676)
				(1252.44,0.0255684)
				(1475.56,0.0231019)
				(1767.37,0.0205325)
				(2165.34,0.0182763)
				(2703.49,0.0159385)
				(3460.91,0.0137415)
				(4543.63,0.0116217)
				(6144.37,0.00965822)
			};

			% CIS ev.std[0] %%%%%%%%%%%%%%%%%%%%%%%%%%%%%%%%%%%%%%%%%%%%%%%%%%%%%%%%%%%%
			\addplot[CIS] coordinates{
				(292.103,0.053676)
				(366.787,0.0468833)
				(436.398,0.0421714)
				(575.972,0.0371591)
				(721.491,0.0339259)
				(783.672,0.0325358)
				(963.875,0.0295259)
				(1083.42,0.0287993)
				(1227.51,0.0266192)
				(1408.35,0.0258363)
				(1424.97,0.0256336)
				(1751.64,0.0230309)
				(2118.16,0.0215785)
				(2717.81,0.0195083)
				(3240.8,0.0180383)
				(3240.8,0.0180383)
				(4824.68,0.0155753)
			};

			% RESEEDS3D ev.std[0] %%%%%%%%%%%%%%%%%%%%%%%%%%%%%%%%%%%%%%%%%%%%%%%%%%%%%%%%%%%%
			\addplot[RESEEDS3D] coordinates{
				(200.035,0.0796512)
				(300.105,0.0317195)
				(400.128,0.0267587)
				(600.233,0.0212192)
				(800.351,0.0184726)
				(1000.04,0.0159796)
				(1200.03,0.0147033)
				(1400.65,0.0137393)
				(1600.08,0.0126341)
				(1792.08,0.0145852)
				(1998.07,0.0119055)
				(2408.16,0.0103032)
				(2801.12,0.00999785)
				(3182.14,0.00905298)
				(3648.05,0.00942395)
				(4257.09,0.00793363)
				(4444.11,0.0077204)
				(4864.04,0.00787708)
			};

			% ERS ev.std[0] %%%%%%%%%%%%%%%%%%%%%%%%%%%%%%%%%%%%%%%%%%%%%%%%%%%%%%%%%%%%
			\addplot[ERS] coordinates{
				(200,0.062406)
				(300,0.0557074)
				(400,0.0516518)
				(600,0.0461574)
				(800,0.0428375)
				(1000,0.0407122)
				(1200,0.0385958)
				(1400,0.0372256)
				(1600,0.0359318)
				(1800,0.0346809)
				(2000,0.0335815)
				(2400,0.0317233)
				(2800,0.0301946)
				(3200,0.0288654)
				(3600,0.0276677)
				(4000,0.0265777)
				(4600,0.0249717)
				(5200,0.0237307)
			};

			% DASP ev.std[0] %%%%%%%%%%%%%%%%%%%%%%%%%%%%%%%%%%%%%%%%%%%%%%%%%%%%%%%%%%%%
			\addplot[DASP] coordinates{
				(417.987,0.0402069)
				(521.05,0.0339864)
				(620.704,0.0305917)
				(812.654,0.0256568)
				(1006.72,0.0227208)
				(1197.12,0.0205209)
				(1388.06,0.0189363)
				(1576.93,0.017634)
				(1767.2,0.0168828)
				(1947.7,0.0159684)
				(2131.65,0.0154002)
				(2494.37,0.0142692)
				(2860.93,0.0133766)
				(3211.97,0.012653)
				(3564.56,0.0120745)
				(3909.24,0.0115161)
				(4418.09,0.0109592)
				(4919.67,0.0104497)
			};

			% MSS ev.std[0] %%%%%%%%%%%%%%%%%%%%%%%%%%%%%%%%%%%%%%%%%%%%%%%%%%%%%%%%%%%%
			\addplot[MSS] coordinates{
				(207.576,0.0624809)
				(308.962,0.052006)
				(436.368,0.0459374)
				(668.784,0.0363031)
				(912.997,0.0313823)
				(1055.23,0.0288045)
				(1363.6,0.0270269)
				(1575.7,0.0252305)
				(1862.23,0.0236325)
				(2222.47,0.0218748)
				(2222.47,0.0218748)
				(2666.45,0.0209734)
				(3236.21,0.0200834)
				(4079.47,0.0189441)
				(4078.85,0.018859)
				(5209.16,0.0180053)
				(5209.16,0.0180053)
				(6994.34,0.0168987)
			};

			% ETPS ev.std[0] %%%%%%%%%%%%%%%%%%%%%%%%%%%%%%%%%%%%%%%%%%%%%%%%%%%%%%%%%%%%
			\addplot[ETPS] coordinates{
				(221,0.0238809)
				(315,0.0204409)
				(432,0.0183366)
				(638,0.015775)
				(850,0.0140227)
				(972,0.0134676)
				(1230,0.0122387)
				(1408,0.0116166)
				(1645,0.0108937)
				(1938,0.0101134)
				(1938,0.0101134)
				(2296,0.00943975)
				(2745,0.00870388)
				(3400,0.00788465)
				(3400,0.00788465)
				(4256,0.0070969)
				(4256,0.0070969)
				(5568,0.00620521)
			};

			% SEAW ev.std[0] %%%%%%%%%%%%%%%%%%%%%%%%%%%%%%%%%%%%%%%%%%%%%%%%%%%%%%%%%%%%
			\addplot[SEAW] coordinates{
				(124.667,0.0780303)
				(349.414,0.0515142)
				(1192.36,0.0304795)
				(4497.65,0.0168103)
			};

			\end{axis}
	\end{tikzpicture}
\end{subfigure}
\\[-8px]
	\begin{subfigure}[t]{0.325\textwidth}\phantomsubcaption\label{subfig:experiments-quantitative-nyuv2-sp.max[0]}
	%%%%%%%%%%%%%%%%%%%%%%%%%%%%%%%%%%%%%%%%%%%%%%%%%%%%%%%%%%%%
	% sp.max[0]
	%%%%%%%%%%%%%%%%%%%%%%%%%%%%%%%%%%%%%%%%%%%%%%%%%%%%%%%%%%%%
	\begin{tikzpicture}
		\begin{axis}[EQNYUV2KMax]

			% CCS sp.max[0] %%%%%%%%%%%%%%%%%%%%%%%%%%%%%%%%%%%%%%%%%%%%%%%%%%%%%%%%%%%%
			\addplot[CCS] coordinates{
				(194.01,209)
				(282.682,305)
				(397.098,432)
				(597.153,654)
				(803.471,856)
				(925.103,976)
				(1177.26,1256)
				(1397.83,1469)
				(1588.43,1671)
				(1878.46,1970)
				(1878.46,1970)
				(2231.81,2326)
				(2677.06,2797)
				(3328.01,3485)
				(3328.01,3485)
				(4313.98,4487)
				(4313.98,4487)
				(5573.99,5781)
			};

			% SEEDS sp.max[0] %%%%%%%%%%%%%%%%%%%%%%%%%%%%%%%%%%%%%%%%%%%%%%%%%%%%%%%%%%%%
			\addplot[SEEDS] coordinates{
				(246.043,309)
				(347.734,390)
				(450.607,498)
				(654.489,696)
				(857.654,902)
				(1057.59,1104)
				(1258.91,1301)
				(1462.49,1504)
				(1661.42,1703)
				(2066.14,2108)
				(2473.07,2517)
				(2864.58,2905)
				(3245.7,3290)
				(3711.17,3758)
				(4314.51,4347)
				(4499.92,4555)
				(4922.37,4959)
			};

			% SLIC sp.max[0] %%%%%%%%%%%%%%%%%%%%%%%%%%%%%%%%%%%%%%%%%%%%%%%%%%%%%%%%%%%%
			\addplot[SLIC] coordinates{
				(184.484,213)
				(279.316,326)
				(385.211,430)
				(596.942,661)
				(819.424,934)
				(939.241,1055)
				(1211.08,1361)
				(1425.96,1604)
				(1642.44,1850)
				(1940.44,2160)
				(1940.44,2160)
				(2316.71,2552)
				(2774.73,3058)
				(3441.91,3787)
				(3441.91,3787)
				(4395.74,4773)
				(4395.74,4773)
				(5687.73,6189)
			};

			% RW sp.max[0] %%%%%%%%%%%%%%%%%%%%%%%%%%%%%%%%%%%%%%%%%%%%%%%%%%%%%%%%%%%%
			\addplot[RW] coordinates{
				(215.912,255)
				(306.055,341)
				(440.393,466)
				(635.679,663)
				(846.511,877)
				(1060.11,1087)
				(1306.99,1343)
				(1464.51,1491)
				(1685.01,1720)
				(1868.17,1900)
				(2109.97,2141)
				(3967.51,4009)
				(4401.89,4441)
				(5110.46,5147)
			};

			% CW sp.max[0] %%%%%%%%%%%%%%%%%%%%%%%%%%%%%%%%%%%%%%%%%%%%%%%%%%%%%%%%%%%%
			\addplot[CW] coordinates{
				(198.614,215)
				(293.173,320)
				(407.073,450)
				(613.509,689)
				(834.84,944)
				(1056.32,1196)
				(1266,1435)
				(1460.01,1656)
				(1700.24,1917)
				(1828.56,2068)
				(2009.84,2284)
				(2449.61,2783)
				(2991.17,3401)
				(3233.9,3677)
				(3707.85,4209)
				(4124.98,4710)
				(5174.47,5939)
				(5281.06,6064)
			};

			% TP sp.max[0] %%%%%%%%%%%%%%%%%%%%%%%%%%%%%%%%%%%%%%%%%%%%%%%%%%%%%%%%%%%%
			\addplot[TP] coordinates{
				(295.99,426)
				(395.489,510)
				(551.496,717)
				(775.682,965)
				(1000.76,1200)
				(1130.49,1321)
				(1401.47,1615)
				(1572.01,1785)
				(1815.33,2044)
				(2110.15,2355)
			};

			% POISE sp.max[0] %%%%%%%%%%%%%%%%%%%%%%%%%%%%%%%%%%%%%%%%%%%%%%%%%%%%%%%%%%%%
			\addplot[POISE] coordinates{
				(204.942,205)
				(306.83,307)
				(408.614,409)
				(612.1,613)
				(815.501,817)
				(1019.01,1021)
				(1222.5,1225)
				(1425.91,1429)
				(1628.65,1633)
				(1830.61,1837)
				(2031.68,2041)
				(2427.44,2449)
				(2807.4,2857)
				(3158.64,3265)
				(3466.96,3673)
				(3736.5,4080)
				(4042.51,4689)
				(4219.03,5277)
			};

			% FH sp.max[0] %%%%%%%%%%%%%%%%%%%%%%%%%%%%%%%%%%%%%%%%%%%%%%%%%%%%%%%%%%%%
			\addplot[FH] coordinates{
				(689.802,1015)
				(763.792,1150)
				(813.815,1242)
				(988.712,1535)
				(1077.66,1594)
				(1359.76,2056)
				(1408.18,1941)
				(1559.12,2060)
				(1923.5,2814)
				(2206.32,2897)
				(2448.19,3059)
				(2699.96,4214)
				(3024.27,5370)
				(3568.29,6471)
				(4199.72,5332)
				(4594.6,5653)
			};

			% EAMS sp.max[0] %%%%%%%%%%%%%%%%%%%%%%%%%%%%%%%%%%%%%%%%%%%%%%%%%%%%%%%%%%%%
			\addplot[EAMS] coordinates{
				(419.068,581)
				(455.657,620)
				(499.416,676)
				(641.188,788)
				(842.328,999)
				(1268.71,1557)
				(2584.6,3268)
				(2683.21,3407)
				(2779.09,3585)
				(2997.7,3876)
				(3180.29,4199)
				(3421.45,4476)
				(3646.16,4899)
				(3943.04,5400)
				(4448.4,6197)
				(5663.91,8289)
				(9098.39,14535)
			};

			% CRS sp.max[0] %%%%%%%%%%%%%%%%%%%%%%%%%%%%%%%%%%%%%%%%%%%%%%%%%%%%%%%%%%%%
			\addplot[CRS] coordinates{
				(254.263,388)
				(396.694,582)
				(514.895,763)
				(764.997,1034)
				(988.589,1326)
				(1232.03,1633)
				(1498.25,1926)
				(1707.65,2171)
				(1968.19,2523)
				(2099.81,2642)
				(2299.23,2804)
				(2706.1,3267)
				(3259.07,3775)
				(3558.75,4242)
				(4055.02,4768)
				(4394.15,5068)
				(5418.11,6265)
				(5695.52,6491)
			};

			% SEAW sp.max[0] %%%%%%%%%%%%%%%%%%%%%%%%%%%%%%%%%%%%%%%%%%%%%%%%%%%%%%%%%%%%
			\addplot[SEAW] coordinates{
				(124.667,250)
				(349.414,533)
				(1192.36,1463)
				(4497.65,4871)
			};

			% RESEEDS sp.max[0] %%%%%%%%%%%%%%%%%%%%%%%%%%%%%%%%%%%%%%%%%%%%%%%%%%%%%%%%%%%%
			\addplot[RESEEDS] coordinates{
				(200.441,204)
				(300.263,303)
				(400.323,403)
				(600.361,603)
				(800.526,804)
				(1000.06,1001)
				(1200.06,1201)
				(1400.96,1405)
				(1600.21,1602)
				(1792.08,1793)
				(1998.13,2000)
				(2408.17,2410)
				(2801.44,2807)
				(3182.16,3184)
				(3648.12,3650)
				(4257.44,4263)
				(4444.16,4447)
				(4864.15,4866)
			};

			% ERGC sp.max[0] %%%%%%%%%%%%%%%%%%%%%%%%%%%%%%%%%%%%%%%%%%%%%%%%%%%%%%%%%%%%
			\addplot[ERGC] coordinates{
				(196,196)
				(289,289)
				(400,400)
				(600,600)
				(812,812)
				(1024,1024)
				(1224,1224)
				(1406,1406)
				(1681,1681)
				(1763,1763)
				(1935,1935)
				(2350,2350)
				(2856,2856)
				(3080,3080)
				(3520,3520)
				(3904,3904)
				(4864,4864)
				(5100,5100)
			};

			% PF sp.max[0] %%%%%%%%%%%%%%%%%%%%%%%%%%%%%%%%%%%%%%%%%%%%%%%%%%%%%%%%%%%%
			\addplot[PF] coordinates{
				(281.19,590)
				(430.376,1019)
				(586.371,1101)
				(907.19,1896)
				(1201.57,2182)
				(1354.07,2602)
				(1718.72,3184)
				(1939.43,3035)
				(2251.38,3474)
				(2592.38,4807)
				(3186.03,5066)
				(4180.57,6271)
				(6182.51,10057)
			};

			% TPS sp.max[0] %%%%%%%%%%%%%%%%%%%%%%%%%%%%%%%%%%%%%%%%%%%%%%%%%%%%%%%%%%%%
			\addplot[TPS] coordinates{
				(230.419,249)
				(313.739,341)
				(450.436,475)
				(638.008,676)
				(857.559,896)
				(1043.09,1082)
				(1317.64,1374)
				(1503.8,1573)
				(1703.97,1750)
				(1873.98,1938)
				(2144.93,2198)
				(2580.57,2633)
				(2894.48,2966)
				(3314.47,3414)
				(3924.13,3957)
				(4380.02,4435)
				(5132.34,5189)
				(5738.91,5792)
			};

			% NC sp.max[0] %%%%%%%%%%%%%%%%%%%%%%%%%%%%%%%%%%%%%%%%%%%%%%%%%%%%%%%%%%%%
			\addplot[NC] coordinates{
				(439.952,483)
				(1154.47,1228)
				(2631.49,2667)
				(3874.82,3914)
			};

			% VC sp.max[0] %%%%%%%%%%%%%%%%%%%%%%%%%%%%%%%%%%%%%%%%%%%%%%%%%%%%%%%%%%%%
			\addplot[VC] coordinates{
				(243.744,820)
				(400.291,1190)
				(531.672,1358)
				(660.84,1499)
				(895.915,1797)
				(1125.96,2231)
				(1348.09,2319)
				(1564.34,2648)
				(1781.1,2946)
				(1995.21,3263)
				(2205.86,3387)
				(2416.36,3658)
				(2820.76,4057)
				(3224.25,4548)
				(3610.4,4948)
				(3988.04,5475)
				(4351.59,5921)
				(4847.23,6429)
				(5262.69,6968)
			};

			% PB sp.max[0] %%%%%%%%%%%%%%%%%%%%%%%%%%%%%%%%%%%%%%%%%%%%%%%%%%%%%%%%%%%%
			\addplot[PB] coordinates{
				(273.248,480)
				(360.188,591)
				(463.193,680)
				(656.962,850)
				(857.526,1029)
				(970.915,1130)
				(1217.27,1368)
				(1387.49,1571)
				(1616.5,1763)
				(1896.95,2029)
				(1896.95,2029)
				(2243.04,2333)
				(2681.85,2793)
				(3316.29,3414)
				(3316.29,3414)
				(4151.92,4232)
				(4151.92,4232)
				(5425.24,5502)
			};

			% VCCS sp.max[0] %%%%%%%%%%%%%%%%%%%%%%%%%%%%%%%%%%%%%%%%%%%%%%%%%%%%%%%%%%%%
			\addplot[VCCS] coordinates{
				(564.123,1166)
				(600.283,1290)
				(648.629,1341)
				(755.632,1544)
				(836.05,1749)
				(1051.64,2425)
				(1109.07,2461)
				(1179.74,2670)
				(1237.23,2771)
				(1322.17,3128)
				(1400.72,3105)
				(1526.22,3271)
				(1630.95,3574)
				(1756.59,3875)
				(1914.09,4207)
				(2024.07,4443)
				(2258.58,4966)
				(2558.8,5714)
				(2780.39,6260)
				(3044.22,6936)
				(3687.7,8413)
				(4171.76,9865)
				(4255.91,9425)
			};

			% PRESLIC sp.max[0] %%%%%%%%%%%%%%%%%%%%%%%%%%%%%%%%%%%%%%%%%%%%%%%%%%%%%%%%%%%%
			\addplot[PRESLIC] coordinates{
				(189.89,216)
				(388.539,431)
				(589.772,642)
				(798.767,872)
				(920.201,991)
				(1188.65,1293)
				(1401.21,1551)
				(1612.25,1760)
				(1904.93,2047)
				(1904.93,2047)
				(2282.12,2441)
				(2738.74,2899)
				(3422.72,3765)
				(3422.72,3765)
				(4393.15,4782)
				(4393.15,4782)
				(5724.22,6035)
			};

			% W sp.max[0] %%%%%%%%%%%%%%%%%%%%%%%%%%%%%%%%%%%%%%%%%%%%%%%%%%%%%%%%%%%%
			\addplot[W] coordinates{
				(193.87,210)
				(303.997,333)
				(397.103,447)
				(621.108,693)
				(870.236,983)
				(959.915,1081)
				(1234.38,1398)
				(1418.74,1614)
				(1652.83,1874)
				(1957.01,2225)
				(1957.01,2225)
				(2345.3,2665)
				(2867.42,3249)
				(3518.42,3997)
				(3518.42,3997)
				(4497.46,5161)
				(4497.46,5161)
				(5940.33,6817)
			};

			% LSC sp.max[0] %%%%%%%%%%%%%%%%%%%%%%%%%%%%%%%%%%%%%%%%%%%%%%%%%%%%%%%%%%%%
			\addplot[LSC] coordinates{
				(383.095,556)
				(552.409,755)
				(718.163,986)
				(1059.33,1404)
				(1414.95,1860)
				(1816.61,2318)
				(2009.32,2545)
				(2263.42,2816)
				(2398.35,2953)
				(2446.8,3047)
				(2588.34,3178)
				(2830.36,3474)
				(3264.3,3880)
				(3438.52,4071)
				(3802.19,4375)
				(4059.6,4544)
				(4900.83,5269)
				(5001.09,5394)
			};

			% WP sp.max[0] %%%%%%%%%%%%%%%%%%%%%%%%%%%%%%%%%%%%%%%%%%%%%%%%%%%%%%%%%%%%
			\addplot[WP] coordinates{
				(204,204)
				(315,315)
				(432,432)
				(638,638)
				(850,850)
				(1064,1064)
				(1230,1230)
				(1408,1408)
				(1645,1645)
				(1938,1938)
				(1938,1938)
				(2296,2296)
				(2745,2745)
				(3400,3400)
				(3400,3400)
				(4256,4256)
				(4256,4256)
				(5568,5568)
			};

			% QS sp.max[0] %%%%%%%%%%%%%%%%%%%%%%%%%%%%%%%%%%%%%%%%%%%%%%%%%%%%%%%%%%%%
			\addplot[QS] coordinates{
				(223.521,722)
				(325.303,952)
				(414.609,1190)
				(663.504,1920)
				(828.486,2382)
				(1077.18,3139)
				(1252.44,3700)
				(1475.56,4217)
				(1767.37,5023)
				(2165.34,6137)
				(2703.49,7446)
				(3460.91,9304)
				(4543.63,11974)
				(6144.37,15990)
			};

			% CIS sp.max[0] %%%%%%%%%%%%%%%%%%%%%%%%%%%%%%%%%%%%%%%%%%%%%%%%%%%%%%%%%%%%
			\addplot[CIS] coordinates{
				(292.103,824)
				(366.787,887)
				(436.398,1003)
				(575.972,1101)
				(721.491,1250)
				(783.672,1356)
				(963.875,1589)
				(1083.42,1715)
				(1227.51,1899)
				(1408.35,2108)
				(1424.97,2178)
				(1751.64,2901)
				(2118.16,3157)
				(2717.81,4752)
				(3240.8,5034)
				(3240.8,5034)
				(4824.68,7850)
			};

			% RESEEDS3D sp.max[0] %%%%%%%%%%%%%%%%%%%%%%%%%%%%%%%%%%%%%%%%%%%%%%%%%%%%%%%%%%%%
			\addplot[RESEEDS3D] coordinates{
				(200.035,203)
				(300.105,302)
				(400.128,403)
				(600.233,603)
				(800.351,803)
				(1000.04,1001)
				(1200.03,1201)
				(1400.65,1404)
				(1600.08,1602)
				(1792.08,1793)
				(1998.07,1999)
				(2408.16,2410)
				(2801.12,2809)
				(3182.14,3184)
				(3648.05,3649)
				(4257.09,4264)
				(4444.11,4446)
				(4864.04,4865)
			};

			% ERS sp.max[0] %%%%%%%%%%%%%%%%%%%%%%%%%%%%%%%%%%%%%%%%%%%%%%%%%%%%%%%%%%%%
			\addplot[ERS] coordinates{
				(200,200)
				(300,300)
				(400,400)
				(600,600)
				(800,800)
				(1000,1000)
				(1200,1200)
				(1400,1400)
				(1600,1600)
				(1800,1800)
				(2000,2000)
				(2400,2400)
				(2800,2800)
				(3200,3200)
				(3600,3600)
				(4000,4000)
				(4600,4600)
				(5200,5200)
			};

			% DASP sp.max[0] %%%%%%%%%%%%%%%%%%%%%%%%%%%%%%%%%%%%%%%%%%%%%%%%%%%%%%%%%%%%
			\addplot[DASP] coordinates{
				(417.987,1125)
				(521.05,1266)
				(620.704,1280)
				(812.654,1502)
				(1006.72,1657)
				(1197.12,1821)
				(1388.06,2002)
				(1576.93,2197)
				(1767.2,2367)
				(1947.7,2549)
				(2131.65,2776)
				(2494.37,3120)
				(2860.93,3443)
				(3211.97,3759)
				(3564.56,4102)
				(3909.24,4421)
				(4418.09,4923)
				(4919.67,5390)
			};

			% MSS sp.max[0] %%%%%%%%%%%%%%%%%%%%%%%%%%%%%%%%%%%%%%%%%%%%%%%%%%%%%%%%%%%%
			\addplot[MSS] coordinates{
				(207.576,271)
				(308.962,403)
				(436.368,573)
				(668.784,887)
				(912.997,1232)
				(1055.23,1403)
				(1363.6,1822)
				(1575.7,2093)
				(1862.23,2469)
				(2222.47,2950)
				(2222.47,2950)
				(2666.45,3510)
				(3236.21,4308)
				(4079.47,5389)
				(4078.85,5388)
				(5209.16,6867)
				(5209.16,6867)
				(6994.34,9122)
			};

			% ETPS sp.max[0] %%%%%%%%%%%%%%%%%%%%%%%%%%%%%%%%%%%%%%%%%%%%%%%%%%%%%%%%%%%%
			\addplot[ETPS] coordinates{
				(221,221)
				(315,315)
				(432,432)
				(638,638)
				(850,850)
				(972,972)
				(1230,1230)
				(1408,1408)
				(1645,1645)
				(1938,1938)
				(1938,1938)
				(2296,2296)
				(2745,2745)
				(3400,3400)
				(3400,3400)
				(4256,4256)
				(4256,4256)
				(5568,5568)
			};

				\end{axis}
	\end{tikzpicture}
\end{subfigure}
\begin{subfigure}[t]{\fullthreetwo\textwidth}\phantomsubcaption\label{subfig:experiments-quantitative-nyuv2-sp.std[0]}
	\vspace{5px}
	\begin{tikzpicture}
		\begin{axis}[EQNYUV2K,%
				symbolic x coords = {
					W,
					EAMS,
					NC,
					FH,
					RW,
					QS,
					PF,
					TP,
					CIS,
					SLIC,
					%vlSLIC,
					%SLIC3D,
					CRS,
					ERS,
					PB,
					DASP,
					SEEDS,
					%reSEEDS,
					%reSEEDS3D,
					TPS,
					VC,
					CCS,
					VCCS,
					CW,
					ERGC,
					MSS,
					preSLIC,
					WP,
					%LRW,
					ETPS,
					LSC,
					POISE,
					SEAW,
					%reFH,
				},
			]

			%%%%%%%%%%%%%%%%%%%%%%%%%%%%%%%%%%%%%%%%%%%%%%%%%%%%%%%%%%%%
			% sp.std[0]
			%%%%%%%%%%%%%%%%%%%%%%%%%%%%%%%%%%%%%%%%%%%%%%%%%%%%%%%%%%%%
			\addplot[fill=blue] coordinates {
				(CCS,3.34244)
				(SEEDS,15.7961)
				(SLIC,10.6837)
				(RW,6.80562)
				(CW,3.29417)
				(TP,30.8523)
				(POISE,0.257694)
				(FH,83.6101)
				(EAMS,40.3982)
				(CRS,18.4147)
				(SEAW,29.7198)
				%(reSEEDS,0.670238)
				(ERGC,0)
				(PF,79.4884)
				(TPS,4.90257)
				(NC,14.0796)
				(VC,108.912)
				(PB,49.5457)
				(VCCS,195.482)
				(preSLIC,8.74598)
				(W,2.82013)
				(LSC,49.7049)
				(WP,0)
				(QS,98.6223)
				(CIS,134.67)
				%(reSEEDS3D,0.216506)
				(ERS,0)
				(DASP,138.984)
				(MSS,15.4095)
				(ETPS,0)
			};

		\end{axis}
	\end{tikzpicture}
	\vspace{-18px}
	\begin{center}{\scriptsize(k)}\end{center}
\end{subfigure}
\\[-4px]
	\caption{Quantitative results on the \NYU dataset; remember that \K denotes the number of generated superpixels.
    The presented experimental results complement the discussion in Figure \ref{fig:experiments-quantitative-bsds500}
    and show that most observations can be confirmed across datasets. Furthermore,
    \DASP and \VCCS show inferior performance suggesting that depth information does
    not necessarily improve performance.
	\textbf{Best viewed in color.}}
	\label{fig:experiments-quantitative-nyuv2}
	\vskip 12px
	% Total: 26 (+2 depth)
%\begin{mdframed}
	{\scriptsize
		\begin{tabularx}{\textwidth}{X X X X X X X X X l}
			\ref{plot:w} \W &
			\ref{plot:eams} \EAMS &
			\ref{plot:nc} \NC &
			\ref{plot:fh} \FH &
			\ref{plot:rw} \RW &
			\ref{plot:qs} \QS &
			\ref{plot:pf} \PF &
			\ref{plot:tp} \TP &
			\ref{plot:cis} \CIS \\
			\ref{plot:slic} \SLIC &
			\ref{plot:crs} \CRS &
			\ref{plot:ers} \ERS &
			\ref{plot:pb} \PB &
			\ref{plot:dasp} \DASP &
			\ref{plot:seeds} \SEEDS &
			\ref{plot:tps} \TPS &
			\ref{plot:vc} \VC &
			\ref{plot:ccs} \CCS \\
			\ref{plot:vccs} \VCCS &
			\ref{plot:cw} \CW &
			\ref{plot:ergc} \ERGC &
			\ref{plot:mss} \MSS &
			\ref{plot:preslic} \preSLIC &
			\ref{plot:wp} \WP &
			\ref{plot:etps} \ETPS &
			\ref{plot:lsc} \LSC &
			\ref{plot:poise} \POISE \\
			\ref{plot:seaw} \SEAW & & & & & & & &
		\end{tabularx}
	}
%\end{mdframed}

\end{figure*}

Performance is determined by \Rec, \UE and \EV. In contrast to most authors,
we will look beyond metric averages. In particular, we consider the
minimum/maximum as well as the standard deviation to get an impression of the behavior of superpixel algorithms.
Furthermore, this allows us to quantify the stability of superpixel algorithms as
also considered by Neubert and Protzel in~\cite{NeubertProtzel:2013}.

\Rec and \UE offer a ground truth dependent overview to assess the performance of
superpixel algorithms. We consider
Figures \ref{subfig:experiments-quantitative-bsds500-rec.mean_min} and \ref{subfig:experiments-quantitative-bsds500-ue_np.mean_max},
showing \Rec and \UE on the \BSDS dataset. With respect to \Rec, we can easily identify top performing
algorithms, such as \ETPSr and \SEEDSr, as well as low performing algorithms,
such as \FHr, \QSr and \PFr. However, the remaining algorithms lie closely together
in between these two extremes, showing (apart from some exceptions) similar performance
especially for large~\K. Still, some algorithms perform consistently better than others,
as for example \ERGCr, \SLICr, \ERSr and \CRSr. For \UE, low performing algorithms,
such as \PFr or \QSr, are still easily identified while the remaining algorithms
tend to lie more closely together. Nevertheless, we can identify algorithms consistently
demonstrating good performance, such as \ERGCr, \ETPSr, \CRSr, \SLICr and \ERSr.
On the \NYU dataset, considering Figures \ref{subfig:experiments-quantitative-nyuv2-rec.mean[0]} and \ref{subfig:experiments-quantitative-nyuv2-ue_np.mean[0]},
these observations can be confirmed except for minor differences as for example the
excellent performance of \ERS regarding \UE or the better performance of \QS regarding \UE.
Overall, \Rec and \UE provide a quick overview of superpixel algorithm performance
but might not be sufficient to reliably discriminate superpixel algorithms.

In contrast to \Rec and \UE, \EV offers a ground truth independent assessment of superpixel algorithms.
Considering Figure \ref{subfig:experiments-quantitative-bsds500-ev.mean_min}, showing \EV on the \BSDS dataset,
we observe that algorithms are dragged apart and even for large \K
significantly different \EV values are attained. This suggests, that considering
ground truth independent metrics may be beneficial for comparison. However, \EV
cannot replace \Rec or \UE, as we can observe when comparing to
Figures \ref{subfig:experiments-quantitative-bsds500-rec.mean_min} and \ref{subfig:experiments-quantitative-bsds500-ue_np.mean_max},
showing \Rec and \UE on the \BSDS dataset; in particular
\QSr, \FHr and \CISr are performing significantly better with respect to \EV than regarding \Rec and \UE.
This suggests that \EV may be used to identify poorly performing algorithms, such as \TPSr, \PFr, \PBr or \NCr.
On the other hand, \EV is not necessarily suited to identify well-performing algorithms
due to the lack of underlying ground truth.
Overall, \EV is suitable to complement the view provided by \Rec and \UE,
however, should not be considered in isolation.

The stability of superpixel algorithms can be quantified by $\min\Rec$, $\max\UE$ and $\min\EV$ considering the behavior for increasing \K.
We consider Figures \ref{subfig:experiments-quantitative-bsds500-rec.min_min}, \ref{subfig:experiments-quantitative-bsds500-ue_np.max_max}
and \ref{subfig:experiments-quantitative-bsds500-ev.min_min},
showing $\min\Rec$, $\max\UE$ and $\min\EV$ on the \BSDS dataset. We define the stability of superpixel algorithms
as follows: an algorithm is considered stable if performance monotonically increases with \K
(\ie monotonically increasing \Rec and \EV and monotonically decreasing \UE).
Furthermore, these experiments can be interpreted as empirical bounds on the performance.
For example algorithms such as \ETPSr, \ERGCr, \ERSr, \CRSr and \SLICr can be considered stable and provide good bounds.
In contrast, algorithms such as \EAMSr, \FHr, \VCr or \POISEr are punished by
considering $\min\Rec$, $\max\UE$ and $\min\EV$ and cannot be described as stable.
Especially oversegmentation algorithms show poor stability. Most strikingly,
\EAMS seems to perform especially poorly on at least one image from the \BSDS dataset.
Overall, we find that $\min\Rec$, $\max\UE$ and $\min\EV$ appropriately reflect
the stability of superpixel algorithms.

The minimum/maximum of \Rec, \UE and \EV captures lower/upper bounds on performance.
In contrast, the corresponding standard deviation can be thought of as the expected
deviation from the average performance.
We consider Figures \ref{subfig:appendix-experiments-bsds500-rec.std[0]},
\ref{subfig:appendix-experiments-bsds500-ue_np.std[0]} and \ref{subfig:appendix-experiments-bsds500-ev.std[0]}
showing the standard deviation of \Rec, \UE and \EV on the \BSDS dataset.
We can observe that in many cases good performing algorithms such as \ETPS, \CRS, \SLIC or \ERS
also demonstrate low standard deviation. Oversegmentation algorithms, on the other hand, show higher standard deviation
-- together with algorithms such as \PF, \TPS, \VC, \CIS and \SEAW.
In this sense, stable algorithms can also be identified by low and monotonically decreasing standard deviation.

The variation in the number of generated superpixels is an important aspect
for many superpixel algorithms. In particular, high standard deviation in the number of generated superpixels can be related to
poor performance regarding \Rec, \UE and \EV. We find that superpixel algorithms ensuring that
the desired number of superpixels is met within appropriate bounds are preferrable. We consider
Figures \ref{subfig:experiments-quantitative-bsds500-sp.max[0]} and \ref{subfig:experiments-quantitative-bsds500-sp.std[0]},
showing $\max\K$ and $\text{std }\K$ for $\K \approx 400$ on the \BSDS dataset. Even after enforcing connectivity
as described in Section \ref{subsec:parameter-optimization-connectivity}, we observe
that several implementations are not always able to meet the desired number of superpixels
within acceptable bounds. Among these algorithms are \QSr, \VCr, \FHr, \CISr and \LSCr.
Except for the latter case, this can be related to poor performance with respect
to \Rec, \UE and~\EV. Conversely, considering algorithms such as \ETPSr, \ERGCr or \ERSr
which guarantee that the desired number of superpixels is met exactly, this can be
related to good performance regarding these metrics. To draw similar conclusions
for algorithms utilizing depth information, \ie \DASP and \VCCS,
the reader is encouraged to consider
Figures \ref{subfig:experiments-quantitative-nyuv2-sp.max[0]} and \ref{subfig:experiments-quantitative-nyuv2-sp.std[0]},
showing $\max\K$ and $\text{std }\K$ for $\K\approx 400$ on the \NYU dataset.
We can conclude that superpixel algorithms with low standard deviation in the number
of generated superpixels are showing better performance in many cases.

Finally, we discuss the proposed metrics \ARec, \AUE and \AEV (computed as the area
below the $\MR = (1 - \Rec)$, \UE and $\UEV = (1 - \EV)$ curves within the interval $[\K_{\min}, \K_{\max}] = [200,5200]$, \ie lower is better).
We find that these metrics appropriately reflect and summarize the performance of superpixel
algorithms independent of \K. As can be seen in Figure \ref{subfig:experiments-quantitative-bsds500-average},
showing \ref{plot:experiments-quantitative-bsds500-average-rec} \ARec, \ref{plot:experiments-quantitative-bsds500-average-ue_np}
\AUE and \ref{plot:experiments-quantitative-bsds500-average-ev} \AEV on the \BSDS dataset, most of the
previous observations can be confirmed. For example, we exemplarily consider \SEEDSr
and observe low \ARec and \AEV which is confirmed by
Figures \ref{subfig:experiments-quantitative-bsds500-rec.mean_min} and \ref{subfig:experiments-quantitative-bsds500-ev.mean_min},
showing \Rec and \EV on the \BSDS dataset, where \SEEDSr consistently outperforms all algorithms except for \ETPSr.
However, we can also observe higher \AUE compared to algorithms such as
\ETPSr, \ERSr or \CRSr wich is also consistent with Figure \ref{subfig:experiments-quantitative-bsds500-ue_np.mean_max},
showing \UE on the \BSDS dataset. We conclude, that \ARec, \AUE and \AEV give an easy-to-understand summary of algorithm performance.
Furthermore, \ARec, \AUE and \AEV can be used to rank the different
algorithms according to the corresponding metrics; we will follow up on this idea in Section \ref{subsec:experiments-ranking}.

The observed \ARec, \AUE and \AEV also properly reflect the difficulty of the different datasets. We
consider Figure \ref{fig:experiments-quantitative-avg} showing \ref{plot:experiments-quantitative-bsds500-average-rec} \ARec,
\ref{plot:experiments-quantitative-bsds500-average-ue_np} \AUE and \ref{plot:experiments-quantitative-bsds500-average-ev}  \AEV for all five datasets.
Concentrating on \SEEDS and \ETPS, we see that the relative
performance (\ie the performance of \SEEDS compared to \ETPS) is consistent across
datasets; \SEEDS usually showing higher \AUE while \ARec and \AEV are usually similar.
Therefore, we observe that these metrics can be used to characterize
the difficulty and ground truth of the datasets. For example, considering
the \Fash dataset, we observe very high \AEV compared
to the other datasets, while \ARec and \AUE are usually very low. This can be
explained by the ground truth shown in Figure~\ref{subfig:datasets-fash},
\ie the ground truth is limited to the foreground (in the case of Figure \ref{subfig:datasets-fash}, the woman),
leaving even complicated background unannotated. Similar arguments can be developed
for the consistently lower \ARec, \AUE and \AEV for the \NYU and \SUNRGBD datasets compared to the \BSDS dataset.
For the \SBD dataset, lower \ARec, \AUE and \AEV can also be explained by the smaller average image size.

In conclusion, \ARec, \AUE and \AEV accurately reflect the performance of superpixel
algorithms and can be used to judge datasets. Across the different datasets, path-based
and density-based algorithms perform poorly, while the remaining classes show mixed
performance. However, some iterative energy optimization, clustering-based and
graph-based algorithms such as \ETPS, \SEEDS, \CRS, \ERS and \SLIC show favorable performance.

\begin{figure*}
	\centering
	%\begin{subfigure}[b]{\fullthreethree\textwidth}\phantomsubcaption\label{subfig:experiments-quantitative-bsds500-average}
%	\begin{tikzpicture}
%		\begin{axis}[EQBSDS500Avg,%
%				symbolic x coords = {
%					W,
%					EAMS,
%					NC,
%					FH,
%					RW,
%					QS,
%					PF,
%					TP,
%					CIS,
%					SLIC,
%					CRS,
%					ERS,
%					PB,
%					SEEDS,
%					TPS,
%					VC,
%					CCS,
%					CW,
%					ERGC,
%					MSS,
%					preSLIC,
%					WP,
%					ETPS,
%					LSC,
%					POISE,
%					SEAW,
%				},
%			]
%
%			%%%%%%%%%%%%%%%%%%%%%%%%%%%%%%%%%%%%%%%%%%%%%%%%%%%%%%%%%%%%
%			% rec
%			%%%%%%%%%%%%%%%%%%%%%%%%%%%%%%%%%%%%%%%%%%%%%%%%%%%%%%%%%%%%
%			\addplot[fill=blue] coordinates {
%				(CCS,9.9318965816336)
%				(SEEDS,2.28232472)
%				(SLIC,8.6849153136699)
%				(RW,13.347694664807)
%				(CW,10.508734794086)
%				(TP,16.91966442)
%				(POISE,13.09785248)
%				(FH,15.02527176)
%				(EAMS,6.9713463075023)
%				(CRS,6.4499133686543)
%				(SEAW,18.718694626414)
%				(ERGC,7.0299239643847)
%				(PF,32.339046914181)
%				(TPS,16.608716929614)
%				(NC,14.5729878)
%				(VC,16.099002920902)
%				(PB,12.475172916609)
%				(preSLIC,10.364926165195)
%				(W,10.512659928738)
%				(LSC,9.11646619)
%				(WP,12.859419948232)
%				(QS,26.357822257892)
%				(CIS,17.742002940095)
%				(ERS,6.227106)
%				(MSS,11.451229616209)
%				(ETPS,2.9666871716831)
%			};
%			\label{plot:experiments-quantitative-bsds500-average-rec}
%
%			%%%%%%%%%%%%%%%%%%%%%%%%%%%%%%%%%%%%%%%%%%%%%%%%%%%%%%%%%%%%
%			% ue_np
%			%%%%%%%%%%%%%%%%%%%%%%%%%%%%%%%%%%%%%%%%%%%%%%%%%%%%%%%%%%%%
%			\addplot[fill=red] coordinates {
%				(CCS,7.5289953858794)
%				(SEEDS,9.446739759)
%				(SLIC,7.6497140156322)
%				(RW,8.6756965619526)
%				(CW,8.4402624332611)
%				(TP,9.271960248)
%				(POISE,8.924878757)
%				(FH,10.243686489)
%				(EAMS,8.2105337749209)
%				(CRS,7.3448022077077)
%				(SEAW,10.305148679693)
%				(ERGC,7.1872745931238)
%				(PF,17.356643633834)
%				(TPS,8.9472800577507)
%				(NC,9.334483981)
%				(VC,10.953437795205)
%				(PB,9.3881740126911)
%				(preSLIC,8.1143337244377)
%				(W,8.7587242552172)
%				(LSC,8.969104772)
%				(WP,8.3393118766649)
%				(QS,16.426937818043)
%				(CIS,7.883995044819)
%				(ERS,7.6528118)
%				(MSS,8.2658572868153)
%				(ETPS,7.0390794600518)
%			};
%			\label{plot:experiments-quantitative-bsds500-average-ue_np}
%
%			%%%%%%%%%%%%%%%%%%%%%%%%%%%%%%%%%%%%%%%%%%%%%%%%%%%%%%%%%%%%
%			% ev
%			%%%%%%%%%%%%%%%%%%%%%%%%%%%%%%%%%%%%%%%%%%%%%%%%%%%%%%%%%%%%
%			\addplot[fill=black] coordinates {
%				(CCS,8.810595522855)
%				(SEEDS,7.32645749)
%				(SLIC,10.695688557583)
%				(RW,12.150049603053)
%				(CW,13.965614906343)
%				(TP,14.73410334)
%				(POISE,14.2154631)
%				(FH,12.70897273)
%				(EAMS,7.0888362851786)
%				(CRS,10.759921739142)
%				(SEAW,11.431681615602)
%				(ERGC,8.1405080708485)
%				(PF,23.576251534528)
%				(TPS,16.153750090356)
%				(NC,16.62427233)
%				(VC,11.799903143852)
%				(PB,15.551225371015)
%				(preSLIC,10.942140607989)
%				(W,14.204407365309)
%				(LSC,11.99522686)
%				(WP,12.032242915515)
%				(QS,14.490709318973)
%				(CIS,9.332430822)
%				(ERS,13.001537)
%				(MSS,13.02655533315)
%				(ETPS,5.3336737754583)
%			};
%			\label{plot:experiments-quantitative-bsds500-average-ev}
%
%		\end{axis}
%	\end{tikzpicture}
%\end{subfigure}
\begin{subfigure}[b]{\fullthreethree\textwidth}\phantomsubcaption\label{subfig:experiments-quantitative-bsds500-average}
	\begin{tikzpicture}
		\begin{axis}[EQBSDS500Avg,%
				symbolic x coords = {
					W,
					EAMS,
					NC,
					FH,
					RW,
					QS,
					PF,
					TP,
					CIS,
					SLIC,
					%vlSLIC,
					%SLIC3D,
					CRS,
					ERS,
					PB,
					(DASP),
					SEEDS,
					%reSEEDS,
					%reSEEDS3D,
					TPS,
					VC,
					CCS,
					(VCCS),
					CW,
					ERGC,
					MSS,
					preSLIC,
					WP,
					%LRW,
					ETPS,
					LSC,
					POISE,
					SEAW,
					%reFH,
				},
			]

			%%%%%%%%%%%%%%%%%%%%%%%%%%%%%%%%%%%%%%%%%%%%%%%%%%%%%%%%%%%%
			% rec
			%%%%%%%%%%%%%%%%%%%%%%%%%%%%%%%%%%%%%%%%%%%%%%%%%%%%%%%%%%%%
			\addplot[fill=blue] coordinates {
				(CCS,9.9318965816336)
				(SEEDS,2.28232472)
				(SLIC,8.6849153136699)
				(RW,13.347694664807)
				(CW,10.508734794086)
				(TP,16.91966442)
				(POISE,13.09785248)
				(FH,15.02527176)
				(EAMS,6.9713463075023)
				(CRS,6.4499133686543)
				(SEAW,18.718694626414)
				%(reSEEDS,2.33844378)
				(ERGC,7.0299239643847)
				(PF,32.339046914181)
				(TPS,16.608716929614)
				(NC,14.5729878)
				(VC,16.099002920902)
				(PB,12.475172916609)
				(preSLIC,10.364926165195)
				(W,10.512659928738)
				(LSC,9.11646619)
				(WP,12.859419948232)
				(QS,26.357822257892)
				%(vlSLIC,9.34130346)
				(CIS,17.742002940095)
				(ERS,6.227106)
				(MSS,11.451229616209)
				(ETPS,2.9666871716831)
				((DASP),0)
				((VCCS),0)
			};
			\label{plot:experiments-quantitative-bsds500-average-rec}

			%%%%%%%%%%%%%%%%%%%%%%%%%%%%%%%%%%%%%%%%%%%%%%%%%%%%%%%%%%%%
			% ue_np
			%%%%%%%%%%%%%%%%%%%%%%%%%%%%%%%%%%%%%%%%%%%%%%%%%%%%%%%%%%%%
			\addplot[fill=red] coordinates {
				(CCS,7.5289953858794)
				(SEEDS,9.446739759)
				(SLIC,7.6497140156322)
				(RW,8.6756965619526)
				(CW,8.4402624332611)
				(TP,9.271960248)
				(POISE,8.924878757)
				(FH,10.243686489)
				(EAMS,8.2105337749209)
				(CRS,7.3448022077077)
				(SEAW,10.305148679693)
				%(reSEEDS,7.807331413)
				(ERGC,7.1872745931238)
				(PF,17.356643633834)
				(TPS,8.9472800577507)
				(NC,9.334483981)
				(VC,10.953437795205)
				(PB,9.3881740126911)
				(preSLIC,8.1143337244377)
				(W,8.7587242552172)
				(LSC,8.969104772)
				(WP,8.3393118766649)
				(QS,16.426937818043)
				%(vlSLIC,8.777162714)
				(CIS,7.883995044819)
				(ERS,7.6528118)
				(MSS,8.2658572868153)
				(ETPS,7.0390794600518)
				((DASP),0)
				((VCCS),0)
			};
			\label{plot:experiments-quantitative-bsds500-average-ue_np}

			%%%%%%%%%%%%%%%%%%%%%%%%%%%%%%%%%%%%%%%%%%%%%%%%%%%%%%%%%%%%
			% ev
			%%%%%%%%%%%%%%%%%%%%%%%%%%%%%%%%%%%%%%%%%%%%%%%%%%%%%%%%%%%%
			\addplot[fill=green] coordinates {
				(CCS,8.810595522855)
				(SEEDS,7.32645749)
				(SLIC,10.695688557583)
				(RW,12.150049603053)
				(CW,13.965614906343)
				(TP,14.73410334)
				(POISE,14.2154631)
				(FH,12.70897273)
				(EAMS,7.0888362851786)
				(CRS,10.759921739142)
				(SEAW,11.431681615602)
				%(reSEEDS,6.46062533)
				(ERGC,8.1405080708485)
				(PF,23.576251534528)
				(TPS,16.153750090356)
				(NC,16.62427233)
				(VC,11.799903143852)
				(PB,15.551225371015)
				(preSLIC,10.942140607989)
				(W,14.204407365309)
				(LSC,11.99522686)
				(WP,12.032242915515)
				(QS,14.490709318973)
				%(vlSLIC,11.65575168)
				(CIS,9.332430822)
				(ERS,13.001537)
				(MSS,13.02655533315)
				(ETPS,5.3336737754583)
				((DASP),0)
				((VCCS),0)
			};
			\label{plot:experiments-quantitative-bsds500-average-ev}

		\end{axis}
	\end{tikzpicture}
\end{subfigure}
\begin{tikzpicture}[overlay]
	\node at (-9.25, 2.15){\rotatebox{90}{\small no depth}};
	\node at (-6.2, 2.15){\rotatebox{90}{\small no depth}};
\end{tikzpicture}\\
	%\begin{subfigure}[b]{0.985\textwidth}\phantomsubcaption\label{subfig:experiments-quantitative-nyuv2-average}
%	\begin{tikzpicture}
%		\begin{axis}[EQNYUV2Avg,%
%				symbolic x coords = {
%					W,
%					EAMS,
%					NC,
%					FH,
%					RW,
%					QS,
%					PF,
%					TP,
%					CIS,
%					SLIC,
%					CRS,
%					ERS,
%					PB,
%					DASP,
%					SEEDS,
%					TPS,
%					VC,
%					CCS,
%					VCCS,
%					CW,
%					ERGC,
%					MSS,
%					preSLIC,
%					WP,
%					ETPS,
%					LSC,
%					POISE,
%					SEAW,
%				},
%			]
%
%				%%%%%%%%%%%%%%%%%%%%%%%%%%%%%%%%%%%%%%%%%%%%%%%%%%%%%%%%%%%%
%			% rec
%			%%%%%%%%%%%%%%%%%%%%%%%%%%%%%%%%%%%%%%%%%%%%%%%%%%%%%%%%%%%%
%			\addplot[fill=blue] coordinates {
%				(CCS,7.9704040194755)
%				(SEEDS,1.99918493)
%				(SLIC,4.8869580065139)
%				(RW,9.7872709)
%				(CW,7.0241612257619)
%				(TP,12.01724336)
%				(POISE,6.71664896)
%				(FH,7.84434987)
%				(EAMS,5.7893706432839)
%				(CRS,4.2017993563281)
%				(SEAW,14.491755414489)
%				(ERGC,5.3342177711828)
%				(PF,26.952844206793)
%				(TPS,12.578696618482)
%				(NC,9.39717382)
%				(VC,8.892891812)
%				(PB,9.5242039277316)
%				(VCCS,10.5670046)
%				(preSLIC,4.7838102442251)
%				(W,7.0637844541329)
%				(LSC,4.90401351)
%				(WP,7.5084420846341)
%				(QS,16.046153933985)
%				(CIS,13.38194051)
%				(ERS,3.477366)
%				(DASP,7.43460309)
%				(MSS,8.5582389358142)
%				(ETPS,2.4733715019512)
%			};
%			\label{plot:experiments-quantitative-nyuv2-average-rec}
%			
%			%%%%%%%%%%%%%%%%%%%%%%%%%%%%%%%%%%%%%%%%%%%%%%%%%%%%%%%%%%%%
%			% ue_np
%			%%%%%%%%%%%%%%%%%%%%%%%%%%%%%%%%%%%%%%%%%%%%%%%%%%%%%%%%%%%%
%			\addplot[fill=red] coordinates {
%				(CCS,8.2978239111225)
%				(SEEDS,11.001924982)
%				(SLIC,8.4907755437663)
%				(RW,9.441367153)
%				(CW,9.2316959318342)
%				(TP,9.73918887)
%				(POISE,10.01700179)
%				(FH,10.094505885)
%				(EAMS,9.1389145184757)
%				(CRS,7.9479839954883)
%				(SEAW,10.778732686267)
%				(ERGC,8.3641164685591)
%				(PF,20.064094800719)
%				(TPS,9.372713628934)
%				(NC,9.32182317)
%				(VC,8.8199022232361)
%				(PB,10.00583010179)
%				(VCCS,10.94676269)
%				(preSLIC,9.0053832940137)
%				(W,9.5520796412727)
%				(LSC,8.527777268)
%				(WP,8.9339077710976)
%				(QS,12.273312979277)
%				(CIS,8.90993327)
%				(ERS,7.7164086)
%				(DASP,8.315646668)
%				(MSS,9.6461625872257)
%				(ETPS,8.0038928736829)
%			};
%			\label{plot:experiments-quantitative-nyuv2-average-ue_np}
%			
%			%%%%%%%%%%%%%%%%%%%%%%%%%%%%%%%%%%%%%%%%%%%%%%%%%%%%%%%%%%%%
%			% ev
%			%%%%%%%%%%%%%%%%%%%%%%%%%%%%%%%%%%%%%%%%%%%%%%%%%%%%%%%%%%%%
%			\addplot[fill=black] coordinates {
%				(CCS,3.2656276782785)
%				(SEEDS,2.9067482)
%				(SLIC,4.3347220783406)
%				(RW,5.90116557)
%				(CW,5.4825582152504)
%				(TP,5.94252827)
%				(POISE,5.46585384)
%				(FH,5.15784339)
%				(EAMS,2.6588718157037)
%				(CRS,4.5231009601953)
%				(SEAW,5.7570158156)
%				(ERGC,3.7341660154839)
%				(PF,14.434696904625)
%				(TPS,7.0409476476568)
%				(NC,7.22154213)
%				(VC,3.1121424192048)
%				(PB,7.3257554894976)
%				(VCCS,6.61132094)
%				(preSLIC,4.8991294197023)
%				(W,5.7317004976434)
%				(LSC,3.39076539)
%				(WP,4.4520422653659)
%				(QS,4.7715113993379)
%				(CIS,4.81193817)
%				(ERS,6.250889)
%				(DASP,3.3488285)
%				(MSS,5.2941333502832)
%				(ETPS,1.8714741002439)
%			};
%			\label{plot:experiments-quantitative-nyuv2-average-ev}
%			
%		\end{axis}
%	\end{tikzpicture}
%\end{subfigure}
\begin{subfigure}[b]{0.985\textwidth}\phantomsubcaption\label{subfig:experiments-quantitative-nyuv2-average}
	\begin{tikzpicture}
		\begin{axis}[EQNYUV2Avg,%
				symbolic x coords = {
					W,
					EAMS,
					NC,
					FH,
					RW,
					QS,
					PF,
					TP,
					CIS,
					SLIC,
					%vlSLIC,
					%SLIC3D,
					CRS,
					ERS,
					PB,
					DASP,
					SEEDS,
					%reSEEDS,
					%reSEEDS3D,
					TPS,
					VC,
					CCS,
					VCCS,
					CW,
					ERGC,
					MSS,
					preSLIC,
					WP,
					%LRW,
					ETPS,
					LSC,
					POISE,
					SEAW,
					%reFH,
				},
			]

				%%%%%%%%%%%%%%%%%%%%%%%%%%%%%%%%%%%%%%%%%%%%%%%%%%%%%%%%%%%%
			% rec
			%%%%%%%%%%%%%%%%%%%%%%%%%%%%%%%%%%%%%%%%%%%%%%%%%%%%%%%%%%%%
			\addplot[fill=blue] coordinates {
				(CCS,7.9704040194755)
				(SEEDS,1.99918493)
				(SLIC,4.8869580065139)
				(RW,9.7872709)
				(CW,7.0241612257619)
				(TP,12.01724336)
				(POISE,6.71664896)
				(FH,7.84434987)
				(EAMS,5.7893706432839)
				(CRS,4.2017993563281)
				(SEAW,14.491755414489)
				%(reSEEDS,1.8787781)
				(ERGC,5.3342177711828)
				(PF,26.952844206793)
				(TPS,12.578696618482)
				(NC,9.39717382)
				(VC,8.892891812)
				(PB,9.5242039277316)
				(VCCS,10.5670046)
				(preSLIC,4.7838102442251)
				(W,7.0637844541329)
				(LSC,4.90401351)
				(WP,7.5084420846341)
				(QS,16.046153933985)
				(CIS,13.38194051)
				%(reSEEDS3D,3.36632442)
				(ERS,3.477366)
				(DASP,7.43460309)
				(MSS,8.5582389358142)
				(ETPS,2.4733715019512)
			};
			\label{plot:experiments-quantitative-nyuv2-average-rec}
			
			%%%%%%%%%%%%%%%%%%%%%%%%%%%%%%%%%%%%%%%%%%%%%%%%%%%%%%%%%%%%
			% ue_np
			%%%%%%%%%%%%%%%%%%%%%%%%%%%%%%%%%%%%%%%%%%%%%%%%%%%%%%%%%%%%
			\addplot[fill=red] coordinates {
				(CCS,8.2978239111225)
				(SEEDS,11.001924982)
				(SLIC,8.4907755437663)
				(RW,9.441367153)
				(CW,9.2316959318342)
				(TP,9.73918887)
				(POISE,10.01700179)
				(FH,10.094505885)
				(EAMS,9.1389145184757)
				(CRS,7.9479839954883)
				(SEAW,10.778732686267)
				%(reSEEDS,8.798189698)
				(ERGC,8.3641164685591)
				(PF,20.064094800719)
				(TPS,9.372713628934)
				(NC,9.32182317)
				(VC,8.8199022232361)
				(PB,10.00583010179)
				(VCCS,10.94676269)
				(preSLIC,9.0053832940137)
				(W,9.5520796412727)
				(LSC,8.527777268)
				(WP,8.9339077710976)
				(QS,12.273312979277)
				(CIS,8.90993327)
				%(reSEEDS3D,8.178461121)
				(ERS,7.7164086)
				(DASP,8.315646668)
				(MSS,9.6461625872257)
				(ETPS,8.0038928736829)
			};
			\label{plot:experiments-quantitative-nyuv2-average-ue_np}
			
			%%%%%%%%%%%%%%%%%%%%%%%%%%%%%%%%%%%%%%%%%%%%%%%%%%%%%%%%%%%%
			% ev
			%%%%%%%%%%%%%%%%%%%%%%%%%%%%%%%%%%%%%%%%%%%%%%%%%%%%%%%%%%%%
			\addplot[fill=green] coordinates {
				(CCS,3.2656276782785)
				(SEEDS,2.9067482)
				(SLIC,4.3347220783406)
				(RW,5.90116557)
				(CW,5.4825582152504)
				(TP,5.94252827)
				(POISE,5.46585384)
				(FH,5.15784339)
				(EAMS,2.6588718157037)
				(CRS,4.5231009601953)
				(SEAW,5.7570158156)
				%(reSEEDS,2.13950859)
				(ERGC,3.7341660154839)
				(PF,14.434696904625)
				(TPS,7.0409476476568)
				(NC,7.22154213)
				(VC,3.1121424192048)
				(PB,7.3257554894976)
				(VCCS,6.61132094)
				(preSLIC,4.8991294197023)
				(W,5.7317004976434)
				(LSC,3.39076539)
				(WP,4.4520422653659)
				(QS,4.7715113993379)
				(CIS,4.81193817)
				%(reSEEDS3D,2.45424108)
				(ERS,6.250889)
				(DASP,3.3488285)
				(MSS,5.2941333502832)
				(ETPS,1.8714741002439)
			};
			\label{plot:experiments-quantitative-nyuv2-average-ev}
			
		\end{axis}
	\end{tikzpicture}
\end{subfigure}
\\
	%\begin{subfigure}[b]{0.985\textwidth}\phantomsubcaption\label{subfig:experiments-quantitative-sbd-average}
%	\begin{tikzpicture}
%		\begin{axis}[EQSBDAvg,%
%				symbolic x coords = {
%					W,
%					EAMS,
%					NC,
%					FH,
%					RW,
%					QS,
%					PF,
%					TP,
%					CIS,
%					SLIC,
%					CRS,
%					ERS,
%					PB,
%					SEEDS,
%					TPS,
%					VC,
%					CCS,
%					CW,
%					ERGC,
%					MSS,
%					preSLIC,
%					WP,
%					ETPS,
%					LSC,
%					POISE,
%					SEAW,
%				},
%			]
%
%				%%%%%%%%%%%%%%%%%%%%%%%%%%%%%%%%%%%%%%%%%%%%%%%%%%%%%%%%%%%%
%			% rec
%			%%%%%%%%%%%%%%%%%%%%%%%%%%%%%%%%%%%%%%%%%%%%%%%%%%%%%%%%%%%%
%			\addplot[fill=blue] coordinates {
%				(CCS,3.99207976)
%				(SEEDS,0.98564913)
%				(SLIC,3.8467011057143)
%				(RW,6.5381300390722)
%				(CW,4.6874497903046)
%				(TP,12.66695784)
%				(POISE,10.6843381)
%				(FH,6.46660202)
%				(EAMS,1.7266924017788)
%				(CRS,3.3346620325286)
%				(SEAW,9.5443596898462)
%				(ERGC,3.2488854819788)
%				(PF,18.64936024)
%				(TPS,9.0273009484173)
%				(NC,6.31007093)
%				(VC,6.39442433)
%				(PB,5.51861166)
%				(preSLIC,4.3352076513208)
%				(W,4.7278366937943)
%				(LSC,5.4246496)
%				(WP,5.5110872544954)
%				(QS,12.26025829483)
%				(CIS,8.60568384)
%				(ERS,2.074466)
%				(MSS,6.3959980483071)
%				(ETPS,1.1757854151376)
%			};
%			\label{plot:experiments-quantitative-sbd-average-rec}
%
%			%%%%%%%%%%%%%%%%%%%%%%%%%%%%%%%%%%%%%%%%%%%%%%%%%%%%%%%%%%%%
%			% ue_np
%			%%%%%%%%%%%%%%%%%%%%%%%%%%%%%%%%%%%%%%%%%%%%%%%%%%%%%%%%%%%%
%			\addplot[fill=red] coordinates {
%				(CCS,6.500144161)
%				(SEEDS,7.255817146)
%				(SLIC,6.6241608762857)
%				(RW,7.4583036919485)
%				(CW,7.2332045751798)
%				(TP,9.126335872)
%				(POISE,8.133637934)
%				(FH,7.801094687)
%				(EAMS,6.2440697203207)
%				(CRS,6.3964248642735)
%				(SEAW,8.146456334)
%				(ERGC,6.5602665651458)
%				(PF,12.47642349)
%				(TPS,7.165171797705)
%				(NC,5.995230859)
%				(VC,7.076657975)
%				(PB,7.431007376)
%				(preSLIC,7.2020326533585)
%				(W,7.4067354291667)
%				(LSC,8.788028426)
%				(WP,6.9608341891835)
%				(QS,9.3828209431597)
%				(CIS,6.567490704)
%				(ERS,6.3167828)
%				(MSS,7.6111817829679)
%				(ETPS,6.2243865069174)
%			};
%			\label{plot:experiments-quantitative-sbd-average-ue_np}
%
%			%%%%%%%%%%%%%%%%%%%%%%%%%%%%%%%%%%%%%%%%%%%%%%%%%%%%%%%%%%%%
%			% ev
%			%%%%%%%%%%%%%%%%%%%%%%%%%%%%%%%%%%%%%%%%%%%%%%%%%%%%%%%%%%%%
%			\addplot[fill=black] coordinates {
%				(CCS,5.1215566)
%				(SEEDS,3.65680745)
%				(SLIC,6.6596438680952)
%				(RW,6.9632355873196)
%				(CW,9.4167775171791)
%				(TP,11.13864972)
%				(POISE,10.16088613)
%				(FH,6.05722211)
%				(EAMS,3.5112406391239)
%				(CRS,7.1839177763178)
%				(SEAW,6.9888148935385)
%				(ERGC,4.5890374901253)
%				(PF,15.33460288)
%				(TPS,9.9312681434532)
%				(NC,10.26760098)
%				(VC,5.53762571)
%				(PB,9.06619777)
%				(preSLIC,7.4900418458491)
%				(W,9.5496883188652)
%				(LSC,9.69159198)
%				(WP,7.2626703088073)
%				(QS,5.6803576142068)
%				(CIS,5.08452376)
%				(ERS,8.269597)
%				(MSS,9.1936434007485)
%				(ETPS,2.9591285079816)
%			};
%			\label{plot:experiments-quantitative-sbd-average-ev}
%
%		\end{axis}
%	\end{tikzpicture}
%\end{subfigure}
\begin{subfigure}[b]{0.985\textwidth}\phantomsubcaption\label{subfig:experiments-quantitative-sbd-average}
	\begin{tikzpicture}
		\begin{axis}[EQSBDAvg,%
				symbolic x coords = {
					W,
					EAMS,
					(NC),
					FH,
					RW,
					QS,
					PF,
					TP,
					CIS,
					SLIC,
					%vlSLIC,
					%SLIC3D,
					CRS,
					ERS,
					PB,
					(DASP),
					SEEDS,
					%reSEEDS,
					%reSEEDS3D,
					TPS,
					VC,
					CCS,
					(VCCS),
					CW,
					ERGC,
					MSS,
					preSLIC,
					WP,
					%LRW,
					ETPS,
					LSC,
					POISE,
					SEAW,
					%reFH,
				},
			]

				%%%%%%%%%%%%%%%%%%%%%%%%%%%%%%%%%%%%%%%%%%%%%%%%%%%%%%%%%%%%
			% rec
			%%%%%%%%%%%%%%%%%%%%%%%%%%%%%%%%%%%%%%%%%%%%%%%%%%%%%%%%%%%%
			\addplot[fill=blue] coordinates {
				(CCS,3.99207976)
				(SEEDS,0.98564913)
				(SLIC,3.8467011057143)
				(RW,6.5381300390722)
				(CW,4.6874497903046)
				(TP,12.66695784)
				(POISE,10.6843381)
				(FH,6.46660202)
				(EAMS,1.7266924017788)
				(CRS,3.3346620325286)
				(SEAW,9.5443596898462)
				%(reSEEDS,0.90183721)
				(ERGC,3.2488854819788)
				(PF,18.64936024)
				(TPS,9.0273009484173)
				((NC),0)
				(VC,6.39442433)
				(PB,5.51861166)
				(preSLIC,4.3352076513208)
				(W,4.7278366937943)
				(LSC,5.4246496)
				(WP,5.5110872544954)
				(QS,12.26025829483)
				%(vlSLIC,5.08759744)
				(CIS,8.60568384)
				(ERS,2.074466)
				(MSS,6.3959980483071)
				(ETPS,1.1757854151376)
				((DASP),0)
				((VCCS),0)
			};
			\label{plot:experiments-quantitative-sbd-average-rec}

			%%%%%%%%%%%%%%%%%%%%%%%%%%%%%%%%%%%%%%%%%%%%%%%%%%%%%%%%%%%%
			% ue_np
			%%%%%%%%%%%%%%%%%%%%%%%%%%%%%%%%%%%%%%%%%%%%%%%%%%%%%%%%%%%%
			\addplot[fill=red] coordinates {
				(CCS,6.500144161)
				(SEEDS,7.255817146)
				(SLIC,6.6241608762857)
				(RW,7.4583036919485)
				(CW,7.2332045751798)
				(TP,9.126335872)
				(POISE,8.133637934)
				(FH,7.801094687)
				(EAMS,6.2440697203207)
				(CRS,6.3964248642735)
				(SEAW,8.146456334)
				%(reSEEDS,6.706953897)
				(ERGC,6.5602665651458)
				(PF,12.47642349)
				(TPS,7.165171797705)
				((NC),0)
				(VC,7.076657975)
				(PB,7.431007376)
				(preSLIC,7.2020326533585)
				(W,7.4067354291667)
				(LSC,8.788028426)
				(WP,6.9608341891835)
				(QS,9.3828209431597)
				%(vlSLIC,8.367953218)
				(CIS,6.567490704)
				(ERS,6.3167828)
				(MSS,7.6111817829679)
				(ETPS,6.2243865069174)
				((DASP),0)
				((VCCS),0)
			};
			\label{plot:experiments-quantitative-sbd-average-ue_np}

			%%%%%%%%%%%%%%%%%%%%%%%%%%%%%%%%%%%%%%%%%%%%%%%%%%%%%%%%%%%%
			% ev
			%%%%%%%%%%%%%%%%%%%%%%%%%%%%%%%%%%%%%%%%%%%%%%%%%%%%%%%%%%%%
			\addplot[fill=green] coordinates {
				(CCS,5.1215566)
				(SEEDS,3.65680745)
				(SLIC,6.6596438680952)
				(RW,6.9632355873196)
				(CW,9.4167775171791)
				(TP,11.13864972)
				(POISE,10.16088613)
				(FH,6.05722211)
				(EAMS,3.5112406391239)
				(CRS,7.1839177763178)
				(SEAW,6.9888148935385)
				%(reSEEDS,3.58203211)
				(ERGC,4.5890374901253)
				(PF,15.33460288)
				(TPS,9.9312681434532)
				((NC),)
				(VC,5.53762571)
				(PB,9.06619777)
				(preSLIC,7.4900418458491)
				(W,9.5496883188652)
				(LSC,9.69159198)
				(WP,7.2626703088073)
				(QS,5.6803576142068)
				%(vlSLIC,8.90496438)
				(CIS,5.08452376)
				(ERS,8.269597)
				(MSS,9.1936434007485)
				(ETPS,2.9591285079816)
				((DASP),0)
				((VCCS),0)
			};
			\label{plot:experiments-quantitative-sbd-average-ev}

		\end{axis}
	\end{tikzpicture}
\end{subfigure}
\begin{tikzpicture}[overlay]
	\node at (-15.95,  1.925){\rotatebox{90}{\small failed}};
	\node at (-9.25, 2.15){\rotatebox{90}{\small no depth}};
	\node at (-6.2, 2.15){\rotatebox{90}{\small no depth}};
\end{tikzpicture}\\
	%\begin{subfigure}[b]{0.985\textwidth}\phantomsubcaption\label{subfig:experiments-quantitative-sunrgbd-average}
%	\begin{tikzpicture}
%		\begin{axis}[EQSUNRGBDAvg,%
%				symbolic x coords = {
%					W,
%					EAMS,
%					FH,
%					QS,
%					PF,
%					TP,
%					CIS,
%					SLIC,
%					CRS,
%					ERS,
%					PB,
%					DASP,
%					SEEDS,
%					TPS,
%					VC,
%					CCS,
%					VCCS,
%					CW,
%					ERGC,
%					MSS,
%					preSLIC,
%					WP,
%					ETPS,
%					LSC,
%					POISE,
%				},
%			]
%
%				%%%%%%%%%%%%%%%%%%%%%%%%%%%%%%%%%%%%%%%%%%%%%%%%%%%%%%%%%%%%
%			% rec
%			%%%%%%%%%%%%%%%%%%%%%%%%%%%%%%%%%%%%%%%%%%%%%%%%%%%%%%%%%%%%
%			\addplot[fill=blue] coordinates {
%				(CCS,9.7793567967647)
%				(SEEDS,2.3172278871707)
%				(SLIC,5.9045951381443)
%				(CW,8.301950156873)
%				(TP,10.73901102)
%				(POISE,8.10561845)
%				(FH,10.229925891852)
%				(EAMS,5.5189034036842)
%				(CRS,5.8588099082847)
%				(ERGC,6.1777726336)
%				(PF,29.342998077107)
%				(TPS,14.937158274754)
%				(VC,10.412830284681)
%				(PB,10.85350273)
%				(VCCS,17.86737119)
%				(preSLIC,6.1443742248148)
%				(W,8.2592052550445)
%				(LSC,6.7356841054902)
%				(WP,8.93473851)
%				(QS,17.416250791579)
%				(CIS,14.52232126)
%				(ERS,4.387237)
%				(DASP,9.0420388)
%				(MSS,10.080962459341)
%				(ETPS,3.44468441)
%			};
%			\label{plot:experiments-quantitative-sunrgbd-average-rec}
%
%				%%%%%%%%%%%%%%%%%%%%%%%%%%%%%%%%%%%%%%%%%%%%%%%%%%%%%%%%%%%%
%			% ue_np
%			%%%%%%%%%%%%%%%%%%%%%%%%%%%%%%%%%%%%%%%%%%%%%%%%%%%%%%%%%%%%
%			\addplot[fill=red] coordinates {
%				(CCS,6.9436340856471)
%				(SEEDS,8.5258507456325)
%				(SLIC,7.0139590942577)
%				(CW,7.4875824507232)
%				(TP,7.512286646)
%				(POISE,8.228683284)
%				(FH,8.6631790703704)
%				(EAMS,7.6068779476316)
%				(CRS,6.7287454884014)
%				(ERGC,6.7503710224)
%				(PF,16.519507852642)
%				(TPS,7.8340534858443)
%				(VC,7.4476611429255)
%				(PB,8.360526273)
%				(VCCS,16.14082359)
%				(preSLIC,7.5937574069206)
%				(W,7.7706754118519)
%				(LSC,7.0259231029804)
%				(WP,7.331545901)
%				(QS,10.196305483789)
%				(CIS,7.28551162)
%				(ERS,6.6387882)
%				(DASP,7.339403359)
%				(MSS,7.9004804341018)
%				(ETPS,6.617327532)
%			};
%			\label{plot:experiments-quantitative-sunrgbd-average-ue_np}
%
%				%%%%%%%%%%%%%%%%%%%%%%%%%%%%%%%%%%%%%%%%%%%%%%%%%%%%%%%%%%%%
%			% ev
%			%%%%%%%%%%%%%%%%%%%%%%%%%%%%%%%%%%%%%%%%%%%%%%%%%%%%%%%%%%%%
%			\addplot[fill=black] coordinates {
%				(CCS,2.9527510860784)
%				(SEEDS,2.3956850803252)
%				(SLIC,4.2642592482474)
%				(CW,5.0632259526441)
%				(TP,5.22223075)
%				(POISE,6.71430356)
%				(FH,4.5491502197037)
%				(EAMS,2.3456012263158)
%				(CRS,4.3902226967581)
%				(ERGC,2.9190980512)
%				(PF,13.130088422932)
%				(TPS,6.3351563696311)
%				(VC,2.9665376725532)
%				(PB,6.97417875)
%				(VCCS,14.38189913)
%				(preSLIC,5.3502755248677)
%				(W,5.2309960849604)
%				(LSC,3.1444705776471)
%				(WP,3.85768184)
%				(QS,4.2290703015789)
%				(CIS,4.09318536)
%				(ERS,6.138697)
%				(DASP,3.31095176)
%				(MSS,4.9316343176647)
%				(ETPS,1.55775485)
%			};
%			\label{plot:experiments-quantitative-sunrgbd-average-ev}
%
%			\end{axis}
%	\end{tikzpicture}
%\end{subfigure}
\begin{subfigure}[b]{0.985\textwidth}\phantomsubcaption\label{subfig:experiments-quantitative-sunrgbd-average}
	\begin{tikzpicture}
		\begin{axis}[EQSUNRGBDAvg,%
				symbolic x coords = {
					W,
					EAMS,
					(NC),
					FH,
					(RW),
					QS,
					PF,
					TP,
					CIS,
					SLIC,
					%vlSLIC,
					%SLIC3D,
					CRS,
					ERS,
					PB,
					DASP,
					SEEDS,
					%reSEEDS,
					%reSEEDS3D,
					TPS,
					VC,
					CCS,
					VCCS,
					CW,
					ERGC,
					MSS,
					preSLIC,
					WP,
					%LRW,
					ETPS,
					LSC,
					POISE,
					(SEAW),
					%{},
					%reFH,
				},
			]

				%%%%%%%%%%%%%%%%%%%%%%%%%%%%%%%%%%%%%%%%%%%%%%%%%%%%%%%%%%%%
			% rec
			%%%%%%%%%%%%%%%%%%%%%%%%%%%%%%%%%%%%%%%%%%%%%%%%%%%%%%%%%%%%
			\addplot[fill=blue] coordinates {
				%(SLIC3D,5.9953141949474)
				(CCS,9.7793567967647)
				(SEEDS,2.3172278871707)
				(SLIC,5.9045951381443)
				(CW,8.301950156873)
				(TP,10.73901102)
				(POISE,8.10561845)
				(FH,10.229925891852)
				(EAMS,5.5189034036842)
				(CRS,5.8588099082847)
				%(reSEEDS,2.3730680338722)
				(ERGC,6.1777726336)
				(PF,29.342998077107)
				(TPS,14.937158274754)
				(VC,10.412830284681)
				(PB,10.85350273)
				(VCCS,17.86737119)
				(preSLIC,6.1443742248148)
				(W,8.2592052550445)
				(LSC,6.7356841054902)
				(WP,8.93473851)
				(QS,17.416250791579)
				%(vlSLIC,6.43582089)
				(CIS,14.52232126)
				%(reSEEDS3D,3.5744616452938)
				(ERS,4.387237)
				(DASP,9.0420388)
				(MSS,10.080962459341)
				(ETPS,3.44468441)
				((SEAW),0)
				((NC),0)
				((RW),0)
				%({},0)
			};
			\label{plot:experiments-quantitative-sunrgbd-average-rec}

				%%%%%%%%%%%%%%%%%%%%%%%%%%%%%%%%%%%%%%%%%%%%%%%%%%%%%%%%%%%%
			% ue_np
			%%%%%%%%%%%%%%%%%%%%%%%%%%%%%%%%%%%%%%%%%%%%%%%%%%%%%%%%%%%%
			\addplot[fill=red] coordinates {
				%(SLIC3D,6.8847576063684)
				(CCS,6.9436340856471)
				(SEEDS,8.5258507456325)
				(SLIC,7.0139590942577)
				(CW,7.4875824507232)
				(TP,7.512286646)
				(POISE,8.228683284)
				(FH,8.6631790703704)
				(EAMS,7.6068779476316)
				(CRS,6.7287454884014)
				%(reSEEDS,7.4470678425419)
				(ERGC,6.7503710224)
				(PF,16.519507852642)
				(TPS,7.8340534858443)
				(VC,7.4476611429255)
				(PB,8.360526273)
				(VCCS,16.14082359)
				(preSLIC,7.5937574069206)
				(W,7.7706754118519)
				(LSC,7.0259231029804)
				(WP,7.331545901)
				(QS,10.196305483789)
				%(vlSLIC,7.542516157)
				(CIS,7.28551162)
				%(reSEEDS3D,6.959893582584)
				(ERS,6.6387882)
				(DASP,7.339403359)
				(MSS,7.9004804341018)
				(ETPS,6.617327532)
				((SEAW),0)
				((NC),0)
				((RW),0)
				%({},0)
			};
			\label{plot:experiments-quantitative-sunrgbd-average-ue_np}

				%%%%%%%%%%%%%%%%%%%%%%%%%%%%%%%%%%%%%%%%%%%%%%%%%%%%%%%%%%%%
			% ev
			%%%%%%%%%%%%%%%%%%%%%%%%%%%%%%%%%%%%%%%%%%%%%%%%%%%%%%%%%%%%
			\addplot[fill=green] coordinates {
				%(SLIC3D,4.1890442532632)
				(CCS,2.9527510860784)
				(SEEDS,2.3956850803252)
				(SLIC,4.2642592482474)
				(CW,5.0632259526441)
				(TP,5.22223075)
				(POISE,6.71430356)
				(FH,4.5491502197037)
				(EAMS,2.3456012263158)
				(CRS,4.3902226967581)
				%(reSEEDS,1.9579262185304)
				(ERGC,2.9190980512)
				(PF,13.130088422932)
				(TPS,6.3351563696311)
				(VC,2.9665376725532)
				(PB,6.97417875)
				(VCCS,14.38189913)
				(preSLIC,5.3502755248677)
				(W,5.2309960849604)
				(LSC,3.1444705776471)
				(WP,3.85768184)
				(QS,4.2290703015789)
				%(vlSLIC,3.78018157)
				(CIS,4.09318536)
				%(reSEEDS3D,2.2059215553396)
				(ERS,6.138697)
				(DASP,3.31095176)
				(MSS,4.9316343176647)
				(ETPS,1.55775485)
				((SEAW),0)
				((NC),0)
				((RW),0)
				%({},0)
			};
			\label{plot:experiments-quantitative-sunrgbd-average-ev}

			\end{axis}
	\end{tikzpicture}
\end{subfigure}
\begin{tikzpicture}[overlay]
	\node at (-16, 1.925){\rotatebox{90}{\small failed}};
	\node at (-14.75, 1.925){\rotatebox{90}{\small failed}};
	\node at (-0.725, 1.925){\rotatebox{90}{\small failed}};
\end{tikzpicture}\\
	%\begin{subfigure}[b]{0.985\textwidth}\phantomsubcaption\label{subfig:experiments-quantitative-sbd-average}
%	\begin{tikzpicture}
%		\begin{axis}[EQFashAvg,%
%				symbolic x coords = {
%					W,
%					EAMS,
%					NC,
%					FH,
%					RW,
%					QS,
%					PF,
%					TP,
%					CIS,
%					SLIC,
%					CRS,
%					ERS,
%					PB,
%					SEEDS,
%					TPS,
%					VC,
%					CCS,
%					CW,
%					ERGC,
%					MSS,
%					preSLIC,
%					WP,
%					ETPS,
%					LSC,
%					POISE,
%					SEAW,
%				},
%			]
%
%				%%%%%%%%%%%%%%%%%%%%%%%%%%%%%%%%%%%%%%%%%%%%%%%%%%%%%%%%%%%%
%			% rec
%			%%%%%%%%%%%%%%%%%%%%%%%%%%%%%%%%%%%%%%%%%%%%%%%%%%%%%%%%%%%%
%			\addplot[fill=blue] coordinates {
%				(CCS,1.30381016)
%				(SEEDS,0.27558563)
%				(SLIC,1.2052600288462)
%				(RW,3.7291897144478)
%				(CW,1.9320466830252)
%				(TP,2.90232317)
%				(POISE,0.63749367)
%				(FH,3.0623073551485)
%				(EAMS,0.89636899)
%				(CRS,0.96158964478261)
%				(SEAW,5.1436382098319)
%				(ERGC,0.95938017666667)
%				(PF,15.903108370669)
%				(TPS,4.0994467796099)
%				(NC,2.25343566)
%				(VC,2.69537224)
%				(PB,3.16592995)
%				(preSLIC,1.8012823817143)
%				(W,2.0076433011981)
%				(LSC,0.84386359)
%				(WP,2.39111976)
%				(QS,6.5004867285691)
%				(CIS,2.99686849)
%				(ERS,0.738335)
%				(MSS,1.7895051124315)
%				(ETPS,0.21114242)
%			};
%			\label{plot:experiments-quantitative-fash-average-rec}
%
%			%%%%%%%%%%%%%%%%%%%%%%%%%%%%%%%%%%%%%%%%%%%%%%%%%%%%%%%%%%%%
%			% ue_np
%			%%%%%%%%%%%%%%%%%%%%%%%%%%%%%%%%%%%%%%%%%%%%%%%%%%%%%%%%%%%%
%			\addplot[fill=red] coordinates {
%				(CCS,2.764525675)
%				(SEEDS,3.694062723)
%				(SLIC,2.7143095554808)
%				(RW,3.7020414663224)
%				(CW,3.5731282445733)
%				(TP,3.977292127)
%				(POISE,2.579006278)
%				(FH,3.9528368212529)
%				(EAMS,2.6387901)
%				(CRS,3.0353119784783)
%				(SEAW,4.2277920126933)
%				(ERGC,2.8178267316022)
%				(PF,8.1170746999841)
%				(TPS,2.637501723848)
%				(NC,2.952651111)
%				(VC,4.076593886)
%				(PB,4.088234941)
%				(preSLIC,3.1908578729905)
%				(W,3.7307536424901)
%				(LSC,2.876269579)
%				(WP,3.277769837)
%				(QS,5.4619018185004)
%				(CIS,2.846133049)
%				(ERS,3.068406)
%				(MSS,3.7181266457414)
%				(ETPS,2.486032333)
%			};
%			\label{plot:experiments-quantitative-fash-average-ue_np}
%
%				%%%%%%%%%%%%%%%%%%%%%%%%%%%%%%%%%%%%%%%%%%%%%%%%%%%%%%%%%%%%
%			% ev
%			%%%%%%%%%%%%%%%%%%%%%%%%%%%%%%%%%%%%%%%%%%%%%%%%%%%%%%%%%%%%
%			\addplot[fill=black] coordinates {
%				(CCS,5.32973367)
%				(SEEDS,4.18608671)
%				(SLIC,5.7260763842308)
%				(RW,8.1070688972537)
%				(CW,8.7929110226793)
%				(TP,9.261288)
%				(POISE,8.75335052)
%				(FH,7.7833847431379)
%				(EAMS,6.895071)
%				(CRS,7.1783888765217)
%				(SEAW,7.9154503245378)
%				(ERGC,5.3435673125806)
%				(PF,15.863401454533)
%				(TPS,10.028079727577)
%				(NC,10.60275725)
%				(VC,6.932652)
%				(PB,10.76750559)
%				(preSLIC,6.8505669954286)
%				(W,8.978665027208)
%				(LSC,5.86400029)
%				(WP,7.23421236)
%				(QS,9.3564948894804)
%				(CIS,6.29582096)
%				(ERS,8.031587)
%				(MSS,8.1725618683075)
%				(ETPS,3.05049199)
%			};
%			\label{plot:experiments-quantitative-fash-average-ev}
%
%			\end{axis}
%	\end{tikzpicture}
%\end{subfigure}
\begin{subfigure}[b]{0.985\textwidth}\phantomsubcaption\label{subfig:experiments-quantitative-sbd-average}
	\begin{tikzpicture}
		\begin{axis}[EQFashAvg,%
				symbolic x coords = {
					W,
					EAMS,
					NC,
					FH,
					RW,
					QS,
					PF,
					TP,
					CIS,
					SLIC,
					%vlSLIC,
					%SLIC3D,
					CRS,
					ERS,
					PB,
					(DASP),
					SEEDS,
					%reSEEDS,
					%reSEEDS3D,
					TPS,
					VC,
					CCS,
					(VCCS),
					CW,
					ERGC,
					MSS,
					preSLIC,
					WP,
					%LRW,
					ETPS,
					LSC,
					POISE,
					SEAW,
					%reFH,
				},
			]

				%%%%%%%%%%%%%%%%%%%%%%%%%%%%%%%%%%%%%%%%%%%%%%%%%%%%%%%%%%%%
			% rec
			%%%%%%%%%%%%%%%%%%%%%%%%%%%%%%%%%%%%%%%%%%%%%%%%%%%%%%%%%%%%
			\addplot[fill=blue] coordinates {
				(CCS,1.30381016)
				(SEEDS,0.27558563)
				(SLIC,1.2052600288462)
				(RW,3.7291897144478)
				(CW,1.9320466830252)
				(TP,2.90232317)
				(POISE,0.63749367)
				(FH,3.0623073551485)
				(EAMS,0.89636899)
				(CRS,0.96158964478261)
				(SEAW,5.1436382098319)
				%(reSEEDS,0.16413133)
				(ERGC,0.95938017666667)
				(PF,15.903108370669)
				(TPS,4.0994467796099)
				(NC,2.25343566)
				(VC,2.69537224)
				(PB,3.16592995)
				(preSLIC,1.8012823817143)
				(W,2.0076433011981)
				(LSC,0.84386359)
				(WP,2.39111976)
				(QS,6.5004867285691)
				%(vlSLIC,1.03357838)
				(CIS,2.99686849)
				(ERS,0.738335)
				(MSS,1.7895051124315)
				(ETPS,0.21114242)
				((DASP),0)
				((VCCS),0)
			};
			\label{plot:experiments-quantitative-fash-average-rec}

			%%%%%%%%%%%%%%%%%%%%%%%%%%%%%%%%%%%%%%%%%%%%%%%%%%%%%%%%%%%%
			% ue_np
			%%%%%%%%%%%%%%%%%%%%%%%%%%%%%%%%%%%%%%%%%%%%%%%%%%%%%%%%%%%%
			\addplot[fill=red] coordinates {
				(CCS,2.764525675)
				(SEEDS,3.694062723)
				(SLIC,2.7143095554808)
				(RW,3.7020414663224)
				(CW,3.5731282445733)
				(TP,3.977292127)
				(POISE,2.579006278)
				(FH,3.9528368212529)
				(EAMS,2.6387901)
				(CRS,3.0353119784783)
				(SEAW,4.2277920126933)
				%(reSEEDS,2.92046343)
				(ERGC,2.8178267316022)
				(PF,8.1170746999841)
				(TPS,2.637501723848)
				(NC,2.952651111)
				(VC,4.076593886)
				(PB,4.088234941)
				(preSLIC,3.1908578729905)
				(W,3.7307536424901)
				(LSC,2.876269579)
				(WP,3.277769837)
				(QS,5.4619018185004)
				%(vlSLIC,2.991763952)
				(CIS,2.846133049)
				(ERS,3.068406)
				(MSS,3.7181266457414)
				(ETPS,2.486032333)
				((DASP),0)
				((VCCS),0)
			};
			\label{plot:experiments-quantitative-fash-average-ue_np}

				%%%%%%%%%%%%%%%%%%%%%%%%%%%%%%%%%%%%%%%%%%%%%%%%%%%%%%%%%%%%
			% ev
			%%%%%%%%%%%%%%%%%%%%%%%%%%%%%%%%%%%%%%%%%%%%%%%%%%%%%%%%%%%%
			\addplot[fill=green] coordinates {
				(CCS,5.32973367)
				(SEEDS,4.18608671)
				(SLIC,5.7260763842308)
				(RW,8.1070688972537)
				(CW,8.7929110226793)
				(TP,9.261288)
				(POISE,8.75335052)
				(FH,7.7833847431379)
				(EAMS,6.895071)
				(CRS,7.1783888765217)
				(SEAW,7.9154503245378)
				%(reSEEDS,3.82138675)
				(ERGC,5.3435673125806)
				(PF,15.863401454533)
				(TPS,10.028079727577)
				(NC,10.60275725)
				(VC,6.932652)
				(PB,10.76750559)
				(preSLIC,6.8505669954286)
				(W,8.978665027208)
				(LSC,5.86400029)
				(WP,7.23421236)
				(QS,9.3564948894804)
				%(vlSLIC,5.61040401)
				(CIS,6.29582096)
				(ERS,8.031587)
				(MSS,8.1725618683075)
				(ETPS,3.05049199)
				((DASP),0)
				((VCCS),0)
			};
			\label{plot:experiments-quantitative-fash-average-ev}

			\end{axis}
	\end{tikzpicture}
\end{subfigure}
\begin{tikzpicture}[overlay]
	\node at (-9.25, 2.15){\rotatebox{90}{\small no depth}};
	\node at (-6.2, 2.15){\rotatebox{90}{\small no depth}};
\end{tikzpicture}
	\caption{\ARec, \AUE and \AEV (lower is better) on the used datasets.
    We find that \ARec, \AUE and \AEV appropriately
    summarize performance independent of the number of generated superpixels. Plausible
    examples to consider are top-performing algorithms such as \ETPS, \ERS, \SLIC or \CRS
    as well as poorly performing ones such as \QS and~\PF.
	\textbf{Best viewed in color.}}
	\label{fig:experiments-quantitative-avg}
\end{figure*}

\subsubsection{Depth}

Depth information does not necessarily improve performance regarding \Rec, \UE and \EV.
We consider Figures \ref{subfig:experiments-quantitative-nyuv2-rec.mean[0]},
\ref{subfig:experiments-quantitative-nyuv2-ue_np.mean[0]} and \ref{subfig:experiments-quantitative-nyuv2-ev.mean[0]}
presenting \Rec, \UE and \EV on the \NYU dataset. In particular, we consider \DASPr and \VCCSr.
We observe, that \DASPr consistently outperforms \VCCSr.
Therefore, we consider the performance of \DASPr and investigate whether depth information
improves performance. Note that \DASPr performs similar to \SLICr, exhibiting slightly
worse \Rec and slightly better \UE and \EV for large \K. However, \DASPr does not
clearly outperform \SLICr. As indicated in Section \ref{sec:algorithms}, \DASP and \SLIC are
both clustering-based algorithms. In particular, both algorithms are based on $k$-means
using color and spatial information and \DASPr additionally utilizes depth information.
This suggests that the clustering approach
does not benefit from depth information. We note that a similar line of thought can be
applied to \VCCS except that \VCCS directly operates within a point cloud, rendering the
comparison problematic. Still we conclude that depth information used in the form of
\DASP does not improve performance. This might be in contrast to
experiments with different superpixel algorithms, \eg a \SLIC variant using depth information
as in \cite{ZhangKanSchwingUrtasun:2013}. We suspect that regarding the used metrics, the number of superpixels ($\K = 200$)
and the used superpixel algorithm, the effect of depth information might be more pronounced 
in the experiments presented in \cite{ZhangKanSchwingUrtasun:2013} compared to ours.
Furthermore, it should be noted that our evaluation is carried out in the 2D image plane, 
which does not directly reflect the segmentation of point clouds.
%We further note that evaluation is carried out in the 2D image plane only.
