\subsection{Robustness}
\label{subsec:experiments-robustness}

Similar to Neubert and Protzel \cite{NeubertProtzel:2012}, we investigate the influence
of noise, blur and affine transformations.
We evaluated all algorithms for $\K \approx 400$ on the \BSDS dataset.
In the following we exemplarily discuss salt and pepper noise and average blurring.

\begin{figure}[t]
	\centering
	\begin{subfigure}[b]{\halfthreeone\textwidth}\phantomsubcaption\label{subfig:experiments-implementations-bsds500-rec.mean_min}
	\begin{tikzpicture}
		\begin{axis}[EIBSDS500Rec,xmode=log]

			% SEEDS rec.mean_min %%%%%%%%%%%%%%%%%%%%%%%%%%%%%%%%%%%%%%%%%%%%%%%%%%%%%%%%%%%%
			\addplot[SEEDS] coordinates{
				(261.62,0.882484)
				(365.675,0.906473)
				(468.81,0.922132)
				(670.57,0.941889)
				(870.75,0.952388)
				(1087.4,0.963175)
				(1270.11,0.967167)
				(1451.85,0.973778)
				(1669.18,0.974282)
				(1873.19,0.980114)
				(2104.62,0.984773)
				(2462.77,0.98341)
				(2793.43,0.989414)
				(3260.86,0.989969)
				(3895.78,0.993267)
				(3895.78,0.993267)
				(4846.12,0.995104)
				(4846.12,0.995104)
			};

			% RESEEDS rec.mean_min %%%%%%%%%%%%%%%%%%%%%%%%%%%%%%%%%%%%%%%%%%%%%%%%%%%%%%%%%%%%
			\addplot[RESEEDSI] coordinates{
				(200.795,0.876167)
				(301.525,0.902782)
				(401.465,0.918521)
				(602.33,0.937366)
				(800.99,0.949909)
				(1020.12,0.957079)
				(1201.34,0.965188)
				(1378.1,0.969268)
				(1601.55,0.97409)
				(1802.11,0.976241)
				(2040.11,0.980008)
				(2402.13,0.984131)
				(2720.13,0.987033)
				(3200.2,0.988506)
				(3840.22,0.992189)
				(3840.22,0.992189)
				(4800.37,0.994628)
				(4800.37,0.994628)
			};

			% FH rec.mean_min %%%%%%%%%%%%%%%%%%%%%%%%%%%%%%%%%%%%%%%%%%%%%%%%%%%%%%%%%%%%
			\addplot[FH] coordinates{
				(628.745,0.663508)
				(799.09,0.706555)
				(963.36,0.748153)
				(1090.39,0.75093)
				(1187.04,0.782457)
				(1605.71,0.808989)
				(2533.01,0.888881)
				(3000.74,0.889299)
				(3219.63,0.904142)
				(3814.42,0.913162)
				(4746.96,0.935408)
			};

			% REFH rec.mean_min %%%%%%%%%%%%%%%%%%%%%%%%%%%%%%%%%%%%%%%%%%%%%%%%%%%%%%%%%%%%
			\addplot[REFHI] coordinates{
				(247.49,0.716919)
				(335.32,0.752939)
				(465.54,0.79338)
				(543.205,0.80981)
				(621.41,0.823826)
				(862.475,0.857431)
				(1059.11,0.876169)
				(1359.22,0.898679)
				(1871.67,0.923443)
				(2930.78,0.950073)
				(3532,0.962835)
				(4508.69,0.975612)
				(4998.21,0.978696)
			};

			% PRESLIC rec.mean_min %%%%%%%%%%%%%%%%%%%%%%%%%%%%%%%%%%%%%%%%%%%%%%%%%%%%%%%%%%%%
			\addplot[PRESLIC] coordinates{
				(369,0.704329)
				(581.54,0.753518)
				(734.575,0.781089)
				(1020.3,0.819649)
				(1229.22,0.839652)
				(1229.22,0.839652)
				(1511.3,0.862751)
				(1838.94,0.883831)
				(1838.94,0.883831)
				(2375.99,0.911038)
				(3059.43,0.934761)
				(3059.43,0.934761)
				(3059.43,0.934761)
				(4218.88,0.962025)
				(4218.88,0.962025)
				(6137.8,0.985124)
			};

			% VLSLIC rec.mean_min %%%%%%%%%%%%%%%%%%%%%%%%%%%%%%%%%%%%%%%%%%%%%%%%%%%%%%%%%%%%
			\addplot[VLSLICI] coordinates{
				(575.975,0.78762)
				(651.86,0.800183)
				(763.62,0.819088)
				(899,0.836123)
				(988.985,0.845203)
				(1193.67,0.860582)
				(1348.08,0.866749)
				(1349.01,0.865282)
				(1579.08,0.87797)
				(1849.65,0.891934)
				(1857.27,0.890967)
				(2307.73,0.909537)
				(2890.56,0.929862)
				(2924.52,0.929379)
				(2954.18,0.9279)
				(3858.98,0.952214)
				(3858.98,0.952214)
				(4809.57,0.969088)
			};

			% SLIC rec.mean_min %%%%%%%%%%%%%%%%%%%%%%%%%%%%%%%%%%%%%%%%%%%%%%%%%%%%%%%%%%%%
			\addplot[SLIC] coordinates{
				(180.32,0.649493)
				(256.725,0.687987)
				(368.57,0.726167)
				(575.335,0.778431)
				(726.36,0.807455)
				(1002.87,0.843839)
				(1203.61,0.868781)
				(1203.61,0.868781)
				(1475.69,0.887329)
				(1814.8,0.908264)
				(1814.8,0.908264)
				(2334.56,0.929101)
				(3038.13,0.949264)
				(3038.13,0.949264)
				(3038.13,0.949264)
				(4188.97,0.970933)
				(4188.97,0.970933)
				(6139.41,0.988423)
			};

		\end{axis}
	\end{tikzpicture}
\end{subfigure}
\begin{subfigure}[b]{\halfthreeone\textwidth}\phantomsubcaption\label{subfig:experiments-implementations-bsds500-ue_np.mean_max}
	\begin{tikzpicture}
		\begin{axis}[EIBSDS500UE,xmode=log]

			% SEEDS ue_np.mean_max %%%%%%%%%%%%%%%%%%%%%%%%%%%%%%%%%%%%%%%%%%%%%%%%%%%%%%%%%%%%
			\addplot[SEEDS] coordinates{
				(261.62,0.223076)
				(365.675,0.19367)
				(468.81,0.166744)
				(670.57,0.144635)
				(870.75,0.127624)
				(1087.4,0.120216)
				(1270.11,0.110512)
				(1451.85,0.104601)
				(1669.18,0.0958062)
				(1873.19,0.091841)
				(2104.62,0.102696)
				(2462.77,0.0825942)
				(2793.43,0.0820788)
				(3260.86,0.0738429)
				(3895.78,0.0707603)
				(3895.78,0.0707603)
				(4846.12,0.0633063)
				(4846.12,0.0633063)
			};

			% SLIC ue_np.mean_max %%%%%%%%%%%%%%%%%%%%%%%%%%%%%%%%%%%%%%%%%%%%%%%%%%%%%%%%%%%%
			\addplot[SLIC] coordinates{
				(180.32,0.174507)
				(256.725,0.14981)
				(368.57,0.133617)
				(575.335,0.113918)
				(726.36,0.10487)
				(1002.87,0.0957898)
				(1203.61,0.0912283)
				(1203.61,0.0912283)
				(1475.69,0.0857781)
				(1814.8,0.0785351)
				(1814.8,0.0785351)
				(2334.56,0.0724657)
				(3038.13,0.0657068)
				(3038.13,0.0657068)
				(3038.13,0.0657068)
				(4188.97,0.0589157)
				(4188.97,0.0589157)
				(6139.41,0.0512524)
			};

			% FH ue_np.mean_max %%%%%%%%%%%%%%%%%%%%%%%%%%%%%%%%%%%%%%%%%%%%%%%%%%%%%%%%%%%%
			\addplot[FH] coordinates{
				(628.745,0.182601)
				(799.09,0.150389)
				(963.36,0.140866)
				(1090.39,0.131649)
				(1187.04,0.125432)
				(1605.71,0.110407)
				(2533.01,0.0877137)
				(3000.74,0.0845623)
				(3219.63,0.0821065)
				(3814.42,0.0819198)
				(4746.96,0.0709097)
			};

			% RESEEDS ue_np.mean_max %%%%%%%%%%%%%%%%%%%%%%%%%%%%%%%%%%%%%%%%%%%%%%%%%%%%%%%%%%%%
			\addplot[RESEEDSI] coordinates{
				(200.795,0.185835)
				(301.525,0.163531)
				(401.465,0.132874)
				(602.33,0.113933)
				(800.99,0.10611)
				(1020.12,0.0994214)
				(1201.34,0.0923218)
				(1378.1,0.087877)
				(1601.55,0.0800663)
				(1802.11,0.0781975)
				(2040.11,0.0832874)
				(2402.13,0.0699599)
				(2720.13,0.0694846)
				(3200.2,0.0644983)
				(3840.22,0.0617858)
				(3840.22,0.0617858)
				(4800.37,0.0561646)
				(4800.37,0.0561646)
			};

			% PRESLIC ue_np.mean_max %%%%%%%%%%%%%%%%%%%%%%%%%%%%%%%%%%%%%%%%%%%%%%%%%%%%%%%%%%%%
			\addplot[PRESLIC] coordinates{
				(369,0.15315)
				(581.54,0.127686)
				(734.575,0.117583)
				(1020.3,0.103212)
				(1229.22,0.0957013)
				(1229.22,0.0957013)
				(1511.3,0.0880409)
				(1838.94,0.0822743)
				(1838.94,0.0822743)
				(2375.99,0.0748201)
				(3059.43,0.068826)
				(3059.43,0.068826)
				(3059.43,0.068826)
				(4218.88,0.0614387)
				(4218.88,0.0614387)
				(6137.8,0.0537188)
			};

			% VLSLIC ue_np.mean_max %%%%%%%%%%%%%%%%%%%%%%%%%%%%%%%%%%%%%%%%%%%%%%%%%%%%%%%%%%%%
			\addplot[VLSLICI] coordinates{
				(575.975,0.146745)
				(651.86,0.133883)
				(763.62,0.12431)
				(899,0.112635)
				(988.985,0.106043)
				(1193.67,0.0981205)
				(1348.08,0.0935383)
				(1349.01,0.0929579)
				(1579.08,0.0889043)
				(1849.65,0.0848381)
				(1857.27,0.0842664)
				(2307.73,0.0796918)
				(2890.56,0.0748332)
				(2924.52,0.0736624)
				(2954.18,0.0727692)
				(3858.98,0.0707121)
				(3858.98,0.0707121)
				(4809.57,0.077104)
			};

			% REFH ue_np.mean_max %%%%%%%%%%%%%%%%%%%%%%%%%%%%%%%%%%%%%%%%%%%%%%%%%%%%%%%%%%%%
			\addplot[REFHI] coordinates{
				(247.49,0.156969)
				(335.32,0.139838)
				(465.54,0.125235)
				(543.205,0.117281)
				(621.41,0.111133)
				(862.475,0.0986539)
				(1059.11,0.0920731)
				(1359.22,0.0849422)
				(1871.67,0.076739)
				(2930.78,0.0663083)
				(3532,0.0618438)
				(4508.69,0.0587328)
				(4998.21,0.0585372)
			};

		\end{axis}
	\end{tikzpicture}
\end{subfigure}
\begin{subfigure}[b]{\halfthreeone\textwidth}\phantomsubcaption\label{subfig:experiments-implementations-bsds500-t}
	\begin{tikzpicture}
		\begin{axis}[EIBSDS500t,ymode=log]

			% RESEEDS %%%%%%%%%%%%%%%%%%%%%%%%%%%%%%%%%%%%%%%%%%%%%%%%%%%%%%%%%%%%
			\addplot[RESEEDSI] coordinates{
				(401.465,0.03832495)
				(1201.34,0.03862495)
				(3840.22,0.03812495)
			};

				% SEEDS %%%%%%%%%%%%%%%%%%%%%%%%%%%%%%%%%%%%%%%%%%%%%%%%%%%%%%%%%%%%
			\addplot[SEEDS] coordinates{
				(468.81,0.450675)
				(1270.11,0.45755)
				(3895.78,0.452425)
			};

			% FH %%%%%%%%%%%%%%%%%%%%%%%%%%%%%%%%%%%%%%%%%%%%%%%%%%%%%%%%%%%%
			\addplot[FH] coordinates{
				(782.42,0.03667495)
				(799.09,0.034075)
				(3219.63,0.02397505)
			};

			% PRESLIC %%%%%%%%%%%%%%%%%%%%%%%%%%%%%%%%%%%%%%%%%%%%%%%%%%%%%%%%%%%%
			\addplot[PRESLIC] coordinates{
				(369,0.01435)
				(1229.22,0.01415)
				(3059.43,0.01335)
			};

			% SLIC %%%%%%%%%%%%%%%%%%%%%%%%%%%%%%%%%%%%%%%%%%%%%%%%%%%%%%%%%%%%
			\addplot[SLIC] coordinates{
				(368.57,0.072925)
				(1203.61,0.07685)
				(3038.13,0.078375)
			};

			% VLSLIC %%%%%%%%%%%%%%%%%%%%%%%%%%%%%%%%%%%%%%%%%%%%%%%%%%%%%%%%%%%%
			\addplot[VLSLICI] coordinates{
				(763.62,0.0588)
				(1348.08,0.05875)
				(2954.18,0.05895)
			};

			% TODO
			% REFH %%%%%%%%%%%%%%%%%%%%%%%%%%%%%%%%%%%%%%%%%%%%%%%%%%%%%%%%%%%%
			\addplot[REFHI] coordinates{
				(465.54, 0.094)
				(862.475, 0.093)
				(2930.78, 0.086)
			};

		\end{axis}
	\end{tikzpicture}
\end{subfigure}

	\caption{\Rec, \UE and runtime in seconds $t$ on the \BSDS dataset for different implementations of \SLIC, \SEEDS and \FH.
	In particular, \reSEEDS and \reFH show slightly better performance which may be explained by
	improved connectivity. \vlSLIC, in contrast, shows similar performance to \SLIC and, indeed,
	the implementations are very similar. Finally, \preSLIC reduces runtime by reducing the number
	of iterations spend on individual superpixels.
    \textbf{Best viewed in color.}}
	\label{fig:experiments-implementations-bsds500}
    \vskip 12px
    % Total: 26 (+2 depth)
%\begin{mdframed}
	{\scriptsize
		\begin{tabularx}{0.475\textwidth}{X X X l}
			\ref{plot:fh} \FH &
			\ref{plot:refh} \reFH &
			\ref{plot:slic} \SLIC &
			\ref{plot:vlslic} \vlSLIC\\
			\ref{plot:seeds} \SEEDS &
			\ref{plot:reseeds} \reSEEDS &
			\ref{plot:preslic} \preSLIC
		\end{tabularx}
	}
%\end{mdframed}

\end{figure}

Most algorithms are robust to salt and pepper noise; blurring, in contrast, tends
to reduce performance. We consider Figure \ref{subfig:experiments-robustness-noise} showing \Rec, \UE and \K
for $p \in \{0, 0.04,$ $0.08, 0.12, 0.16\}$ being the probability of a pixel being salt or pepper.
Note that Figure \ref{subfig:experiments-robustness-noise} shows the number of
superpixels $K$ before enforcing connectivity as described in Section \ref{subsec:parameter-optimization-connectivity}.
As we can deduce, salt and pepper noise only slightly reduces \Rec and \UE for most algorithms.
Some algorithms compensate the noise by generating more superpixels such as \VCr or \SEAWr
while only slightly reducing performance. In contrast, for \QSr the performance even increases
-- a result of the strongly increasing number of superpixels.
Similar results can be obtained for Gaussian additive noise.
Turning to Figure \ref{subfig:experiments-robustness-blur}
showing \Rec, \UE and \K for $k \in \{0, 5, 9, 13, 17\}$ being the size of a box
filter used for average blurring. As expected, blurring leads to reduced performance with
respect to both \Rec and \UE. Furthermore, it leads to a reduced number of generated
superpixels for algorithms such as \QSr or \VCr. Similar observations can be made for motion blur as well as Gaussian blur.

\begin{figure}[t]
    \centering
    \begin{subfigure}[b]{\halfthreeone\textwidth}\phantomsubcaption\label{subfig:experiments-robustness-salt-pepper.rec.mean_min}
	%%%%%%%%%%%%%%%%%%%%%%%%%%%%%%%%%%%%%%%%%%%%%%%%%%%%%%%%%%%%
	% salt-pepper.rec.mean_min
	%%%%%%%%%%%%%%%%%%%%%%%%%%%%%%%%%%%%%%%%%%%%%%%%%%%%%%%%%%%%
	\begin{tikzpicture}
		\begin{axis}[EQBSDS500RobustnessRec,
			ylabel=\Rec,
			symbolic x coords= {
				0,
				0.04,
				0.08,
				0.12,
				0.16,
			},
			xtick={0,0.08,0.16}]

			% CCS rec.mean_min %%%%%%%%%%%%%%%%%%%%%%%%%%%%%%%%%%%%%%%%%%%%%%%%%%%%%%%%%%%%
			\addplot[CCS] coordinates{
				(0,0.706618)
				(0.04,0.70955)
				(0.08,0.707202)
				(0.12,0.697276)
				(0.16,0.687284)
			};

			% LSC rec.mean_min %%%%%%%%%%%%%%%%%%%%%%%%%%%%%%%%%%%%%%%%%%%%%%%%%%%%%%%%%%%%
			\addplot[LSC] coordinates{
				(0,0.843883)
				(0.04,0.821698)
				(0.08,0.791685)
				(0.12,0.76977)
				(0.16,0.747131)
			};

			% POISE rec.mean_min %%%%%%%%%%%%%%%%%%%%%%%%%%%%%%%%%%%%%%%%%%%%%%%%%%%%%%%%%%%%
			\addplot[POISE] coordinates{
				(0,0.776481)
				(0.04,0.768781)
				(0.08,0.761137)
				(0.12,0.750757)
				(0.16,0.741739)
			};

			% RW rec.mean_min %%%%%%%%%%%%%%%%%%%%%%%%%%%%%%%%%%%%%%%%%%%%%%%%%%%%%%%%%%%%
			\addplot[RW] coordinates{
				(0,0)
				(0.04,0.593195)
				(0.08,0.58393)
				(0.12,0.580095)
				(0.16,0.571446)
			};

			% WP rec.mean_min %%%%%%%%%%%%%%%%%%%%%%%%%%%%%%%%%%%%%%%%%%%%%%%%%%%%%%%%%%%%
			\addplot[WP] coordinates{
				(0,0.668597)
				(0.04,0.652255)
				(0.08,0.624842)
				(0.12,0.58217)
				(0.16,0.547849)
			};

			% CIS rec.mean_min %%%%%%%%%%%%%%%%%%%%%%%%%%%%%%%%%%%%%%%%%%%%%%%%%%%%%%%%%%%%
			\addplot[CIS] coordinates{
				(0,0.636139)
				(0.04,0.666544)
				(0.08,0.677006)
				(0.12,0.668329)
				(0.16,0.657536)
			};

			% CRS rec.mean_min %%%%%%%%%%%%%%%%%%%%%%%%%%%%%%%%%%%%%%%%%%%%%%%%%%%%%%%%%%%%
			\addplot[CRS] coordinates{
				(0,0.834591)
				(0.04,0.811842)
				(0.08,0.79571)
				(0.12,0.78372)
				(0.16,0.777032)
			};

			% CW rec.mean_min %%%%%%%%%%%%%%%%%%%%%%%%%%%%%%%%%%%%%%%%%%%%%%%%%%%%%%%%%%%%
			\addplot[CW] coordinates{
				(0,0.699117)
				(0.04,0.688181)
				(0.08,0.678549)
				(0.12,0.667058)
				(0.16,0.655728)
			};

			% EAMS rec.mean_min %%%%%%%%%%%%%%%%%%%%%%%%%%%%%%%%%%%%%%%%%%%%%%%%%%%%%%%%%%%%
			\addplot[EAMS] coordinates{
				(0,0.785763)
				(0.04,0.78614)
				(0.08,0.786193)
				(0.12,0.787719)
				(0.16,0.785319)
			};

			% ERGC rec.mean_min %%%%%%%%%%%%%%%%%%%%%%%%%%%%%%%%%%%%%%%%%%%%%%%%%%%%%%%%%%%%
			\addplot[ERGC] coordinates{
				(0,0.756603)
				(0.04,0.736551)
				(0.08,0.722323)
				(0.12,0.707755)
				(0.16,0.695918)
			};

			% ERS rec.mean_min %%%%%%%%%%%%%%%%%%%%%%%%%%%%%%%%%%%%%%%%%%%%%%%%%%%%%%%%%%%%
			\addplot[ERS] coordinates{
				(0,0.77887)
				(0.04,0.773266)
				(0.08,0.773746)
				(0.12,0.771356)
				(0.16,0.763404)
			};

			% ETPS rec.mean_min %%%%%%%%%%%%%%%%%%%%%%%%%%%%%%%%%%%%%%%%%%%%%%%%%%%%%%%%%%%%
			\addplot[ETPS] coordinates{
				(0,0.901663)
				(0.04,0.910417)
				(0.08,0.921386)
				(0.12,0.930002)
				(0.16,0.936541)
			};

			% FH rec.mean_min %%%%%%%%%%%%%%%%%%%%%%%%%%%%%%%%%%%%%%%%%%%%%%%%%%%%%%%%%%%%
			\addplot[FH] coordinates{
				(0,0.653246)
				(0.04,0.648109)
				(0.08,0.642653)
				(0.12,0.641582)
				(0.16,0.638115)
			};

			% MSS rec.mean_min %%%%%%%%%%%%%%%%%%%%%%%%%%%%%%%%%%%%%%%%%%%%%%%%%%%%%%%%%%%%
			\addplot[MSS] coordinates{
				(0,0.730153)
				(0.04,0.724668)
				(0.08,0.720373)
				(0.12,0.71548)
				(0.16,0.710135)
			};

			% PB rec.mean_min %%%%%%%%%%%%%%%%%%%%%%%%%%%%%%%%%%%%%%%%%%%%%%%%%%%%%%%%%%%%
			\addplot[PB] coordinates{
				(0,0.615142)
				(0.04,0.65845)
				(0.08,0.66913)
				(0.12,0.683065)
				(0.16,0.701987)
			};

			% PRESLIC rec.mean_min %%%%%%%%%%%%%%%%%%%%%%%%%%%%%%%%%%%%%%%%%%%%%%%%%%%%%%%%%%%%
			\addplot[PRESLIC] coordinates{
				(0,0.704329)
				(0.04,0.68981)
				(0.08,0.671746)
				(0.12,0.656161)
				(0.16,0.638995)
			};

			% QS rec.mean_min %%%%%%%%%%%%%%%%%%%%%%%%%%%%%%%%%%%%%%%%%%%%%%%%%%%%%%%%%%%%
			\addplot[QS] coordinates{
				(0,0.786206)
				(0.04,0.888975)
				(0.08,0.941518)
				(0.12,0.969838)
				(0.16,0.983516)
			};

			% RESEEDS rec.mean_min %%%%%%%%%%%%%%%%%%%%%%%%%%%%%%%%%%%%%%%%%%%%%%%%%%%%%%%%%%%%
			\addplot[RESEEDS] coordinates{
				(0,0.918521)
				(0.04,0.922841)
				(0.08,0.926112)
				(0.12,0.931436)
				(0.16,0.912143)
			};

			% SEAW rec.mean_min %%%%%%%%%%%%%%%%%%%%%%%%%%%%%%%%%%%%%%%%%%%%%%%%%%%%%%%%%%%%
			\addplot[SEAW] coordinates{
				(0,0.761138)
				(0.04,0.704067)
				(0.08,0.708204)
				(0.12,0.713274)
				(0.16,0.717825)
			};

			% SEEDS rec.mean_min %%%%%%%%%%%%%%%%%%%%%%%%%%%%%%%%%%%%%%%%%%%%%%%%%%%%%%%%%%%%
			\addplot[SEEDS] coordinates{
				(0,0.922132)
				(0.04,0.92937)
				(0.08,0.929101)
				(0.12,0.930778)
				(0.16,0.931857)
			};

			% SLIC rec.mean_min %%%%%%%%%%%%%%%%%%%%%%%%%%%%%%%%%%%%%%%%%%%%%%%%%%%%%%%%%%%%
			\addplot[SLIC] coordinates{
				(0,0.726167)
				(0.04,0.702687)
				(0.08,0.676811)
				(0.12,0.654143)
				(0.16,0.632632)
			};

			% TP rec.mean_min %%%%%%%%%%%%%%%%%%%%%%%%%%%%%%%%%%%%%%%%%%%%%%%%%%%%%%%%%%%%
			\addplot[TP] coordinates{
				(0,0.60368)
				(0.04,0.668362)
				(0.08,0.665018)
				(0.12,0.658363)
				(0.16,0.65019)
			};

			% TPS rec.mean_min %%%%%%%%%%%%%%%%%%%%%%%%%%%%%%%%%%%%%%%%%%%%%%%%%%%%%%%%%%%%
			\addplot[TPS] coordinates{
				(0,0.620363)
				(0.04,0.605397)
				(0.08,0.590221)
				(0.12,0.582724)
				(0.16,0.571291)
			};

			% VC rec.mean_min %%%%%%%%%%%%%%%%%%%%%%%%%%%%%%%%%%%%%%%%%%%%%%%%%%%%%%%%%%%%
			\addplot[VC] coordinates{
				(0,0.856154)
				(0.04,0.842539)
				(0.08,0.837406)
				(0.12,0.833472)
				(0.16,0.835234)
			};

			% VLSLIC rec.mean_min %%%%%%%%%%%%%%%%%%%%%%%%%%%%%%%%%%%%%%%%%%%%%%%%%%%%%%%%%%%%
			\addplot[VLSLIC] coordinates{
				(0,0.819088)
				(0.04,0.777578)
				(0.08,0.742826)
				(0.12,0.706885)
				(0.16,0.684141)
			};

			% W rec.mean_min %%%%%%%%%%%%%%%%%%%%%%%%%%%%%%%%%%%%%%%%%%%%%%%%%%%%%%%%%%%%
			\addplot[W] coordinates{
				(0,0.692999)
				(0.04,0.676969)
				(0.08,0.665391)
				(0.12,0.654977)
				(0.16,0.643579)
			};

			\end{axis}
	\end{tikzpicture}
\end{subfigure}
\begin{subfigure}[b]{\halfthreeone\textwidth}\phantomsubcaption\label{subfig:experiments-robustness-salt-pepper.ue_np.mean_max}
	%%%%%%%%%%%%%%%%%%%%%%%%%%%%%%%%%%%%%%%%%%%%%%%%%%%%%%%%%%%%
	% salt-pepper.ue_np.mean_max
	%%%%%%%%%%%%%%%%%%%%%%%%%%%%%%%%%%%%%%%%%%%%%%%%%%%%%%%%%%%%
	\begin{tikzpicture}
		\begin{axis}[EQBSDS500RobustnessUE,
			ylabel=\UE,
			symbolic x coords= {
				0,
				0.04,
				0.08,
				0.12,
				0.16,
			},
			xtick={0,0.08,0.16}]

			% CCS ue_np.mean_max %%%%%%%%%%%%%%%%%%%%%%%%%%%%%%%%%%%%%%%%%%%%%%%%%%%%%%%%%%%%
			\addplot[CCS] coordinates{
				(0,0.129268)
				(0.04,0.130222)
				(0.08,0.132634)
				(0.12,0.136609)
				(0.16,0.140972)
			};

			% LSC ue_np.mean_max %%%%%%%%%%%%%%%%%%%%%%%%%%%%%%%%%%%%%%%%%%%%%%%%%%%%%%%%%%%%
			\addplot[LSC] coordinates{
				(0,0.107074)
				(0.04,0.116122)
				(0.08,0.125957)
				(0.12,0.135518)
				(0.16,0.143323)
			};

			% POISE ue_np.mean_max %%%%%%%%%%%%%%%%%%%%%%%%%%%%%%%%%%%%%%%%%%%%%%%%%%%%%%%%%%%%
			\addplot[POISE] coordinates{
				(0,0.128172)
				(0.04,0.139034)
				(0.08,0.151428)
				(0.12,0.162812)
				(0.16,0.172363)
			};

			% RW ue_np.mean_max %%%%%%%%%%%%%%%%%%%%%%%%%%%%%%%%%%%%%%%%%%%%%%%%%%%%%%%%%%%%
			\addplot[RW] coordinates{
				(0,0)
				(0.04,0.147155)
				(0.08,0.151081)
				(0.12,0.154979)
				(0.16,0.160228)
			};

			% WP ue_np.mean_max %%%%%%%%%%%%%%%%%%%%%%%%%%%%%%%%%%%%%%%%%%%%%%%%%%%%%%%%%%%%
			\addplot[WP] coordinates{
				(0,0.148291)
				(0.04,0.160305)
				(0.08,0.178691)
				(0.12,0.207448)
				(0.16,0.226226)
			};

			% CIS ue_np.mean_max %%%%%%%%%%%%%%%%%%%%%%%%%%%%%%%%%%%%%%%%%%%%%%%%%%%%%%%%%%%%
			\addplot[CIS] coordinates{
				(0,0.0617578)
				(0.04,0.127078)
				(0.08,0.125542)
				(0.12,0.127834)
				(0.16,0.13048)
			};

			% CRS ue_np.mean_max %%%%%%%%%%%%%%%%%%%%%%%%%%%%%%%%%%%%%%%%%%%%%%%%%%%%%%%%%%%%
			\addplot[CRS] coordinates{
				(0,0.108806)
				(0.04,0.114773)
				(0.08,0.12009)
				(0.12,0.123887)
				(0.16,0.128148)
			};

			% CW ue_np.mean_max %%%%%%%%%%%%%%%%%%%%%%%%%%%%%%%%%%%%%%%%%%%%%%%%%%%%%%%%%%%%
			\addplot[CW] coordinates{
				(0,0.144929)
				(0.04,0.161268)
				(0.08,0.166307)
				(0.12,0.161921)
				(0.16,0.170232)
			};

			% EAMS ue_np.mean_max %%%%%%%%%%%%%%%%%%%%%%%%%%%%%%%%%%%%%%%%%%%%%%%%%%%%%%%%%%%%
			\addplot[EAMS] coordinates{
				(0,0.163536)
				(0.04,0.162229)
				(0.08,0.172235)
				(0.12,0.183464)
				(0.16,0.193819)
			};

			% ERGC ue_np.mean_max %%%%%%%%%%%%%%%%%%%%%%%%%%%%%%%%%%%%%%%%%%%%%%%%%%%%%%%%%%%%
			\addplot[ERGC] coordinates{
				(0,0.121812)
				(0.04,0.127053)
				(0.08,0.131103)
				(0.12,0.136648)
				(0.16,0.142495)
			};

			% ERS ue_np.mean_max %%%%%%%%%%%%%%%%%%%%%%%%%%%%%%%%%%%%%%%%%%%%%%%%%%%%%%%%%%%%
			\addplot[ERS] coordinates{
				(0,0.128528)
				(0.04,0.139632)
				(0.08,0.149631)
				(0.12,0.161474)
				(0.16,0.179252)
			};

			% ETPS ue_np.mean_max %%%%%%%%%%%%%%%%%%%%%%%%%%%%%%%%%%%%%%%%%%%%%%%%%%%%%%%%%%%%
			\addplot[ETPS] coordinates{
				(0,0.120575)
				(0.04,0.129498)
				(0.08,0.137519)
				(0.12,0.16954)
				(0.16,0.163737)
			};

			% FH ue_np.mean_max %%%%%%%%%%%%%%%%%%%%%%%%%%%%%%%%%%%%%%%%%%%%%%%%%%%%%%%%%%%%
			\addplot[FH] coordinates{
				(0,0.0895417)
				(0.04,0.183884)
				(0.08,0.1907)
				(0.12,0.193973)
				(0.16,0.199243)
			};

			% MSS ue_np.mean_max %%%%%%%%%%%%%%%%%%%%%%%%%%%%%%%%%%%%%%%%%%%%%%%%%%%%%%%%%%%%
			\addplot[MSS] coordinates{
				(0,0.135999)
				(0.04,0.136819)
				(0.08,0.139058)
				(0.12,0.140357)
				(0.16,0.14284)
			};

			% PB ue_np.mean_max %%%%%%%%%%%%%%%%%%%%%%%%%%%%%%%%%%%%%%%%%%%%%%%%%%%%%%%%%%%%
			\addplot[PB] coordinates{
				(0,0.209432)
				(0.04,0.190183)
				(0.08,0.197812)
				(0.12,0.204181)
				(0.16,0.209512)
			};

			% PRESLIC ue_np.mean_max %%%%%%%%%%%%%%%%%%%%%%%%%%%%%%%%%%%%%%%%%%%%%%%%%%%%%%%%%%%%
			\addplot[PRESLIC] coordinates{
				(0,0.16315)
				(0.04,0.160388)
				(0.08,0.16632)
				(0.12,0.174213)
				(0.16,0.182283)
			};

			% QS ue_np.mean_max %%%%%%%%%%%%%%%%%%%%%%%%%%%%%%%%%%%%%%%%%%%%%%%%%%%%%%%%%%%%
			\addplot[QS] coordinates{
				(0,0.0480242)
				(0.04,0.10741)
				(0.08,0.103501)
				(0.12,0.0995347)
				(0.16,0.0956536)
			};

			% RESEEDS ue_np.mean_max %%%%%%%%%%%%%%%%%%%%%%%%%%%%%%%%%%%%%%%%%%%%%%%%%%%%%%%%%%%%
			\addplot[RESEEDS] coordinates{
				(0,0.132874)
				(0.04,0.140919)
				(0.08,0.147707)
				(0.12,0.161776)
				(0.16,0.294496)
			};

			% SEAW ue_np.mean_max %%%%%%%%%%%%%%%%%%%%%%%%%%%%%%%%%%%%%%%%%%%%%%%%%%%%%%%%%%%%
			\addplot[SEAW] coordinates{
				(0,0.0882449)
				(0.04,0.133313)
				(0.08,0.136773)
				(0.12,0.140114)
				(0.16,0.144389)
			};

			% SEEDS ue_np.mean_max %%%%%%%%%%%%%%%%%%%%%%%%%%%%%%%%%%%%%%%%%%%%%%%%%%%%%%%%%%%%
			\addplot[SEEDS] coordinates{
				(0,0.166744)
				(0.04,0.172316)
				(0.08,0.175096)
				(0.12,0.177122)
				(0.16,0.180743)
			};

			% SLIC ue_np.mean_max %%%%%%%%%%%%%%%%%%%%%%%%%%%%%%%%%%%%%%%%%%%%%%%%%%%%%%%%%%%%
			\addplot[SLIC] coordinates{
				(0,0.133617)
				(0.04,0.142199)
				(0.08,0.160968)
				(0.12,0.16971)
				(0.16,0.167733)
			};

			% TP ue_np.mean_max %%%%%%%%%%%%%%%%%%%%%%%%%%%%%%%%%%%%%%%%%%%%%%%%%%%%%%%%%%%%
			\addplot[TP] coordinates{
				(0,0.146071)
				(0.04,0.12468)
				(0.08,0.128784)
				(0.12,0.133796)
				(0.16,0.139391)
			};

			% TPS ue_np.mean_max %%%%%%%%%%%%%%%%%%%%%%%%%%%%%%%%%%%%%%%%%%%%%%%%%%%%%%%%%%%%
			\addplot[TPS] coordinates{
				(0,0.0666233)
				(0.04,0.164594)
				(0.08,0.169455)
				(0.12,0.163848)
				(0.16,0.166722)
			};

			% VC ue_np.mean_max %%%%%%%%%%%%%%%%%%%%%%%%%%%%%%%%%%%%%%%%%%%%%%%%%%%%%%%%%%%%
			\addplot[VC] coordinates{
				(0,0.0767202)
				(0.04,0.11359)
				(0.08,0.12144)
				(0.12,0.128789)
				(0.16,0.131813)
			};

			% VLSLIC ue_np.mean_max %%%%%%%%%%%%%%%%%%%%%%%%%%%%%%%%%%%%%%%%%%%%%%%%%%%%%%%%%%%%
			\addplot[VLSLIC] coordinates{
				(0,0.12431)
				(0.04,0.139033)
				(0.08,0.162061)
				(0.12,0.163454)
				(0.16,0.173809)
			};

			% W ue_np.mean_max %%%%%%%%%%%%%%%%%%%%%%%%%%%%%%%%%%%%%%%%%%%%%%%%%%%%%%%%%%%%
			\addplot[W] coordinates{
				(0,0.162436)
				(0.04,0.170377)
				(0.08,0.177472)
				(0.12,0.184894)
				(0.16,0.193174)
			};

			\end{axis}
	\end{tikzpicture}
\end{subfigure}
\begin{subfigure}[b]{\halfthreeone\textwidth}\phantomsubcaption\label{subfig:experiments-robustness-salt-pepper.sp.mean_min}
	%%%%%%%%%%%%%%%%%%%%%%%%%%%%%%%%%%%%%%%%%%%%%%%%%%%%%%%%%%%%
	% salt-pepper.sp.mean_min
	%%%%%%%%%%%%%%%%%%%%%%%%%%%%%%%%%%%%%%%%%%%%%%%%%%%%%%%%%%%%
	\begin{tikzpicture}
		\begin{axis}[EQBSDS500RobustnessK,
			ylabel=\K,
			symbolic x coords= {
				0,
				0.04,
				0.08,
				0.12,
				0.16,
			},
			xtick={0,0.08,0.16}]

			% CCS sp.mean_min %%%%%%%%%%%%%%%%%%%%%%%%%%%%%%%%%%%%%%%%%%%%%%%%%%%%%%%%%%%%
			\addplot[CCS] coordinates{
				(0,453.82)
				(0.04,484.136)
				(0.08,505.638)
				(0.12,526.191)
				(0.16,558.08)
			};

			% POISE sp.mean_min %%%%%%%%%%%%%%%%%%%%%%%%%%%%%%%%%%%%%%%%%%%%%%%%%%%%%%%%%%%%
			\addplot[POISE] coordinates{
				(0,408.385)
				(0.04,408.784)
				(0.08,408.794)
				(0.12,408.749)
				(0.16,408.613)
			};

			% RW sp.mean_min %%%%%%%%%%%%%%%%%%%%%%%%%%%%%%%%%%%%%%%%%%%%%%%%%%%%%%%%%%%%
			\addplot[RW] coordinates{
				(0,0)
				(0.04,408.833)
				(0.08,411.875)
				(0.12,416.292)
				(0.16,419.583)
			};

			% LSC sp.mean_min %%%%%%%%%%%%%%%%%%%%%%%%%%%%%%%%%%%%%%%%%%%%%%%%%%%%%%%%%%%%
			\addplot[LSC] coordinates{
				(0,1045.31)
				(0.04,1136.31)
				(0.08,948.905)
				(0.12,825.91)
				(0.16,758.191)
			};

			% WP sp.mean_min %%%%%%%%%%%%%%%%%%%%%%%%%%%%%%%%%%%%%%%%%%%%%%%%%%%%%%%%%%%%
			\addplot[WP] coordinates{
				(0,384)
				(0.04,384)
				(0.08,384)
				(0.12,384)
				(0.16,384)
			};

			% CIS sp.mean_min %%%%%%%%%%%%%%%%%%%%%%%%%%%%%%%%%%%%%%%%%%%%%%%%%%%%%%%%%%%%
			\addplot[CIS] coordinates{
				(0,472.685)
				(0.04,677.266)
				(0.08,837.894)
				(0.12,892.171)
				(0.16,922.025)
			};

			% CRS sp.mean_min %%%%%%%%%%%%%%%%%%%%%%%%%%%%%%%%%%%%%%%%%%%%%%%%%%%%%%%%%%%%
			\addplot[CRS] coordinates{
				(0,635.865)
				(0.04,589.417)
				(0.08,649.236)
				(0.12,728.121)
				(0.16,821.256)
			};

			% CW sp.mean_min %%%%%%%%%%%%%%%%%%%%%%%%%%%%%%%%%%%%%%%%%%%%%%%%%%%%%%%%%%%%
			\addplot[CW] coordinates{
				(0,405.935)
				(0.04,402.734)
				(0.08,402.839)
				(0.12,402.869)
				(0.16,402.869)
			};

			% EAMS sp.mean_min %%%%%%%%%%%%%%%%%%%%%%%%%%%%%%%%%%%%%%%%%%%%%%%%%%%%%%%%%%%%
			\addplot[EAMS] coordinates{
				(0,309.87)
				(0.04,336.905)
				(0.08,342.03)
				(0.12,353.407)
				(0.16,372.633)
			};

			% ERGC sp.mean_min %%%%%%%%%%%%%%%%%%%%%%%%%%%%%%%%%%%%%%%%%%%%%%%%%%%%%%%%%%%%
			\addplot[ERGC] coordinates{
				(0,400)
				(0.04,400)
				(0.08,400)
				(0.12,400)
				(0.16,400)
			};

			% ERS sp.mean_min %%%%%%%%%%%%%%%%%%%%%%%%%%%%%%%%%%%%%%%%%%%%%%%%%%%%%%%%%%%%
			\addplot[ERS] coordinates{
				(0,400)
				(0.04,400)
				(0.08,400)
				(0.12,400)
				(0.16,400)
			};

			% ETPS sp.mean_min %%%%%%%%%%%%%%%%%%%%%%%%%%%%%%%%%%%%%%%%%%%%%%%%%%%%%%%%%%%%
			\addplot[ETPS] coordinates{
				(0,425)
				(0.04,425)
				(0.08,425)
				(0.12,425)
				(0.16,425)
			};

			% FH sp.mean_min %%%%%%%%%%%%%%%%%%%%%%%%%%%%%%%%%%%%%%%%%%%%%%%%%%%%%%%%%%%%
			\addplot[FH] coordinates{
				(0,566.065)
				(0.04,1056.34)
				(0.08,1142.69)
				(0.12,1248.58)
				(0.16,1381.2)
			};

			% MSS sp.mean_min %%%%%%%%%%%%%%%%%%%%%%%%%%%%%%%%%%%%%%%%%%%%%%%%%%%%%%%%%%%%
			\addplot[MSS] coordinates{
				(0,420.66)
				(0.04,400.482)
				(0.08,396.533)
				(0.12,395.116)
				(0.16,393.598)
			};

			% PB sp.mean_min %%%%%%%%%%%%%%%%%%%%%%%%%%%%%%%%%%%%%%%%%%%%%%%%%%%%%%%%%%%%
			\addplot[PB] coordinates{
				(0,398.425)
				(0.04,545.864)
				(0.08,628.794)
				(0.12,740.477)
				(0.16,887.899)
			};

			% PRESLIC sp.mean_min %%%%%%%%%%%%%%%%%%%%%%%%%%%%%%%%%%%%%%%%%%%%%%%%%%%%%%%%%%%%
			\addplot[PRESLIC] coordinates{
				(0,369)
				(0.04,350.337)
				(0.08,336.085)
				(0.12,321.633)
				(0.16,307.231)
			};

			% QS sp.mean_min %%%%%%%%%%%%%%%%%%%%%%%%%%%%%%%%%%%%%%%%%%%%%%%%%%%%%%%%%%%%
			\addplot[QS] coordinates{
				(0,2626.4)
				(0.04,7909.63)
				(0.08,12967.1)
				(0.12,17779.7)
				(0.16,22272.2)
			};

			% RESEEDS sp.mean_min %%%%%%%%%%%%%%%%%%%%%%%%%%%%%%%%%%%%%%%%%%%%%%%%%%%%%%%%%%%%
			\addplot[RESEEDS] coordinates{
				(0,401.465)
				(0.04,401.633)
				(0.08,401.377)
				(0.12,401.518)
				(0.16,400.829)
			};

			% SEAW sp.mean_min %%%%%%%%%%%%%%%%%%%%%%%%%%%%%%%%%%%%%%%%%%%%%%%%%%%%%%%%%%%%
			\addplot[SEAW] coordinates{
				(0,923.49)
				(0.04,1108.94)
				(0.08,1359.16)
				(0.12,1676.77)
				(0.16,2051.44)
			};

			% SEEDS sp.mean_min %%%%%%%%%%%%%%%%%%%%%%%%%%%%%%%%%%%%%%%%%%%%%%%%%%%%%%%%%%%%
			\addplot[SEEDS] coordinates{
				(0,468.81)
				(0.04,504.02)
				(0.08,516.618)
				(0.12,527.593)
				(0.16,536.407)
			};

			% SLIC sp.mean_min %%%%%%%%%%%%%%%%%%%%%%%%%%%%%%%%%%%%%%%%%%%%%%%%%%%%%%%%%%%%
			\addplot[SLIC] coordinates{
				(0,368.57)
				(0.04,319.528)
				(0.08,282.266)
				(0.12,258.367)
				(0.16,241.693)
			};

			% TP sp.mean_min %%%%%%%%%%%%%%%%%%%%%%%%%%%%%%%%%%%%%%%%%%%%%%%%%%%%%%%%%%%%
			\addplot[TP] coordinates{
				(0,381.285)
				(0.04,560.945)
				(0.08,566.658)
				(0.12,574.005)
				(0.16,586.769)
			};

			% TPS sp.mean_min %%%%%%%%%%%%%%%%%%%%%%%%%%%%%%%%%%%%%%%%%%%%%%%%%%%%%%%%%%%%
			\addplot[TPS] coordinates{
				(0,445.405)
				(0.04,455.799)
				(0.08,457.065)
				(0.12,458.608)
				(0.16,458.608)
			};

			% VC sp.mean_min %%%%%%%%%%%%%%%%%%%%%%%%%%%%%%%%%%%%%%%%%%%%%%%%%%%%%%%%%%%%
			\addplot[VC] coordinates{
				(0,1375.27)
				(0.04,3058.35)
				(0.08,3213.37)
				(0.12,3499.17)
				(0.16,3868.75)
			};

			% VLSLIC sp.mean_min %%%%%%%%%%%%%%%%%%%%%%%%%%%%%%%%%%%%%%%%%%%%%%%%%%%%%%%%%%%%
			\addplot[VLSLIC] coordinates{
				(0,763.62)
				(0.04,617.724)
				(0.08,537.874)
				(0.12,481.9)
				(0.16,439.975)
			};

			% W sp.mean_min %%%%%%%%%%%%%%%%%%%%%%%%%%%%%%%%%%%%%%%%%%%%%%%%%%%%%%%%%%%%
			\addplot[W] coordinates{
				(0,387.92)
				(0.04,386.724)
				(0.08,386.291)
				(0.12,386.357)
				(0.16,386.487)
			};

			\end{axis}
	\end{tikzpicture}
\end{subfigure}

    \caption{The influence of salt and pepper noise for $p \in \{0, 0.04, 0.08, 0.12, 0.16\}$ being the
    probability of salt or pepper. Regarding \Rec and \UE, most algorithms are not
    significantly influence by salt and pepper noise. Algorithms such as
    \QS and \VC compensate the noise by generating additional superpixels.
    \textbf{Best viewed in color.}}
    \label{subfig:experiments-robustness-noise}
\end{figure}
\begin{figure}[t]
	\centering
    \begin{subfigure}[b]{\halfthreeone\textwidth}\phantomsubcaption\label{subfig:experiments-robustness-blur.rec.mean_min}
	%%%%%%%%%%%%%%%%%%%%%%%%%%%%%%%%%%%%%%%%%%%%%%%%%%%%%%%%%%%%
	% blur.rec.mean_min
	%%%%%%%%%%%%%%%%%%%%%%%%%%%%%%%%%%%%%%%%%%%%%%%%%%%%%%%%%%%%
	\begin{tikzpicture}
		\begin{axis}[EQBSDS500RobustnessRec,
			ylabel=\Rec,
			symbolic x coords= {
				0,
				5,
				9,
				13,
				17,
			},
			xtick=data]

			% CCS rec.mean_min %%%%%%%%%%%%%%%%%%%%%%%%%%%%%%%%%%%%%%%%%%%%%%%%%%%%%%%%%%%%
			\addplot[CCS] coordinates{
				(0,0.706618)
				(5,0.658041)
				(9,0.584521)
				(13,0.508144)
				(17,0.457041)
			};

			% LSC rec.mean_min %%%%%%%%%%%%%%%%%%%%%%%%%%%%%%%%%%%%%%%%%%%%%%%%%%%%%%%%%%%%
			\addplot[LSC] coordinates{
				(0,0.843883)
				(5,0.845213)
				(9,0.803319)
				(13,0.747817)
				(17,0.698278)
			};

			% POISE rec.mean_min %%%%%%%%%%%%%%%%%%%%%%%%%%%%%%%%%%%%%%%%%%%%%%%%%%%%%%%%%%%%
			\addplot[POISE] coordinates{
				(0,0.776481)
				(5,0.732512)
				(9,0.673233)
				(13,0.603598)
				(17,0.553914)
			};

			% RW rec.mean_min %%%%%%%%%%%%%%%%%%%%%%%%%%%%%%%%%%%%%%%%%%%%%%%%%%%%%%%%%%%%
			\addplot[RW] coordinates{
				(0,0)
				(5,0.642622)
				(9,0.586494)
				(13,0.535793)
				(17,0.502067)
			};

			% WP rec.mean_min %%%%%%%%%%%%%%%%%%%%%%%%%%%%%%%%%%%%%%%%%%%%%%%%%%%%%%%%%%%%
			\addplot[WP] coordinates{
				(0,0.668597)
				(5,0.654304)
				(9,0.560971)
				(13,0.493219)
				(17,0.461277)
			};

			% CIS rec.mean_min %%%%%%%%%%%%%%%%%%%%%%%%%%%%%%%%%%%%%%%%%%%%%%%%%%%%%%%%%%%%
			\addplot[CIS] coordinates{
				(0,0.636139)
				(5,0.516581)
				(9,0.40546)
				(13,0.340946)
				(17,0.304633)
			};

			% CRS rec.mean_min %%%%%%%%%%%%%%%%%%%%%%%%%%%%%%%%%%%%%%%%%%%%%%%%%%%%%%%%%%%%
			\addplot[CRS] coordinates{
				(0,0.834591)
				(5,0.810699)
				(9,0.767247)
				(13,0.727603)
				(17,0.690951)
			};

			% CW rec.mean_min %%%%%%%%%%%%%%%%%%%%%%%%%%%%%%%%%%%%%%%%%%%%%%%%%%%%%%%%%%%%
			\addplot[CW] coordinates{
				(0,0.699117)
				(5,0.693248)
				(9,0.617857)
				(13,0.563714)
				(17,0.529397)
			};

			% EAMS rec.mean_min %%%%%%%%%%%%%%%%%%%%%%%%%%%%%%%%%%%%%%%%%%%%%%%%%%%%%%%%%%%%
			\addplot[EAMS] coordinates{
				(0,0.785763)
				(5,0.769758)
				(9,0.742642)
				(13,0.727944)
				(17,0.720178)
			};

			% ERGC rec.mean_min %%%%%%%%%%%%%%%%%%%%%%%%%%%%%%%%%%%%%%%%%%%%%%%%%%%%%%%%%%%%
			\addplot[ERGC] coordinates{
				(0,0.756603)
				(5,0.763287)
				(9,0.730626)
				(13,0.688447)
				(17,0.656012)
			};

			% ERS rec.mean_min %%%%%%%%%%%%%%%%%%%%%%%%%%%%%%%%%%%%%%%%%%%%%%%%%%%%%%%%%%%%
			\addplot[ERS] coordinates{
				(0,0.77887)
				(5,0.743717)
				(9,0.693797)
				(13,0.681632)
				(17,0.685176)
			};

			% ETPS rec.mean_min %%%%%%%%%%%%%%%%%%%%%%%%%%%%%%%%%%%%%%%%%%%%%%%%%%%%%%%%%%%%
			\addplot[ETPS] coordinates{
				(0,0.901663)
				(5,0.878087)
				(9,0.844627)
				(13,0.807595)
				(17,0.768999)
			};

			% FH rec.mean_min %%%%%%%%%%%%%%%%%%%%%%%%%%%%%%%%%%%%%%%%%%%%%%%%%%%%%%%%%%%%
			\addplot[FH] coordinates{
				(0,0.653246)
				(5,0.642081)
				(9,0.585137)
				(13,0.543187)
				(17,0.516094)
			};

			% MSS rec.mean_min %%%%%%%%%%%%%%%%%%%%%%%%%%%%%%%%%%%%%%%%%%%%%%%%%%%%%%%%%%%%
			\addplot[MSS] coordinates{
				(0,0.730153)
				(5,0.70094)
				(9,0.622515)
				(13,0.555875)
				(17,0.507496)
			};

			% PB rec.mean_min %%%%%%%%%%%%%%%%%%%%%%%%%%%%%%%%%%%%%%%%%%%%%%%%%%%%%%%%%%%%
			\addplot[PB] coordinates{
				(0,0.615142)
				(5,0.594889)
				(9,0.499257)
				(13,0.441529)
				(17,0.418764)
			};

			% PRESLIC rec.mean_min %%%%%%%%%%%%%%%%%%%%%%%%%%%%%%%%%%%%%%%%%%%%%%%%%%%%%%%%%%%%
			\addplot[PRESLIC] coordinates{
				(0,0.704329)
				(5,0.694292)
				(9,0.626805)
				(13,0.555692)
				(17,0.499601)
			};

			% QS rec.mean_min %%%%%%%%%%%%%%%%%%%%%%%%%%%%%%%%%%%%%%%%%%%%%%%%%%%%%%%%%%%%
			\addplot[QS] coordinates{
				(0,0.786206)
				(5,0.703174)
				(9,0.617742)
				(13,0.541993)
				(17,0.484745)
			};

			% RESEEDS rec.mean_min %%%%%%%%%%%%%%%%%%%%%%%%%%%%%%%%%%%%%%%%%%%%%%%%%%%%%%%%%%%%
			\addplot[RESEEDS] coordinates{
				(0,0.918521)
				(5,0.898892)
				(9,0.875693)
				(13,0.856144)
				(17,0.740574)
			};

			% SEAW rec.mean_min %%%%%%%%%%%%%%%%%%%%%%%%%%%%%%%%%%%%%%%%%%%%%%%%%%%%%%%%%%%%
			\addplot[SEAW] coordinates{
				(0,0.761138)
				(5,0.641337)
				(9,0.581643)
				(13,0.527588)
				(17,0.490409)
			};

			% SEEDS rec.mean_min %%%%%%%%%%%%%%%%%%%%%%%%%%%%%%%%%%%%%%%%%%%%%%%%%%%%%%%%%%%%
			\addplot[SEEDS] coordinates{
				(0,0.922132)
				(5,0.889552)
				(9,0.861879)
				(13,0.843579)
				(17,0.833363)
			};

			% SLIC rec.mean_min %%%%%%%%%%%%%%%%%%%%%%%%%%%%%%%%%%%%%%%%%%%%%%%%%%%%%%%%%%%%
			\addplot[SLIC] coordinates{
				(0,0.726167)
				(5,0.721058)
				(9,0.659831)
				(13,0.575974)
				(17,0.505316)
			};

			% TP rec.mean_min %%%%%%%%%%%%%%%%%%%%%%%%%%%%%%%%%%%%%%%%%%%%%%%%%%%%%%%%%%%%
			\addplot[TP] coordinates{
				(0,0.60368)
				(5,0.607834)
				(9,0.536232)
				(13,0.483669)
				(17,0.439846)
			};

			% TPS rec.mean_min %%%%%%%%%%%%%%%%%%%%%%%%%%%%%%%%%%%%%%%%%%%%%%%%%%%%%%%%%%%%
			\addplot[TPS] coordinates{
				(0,0.620363)
				(5,0.608853)
				(9,0.581325)
				(13,0.529433)
				(17,0.473807)
			};

			% VC rec.mean_min %%%%%%%%%%%%%%%%%%%%%%%%%%%%%%%%%%%%%%%%%%%%%%%%%%%%%%%%%%%%
			\addplot[VC] coordinates{
				(0,0.856154)
				(5,0.808309)
				(9,0.776584)
				(13,0.73511)
				(17,0.689449)
			};

			% VLSLIC rec.mean_min %%%%%%%%%%%%%%%%%%%%%%%%%%%%%%%%%%%%%%%%%%%%%%%%%%%%%%%%%%%%
			\addplot[VLSLIC] coordinates{
				(0,0.819088)
				(5,0.843078)
				(9,0.794607)
				(13,0.739173)
				(17,0.690183)
			};

			% W rec.mean_min %%%%%%%%%%%%%%%%%%%%%%%%%%%%%%%%%%%%%%%%%%%%%%%%%%%%%%%%%%%%
			\addplot[W] coordinates{
				(0,0.692999)
				(5,0.697923)
				(9,0.638138)
				(13,0.593977)
				(17,0.566149)
			};

			\end{axis}
	\end{tikzpicture}
\end{subfigure}
\begin{subfigure}[b]{\halfthreeone\textwidth}\phantomsubcaption\label{subfig:experiments-robustness-blur.ue_np.mean_max}
	%%%%%%%%%%%%%%%%%%%%%%%%%%%%%%%%%%%%%%%%%%%%%%%%%%%%%%%%%%%%
	% blur.ue_np.mean_max
	%%%%%%%%%%%%%%%%%%%%%%%%%%%%%%%%%%%%%%%%%%%%%%%%%%%%%%%%%%%%
	\begin{tikzpicture}
		\begin{axis}[EQBSDS500RobustnessUE,
			ylabel=\UE,
			symbolic x coords= {
				0,
				5,
				9,
				13,
				17,
			},
			xtick=data,
			title=\BSDS]

			% CCS ue_np.mean_max %%%%%%%%%%%%%%%%%%%%%%%%%%%%%%%%%%%%%%%%%%%%%%%%%%%%%%%%%%%%
			\addplot[CCS] coordinates{
				(0,0.129268)
				(5,0.131994)
				(9,0.144176)
				(13,0.158251)
				(17,0.171155)
			};

			% LSC ue_np.mean_max %%%%%%%%%%%%%%%%%%%%%%%%%%%%%%%%%%%%%%%%%%%%%%%%%%%%%%%%%%%%
			\addplot[LSC] coordinates{
				(0,0.107074)
				(5,0.111006)
				(9,0.125107)
				(13,0.139784)
				(17,0.152991)
			};

			% POISE ue_np.mean_max %%%%%%%%%%%%%%%%%%%%%%%%%%%%%%%%%%%%%%%%%%%%%%%%%%%%%%%%%%%%
			\addplot[POISE] coordinates{
				(0,0.128172)
				(5,0.133933)
				(9,0.162836)
				(13,0.204573)
				(17,0.257432)
			};

			% RW ue_np.mean_max %%%%%%%%%%%%%%%%%%%%%%%%%%%%%%%%%%%%%%%%%%%%%%%%%%%%%%%%%%%%
			\addplot[RW] coordinates{
				(0,0)
				(5,0.134802)
				(9,0.142051)
				(13,0.149658)
				(17,0.159095)
			};

			% WP ue_np.mean_max %%%%%%%%%%%%%%%%%%%%%%%%%%%%%%%%%%%%%%%%%%%%%%%%%%%%%%%%%%%%
			\addplot[WP] coordinates{
				(0,0.148291)
				(5,0.143006)
				(9,0.158985)
				(13,0.175147)
				(17,0.190111)
			};

			% CIS ue_np.mean_max %%%%%%%%%%%%%%%%%%%%%%%%%%%%%%%%%%%%%%%%%%%%%%%%%%%%%%%%%%%%
			\addplot[CIS] coordinates{
				(0,0.0617578)
				(5,0.165571)
				(9,0.19469)
				(13,0.216579)
				(17,0.234765)
			};

			% CRS ue_np.mean_max %%%%%%%%%%%%%%%%%%%%%%%%%%%%%%%%%%%%%%%%%%%%%%%%%%%%%%%%%%%%
			\addplot[CRS] coordinates{
				(0,0.108806)
				(5,0.119105)
				(9,0.130749)
				(13,0.142936)
				(17,0.165004)
			};

			% CW ue_np.mean_max %%%%%%%%%%%%%%%%%%%%%%%%%%%%%%%%%%%%%%%%%%%%%%%%%%%%%%%%%%%%
			\addplot[CW] coordinates{
				(0,0.144929)
				(5,0.141971)
				(9,0.167743)
				(13,0.172992)
				(17,0.185232)
			};

			% EAMS ue_np.mean_max %%%%%%%%%%%%%%%%%%%%%%%%%%%%%%%%%%%%%%%%%%%%%%%%%%%%%%%%%%%%
			\addplot[EAMS] coordinates{
				(0,0.163536)
				(5,0.165456)
				(9,0.171368)
				(13,0.18598)
				(17,0.198106)
			};

			% ERGC ue_np.mean_max %%%%%%%%%%%%%%%%%%%%%%%%%%%%%%%%%%%%%%%%%%%%%%%%%%%%%%%%%%%%
			\addplot[ERGC] coordinates{
				(0,0.121812)
				(5,0.121108)
				(9,0.132648)
				(13,0.144161)
				(17,0.166296)
			};

			% ERS ue_np.mean_max %%%%%%%%%%%%%%%%%%%%%%%%%%%%%%%%%%%%%%%%%%%%%%%%%%%%%%%%%%%%
			\addplot[ERS] coordinates{
				(0,0.128528)
				(5,0.132904)
				(9,0.167413)
				(13,0.180684)
				(17,0.19883)
			};

			% ETPS ue_np.mean_max %%%%%%%%%%%%%%%%%%%%%%%%%%%%%%%%%%%%%%%%%%%%%%%%%%%%%%%%%%%%
			\addplot[ETPS] coordinates{
				(0,0.120575)
				(5,0.121042)
				(9,0.129651)
				(13,0.140211)
				(17,0.161529)
			};

			% FH ue_np.mean_max %%%%%%%%%%%%%%%%%%%%%%%%%%%%%%%%%%%%%%%%%%%%%%%%%%%%%%%%%%%%
			\addplot[FH] coordinates{
				(0,0.0895417)
				(5,0.174238)
				(9,0.187288)
				(13,0.202482)
				(17,0.21611)
			};

			% MSS ue_np.mean_max %%%%%%%%%%%%%%%%%%%%%%%%%%%%%%%%%%%%%%%%%%%%%%%%%%%%%%%%%%%%
			\addplot[MSS] coordinates{
				(0,0.135999)
				(5,0.143575)
				(9,0.165265)
				(13,0.183659)
				(17,0.199077)
			};

			% PB ue_np.mean_max %%%%%%%%%%%%%%%%%%%%%%%%%%%%%%%%%%%%%%%%%%%%%%%%%%%%%%%%%%%%
			\addplot[PB] coordinates{
				(0,0.209432)
				(5,0.173686)
				(9,0.187231)
				(13,0.201334)
				(17,0.211727)
			};

			% PRESLIC ue_np.mean_max %%%%%%%%%%%%%%%%%%%%%%%%%%%%%%%%%%%%%%%%%%%%%%%%%%%%%%%%%%%%
			\addplot[PRESLIC] coordinates{
				(0,0.16315)
				(5,0.143618)
				(9,0.162445)
				(13,0.163878)
				(17,0.174892)
			};

			% QS ue_np.mean_max %%%%%%%%%%%%%%%%%%%%%%%%%%%%%%%%%%%%%%%%%%%%%%%%%%%%%%%%%%%%
			\addplot[QS] coordinates{
				(0,0.0480242)
				(5,0.124901)
				(9,0.14721)
				(13,0.170173)
				(17,0.19539)
			};

			% RESEEDS ue_np.mean_max %%%%%%%%%%%%%%%%%%%%%%%%%%%%%%%%%%%%%%%%%%%%%%%%%%%%%%%%%%%%
			\addplot[RESEEDS] coordinates{
				(0,0.132874)
				(5,0.137746)
				(9,0.16163)
				(13,0.165247)
				(17,0.257798)
			};

			% SEAW ue_np.mean_max %%%%%%%%%%%%%%%%%%%%%%%%%%%%%%%%%%%%%%%%%%%%%%%%%%%%%%%%%%%%
			\addplot[SEAW] coordinates{
				(0,0.0882449)
				(5,0.13152)
				(9,0.14122)
				(13,0.161811)
				(17,0.161145)
			};

			% SEEDS ue_np.mean_max %%%%%%%%%%%%%%%%%%%%%%%%%%%%%%%%%%%%%%%%%%%%%%%%%%%%%%%%%%%%
			\addplot[SEEDS] coordinates{
				(0,0.166744)
				(5,0.180029)
				(9,0.195996)
				(13,0.210851)
				(17,0.222971)
			};

			% SLIC ue_np.mean_max %%%%%%%%%%%%%%%%%%%%%%%%%%%%%%%%%%%%%%%%%%%%%%%%%%%%%%%%%%%%
			\addplot[SLIC] coordinates{
				(0,0.133617)
				(5,0.122104)
				(9,0.132949)
				(13,0.146592)
				(17,0.160678)
			};

			% TP ue_np.mean_max %%%%%%%%%%%%%%%%%%%%%%%%%%%%%%%%%%%%%%%%%%%%%%%%%%%%%%%%%%%%
			\addplot[TP] coordinates{
				(0,0.146071)
				(5,0.131207)
				(9,0.147289)
				(13,0.161574)
				(17,0.177116)
			};

			% TPS ue_np.mean_max %%%%%%%%%%%%%%%%%%%%%%%%%%%%%%%%%%%%%%%%%%%%%%%%%%%%%%%%%%%%
			\addplot[TPS] coordinates{
				(0,0.0666233)
				(5,0.162483)
				(9,0.161282)
				(13,0.175236)
				(17,0.191546)
			};

			% VC ue_np.mean_max %%%%%%%%%%%%%%%%%%%%%%%%%%%%%%%%%%%%%%%%%%%%%%%%%%%%%%%%%%%%
			\addplot[VC] coordinates{
				(0,0.0767202)
				(5,0.121124)
				(9,0.138588)
				(13,0.164604)
				(17,0.166716)
			};

			% VLSLIC ue_np.mean_max %%%%%%%%%%%%%%%%%%%%%%%%%%%%%%%%%%%%%%%%%%%%%%%%%%%%%%%%%%%%
			\addplot[VLSLIC] coordinates{
				(0,0.12431)
				(5,0.119889)
				(9,0.135402)
				(13,0.149971)
				(17,0.16259)
			};

			% W ue_np.mean_max %%%%%%%%%%%%%%%%%%%%%%%%%%%%%%%%%%%%%%%%%%%%%%%%%%%%%%%%%%%%
			\addplot[W] coordinates{
				(0,0.162436)
				(5,0.166853)
				(9,0.170368)
				(13,0.182225)
				(17,0.194483)
			};

			\end{axis}
	\end{tikzpicture}
\end{subfigure}
\begin{subfigure}[b]{\halfthreeone\textwidth}\phantomsubcaption\label{subfig:experiments-robustness-blur.sp.mean_min}
	%%%%%%%%%%%%%%%%%%%%%%%%%%%%%%%%%%%%%%%%%%%%%%%%%%%%%%%%%%%%
	% blur.sp.mean_min
	%%%%%%%%%%%%%%%%%%%%%%%%%%%%%%%%%%%%%%%%%%%%%%%%%%%%%%%%%%%%
	\begin{tikzpicture}
		\begin{axis}[EQBSDS500RobustnessK,
			ylabel=\K,
			ymax=3000,
			symbolic x coords= {
				0,
				5,
				9,
				13,
				17,
			},
			xtick=data]

			% CCS sp.mean_min %%%%%%%%%%%%%%%%%%%%%%%%%%%%%%%%%%%%%%%%%%%%%%%%%%%%%%%%%%%%
			\addplot[CCS] coordinates{
				(0,453.82)
				(5,384.784)
				(9,383.432)
				(13,383.573)
				(17,383.744)
			};

			% LSC sp.mean_min %%%%%%%%%%%%%%%%%%%%%%%%%%%%%%%%%%%%%%%%%%%%%%%%%%%%%%%%%%%%
			\addplot[LSC] coordinates{
				(0,1045.31)
				(5,658.01)
				(9,545.497)
				(13,494.09)
				(17,467.487)
			};

			% POISE sp.mean_min %%%%%%%%%%%%%%%%%%%%%%%%%%%%%%%%%%%%%%%%%%%%%%%%%%%%%%%%%%%%
			\addplot[POISE] coordinates{
				(0,408.385)
				(5,408.02)
				(9,407.085)
				(13,403.854)
				(17,400.714)
			};

			% RW sp.mean_min %%%%%%%%%%%%%%%%%%%%%%%%%%%%%%%%%%%%%%%%%%%%%%%%%%%%%%%%%%%%
			\addplot[RW] coordinates{
				(0,0)
				(5,410.917)
				(9,407.167)
				(13,404.458)
				(17,405.583)
			};

			% WP sp.mean_min %%%%%%%%%%%%%%%%%%%%%%%%%%%%%%%%%%%%%%%%%%%%%%%%%%%%%%%%%%%%
			\addplot[WP] coordinates{
				(0,384)
				(5,384)
				(9,384)
				(13,384)
				(17,384)
			};

			% CIS sp.mean_min %%%%%%%%%%%%%%%%%%%%%%%%%%%%%%%%%%%%%%%%%%%%%%%%%%%%%%%%%%%%
			\addplot[CIS] coordinates{
				(0,472.685)
				(5,274.226)
				(9,238.633)
				(13,226.05)
				(17,222.191)
			};

			% CRS sp.mean_min %%%%%%%%%%%%%%%%%%%%%%%%%%%%%%%%%%%%%%%%%%%%%%%%%%%%%%%%%%%%
			\addplot[CRS] coordinates{
				(0,635.865)
				(5,510.402)
				(9,477.347)
				(13,464.05)
				(17,459.462)
			};

			% CW sp.mean_min %%%%%%%%%%%%%%%%%%%%%%%%%%%%%%%%%%%%%%%%%%%%%%%%%%%%%%%%%%%%
			\addplot[CW] coordinates{
				(0,405.935)
				(5,408.874)
				(9,410.9)
				(13,412.236)
				(17,413.95)
			};

			% EAMS sp.mean_min %%%%%%%%%%%%%%%%%%%%%%%%%%%%%%%%%%%%%%%%%%%%%%%%%%%%%%%%%%%%
			\addplot[EAMS] coordinates{
				(0,309.87)
				(5,307.397)
				(9,326.005)
				(13,341.884)
				(17,356.397)
			};

			% ERGC sp.mean_min %%%%%%%%%%%%%%%%%%%%%%%%%%%%%%%%%%%%%%%%%%%%%%%%%%%%%%%%%%%%
			\addplot[ERGC] coordinates{
				(0,400)
				(5,400)
				(9,400)
				(13,400)
				(17,400)
			};

			% ERS sp.mean_min %%%%%%%%%%%%%%%%%%%%%%%%%%%%%%%%%%%%%%%%%%%%%%%%%%%%%%%%%%%%
			\addplot[ERS] coordinates{
				(0,400)
				(5,400)
				(9,400)
				(13,400)
				(17,400)
			};

			% ETPS sp.mean_min %%%%%%%%%%%%%%%%%%%%%%%%%%%%%%%%%%%%%%%%%%%%%%%%%%%%%%%%%%%%
			\addplot[ETPS] coordinates{
				(0,425)
				(5,425)
				(9,425)
				(13,425)
				(17,425)
			};

			% FH sp.mean_min %%%%%%%%%%%%%%%%%%%%%%%%%%%%%%%%%%%%%%%%%%%%%%%%%%%%%%%%%%%%
			\addplot[FH] coordinates{
				(0,566.065)
				(5,515.477)
				(9,517.834)
				(13,507.452)
				(17,492.533)
			};

			% MSS sp.mean_min %%%%%%%%%%%%%%%%%%%%%%%%%%%%%%%%%%%%%%%%%%%%%%%%%%%%%%%%%%%%
			\addplot[MSS] coordinates{
				(0,420.66)
				(5,434.191)
				(9,455.06)
				(13,481.573)
				(17,508.688)
			};

			% PB sp.mean_min %%%%%%%%%%%%%%%%%%%%%%%%%%%%%%%%%%%%%%%%%%%%%%%%%%%%%%%%%%%%
			\addplot[PB] coordinates{
				(0,398.425)
				(5,393.261)
				(9,385.573)
				(13,384.437)
				(17,384.085)
			};

			% PRESLIC sp.mean_min %%%%%%%%%%%%%%%%%%%%%%%%%%%%%%%%%%%%%%%%%%%%%%%%%%%%%%%%%%%%
			\addplot[PRESLIC] coordinates{
				(0,369)
				(5,389.894)
				(9,387.442)
				(13,385.427)
				(17,384.558)
			};

			% QS sp.mean_min %%%%%%%%%%%%%%%%%%%%%%%%%%%%%%%%%%%%%%%%%%%%%%%%%%%%%%%%%%%%
			\addplot[QS] coordinates{
				(0,2626.4)
				(5,648.719)
				(9,453.402)
				(13,372.92)
				(17,326.573)
			};

			% RESEEDS sp.mean_min %%%%%%%%%%%%%%%%%%%%%%%%%%%%%%%%%%%%%%%%%%%%%%%%%%%%%%%%%%%%
			\addplot[RESEEDS] coordinates{
				(0,401.465)
				(5,401.487)
				(9,401.352)
				(13,401.126)
				(17,400.442)
			};

			% SEAW sp.mean_min %%%%%%%%%%%%%%%%%%%%%%%%%%%%%%%%%%%%%%%%%%%%%%%%%%%%%%%%%%%%
			\addplot[SEAW] coordinates{
				(0,923.49)
				(5,702.935)
				(9,669.286)
				(13,661.603)
				(17,658.879)
			};

			% SEEDS sp.mean_min %%%%%%%%%%%%%%%%%%%%%%%%%%%%%%%%%%%%%%%%%%%%%%%%%%%%%%%%%%%%
			\addplot[SEEDS] coordinates{
				(0,468.81)
				(5,437.784)
				(9,429.714)
				(13,424.739)
				(17,422.477)
			};

			% SLIC sp.mean_min %%%%%%%%%%%%%%%%%%%%%%%%%%%%%%%%%%%%%%%%%%%%%%%%%%%%%%%%%%%%
			\addplot[SLIC] coordinates{
				(0,368.57)
				(5,389.146)
				(9,388.035)
				(13,386.201)
				(17,385.216)
			};

			% TP sp.mean_min %%%%%%%%%%%%%%%%%%%%%%%%%%%%%%%%%%%%%%%%%%%%%%%%%%%%%%%%%%%%
			\addplot[TP] coordinates{
				(0,381.285)
				(5,507.005)
				(9,463.397)
				(13,450.553)
				(17,446.09)
			};

			% TPS sp.mean_min %%%%%%%%%%%%%%%%%%%%%%%%%%%%%%%%%%%%%%%%%%%%%%%%%%%%%%%%%%%%
			\addplot[TPS] coordinates{
				(0,445.405)
				(5,459.186)
				(9,464.372)
				(13,467.161)
				(17,469.246)
			};

			% VC sp.mean_min %%%%%%%%%%%%%%%%%%%%%%%%%%%%%%%%%%%%%%%%%%%%%%%%%%%%%%%%%%%%
			\addplot[VC] coordinates{
				(0,1375.27)
				(5,771.136)
				(9,488.412)
				(13,416.97)
				(17,402.638)
			};

			% VLSLIC sp.mean_min %%%%%%%%%%%%%%%%%%%%%%%%%%%%%%%%%%%%%%%%%%%%%%%%%%%%%%%%%%%%
			\addplot[VLSLIC] coordinates{
				(0,763.62)
				(5,788.658)
				(9,642.095)
				(13,560.407)
				(17,517.864)
			};

			% W sp.mean_min %%%%%%%%%%%%%%%%%%%%%%%%%%%%%%%%%%%%%%%%%%%%%%%%%%%%%%%%%%%%
			\addplot[W] coordinates{
				(0,387.92)
				(5,389.588)
				(9,390.638)
				(13,390.995)
				(17,390.598)
			};

			\end{axis}
	\end{tikzpicture}
\end{subfigure}

    \caption{The influence of average blur for $k \in \{0, 5, 9, 13, 17\}$ being
    the filter size. As can be seen, blurring gradually reduces performance
    -- which may be explained by vanishing image boundaries. In addition, for
    algorithms such as \VC and \QS, blurring also leads to fewer superpixels being generated.
    \textbf{Best viewed in color.}}
    \label{subfig:experiments-robustness-blur}
    \vskip 12px
	% Total: 26 (+2 depth)
%\begin{mdframed}
	{\scriptsize
		\begin{tabularx}{0.475\textwidth}{X X X l}
			\ref{plot:w} \W &
			\ref{plot:eams} \EAMS &
			\ref{plot:nc} \NC &
			\ref{plot:fh} \FH\\
			\ref{plot:rw} \RW &
			\ref{plot:qs} \QS &
			\ref{plot:pf} \PF &
			\ref{plot:tp} \TP\\
			\ref{plot:cis} \CIS &
			\ref{plot:slic} \SLIC &
			\ref{plot:crs} \CRS &
			\ref{plot:ers} \ERS\\
			\ref{plot:pb} \PB &
			\ref{plot:seeds} \SEEDS &
			\ref{plot:tps} \TPS &
			\ref{plot:vc} \VC\\
			\ref{plot:ccs} \CCS &
			\ref{plot:cw} \CW &
			\ref{plot:ergc} \ERGC &
			\ref{plot:mss} \MSS\\
			\ref{plot:preslic} \preSLIC &
			\ref{plot:wp} \WP &
			\ref{plot:etps} \ETPS &
			\ref{plot:lsc} \LSC\\
			\ref{plot:poise} \POISE &
			\ref{plot:seaw} \SEAW &
		\end{tabularx}
	}
%\end{mdframed}

\end{figure}

Overall, most superpixel algorithms are robust to the considered noise models,
while blurring tends to reduce performance. Although the corresponding experiments are omitted for brevity,
we found that affine transformations do not influence performance.