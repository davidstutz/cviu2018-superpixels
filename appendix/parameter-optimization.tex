\section{Parameter Optimization}

We discuss the following two topics concerning parameter optimization in more detail:
color spaces and controlling the number of superpixels in a consistent manner.
Overall, we find that together with Section \ref{sec:parameter-optimization},
the described parameter optimization procedure ensures fair comparison as far as possible.

\subsection{Color Spaces}

The used color space inherently influences the performance of superpixel algorithms as
the majority of superpixel algorithms depend on comparing pixels within this color space.
To ensure fair comparison, we included the color space in parameter optimization.
In particular, we ensured that all algorithms support RGB color space and considered
different color spaces only if reported in the corresponding publications or
supported by the respective implementation. While some algorithms may benefit from
different color spaces not mentioned in the corresponding publications, we decided to
not consider additional color spaces for simplicity and to avoid additional overhead
during parameter optimization. Parameter optimization yielded the color spaces
highlighted in Table \ref{table:algorithms}.

\subsection{Controlling the Number of Generated Superpixels}

\DeclarePairedDelimiter\ceil{\lceil}{\rceil}
\DeclarePairedDelimiter\floor{\lfloor}{\rfloor}

Some implementations, for example \ERS and \POISE, control the number of superpixels directly
-- for example by stopping the merging of pixels as soon as the desired number of superpixels is met.
In contrast, clustering-based algorithms (except for \DASP), contour evolution algorithms, watershed-based algorithms
as well as path-based algorithms utilize a regular grid to initialize superpixels.
Some algorithms allow to adapt the grid in both horizontal and vertical direction,
while others require a Cartesian grid. We expected this difference to be reflected in
the experimental results, however, this is not the case. We standardized initialization in both cases.