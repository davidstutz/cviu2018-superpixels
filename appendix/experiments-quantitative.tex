\subsection{Quantitative}
\label{subsec:appendix-experiments-quantitative}

\begin{figure*}
	\centering
	\begin{subfigure}[b]{\fullthreeone\textwidth}\phantomsubcaption\label{subfig:appendix-experiments-sbd-rec.mean[0]}
	%%%%%%%%%%%%%%%%%%%%%%%%%%%%%%%%%%%%%%%%%%%%%%%%%%%%%%%%%%%%
	% rec.mean[0]
	%%%%%%%%%%%%%%%%%%%%%%%%%%%%%%%%%%%%%%%%%%%%%%%%%%%%%%%%%%%%
	\begin{tikzpicture}
		\begin{axis}[EQSBDRec,xmode=log]

			% CCS rec.mean[0] %%%%%%%%%%%%%%%%%%%%%%%%%%%%%%%%%%%%%%%%%%%%%%%%%%%%%%%%%%%%
			\addplot[CCS] coordinates{
				(220.635,0.716247)
				(342.132,0.784563)
				(430.83,0.816746)
				(675.782,0.876725)
				(871.099,0.904536)
				(1043.04,0.923516)
				(1299.21,0.943056)
				(1637.83,0.961266)
				(1654.08,0.961819)
				(1750.56,0.964778)
				(2242.9,0.979454)
				(2403,0.981773)
				(3193.08,0.992247)
				(3212.3,0.992384)
				(3500.9,0.993633)
				(4893.46,0.998929)
				(4928.72,0.998989)
				(4976.43,0.999058)
			};

			% SEEDS rec.mean[0] %%%%%%%%%%%%%%%%%%%%%%%%%%%%%%%%%%%%%%%%%%%%%%%%%%%%%%%%%%%%
			\addplot[SEEDS] coordinates{
				(240.631,0.921457)
				(342.373,0.942666)
				(444.631,0.955292)
				(644.918,0.968173)
				(844.866,0.976899)
				(1067.49,0.985503)
				(1243.06,0.986197)
				(1412.66,0.989482)
				(1654.68,0.991697)
				(1939.07,0.993667)
				(1991.07,0.993967)
				(2393.72,0.995822)
				(3124.71,0.997605)
				(3165.12,0.997541)
				(3399.34,0.997742)
				(4670.56,0.999221)
				(4692.48,0.999234)
				(4709.34,0.999242)
			};

			% SLIC rec.mean[0] %%%%%%%%%%%%%%%%%%%%%%%%%%%%%%%%%%%%%%%%%%%%%%%%%%%%%%%%%%%%
			\addplot[SLIC] coordinates{
				(180.975,0.731903)
				(285.107,0.786101)
				(368.765,0.818445)
				(592.151,0.871412)
				(762.501,0.897095)
				(933,0.915562)
				(1164.08,0.934304)
				(1501.76,0.953789)
				(1517.3,0.954473)
				(1604.98,0.958084)
				(2076.4,0.973651)
				(2234.76,0.977102)
				(3030.82,0.989512)
				(3049.63,0.98961)
				(3325.09,0.991433)
				(4682.36,0.998148)
				(4660.1,0.997937)
				(4743.6,0.997099)
			};

			% RW rec.mean[0] %%%%%%%%%%%%%%%%%%%%%%%%%%%%%%%%%%%%%%%%%%%%%%%%%%%%%%%%%%%%
			\addplot[RW] coordinates{
				(417.935,0.764126)
				(1287.57,0.910768)
				(3875.47,0.99179)
				(6300.55,0.999159)
			};

			% CW rec.mean[0] %%%%%%%%%%%%%%%%%%%%%%%%%%%%%%%%%%%%%%%%%%%%%%%%%%%%%%%%%%%%
			\addplot[CW] coordinates{
				(200,0.69954)
				(313.027,0.756714)
				(411.784,0.789196)
				(622.153,0.842362)
				(842.505,0.878747)
				(1018.83,0.901529)
				(1258.45,0.922642)
				(1484.27,0.939319)
				(1729.03,0.953595)
				(1973.84,0.963433)
				(2098.31,0.969565)
				(2713.82,0.983433)
				(3262.37,0.990897)
				(3549.06,0.994099)
				(4239.54,0.99735)
				(4326.67,0.997567)
				(5705.08,0.99959)
				(5973.93,0.999638)
			};

			% TP rec.mean[0] %%%%%%%%%%%%%%%%%%%%%%%%%%%%%%%%%%%%%%%%%%%%%%%%%%%%%%%%%%%%
			\addplot[TP] coordinates{
				(237.237,0.648684)
				(350.939,0.701824)
				(494.558,0.762243)
				(733.208,0.811693)
				(888.47,0.837833)
				(1039.27,0.852864)
				(1200.89,0.874452)
				(1420.36,0.899397)
				(1425.98,0.899603)
				(1441.29,0.899088)
				(1526.38,0.89332)
				%(237.237,0.648684)
				%(350.939,0.701824)
				%(494.558,0.762243)
				%(733.208,0.811693)
				%(888.47,0.837833)
				%(1039.27,0.852864)
				%(1200.89,0.874452)
				%(1420.36,0.899397)
				%(1425.98,0.899603)
				%(1441.29,0.899088)
				%(1526.38,0.89332)
				%(1448.73,0.891117)
				%(1054.21,0.885068)
				%(1045.06,0.882873)
				%(918.268,0.85269)
				%(289.713,0.714412)
				%(278.998,0.712147)
				%(266.218,0.708081)
			};

			% POISE rec.mean[0] %%%%%%%%%%%%%%%%%%%%%%%%%%%%%%%%%%%%%%%%%%%%%%%%%%%%%%%%%%%%
			\addplot[POISE] coordinates{
				(204.558,0.781508)
				(305.807,0.817922)
				(406.413,0.84188)
				(602.6,0.871753)
				(786.815,0.88729)
				(945.363,0.894879)
				(1056.29,0.898133)
				(1110.86,0.899166)
				(1129.19,0.899386)
				(1131.7,0.899403)
				(1132.06,0.899407)
				(1132.36,0.899408)
				(1132.36,0.899408)
				(1132.36,0.899408)
				(1132.36,0.899408)
				(1132.36,0.899408)
				(1132.36,0.899408)
				(1132.36,0.899408)
			};

			% FH rec.mean[0] %%%%%%%%%%%%%%%%%%%%%%%%%%%%%%%%%%%%%%%%%%%%%%%%%%%%%%%%%%%%
			\addplot[FH] coordinates{
				(254.208,0.661729)
				(303.964,0.688237)
				(409.642,0.750306)
				(714.916,0.836201)
				(1191.35,0.906744)
				(1477.74,0.927607)
				(1820.43,0.939767)
				(2008.15,0.949702)
				(2277.16,0.958279)
				(2725.59,0.965692)
				(4288.99,0.988075)
			};

			% EAMS rec.mean[0] %%%%%%%%%%%%%%%%%%%%%%%%%%%%%%%%%%%%%%%%%%%%%%%%%%%%%%%%%%%%
			\addplot[EAMS] coordinates{
				(620.832,0.928307)
				(654.836,0.932751)
				(692.407,0.937221)
				(802.34,0.947347)
				(928.046,0.956365)
				(1100.62,0.96521)
				(1352.02,0.974952)
				(1417.4,0.976844)
				(1475.78,0.978469)
				(1615.99,0.982023)
				(1697.79,0.983732)
				(1890.75,0.986848)
				(2299.43,0.991464)
				(2651.12,0.993898)
				(3435.61,0.99663)
				(4355.4,0.997969)
				(9229,0.999394)
				(54049,0.99968)
			};

			% CRS rec.mean[0] %%%%%%%%%%%%%%%%%%%%%%%%%%%%%%%%%%%%%%%%%%%%%%%%%%%%%%%%%%%%
			\addplot[CRS] coordinates{
				(265.308,0.778439)
				(397.931,0.828964)
				(497.593,0.852981)
				(721.258,0.900939)
				(961.564,0.927323)
				(1140.37,0.942169)
				(1467.36,0.961836)
				(1670.97,0.970554)
				(1883.05,0.975857)
				(2143.85,0.982266)
				(2266.56,0.984474)
				(2938.4,0.99226)
				(3468.68,0.995328)
				(3772.34,0.996867)
				(4396.39,0.998275)
				(4480.97,0.998355)
				(5789.22,0.999473)
				(6015.14,0.999548)
			};

			% SEAW rec.mean[0] %%%%%%%%%%%%%%%%%%%%%%%%%%%%%%%%%%%%%%%%%%%%%%%%%%%%%%%%%%%%
			\addplot[SEAW] coordinates{
				(67.4528,0.393994)
				(154.597,0.544126)
				(414.981,0.711147)
				(1371.13,0.890352)
				(5128.07,0.99793)
			};

			% RESEEDS rec.mean[0] %%%%%%%%%%%%%%%%%%%%%%%%%%%%%%%%%%%%%%%%%%%%%%%%%%%%%%%%%%%%
			\addplot[RESEEDS] coordinates{
				(200.935,0.922791)
				(301.214,0.941305)
				(400.572,0.954725)
				(600.709,0.968335)
				(800.713,0.977257)
				(1022.1,0.984763)
				(1201.9,0.987)
				(1368.41,0.988639)
				(1611.62,0.991379)
				(1898.44,0.993663)
				(1950.53,0.993977)
				(2357.59,0.996142)
				(3094.96,0.998152)
				(3136.77,0.998194)
				(3373.89,0.998348)
				(4659.04,0.999473)
				(4681.23,0.999489)
				(4698.34,0.999495)
			};

			% ERGC rec.mean[0] %%%%%%%%%%%%%%%%%%%%%%%%%%%%%%%%%%%%%%%%%%%%%%%%%%%%%%%%%%%%
			\addplot[ERGC] coordinates{
				(196.732,0.740576)
				(305.558,0.800066)
				(398.958,0.83487)
				(600.558,0.88321)
				(804.411,0.914403)
				(964.717,0.931136)
				(1225.23,0.949281)
				(1429.78,0.961293)
				(1618.26,0.968449)
				(1848.19,0.975769)
				(1951.28,0.979006)
				(2499.77,0.988001)
				(2981.51,0.992642)
				(3207.91,0.994594)
				(3794.68,0.997019)
				(3867.02,0.997193)
				(5004.8,0.999383)
				(5201.44,0.999438)
			};

			% PF rec.mean[0] %%%%%%%%%%%%%%%%%%%%%%%%%%%%%%%%%%%%%%%%%%%%%%%%%%%%%%%%%%%%
			\addplot[PF] coordinates{
				(216.65,0.470988)
				(339.635,0.532309)
				(456.474,0.570581)
				(712.126,0.637168)
				(874.44,0.670119)
				(1073.21,0.701853)
				(1285.37,0.730559)
				(1610,0.778742)
				(2385.75,0.836764)
				(3007.38,0.868949)
				(4555.92,0.921333)
				(5051.5,0.929546)
			};

			% TPS rec.mean[0] %%%%%%%%%%%%%%%%%%%%%%%%%%%%%%%%%%%%%%%%%%%%%%%%%%%%%%%%%%%%
			\addplot[TPS] coordinates{
				(405.208,0.691068)
				(827.191,0.792805)
				(1275.26,0.850302)
				(1633.84,0.889449)
				(2206.68,0.926259)
				(2464,0.936898)
				(3188.7,0.963321)
				(3314.33,0.964609)
				(3867.77,0.973633)
				(4417.48,0.981854)
				(5172.02,0.986867)
				(6284.27,0.997244)
			};

			% NC rec.mean[0] %%%%%%%%%%%%%%%%%%%%%%%%%%%%%%%%%%%%%%%%%%%%%%%%%%%%%%%%%%%%
			%\addplot[NC] coordinates{
			%	(389,0.777263)
			%	(996.16,0.860312)
			%	(2879.2,0.982364)
			%	(3933.76,0.996658)
			%};

			% VC rec.mean[0] %%%%%%%%%%%%%%%%%%%%%%%%%%%%%%%%%%%%%%%%%%%%%%%%%%%%%%%%%%%%
			\addplot[VC] coordinates{
				(415.908,0.761101)
				(605.48,0.828416)
				(759.394,0.859502)
				(898.063,0.880371)
				(1157.92,0.905941)
				(1395.79,0.922213)
				(1619.19,0.933691)
				(1823.04,0.942407)
				(2019.03,0.949774)
				(2189.37,0.954081)
				(2481.19,0.961501)
				(2975.07,0.967654)
				(3107.27,0.969821)
				(3429.99,0.970992)
				(4202.78,0.972656)
				(4381.11,0.972898)
				(4529.5,0.973856)
			};

			% PB rec.mean[0] %%%%%%%%%%%%%%%%%%%%%%%%%%%%%%%%%%%%%%%%%%%%%%%%%%%%%%%%%%%%
			\addplot[PB] coordinates{
				(242.509,0.67935)
				(336.132,0.730745)
				(432.128,0.770726)
				(642.864,0.829087)
				(779.57,0.857786)
				(969.218,0.88512)
				(1156.66,0.908912)
				(1518.25,0.935666)
				(1533.52,0.936456)
				(1620.65,0.940592)
				(2041.79,0.96105)
				(2182.61,0.964419)
				(2878.04,0.980981)
				(2894.86,0.981175)
				(3142.19,0.983239)
				(4314.18,0.99324)
				(4343.25,0.993373)
				(4382.33,0.993508)
			};

			% PRESLIC rec.mean[0] %%%%%%%%%%%%%%%%%%%%%%%%%%%%%%%%%%%%%%%%%%%%%%%%%%%%%%%%%%%%
			\addplot[PRESLIC] coordinates{
				(172.769,0.672536)
				(278.95,0.756824)
				(360.824,0.795486)
				(580.759,0.854411)
				(751.065,0.880874)
				(913.027,0.903527)
				(1146.3,0.923598)
				(1471.5,0.946008)
				(1484.69,0.946658)
				(1569.68,0.951179)
				(2028.86,0.968321)
				(2180.05,0.972437)
				(2963.54,0.987574)
				(2977.44,0.987608)
				(3243.14,0.989783)
				(4562.43,0.99719)
				(4597.26,0.997329)
				(4645.64,0.997449)
			};

			% W rec.mean[0] %%%%%%%%%%%%%%%%%%%%%%%%%%%%%%%%%%%%%%%%%%%%%%%%%%%%%%%%%%%%
			\addplot[W] coordinates{
				(197.992,0.694541)
				(305.086,0.748228)
				(402.275,0.788917)
				(641.101,0.849203)
				(809.74,0.877431)
				(1006.25,0.900583)
				(1238.6,0.925276)
				(1612.96,0.949133)
				(1629.74,0.950048)
				(1724.79,0.953937)
				(2219.71,0.972009)
				(2396.42,0.975481)
				(3271.88,0.990943)
				(3295.97,0.991112)
				(3637.87,0.992685)
				(5329.19,0.99891)
				(5373.04,0.998995)
				(5433.59,0.999108)
			};

			% LSC rec.mean[0] %%%%%%%%%%%%%%%%%%%%%%%%%%%%%%%%%%%%%%%%%%%%%%%%%%%%%%%%%%%%
			\addplot[LSC] coordinates{
				(441.153,0.836722)
				(632.558,0.876561)
				(804.126,0.89868)
				(968.442,0.914992)
				(1136.86,0.92696)
				(1269,0.934328)
				(1404.39,0.940479)
				(1516.23,0.944679)
				(1635.31,0.948959)
				(1752.66,0.952427)
				(1823.92,0.954581)
				(2133.24,0.962114)
				(2369.8,0.966182)
			};

			% WP rec.mean[0] %%%%%%%%%%%%%%%%%%%%%%%%%%%%%%%%%%%%%%%%%%%%%%%%%%%%%%%%%%%%
			\addplot[WP] coordinates{
				(196.864,0.688169)
				(305.092,0.741439)
				(397.694,0.775121)
				(627.704,0.829985)
				(787.436,0.857514)
				(992.365,0.882397)
				(1223.32,0.904543)
				(1570.9,0.927886)
				(1660.13,0.932147)
				(2109.92,0.954674)
				(2117.36,0.955414)
				(2213.89,0.959528)
				(2677.06,0.978394)
				(2696.13,0.978839)
				(2934.56,0.982137)
				(4159.73,0.996043)
				(4182.61,0.996155)
				(4254.45,0.996411)
			};

			% QS rec.mean[0] %%%%%%%%%%%%%%%%%%%%%%%%%%%%%%%%%%%%%%%%%%%%%%%%%%%%%%%%%%%%
			\addplot[QS] coordinates{
				(206.314,0.448203)
				(339.805,0.58598)
				(588.734,0.706251)
				(646.82,0.725197)
				(804.377,0.764353)
				(912.229,0.783911)
				(1048.4,0.804891)
				(1230.61,0.828083)
				(1476.88,0.851935)
				(1819.53,0.877294)
				(2310.25,0.902672)
				(3027.82,0.927263)
				(4157.32,0.950935)
				(5985.26,0.971048)
			};

			% VLSLIC rec.mean[0] %%%%%%%%%%%%%%%%%%%%%%%%%%%%%%%%%%%%%%%%%%%%%%%%%%%%%%%%%%%%
			\addplot[VLSLIC] coordinates{
				(361.05,0.827495)
				(448.583,0.848419)
				(527.002,0.862879)
				(675.243,0.882687)
				(769.256,0.894398)
				(913.256,0.907211)
				(1072.97,0.919225)
				(1367.59,0.934969)
				(1414.3,0.936883)
				(1505.73,0.940565)
				(1867.41,0.956046)
				(2010.99,0.959362)
				(2531.28,0.973838)
				(2581.1,0.97322)
			};

			% CIS rec.mean[0] %%%%%%%%%%%%%%%%%%%%%%%%%%%%%%%%%%%%%%%%%%%%%%%%%%%%%%%%%%%%
			\addplot[CIS] coordinates{
				(248.342,0.658671)
				(329.042,0.698389)
				(401.874,0.726367)
				(572.715,0.774319)
				(678.086,0.796111)
				(804.69,0.819777)
				(943.665,0.839095)
				(1284.3,0.87036)
				(1322.71,0.87304)
				(1423.88,0.880651)
				(1847.32,0.909196)
				(2034.14,0.917329)
				(2741.07,0.946713)
				(3318.96,0.958045)
				(4404.2,0.978837)
				(4431.82,0.979158)
				(4473.08,0.979484)
			};

			% ERS rec.mean[0] %%%%%%%%%%%%%%%%%%%%%%%%%%%%%%%%%%%%%%%%%%%%%%%%%%%%%%%%%%%%
			\addplot[ERS] coordinates{
				(200,0.796779)
				(300,0.843364)
				(400,0.875669)
				(600,0.917059)
				(800,0.942881)
				(1000,0.959524)
				(1200,0.970695)
				(1400,0.97817)
				(1600,0.984052)
				(1800,0.988145)
				(2000,0.99117)
				(2400,0.995039)
				(2800,0.9971)
				(3200,0.998296)
				(3600,0.999033)
				(4000,0.999416)
				(4600,0.999732)
				(5200,0.999868)
			};

			% MSS rec.mean[0] %%%%%%%%%%%%%%%%%%%%%%%%%%%%%%%%%%%%%%%%%%%%%%%%%%%%%%%%%%%%
			\addplot[MSS] coordinates{
				(183.048,0.707136)
				(297.333,0.763926)
				(419.465,0.795044)
				(680.717,0.84621)
				(844.541,0.865187)
				(1087.43,0.890093)
				(1321.31,0.907672)
				(1806.84,0.932612)
				(1828.12,0.933541)
				(1948.18,0.93819)
				(2526.57,0.957424)
				(2729.61,0.960582)
				(3749.83,0.977597)
				(3777.69,0.977742)
				(4207.84,0.980665)
				(6286.37,0.992928)
				(6342.3,0.993237)
			};

			% ETPS rec.mean[0] %%%%%%%%%%%%%%%%%%%%%%%%%%%%%%%%%%%%%%%%%%%%%%%%%%%%%%%%%%%%
			\addplot[ETPS] coordinates{
				(196.618,0.902922)
				(305.541,0.925979)
				(415.836,0.943292)
				(648.057,0.965966)
				(787.866,0.970214)
				(992.331,0.977405)
				(1185.61,0.980954)
				(1570.96,0.988187)
				(1587.11,0.98837)
				(1679.34,0.98916)
				(2122.77,0.993368)
				(2271,0.994001)
				(3010.56,0.997496)
				(3029.14,0.99752)
				(3309.82,0.997775)
				(4675.49,0.999224)
				(4710.48,0.999244)
				(4758.49,0.999271)
			};

			\end{axis}
	\end{tikzpicture}
\end{subfigure}
\begin{subfigure}[b]{\fullthreeone\textwidth}\phantomsubcaption\label{subfig:appendix-experiments-sbd-ue_np.mean[0]}
	%%%%%%%%%%%%%%%%%%%%%%%%%%%%%%%%%%%%%%%%%%%%%%%%%%%%%%%%%%%%
	% ue_np.mean[0]
	%%%%%%%%%%%%%%%%%%%%%%%%%%%%%%%%%%%%%%%%%%%%%%%%%%%%%%%%%%%%
	\begin{tikzpicture}
		\begin{axis}[EQSBDUE,xmode=log]

			% CCS ue_np.mean[0] %%%%%%%%%%%%%%%%%%%%%%%%%%%%%%%%%%%%%%%%%%%%%%%%%%%%%%%%%%%%
			\addplot[CCS] coordinates{
				(220.635,0.142854)
				(342.132,0.120055)
				(430.83,0.110624)
				(675.782,0.0941502)
				(871.099,0.0868585)
				(1043.04,0.0813288)
				(1299.21,0.0753397)
				(1637.83,0.0694327)
				(1654.08,0.069174)
				(1750.56,0.0680904)
				(2242.9,0.0621414)
				(2403,0.0609485)
				(3193.08,0.0542281)
				(3212.3,0.0541254)
				(3500.9,0.0527154)
				(4893.46,0.0457451)
				(4928.72,0.0455982)
				(4976.43,0.0454133)
			};

			% SEEDS ue_np.mean[0] %%%%%%%%%%%%%%%%%%%%%%%%%%%%%%%%%%%%%%%%%%%%%%%%%%%%%%%%%%%%
			\addplot[SEEDS] coordinates{
				(240.631,0.163946)
				(342.373,0.137966)
				(444.631,0.130822)
				(644.918,0.109737)
				(844.866,0.100408)
				(1067.49,0.0984371)
				(1243.06,0.0855888)
				(1412.66,0.0841884)
				(1654.68,0.0786565)
				(1939.07,0.0728633)
				(1991.07,0.0725012)
				(2393.72,0.0662634)
				(3124.71,0.0594566)
				(3165.12,0.0586358)
				(3399.34,0.0574136)
				(4670.56,0.0498003)
				(4692.48,0.0496924)
				(4709.34,0.049605)
			};

			% SLIC ue_np.mean[0] %%%%%%%%%%%%%%%%%%%%%%%%%%%%%%%%%%%%%%%%%%%%%%%%%%%%%%%%%%%%
			\addplot[SLIC] coordinates{
				(180.975,0.154828)
				(285.107,0.13225)
				(368.765,0.119456)
				(592.151,0.101104)
				(762.501,0.093346)
				(933,0.0869098)
				(1164.08,0.0806507)
				(1501.76,0.0735794)
				(1517.3,0.0733073)
				(1604.98,0.0720287)
				(2076.4,0.0653292)
				(2234.76,0.0636631)
				(3030.82,0.0565109)
				(3049.63,0.0563985)
				(3325.09,0.0548478)
				(4682.36,0.0476234)
				(4660.1,0.0425381)
				(4743.6,0.0511857)
			};

			% RW ue_np.mean[0] %%%%%%%%%%%%%%%%%%%%%%%%%%%%%%%%%%%%%%%%%%%%%%%%%%%%%%%%%%%%
			\addplot[RW] coordinates{
				(417.935,0.127559)
				(1287.57,0.0848659)
				(3875.47,0.0559665)
				(6300.55,0.0461244)
			};

			% CW ue_np.mean[0] %%%%%%%%%%%%%%%%%%%%%%%%%%%%%%%%%%%%%%%%%%%%%%%%%%%%%%%%%%%%
			\addplot[CW] coordinates{
				(200,0.159925)
				(313.027,0.13542)
				(411.784,0.123894)
				(622.153,0.107589)
				(842.505,0.0968355)
				(1018.83,0.0903742)
				(1258.45,0.0847412)
				(1484.27,0.0802683)
				(1729.03,0.0756241)
				(1973.84,0.0723399)
				(2098.31,0.0711134)
				(2713.82,0.0647386)
				(3262.37,0.0603617)
				(3549.06,0.0587015)
				(4239.54,0.0547292)
				(4326.67,0.0544301)
				(5705.08,0.0488556)
				(5973.93,0.0479634)
			};

			% TP ue_np.mean[0] %%%%%%%%%%%%%%%%%%%%%%%%%%%%%%%%%%%%%%%%%%%%%%%%%%%%%%%%%%%%
			\addplot[TP] coordinates{
				(237.237,0.14075)
				(350.939,0.121789)
				(494.558,0.106952)
				(733.208,0.0949444)
				(888.47,0.107043)
				(1039.27,0.100387)
				(1200.89,0.0923956)
				(1420.36,0.084059)
				(1425.98,0.0839185)
				(1441.29,0.08425)
				(1526.38,0.087109)
				%(237.237,0.14075)
				%(350.939,0.121789)
				%(494.558,0.106952)
				%(733.208,0.0949444)
				%(888.47,0.107043)
				%(1039.27,0.100387)
				%(1200.89,0.0923956)
				%(1420.36,0.084059)
				%(1425.98,0.0839185)
				%(1441.29,0.08425)
				%(1526.38,0.087109)
				%(1448.73,0.0877533)
				%(1054.21,0.0899739)
				%(1045.06,0.0908787)
				%(918.268,0.100383)
				%(289.713,0.153365)
				%(278.998,0.154079)
				%(266.218,0.156906)
			};

			% POISE ue_np.mean[0] %%%%%%%%%%%%%%%%%%%%%%%%%%%%%%%%%%%%%%%%%%%%%%%%%%%%%%%%%%%%
			\addplot[POISE] coordinates{
				(204.558,0.129648)
				(305.807,0.111754)
				(406.413,0.101646)
				(602.6,0.0902116)
				(786.815,0.0841201)
				(945.363,0.0809409)
				(1056.29,0.0794252)
				(1110.86,0.0788936)
				(1129.19,0.0787531)
				(1131.7,0.0787438)
				(1132.06,0.0787433)
				(1132.36,0.0787423)
				(1132.36,0.0787423)
				(1132.36,0.0787423)
				(1132.36,0.0787423)
				(1132.36,0.0787423)
				(1132.36,0.0787423)
				(1132.36,0.0787424)
			};

			% FH ue_np.mean[0] %%%%%%%%%%%%%%%%%%%%%%%%%%%%%%%%%%%%%%%%%%%%%%%%%%%%%%%%%%%%
			\addplot[FH] coordinates{
				(254.208,0.189803)
				(303.964,0.175995)
				(409.642,0.147043)
				(714.916,0.114643)
				(1191.35,0.0898786)
				(1477.74,0.0821923)
				(1820.43,0.0771543)
				(2008.15,0.0757201)
				(2277.16,0.0678295)
				(2725.59,0.0685569)
				(4288.99,0.05615)
			};

			% EAMS ue_np.mean[0] %%%%%%%%%%%%%%%%%%%%%%%%%%%%%%%%%%%%%%%%%%%%%%%%%%%%%%%%%%%%
			\addplot[EAMS] coordinates{
				(620.832,0.0977106)
				(654.836,0.0957933)
				(692.407,0.0937835)
				(802.34,0.0890372)
				(928.046,0.0845459)
				(1100.62,0.0797737)
				(1352.02,0.0742813)
				(1417.4,0.0732127)
				(1475.78,0.0721243)
				(1615.99,0.0697221)
				(1697.79,0.0684119)
				(1890.75,0.0656129)
				(2299.43,0.0606672)
				(2651.12,0.05717)
				(3435.61,0.0512148)
				(4355.4,0.0465036)
				(9229,0.0316129)
				(54049,0.00167793)
			};

			% CRS ue_np.mean[0] %%%%%%%%%%%%%%%%%%%%%%%%%%%%%%%%%%%%%%%%%%%%%%%%%%%%%%%%%%%%
			\addplot[CRS] coordinates{
				(265.308,0.131004)
				(397.931,0.11419)
				(497.593,0.106366)
				(721.258,0.0923027)
				(961.564,0.0834151)
				(1140.37,0.0784965)
				(1467.36,0.0718062)
				(1670.97,0.0682658)
				(1883.05,0.0656311)
				(2143.85,0.0621275)
				(2266.56,0.0610565)
				(2938.4,0.0549646)
				(3468.68,0.0513316)
				(3772.34,0.0499951)
				(4396.39,0.0466681)
				(4480.97,0.0464034)
				(5789.22,0.0413502)
				(6015.14,0.0407807)
			};

			% SEAW ue_np.mean[0] %%%%%%%%%%%%%%%%%%%%%%%%%%%%%%%%%%%%%%%%%%%%%%%%%%%%%%%%%%%%
			\addplot[SEAW] coordinates{
				(67.4528,0.404772)
				(154.597,0.241273)
				(414.981,0.14376)
				(1371.13,0.0859179)
				(5128.07,0.049563)
			};

			% RESEEDS ue_np.mean[0] %%%%%%%%%%%%%%%%%%%%%%%%%%%%%%%%%%%%%%%%%%%%%%%%%%%%%%%%%%%%
			\addplot[RESEEDS] coordinates{
				(200.935,0.159789)
				(301.214,0.129816)
				(400.572,0.124881)
				(600.709,0.103788)
				(800.713,0.0943406)
				(1022.1,0.0881259)
				(1201.9,0.0800235)
				(1368.41,0.077574)
				(1611.62,0.0731554)
				(1898.44,0.067855)
				(1950.53,0.0676127)
				(2357.59,0.0622787)
				(3094.96,0.0553856)
				(3136.77,0.0550235)
				(3373.89,0.0538624)
				(4659.04,0.0467299)
				(4681.23,0.0466557)
				(4698.34,0.0465889)
			};

			% ERGC ue_np.mean[0] %%%%%%%%%%%%%%%%%%%%%%%%%%%%%%%%%%%%%%%%%%%%%%%%%%%%%%%%%%%%
			\addplot[ERGC] coordinates{
				(196.732,0.142434)
				(305.558,0.122087)
				(398.958,0.111758)
				(600.558,0.0974328)
				(804.411,0.0883628)
				(964.717,0.0830879)
				(1225.23,0.0769119)
				(1429.78,0.0733828)
				(1618.26,0.07027)
				(1848.19,0.0673309)
				(1951.28,0.0661962)
				(2499.77,0.0607292)
				(2981.51,0.0569308)
				(3207.91,0.0558284)
				(3794.68,0.0523062)
				(3867.02,0.0520395)
				(5004.8,0.0471777)
				(5201.44,0.0465274)
			};

			% PF ue_np.mean[0] %%%%%%%%%%%%%%%%%%%%%%%%%%%%%%%%%%%%%%%%%%%%%%%%%%%%%%%%%%%%
			\addplot[PF] coordinates{
				(216.65,0.259891)
				(339.635,0.223758)
				(456.474,0.205009)
				(712.126,0.176407)
				(874.44,0.163465)
				(1073.21,0.152555)
				(1285.37,0.142956)
				(1610,0.134082)
				(2385.75,0.116562)
				(3007.38,0.107984)
				(4555.92,0.0926527)
				(5051.5,0.0893423)
			};

			% TPS ue_np.mean[0] %%%%%%%%%%%%%%%%%%%%%%%%%%%%%%%%%%%%%%%%%%%%%%%%%%%%%%%%%%%%
			\addplot[TPS] coordinates{
				(405.208,0.127471)
				(827.191,0.0999583)
				(1275.26,0.0864892)
				(1633.84,0.0767003)
				(2206.68,0.0684227)
				(2464,0.0661617)
				(3188.7,0.0590065)
				(3314.33,0.0588373)
				(3867.77,0.0558)
				(4417.48,0.0525208)
				(5172.02,0.0497658)
				(6284.27,0.0442337)
			};

			% NC ue_np.mean[0] %%%%%%%%%%%%%%%%%%%%%%%%%%%%%%%%%%%%%%%%%%%%%%%%%%%%%%%%%%%%
			%\addplot[NC] coordinates{
			%	(389,0.0920053)
			%	(996.16,0.0746993)
			%	(2879.2,0.0529796)
			%	(3933.76,0.0466264)
			%};

			% VC ue_np.mean[0] %%%%%%%%%%%%%%%%%%%%%%%%%%%%%%%%%%%%%%%%%%%%%%%%%%%%%%%%%%%%
			\addplot[VC] coordinates{
				(415.908,0.160848)
				(605.48,0.123838)
				(759.394,0.107752)
				(898.063,0.0985343)
				(1157.92,0.0869482)
				(1395.79,0.0795807)
				(1619.19,0.0743823)
				(1823.04,0.0704161)
				(2019.03,0.0671415)
				(2189.37,0.064663)
				(2481.19,0.0606194)
				(2975.07,0.055432)
				(3107.27,0.053916)
				(3429.99,0.0524726)
				(4202.78,0.0503103)
				(4381.11,0.0492317)
				(4529.5,0.0482183)
			};

			% PB ue_np.mean[0] %%%%%%%%%%%%%%%%%%%%%%%%%%%%%%%%%%%%%%%%%%%%%%%%%%%%%%%%%%%%
			\addplot[PB] coordinates{
				(242.509,0.184809)
				(336.132,0.154345)
				(432.128,0.136144)
				(642.864,0.113753)
				(779.57,0.104094)
				(969.218,0.0956383)
				(1156.66,0.0892478)
				(1518.25,0.0796617)
				(1533.52,0.0794208)
				(1620.65,0.0780956)
				(2041.79,0.0710335)
				(2182.61,0.0695625)
				(2878.04,0.0626669)
				(2894.86,0.0625632)
				(3142.19,0.0610197)
				(4314.18,0.0534561)
				(4343.25,0.0533526)
				(4382.33,0.0531896)
			};

			% PRESLIC ue_np.mean[0] %%%%%%%%%%%%%%%%%%%%%%%%%%%%%%%%%%%%%%%%%%%%%%%%%%%%%%%%%%%%
			\addplot[PRESLIC] coordinates{
				(172.769,0.169705)
				(278.95,0.130256)
				(360.824,0.133049)
				(580.759,0.110733)
				(751.065,0.101666)
				(913.027,0.0941636)
				(1146.3,0.086904)
				(1471.5,0.0795596)
				(1484.69,0.0794496)
				(1569.68,0.0780794)
				(2028.86,0.0714447)
				(2180.05,0.0699609)
				(2963.54,0.0618304)
				(2977.44,0.0620237)
				(3243.14,0.0604102)
				(4562.43,0.0523122)
				(4597.26,0.0521506)
				(4645.64,0.0519391)
			};

			% W ue_np.mean[0] %%%%%%%%%%%%%%%%%%%%%%%%%%%%%%%%%%%%%%%%%%%%%%%%%%%%%%%%%%%%
			\addplot[W] coordinates{
				(197.992,0.172881)
				(305.086,0.146155)
				(402.275,0.131387)
				(641.101,0.110055)
				(809.74,0.101231)
				(1006.25,0.0933883)
				(1238.6,0.0854718)
				(1612.96,0.0787438)
				(1629.74,0.0784007)
				(1724.79,0.0772037)
				(2219.71,0.070289)
				(2396.42,0.0687517)
				(3271.88,0.0611002)
				(3295.97,0.0609777)
				(3637.87,0.0593011)
				(5329.19,0.0508129)
				(5373.04,0.0506761)
				(5433.59,0.0504515)
			};

			% LSC ue_np.mean[0] %%%%%%%%%%%%%%%%%%%%%%%%%%%%%%%%%%%%%%%%%%%%%%%%%%%%%%%%%%%%
			\addplot[LSC] coordinates{
				(441.153,0.119983)
				(632.558,0.104392)
				(804.126,0.0968016)
				(968.442,0.0911962)
				(1136.86,0.0879687)
				(1269,0.0865815)
				(1404.39,0.086141)
				(1516.23,0.0858128)
				(1635.31,0.0847393)
				(1752.66,0.0831521)
				(1823.92,0.0841551)
				(2133.24,0.0828517)
				(2369.8,0.0836969)
			};

			% WP ue_np.mean[0] %%%%%%%%%%%%%%%%%%%%%%%%%%%%%%%%%%%%%%%%%%%%%%%%%%%%%%%%%%%%
			\addplot[WP] coordinates{
				(196.864,0.153051)
				(305.092,0.130665)
				(397.694,0.119256)
				(627.704,0.10186)
				(787.436,0.0948724)
				(992.365,0.0883074)
				(1223.32,0.082569)
				(1570.9,0.0768808)
				(1660.13,0.0756364)
				(2109.92,0.0698599)
				(2117.36,0.0694212)
				(2213.89,0.0678506)
				(2677.06,0.0613129)
				(2696.13,0.0612032)
				(2934.56,0.059644)
				(4159.73,0.0520061)
				(4182.61,0.0519345)
				(4254.45,0.0516447)
			};

			% QS ue_np.mean[0] %%%%%%%%%%%%%%%%%%%%%%%%%%%%%%%%%%%%%%%%%%%%%%%%%%%%%%%%%%%%
			\addplot[QS] coordinates{
				(206.314,0.419245)
				(339.805,0.268363)
				(588.734,0.175264)
				(646.82,0.163679)
				(804.377,0.140792)
				(912.229,0.130626)
				(1048.4,0.120322)
				(1230.61,0.110099)
				(1476.88,0.099259)
				(1819.53,0.0888994)
				(2310.25,0.0784424)
				(3027.82,0.0680684)
				(4157.32,0.0573804)
				(5985.26,0.0464348)
			};

			% VLSLIC ue_np.mean[0] %%%%%%%%%%%%%%%%%%%%%%%%%%%%%%%%%%%%%%%%%%%%%%%%%%%%%%%%%%%%
			\addplot[VLSLIC] coordinates{
				(361.05,0.131217)
				(448.583,0.119681)
				(527.002,0.113373)
				(675.243,0.105077)
				(769.256,0.100179)
				(913.256,0.0958071)
				(1072.97,0.0907379)
				(1367.59,0.0861765)
				(1414.3,0.0839373)
				(1505.73,0.0818799)
				(1867.41,0.0781001)
				(2010.99,0.0759732)
				(2531.28,0.0764805)
				(2581.1,0.076839)
			};

			% CIS ue_np.mean[0] %%%%%%%%%%%%%%%%%%%%%%%%%%%%%%%%%%%%%%%%%%%%%%%%%%%%%%%%%%%%
			\addplot[CIS] coordinates{
				(248.342,0.1354)
				(329.042,0.120035)
				(401.874,0.111276)
				(572.715,0.0977906)
				(678.086,0.0924246)
				(804.69,0.0870386)
				(943.665,0.0828081)
				(1284.3,0.0757068)
				(1322.71,0.0751602)
				(1423.88,0.0738251)
				(1847.32,0.0679354)
				(2034.14,0.0661013)
				(2741.07,0.0584075)
				(3318.96,0.0549275)
				(4404.2,0.0491573)
				(4431.82,0.0490279)
				(4473.08,0.0488624)
			};

			% ERS ue_np.mean[0] %%%%%%%%%%%%%%%%%%%%%%%%%%%%%%%%%%%%%%%%%%%%%%%%%%%%%%%%%%%%
			\addplot[ERS] coordinates{
				(200,0.142302)
				(300,0.124229)
				(400,0.11312)
				(600,0.0988239)
				(800,0.0895085)
				(1000,0.0827686)
				(1200,0.0775869)
				(1400,0.0732377)
				(1600,0.0696693)
				(1800,0.0663848)
				(2000,0.0636983)
				(2400,0.059085)
				(2800,0.0553399)
				(3200,0.0520827)
				(3600,0.0494059)
				(4000,0.0470198)
				(4600,0.0439148)
				(5200,0.0413451)
			};

			% MSS ue_np.mean[0] %%%%%%%%%%%%%%%%%%%%%%%%%%%%%%%%%%%%%%%%%%%%%%%%%%%%%%%%%%%%
			\addplot[MSS] coordinates{
				(183.048,0.164039)
				(297.333,0.135072)
				(419.465,0.120607)
				(680.717,0.102256)
				(844.541,0.0970849)
				(1087.43,0.089637)
				(1321.31,0.0841913)
				(1806.84,0.0778697)
				(1828.12,0.0775898)
				(1948.18,0.0763783)
				(2526.57,0.0704064)
				(2729.61,0.0693547)
				(3749.83,0.0631436)
				(3777.69,0.0631061)
				(4207.84,0.0614792)
				(6286.37,0.0538059)
				(6342.3,0.0536714)
			};

			% ETPS ue_np.mean[0] %%%%%%%%%%%%%%%%%%%%%%%%%%%%%%%%%%%%%%%%%%%%%%%%%%%%%%%%%%%%
			\addplot[ETPS] coordinates{
				(196.618,0.138688)
				(305.541,0.118972)
				(415.836,0.108189)
				(648.057,0.0927966)
				(787.866,0.0860597)
				(992.331,0.0797771)
				(1185.61,0.0738707)
				(1570.96,0.0681734)
				(1587.11,0.0679451)
				(1679.34,0.0668066)
				(2122.77,0.0603482)
				(2271,0.0593489)
				(3010.56,0.0530273)
				(3029.14,0.0529392)
				(3309.82,0.0516031)
				(4675.49,0.044303)
				(4710.48,0.0441743)
				(4758.49,0.0440051)
			};

			\end{axis}
	\end{tikzpicture}
\end{subfigure}
\begin{subfigure}[b]{\fullthreeone\textwidth}\phantomsubcaption\label{subfig:appendix-experiments-sbd-ev.mean[0]}
	%%%%%%%%%%%%%%%%%%%%%%%%%%%%%%%%%%%%%%%%%%%%%%%%%%%%%%%%%%%%
	% ev.mean[0]
	%%%%%%%%%%%%%%%%%%%%%%%%%%%%%%%%%%%%%%%%%%%%%%%%%%%%%%%%%%%%
	\begin{tikzpicture}
		\begin{axis}[EQSBDEV,xmode=log]

			% CCS ev.mean[0] %%%%%%%%%%%%%%%%%%%%%%%%%%%%%%%%%%%%%%%%%%%%%%%%%%%%%%%%%%%%
			\addplot[CCS] coordinates{
				(220.635,0.881183)
				(342.132,0.900839)
				(430.83,0.909047)
				(675.782,0.923774)
				(871.099,0.930471)
				(1043.04,0.935)
				(1299.21,0.940107)
				(1637.83,0.945292)
				(1654.08,0.945523)
				(1750.56,0.946581)
				(2242.9,0.951382)
				(2403,0.952639)
				(3193.08,0.958046)
				(3212.3,0.958125)
				(3500.9,0.959595)
				(4893.46,0.964709)
				(4928.72,0.964809)
				(4976.43,0.964943)
			};

			% SEEDS ev.mean[0] %%%%%%%%%%%%%%%%%%%%%%%%%%%%%%%%%%%%%%%%%%%%%%%%%%%%%%%%%%%%
			\addplot[SEEDS] coordinates{
				(240.631,0.921404)
				(342.373,0.932982)
				(444.631,0.934325)
				(644.918,0.944368)
				(844.866,0.948355)
				(1067.49,0.948773)
				(1243.06,0.95646)
				(1412.66,0.95526)
				(1654.68,0.958394)
				(1939.07,0.961926)
				(1991.07,0.962086)
				(2393.72,0.965907)
				(3124.71,0.970278)
				(3165.12,0.970625)
				(3399.34,0.971713)
				(4670.56,0.976032)
				(4692.48,0.976122)
				(4709.34,0.976168)
			};

			% SLIC ev.mean[0] %%%%%%%%%%%%%%%%%%%%%%%%%%%%%%%%%%%%%%%%%%%%%%%%%%%%%%%%%%%%
			\addplot[SLIC] coordinates{
				(180.975,0.839819)
				(285.107,0.862773)
				(368.765,0.877799)
				(592.151,0.896931)
				(762.501,0.905046)
				(933,0.914008)
				(1164.08,0.919372)
				(1501.76,0.928855)
				(1517.3,0.929177)
				(1604.98,0.930733)
				(2076.4,0.937144)
				(2234.76,0.93939)
				(3030.82,0.947669)
				(3049.63,0.947777)
				(3325.09,0.94986)
				(4682.36,0.95689)
				(4660.1,0.94967)
				(4743.6,0.938934)
			};

			% RW ev.mean[0] %%%%%%%%%%%%%%%%%%%%%%%%%%%%%%%%%%%%%%%%%%%%%%%%%%%%%%%%%%%%
			\addplot[RW] coordinates{
				(417.935,0.865029)
				(1287.57,0.916683)
				(3875.47,0.954038)
				(6300.55,0.965831)
			};

			% CW ev.mean[0] %%%%%%%%%%%%%%%%%%%%%%%%%%%%%%%%%%%%%%%%%%%%%%%%%%%%%%%%%%%%
			\addplot[CW] coordinates{
				(200,0.823089)
				(313.027,0.844366)
				(411.784,0.856354)
				(622.153,0.872275)
				(842.505,0.882437)
				(1018.83,0.88831)
				(1258.45,0.894155)
				(1484.27,0.898492)
				(1729.03,0.902771)
				(1973.84,0.906237)
				(2098.31,0.907163)
				(2713.82,0.913121)
				(3262.37,0.917094)
				(3549.06,0.918659)
				(4239.54,0.92257)
				(4326.67,0.922938)
				(5705.08,0.928465)
				(5973.93,0.929609)
			};

			% TP ev.mean[0] %%%%%%%%%%%%%%%%%%%%%%%%%%%%%%%%%%%%%%%%%%%%%%%%%%%%%%%%%%%%
			\addplot[TP] coordinates{
				(237.237,0.849777)
				(350.939,0.866063)
				(494.558,0.880356)
				(733.208,0.891035)
				(888.47,0.875823)
				(1039.27,0.88068)
				(1200.89,0.888014)
				(1420.36,0.8969)
				(1425.98,0.896982)
				(1441.29,0.896144)
				(1526.38,0.891406)
				%(237.237,0.849777)
				%(350.939,0.866063)
				%(494.558,0.880356)
				%733.208,0.891035)
				%(888.47,0.875823)
				%(1039.27,0.88068)
				%(1200.89,0.888014)
				%(1420.36,0.8969)
				%(1425.98,0.896982)
				%(1441.29,0.896144)
				%(1526.38,0.891406)
				%(1448.73,0.890566)
				%(1054.21,0.886134)
				%(1045.06,0.885632)
				%(918.268,0.878335)
				%(289.713,0.840179)
				%(278.998,0.839645)
				%(266.218,0.837467)
			};

			% POISE ev.mean[0] %%%%%%%%%%%%%%%%%%%%%%%%%%%%%%%%%%%%%%%%%%%%%%%%%%%%%%%%%%%%
			\addplot[POISE] coordinates{
				(204.558,0.874031)
				(305.807,0.884585)
				(406.413,0.889747)
				(602.6,0.895054)
				(786.815,0.89755)
				(945.363,0.898715)
				(1056.29,0.899289)
				(1110.86,0.899475)
				(1129.19,0.899519)
				(1131.7,0.899525)
				(1132.06,0.899525)
				(1132.36,0.899525)
				(1132.36,0.899525)
				(1132.36,0.899525)
				(1132.36,0.899525)
				(1132.36,0.899525)
				(1132.36,0.899525)
				(1132.36,0.899525)
			};

			% FH ev.mean[0] %%%%%%%%%%%%%%%%%%%%%%%%%%%%%%%%%%%%%%%%%%%%%%%%%%%%%%%%%%%%
			\addplot[FH] coordinates{
				(254.208,0.799969)
				(303.964,0.819364)
				(409.642,0.847716)
				(714.916,0.889759)
				(1191.35,0.922132)
				(1477.74,0.937706)
				(1820.43,0.946116)
				(2008.15,0.94119)
				(2277.16,0.954712)
				(2725.59,0.951358)
				(4288.99,0.965053)
			};

			% EAMS ev.mean[0] %%%%%%%%%%%%%%%%%%%%%%%%%%%%%%%%%%%%%%%%%%%%%%%%%%%%%%%%%%%%
			\addplot[EAMS] coordinates{
				(620.832,0.938646)
				(654.836,0.94007)
				(692.407,0.941587)
				(802.34,0.945212)
				(928.046,0.948641)
				(1100.62,0.952399)
				(1352.02,0.956636)
				(1417.4,0.957478)
				(1475.78,0.958218)
				(1615.99,0.960015)
				(1697.79,0.961)
				(1890.75,0.963019)
				(2299.43,0.966433)
				(2651.12,0.968916)
				(3435.61,0.973139)
				(4355.4,0.976328)
				(9229,0.985897)
				(54049,0.999958)
			};

			% CRS ev.mean[0] %%%%%%%%%%%%%%%%%%%%%%%%%%%%%%%%%%%%%%%%%%%%%%%%%%%%%%%%%%%%
			\addplot[CRS] coordinates{
				(265.308,0.857758)
				(397.931,0.873588)
				(497.593,0.880407)
				(721.258,0.896353)
				(961.564,0.905122)
				(1140.37,0.910484)
				(1467.36,0.918712)
				(1670.97,0.922191)
				(1883.05,0.924738)
				(2143.85,0.929428)
				(2266.56,0.930318)
				(2938.4,0.938196)
				(3468.68,0.942421)
				(3772.34,0.944096)
				(4396.39,0.948126)
				(4480.97,0.94865)
				(5789.22,0.955572)
				(6015.14,0.95641)
			};

			% SEAW ev.mean[0] %%%%%%%%%%%%%%%%%%%%%%%%%%%%%%%%%%%%%%%%%%%%%%%%%%%%%%%%%%%%
			\addplot[SEAW] coordinates{
				(67.4528,0.691576)
				(154.597,0.804686)
				(414.981,0.873928)
				(1371.13,0.924433)
				(5128.07,0.960048)
			};

			% RESEEDS ev.mean[0] %%%%%%%%%%%%%%%%%%%%%%%%%%%%%%%%%%%%%%%%%%%%%%%%%%%%%%%%%%%%
			\addplot[RESEEDS] coordinates{
				(200.935,0.914764)
				(301.214,0.92976)
				(400.572,0.931556)
				(600.709,0.942393)
				(800.713,0.947512)
				(1022.1,0.950492)
				(1201.9,0.955603)
				(1368.41,0.956998)
				(1611.62,0.959859)
				(1898.44,0.963074)
				(1950.53,0.963368)
				(2357.59,0.966555)
				(3094.96,0.970882)
				(3136.77,0.971105)
				(3373.89,0.972277)
				(4659.04,0.976626)
				(4681.23,0.976713)
				(4698.34,0.97676)
			};

			% ERGC ev.mean[0] %%%%%%%%%%%%%%%%%%%%%%%%%%%%%%%%%%%%%%%%%%%%%%%%%%%%%%%%%%%%
			\addplot[ERGC] coordinates{
				(196.732,0.876913)
				(305.558,0.896709)
				(398.958,0.907698)
				(600.558,0.922048)
				(804.411,0.931362)
				(964.717,0.936606)
				(1225.23,0.943217)
				(1429.78,0.947045)
				(1618.26,0.949984)
				(1848.19,0.953032)
				(1951.28,0.954035)
				(2499.77,0.959345)
				(2981.51,0.96284)
				(3207.91,0.964146)
				(3794.68,0.967249)
				(3867.02,0.967566)
				(5004.8,0.9717)
				(5201.44,0.972453)
			};

			% PF ev.mean[0] %%%%%%%%%%%%%%%%%%%%%%%%%%%%%%%%%%%%%%%%%%%%%%%%%%%%%%%%%%%%
			\addplot[PF] coordinates{
				(216.65,0.708211)
				(339.635,0.741174)
				(456.474,0.759301)
				(712.126,0.786546)
				(874.44,0.799538)
				(1073.21,0.810853)
				(1285.37,0.821074)
				(1610,0.834951)
				(2385.75,0.854397)
				(3007.38,0.865492)
				(4555.92,0.884559)
				(5051.5,0.889023)
			};

			% TPS ev.mean[0] %%%%%%%%%%%%%%%%%%%%%%%%%%%%%%%%%%%%%%%%%%%%%%%%%%%%%%%%%%%%
			\addplot[TPS] coordinates{
				(405.208,0.832051)
				(827.191,0.865628)
				(1275.26,0.882695)
				(1633.84,0.895142)
				(2206.68,0.905066)
				(2464,0.907838)
				(3188.7,0.915845)
				(3314.33,0.916398)
				(3867.77,0.920316)
				(4417.48,0.923538)
				(5172.02,0.927302)
				(6284.27,0.93327)
			};

			% NC ev.mean[0] %%%%%%%%%%%%%%%%%%%%%%%%%%%%%%%%%%%%%%%%%%%%%%%%%%%%%%%%%%%%
			%\addplot[NC] coordinates{
			%	(389,0.85945)
			%	(996.16,0.880397)
			%	(2879.2,0.90549)
			%	(3933.76,0.912784)
			%};

			% VC ev.mean[0] %%%%%%%%%%%%%%%%%%%%%%%%%%%%%%%%%%%%%%%%%%%%%%%%%%%%%%%%%%%%
			\addplot[VC] coordinates{
				(415.908,0.855574)
				(605.48,0.889472)
				(759.394,0.904756)
				(898.063,0.914069)
				(1157.92,0.926206)
				(1395.79,0.93336)
				(1619.19,0.93868)
				(1823.04,0.942723)
				(2019.03,0.94611)
				(2189.37,0.949054)
				(2481.19,0.953525)
				(2975.07,0.960147)
				(3107.27,0.961667)
				(3429.99,0.963836)
				(4202.78,0.967654)
				(4381.11,0.968869)
				(4529.5,0.969962)
			};

			% PB ev.mean[0] %%%%%%%%%%%%%%%%%%%%%%%%%%%%%%%%%%%%%%%%%%%%%%%%%%%%%%%%%%%%
			\addplot[PB] coordinates{
				(242.509,0.763033)
				(336.132,0.797407)
				(432.128,0.818953)
				(642.864,0.848112)
				(779.57,0.860183)
				(969.218,0.87339)
				(1156.66,0.883428)
				(1518.25,0.897774)
				(1533.52,0.898338)
				(1620.65,0.900871)
				(2041.79,0.911965)
				(2182.61,0.914684)
				(2878.04,0.92665)
				(2894.86,0.926839)
				(3142.19,0.929834)
				(4314.18,0.942301)
				(4343.25,0.942509)
				(4382.33,0.942771)
			};

			% PRESLIC ev.mean[0] %%%%%%%%%%%%%%%%%%%%%%%%%%%%%%%%%%%%%%%%%%%%%%%%%%%%%%%%%%%%
			\addplot[PRESLIC] coordinates{
				(172.769,0.803841)
				(278.95,0.846457)
				(360.824,0.862414)
				(580.759,0.88496)
				(751.065,0.895299)
				(913.027,0.903144)
				(1146.3,0.911464)
				(1471.5,0.919104)
				(1484.69,0.918818)
				(1569.68,0.920099)
				(2028.86,0.927659)
				(2180.05,0.928697)
				(2963.54,0.93654)
				(2977.44,0.935297)
				(3243.14,0.936389)
				(4562.43,0.945164)
				(4597.26,0.945327)
				(4645.64,0.945579)
			};

			% W ev.mean[0] %%%%%%%%%%%%%%%%%%%%%%%%%%%%%%%%%%%%%%%%%%%%%%%%%%%%%%%%%%%%
			\addplot[W] coordinates{
				(197.992,0.811999)
				(305.086,0.8375)
				(402.275,0.850465)
				(641.101,0.870856)
				(809.74,0.87929)
				(1006.25,0.886369)
				(1238.6,0.893562)
				(1612.96,0.900393)
				(1629.74,0.900773)
				(1724.79,0.902019)
				(2219.71,0.908288)
				(2396.42,0.909882)
				(3271.88,0.916513)
				(3295.97,0.9166)
				(3637.87,0.91837)
				(5329.19,0.925719)
				(5373.04,0.925864)
				(5433.59,0.926056)
			};

			% LSC ev.mean[0] %%%%%%%%%%%%%%%%%%%%%%%%%%%%%%%%%%%%%%%%%%%%%%%%%%%%%%%%%%%%
			\addplot[LSC] coordinates{
				(441.153,0.902788)
				(632.558,0.911749)
				(804.126,0.913358)
				(968.442,0.914095)
				(1136.86,0.912035)
				(1269,0.909654)
				(1404.39,0.907009)
				(1516.23,0.905143)
				(1635.31,0.903986)
				(1752.66,0.904989)
				(1823.92,0.903722)
				(2133.24,0.90261)
				(2369.8,0.900285)
			};

			% WP ev.mean[0] %%%%%%%%%%%%%%%%%%%%%%%%%%%%%%%%%%%%%%%%%%%%%%%%%%%%%%%%%%%%
			\addplot[WP] coordinates{
				(196.864,0.841929)
				(305.092,0.864607)
				(397.694,0.876322)
				(627.704,0.894598)
				(787.436,0.902573)
				(992.365,0.909752)
				(1223.32,0.916108)
				(1570.9,0.922396)
				(1660.13,0.923742)
				(2109.92,0.929255)
				(2117.36,0.92914)
				(2213.89,0.930175)
				(2677.06,0.936472)
				(2696.13,0.935437)
				(2934.56,0.936073)
				(4159.73,0.944033)
				(4182.61,0.944128)
				(4254.45,0.944511)
			};

			% QS ev.mean[0] %%%%%%%%%%%%%%%%%%%%%%%%%%%%%%%%%%%%%%%%%%%%%%%%%%%%%%%%%%%%
			\addplot[QS] coordinates{
				(206.314,0.634728)
				(339.805,0.785803)
				(588.734,0.871004)
				(646.82,0.880714)
				(804.377,0.900479)
				(912.229,0.909877)
				(1048.4,0.919282)
				(1230.61,0.92911)
				(1476.88,0.938785)
				(1819.53,0.948333)
				(2310.25,0.957845)
				(3027.82,0.967085)
				(4157.32,0.975954)
				(5985.26,0.983997)
			};

			% VLSLIC ev.mean[0] %%%%%%%%%%%%%%%%%%%%%%%%%%%%%%%%%%%%%%%%%%%%%%%%%%%%%%%%%%%%
			\addplot[VLSLIC] coordinates{
				(361.05,0.894587)
				(448.583,0.896871)
				(527.002,0.897337)
				(675.243,0.896858)
				(769.256,0.898979)
				(913.256,0.901493)
				(1072.97,0.905662)
				(1367.59,0.908689)
				(1414.3,0.913584)
				(1505.73,0.91678)
				(1867.41,0.917539)
				(2010.99,0.920956)
				(2531.28,0.912224)
				(2581.1,0.912977)
			};

			% CIS ev.mean[0] %%%%%%%%%%%%%%%%%%%%%%%%%%%%%%%%%%%%%%%%%%%%%%%%%%%%%%%%%%%%
			\addplot[CIS] coordinates{
				(248.342,0.87676)
				(329.042,0.8889)
				(401.874,0.898421)
				(572.715,0.912619)
				(678.086,0.915881)
				(804.69,0.921916)
				(943.665,0.924558)
				(1284.3,0.936446)
				(1322.71,0.937483)
				(1423.88,0.939223)
				(1847.32,0.944474)
				(2034.14,0.948064)
				(2741.07,0.959058)
				(3318.96,0.96363)
				(4404.2,0.967634)
				(4431.82,0.967746)
				(4473.08,0.967922)
			};

			% ERS ev.mean[0] %%%%%%%%%%%%%%%%%%%%%%%%%%%%%%%%%%%%%%%%%%%%%%%%%%%%%%%%%%%%
			\addplot[ERS] coordinates{
				(200,0.82779)
				(300,0.845645)
				(400,0.857143)
				(600,0.872337)
				(800,0.882752)
				(1000,0.890563)
				(1200,0.896879)
				(1400,0.902188)
				(1600,0.906697)
				(1800,0.910786)
				(2000,0.914515)
				(2400,0.920862)
				(2800,0.926242)
				(3200,0.930892)
				(3600,0.935011)
				(4000,0.938703)
				(4600,0.943679)
				(5200,0.948127)
			};

			% MSS ev.mean[0] %%%%%%%%%%%%%%%%%%%%%%%%%%%%%%%%%%%%%%%%%%%%%%%%%%%%%%%%%%%%
			\addplot[MSS] coordinates{
				(183.048,0.834659)
				(297.333,0.85968)
				(419.465,0.873917)
				(680.717,0.888617)
				(844.541,0.892361)
				(1087.43,0.89861)
				(1321.31,0.90166)
				(1806.84,0.907088)
				(1828.12,0.90744)
				(1948.18,0.908423)
				(2526.57,0.912649)
				(2729.61,0.913805)
				(3749.83,0.918095)
				(3777.69,0.918053)
				(4207.84,0.919163)
				(6286.37,0.923436)
				(6342.3,0.923462)
			};

			% ETPS ev.mean[0] %%%%%%%%%%%%%%%%%%%%%%%%%%%%%%%%%%%%%%%%%%%%%%%%%%%%%%%%%%%%
			\addplot[ETPS] coordinates{
				(196.618,0.934797)
				(305.541,0.942848)
				(415.836,0.94733)
				(648.057,0.954844)
				(787.866,0.958028)
				(992.331,0.961016)
				(1185.61,0.963771)
				(1570.96,0.966599)
				(1587.11,0.966773)
				(1679.34,0.967653)
				(2122.77,0.971068)
				(2271,0.971797)
				(3010.56,0.97478)
				(3029.14,0.974834)
				(3309.82,0.976008)
				(4675.49,0.979902)
				(4710.48,0.979968)
				(4758.49,0.980072)
			};

			\end{axis}
	\end{tikzpicture}
\end{subfigure}

	\begin{subfigure}[b]{\fullthreeone\textwidth}\phantomsubcaption\label{subfig:appendix-experiments-sunrgbd-rec.mean[0]}
	%%%%%%%%%%%%%%%%%%%%%%%%%%%%%%%%%%%%%%%%%%%%%%%%%%%%%%%%%%%%
	% rec.mean[0]
	%%%%%%%%%%%%%%%%%%%%%%%%%%%%%%%%%%%%%%%%%%%%%%%%%%%%%%%%%%%%
	\begin{tikzpicture}
		\begin{axis}[EQSUNRGBDRec,xmode=log]		
	
			% SLIC3D rec.mean[0] %%%%%%%%%%%%%%%%%%%%%%%%%%%%%%%%%%%%%%%%%%%%%%%%%%%%%%%%%%%%
			\addplot[SLIC3D] coordinates{
				(183.423,0.700622)
				(278.67,0.753145)
				(364.19,0.782654)
				(573.713,0.83453)
				(765.372,0.864302)
				(943.992,0.88461)
				(1139.02,0.901059)
				(1296.29,0.911775)
				(1484.66,0.923662)
				(1696.09,0.933966)
				(1993.85,0.944689)
				(2379.4,0.956282)
				(2638.21,0.962835)
				(3112.2,0.97101)
				(3533.8,0.97696)
				(3850.01,0.980152)
				(4402.95,0.984754)
				(4982.22,0.987785)
			};
	
			% CCS rec.mean[0] %%%%%%%%%%%%%%%%%%%%%%%%%%%%%%%%%%%%%%%%%%%%%%%%%%%%%%%%%%%%
			\addplot[CCS] coordinates{
				(195.753,0.560687)
				(297.047,0.633326)
				(384.925,0.674029)
				(613.1,0.749667)
				(806.407,0.788443)
				(983.122,0.815255)
				(1186.45,0.842207)
				(1344.63,0.857819)
				(1535.62,0.874547)
				(1750.9,0.890383)
				(2053.87,0.907824)
				(2449.29,0.927156)
				(2720.31,0.938129)
				(3199.35,0.952499)
				(3647.26,0.963447)
				(3966.9,0.969114)
				(4514.55,0.97774)
				(5088.07,0.983047)
			};
	
			% SEEDS rec.mean[0] %%%%%%%%%%%%%%%%%%%%%%%%%%%%%%%%%%%%%%%%%%%%%%%%%%%%%%%%%%%%
			\addplot[SEEDS] coordinates{
				(242.32,0.852534)
				(347.67,0.88443)
				(448.677,0.907378)
				(664.97,0.938338)
				(862.708,0.950694)
				(1070.34,0.962268)
				(1272.95,0.969124)
				(1468.56,0.975018)
				(1674.56,0.977372)
				(1891.19,0.981847)
				(2065.22,0.982215)
				(2491.02,0.98841)
				(2891.48,0.992069)
				(3275.06,0.992085)
				(3683.12,0.993916)
				(4069.33,0.993205)
				(4676.44,0.995841)
				(5291.02,0.99702)
			};
	
			% SLIC rec.mean[0] %%%%%%%%%%%%%%%%%%%%%%%%%%%%%%%%%%%%%%%%%%%%%%%%%%%%%%%%%%%%
			\addplot[SLIC] coordinates{
				(185.398,0.700627)
				(282.175,0.752002)
				(367.93,0.782185)
				(579.227,0.833694)
				(773.253,0.865831)
				(953.495,0.884366)
				(1150.35,0.902806)
				(1308.64,0.913413)
				(1499.73,0.925743)
				(1711.93,0.935342)
				(2012.34,0.946899)
				(2399.39,0.958045)
				(2660.45,0.964273)
				(3138.66,0.972826)
				(3565.77,0.979067)
				(3884.39,0.98199)
				(4439.6,0.986507)
				(5020.27,0.989432)
			};
	
			% CW rec.mean[0] %%%%%%%%%%%%%%%%%%%%%%%%%%%%%%%%%%%%%%%%%%%%%%%%%%%%%%%%%%%%
			\addplot[CW] coordinates{
				(198.477,0.659511)
				(300.233,0.707499)
				(408.348,0.745343)
				(613.44,0.791215)
				(818.883,0.822555)
				(1012.03,0.847804)
				(1244.05,0.869345)
				(1427.65,0.884316)
				(1623.65,0.897439)
				(1876.75,0.911352)
				(2098.07,0.921332)
				(2481.19,0.937713)
				(2859.2,0.950445)
				(3434.78,0.963418)
				(3867.11,0.970587)
				(4348.77,0.977637)
				(4833.84,0.982137)
				(5520.31,0.987878)
			};
	
			% TP rec.mean[0] %%%%%%%%%%%%%%%%%%%%%%%%%%%%%%%%%%%%%%%%%%%%%%%%%%%%%%%%%%%%
			\addplot[TP] coordinates{
				(306.073,0.676591)
				(430.46,0.720589)
				(526.722,0.740442)
				(783.747,0.789898)
				(1003.49,0.815663)
				(1410.02,0.852698)
				(1771.18,0.878242)
				(2287.71,0.906526)
				(2650.15,0.920307)
				(2914.68,0.930293)
				(3348.8,0.94482)
				(3709.81,0.954598)
			};
	
			% POISE rec.mean[0] %%%%%%%%%%%%%%%%%%%%%%%%%%%%%%%%%%%%%%%%%%%%%%%%%%%%%%%%%%%%
			\addplot[POISE] coordinates{
				(204.91,0.765097)
				(306.737,0.798727)
				(408.505,0.822588)
				(611.875,0.855228)
				(815.405,0.876701)
				(1018.5,0.892556)
				(1220.88,0.904279)
				(1421.58,0.913227)
				(1618.95,0.919698)
				(1813.59,0.924591)
				(2002.57,0.928431)
				(2356.21,0.933498)
				(2669.72,0.936769)
				(2932.32,0.93872)
				(3135.23,0.939846)
				(3284.75,0.940536)
				(3421.31,0.941032)
				(3494.31,0.941284)
			};
	
			% FH rec.mean[0] %%%%%%%%%%%%%%%%%%%%%%%%%%%%%%%%%%%%%%%%%%%%%%%%%%%%%%%%%%%%
			\addplot[FH] coordinates{
				(841.75,0.793509)
				(1061.4,0.82821)
				(1277.38,0.852403)
				(1349.58,0.859137)
				(1858.42,0.874133)
				(2440.8,0.919915)
				(2688.55,0.913184)
				(2806.41,0.922048)
				(3771.37,0.948512)
				(4117.87,0.945379)
				(4886.07,0.950423)
				(5426.45,0.972643)
				(6202.02,0.965731)
			};
	
			% EAMS rec.mean[0] %%%%%%%%%%%%%%%%%%%%%%%%%%%%%%%%%%%%%%%%%%%%%%%%%%%%%%%%%%%%
			\addplot[EAMS] coordinates{
				(586.677,0.839123)
				(632.232,0.846755)
				(687.737,0.855569)
				(836.78,0.874128)
				(1027.26,0.901404)
				(1478.62,0.928738)
				(3028.83,0.973067)
				(3155.03,0.974276)
				(3284.1,0.975485)
				(3576.09,0.977955)
				(3934.32,0.9802)
				(4483.62,0.982865)
				(5080.25,0.984828)
				(5840.7,0.986697)
				(6948.21,0.988409)
				(9692.98,0.990283)
				(17957,0.992008)
			};
	
			% CRS rec.mean[0] %%%%%%%%%%%%%%%%%%%%%%%%%%%%%%%%%%%%%%%%%%%%%%%%%%%%%%%%%%%%
			\addplot[CRS] coordinates{
				(283.467,0.720209)
				(399.272,0.776226)
				(530.128,0.816102)
				(798.34,0.865994)
				(1037.19,0.892096)
				(1259.96,0.913062)
				(1495.45,0.927582)
				(1694.58,0.936232)
				(1947.02,0.947555)
				(2248.88,0.957277)
				(2455.29,0.960772)
				(2859.61,0.970999)
				(3297.01,0.977913)
				(3904.05,0.984552)
				(4297.22,0.987274)
				(4803.79,0.990714)
				(5386.35,0.993234)
				(6068.33,0.99549)
			};
	
			% RESEEDS rec.mean[0] %%%%%%%%%%%%%%%%%%%%%%%%%%%%%%%%%%%%%%%%%%%%%%%%%%%%%%%%%%%%
			\addplot[RESEEDS] coordinates{
				(199.387,0.855135)
				(300.128,0.891713)
				(399.04,0.909161)
				(602.485,0.93376)
				(801.48,0.947192)
				(1004.03,0.957047)
				(1203.35,0.964074)
				(1396.1,0.968801)
				(1601.84,0.973806)
				(1818.54,0.977218)
				(1992.37,0.980563)
				(2412.1,0.984631)
				(2809.68,0.987917)
				(3201.43,0.990132)
				(3613.63,0.991582)
				(4004.66,0.993753)
				(4612.95,0.995324)
				(5237.44,0.996452)
			};
	
			% ERGC rec.mean[0] %%%%%%%%%%%%%%%%%%%%%%%%%%%%%%%%%%%%%%%%%%%%%%%%%%%%%%%%%%%%
			\addplot[ERGC] coordinates{
				(196,0.68785)
				(296.905,0.742271)
				(401.35,0.780467)
				(600,0.826816)
				(818.51,0.861144)
				(982.7,0.880576)
				(1206.2,0.900211)
				(1373.27,0.912149)
				(1564.46,0.923748)
				(1821.57,0.937056)
				(2003.95,0.943872)
				(2374.46,0.95639)
				(2728.29,0.965452)
				(3264.02,0.97506)
				(3663.22,0.979923)
				(4101.08,0.984632)
				(4545.42,0.987899)
				(5170.11,0.991822)
			};
	
			% PF rec.mean[0] %%%%%%%%%%%%%%%%%%%%%%%%%%%%%%%%%%%%%%%%%%%%%%%%%%%%%%%%%%%%
			\addplot[PF] coordinates{
				(297.922,0.416929)
				(446.965,0.469455)
				(576.292,0.503612)
				(890.323,0.56271)
				(1179.05,0.604769)
				(1446.47,0.634916)
				(1709.54,0.66062)
				(1918.02,0.678414)
				(2158.64,0.69908)
				(2441.51,0.716731)
				(2816.51,0.738985)
				(3392.35,0.782887)
				(4335.67,0.819315)
				(6371.69,0.87478)
			};
	
			% TPS rec.mean[0] %%%%%%%%%%%%%%%%%%%%%%%%%%%%%%%%%%%%%%%%%%%%%%%%%%%%%%%%%%%%
			\addplot[TPS] coordinates{
				(232.273,0.547525)
				(332.385,0.596799)
				(444.122,0.636151)
				(657.893,0.689531)
				(889.677,0.731255)
				(1069.44,0.758258)
				(1277.73,0.781735)
				(1526.67,0.806275)
				(1751.96,0.827231)
				(1970.22,0.842121)
				(2152.95,0.851699)
				(2501.68,0.872627)
				(3072.82,0.902075)
				(3436.22,0.91394)
				(3853.6,0.927704)
				(4314.69,0.937937)
				(4837.54,0.947601)
				(5569.37,0.965293)
			};
	
			% VC rec.mean[0] %%%%%%%%%%%%%%%%%%%%%%%%%%%%%%%%%%%%%%%%%%%%%%%%%%%%%%%%%%%%
			\addplot[VC] coordinates{
				(264.483,0.658917)
				(419.552,0.729946)
				(553.5,0.768846)
				(681.035,0.793551)
				(922.085,0.825879)
				(1156.54,0.847948)
				(1382.11,0.863366)
				(1597.11,0.876405)
				(1816.61,0.886978)
				(2034.93,0.896666)
				(2248.42,0.904554)
				(2454.14,0.912612)
				(2869.61,0.924615)
				(3276.23,0.934464)
				(3673.47,0.942799)
				(4059.96,0.949485)
				(4435.44,0.955801)
				(4955.44,0.962243)
				(5425.75,0.967251)
			};
	
			% PB rec.mean[0] %%%%%%%%%%%%%%%%%%%%%%%%%%%%%%%%%%%%%%%%%%%%%%%%%%%%%%%%%%%%
			\addplot[PB] coordinates{
				(298.603,0.607986)
				(397.12,0.65879)
				(479.885,0.689284)
				(680.155,0.742881)
				(870.438,0.780787)
				(1047.06,0.804719)
				(1237.46,0.829873)
				(1390.23,0.847182)
				(1569.82,0.862507)
				(1783.13,0.879817)
				(2072.57,0.897123)
				(2461.61,0.916896)
				(2716.23,0.927393)
				(3159.05,0.94268)
				(3547.41,0.953634)
				(3882.6,0.96087)
				(4461.39,0.971619)
				(4997.73,0.977419)
			};
	
			% VCCS rec.mean[0] %%%%%%%%%%%%%%%%%%%%%%%%%%%%%%%%%%%%%%%%%%%%%%%%%%%%%%%%%%%%
			\addplot[VCCS] coordinates{
				(600.688,0.633666)
				(850.68,0.696416)
				(947.458,0.715684)
				(1254.7,0.737677)
				(1340.03,0.74913)
				(1426.25,0.759409)
				(1495.03,0.766674)
				(1622.35,0.77809)
				(1724.65,0.787667)
				(1876.49,0.799458)
				(2020.57,0.809674)
				(2184.95,0.820499)
				(2394.23,0.83396)
				(2551.26,0.838367)
				(2841.62,0.852579)
				(3185.59,0.869968)
				(3477.67,0.878386)
				(3842.61,0.89037)
				(4566.27,0.917609)
				(5200.63,0.933403)
				(5463.7,0.928775)
			};
	
			% PRESLIC rec.mean[0] %%%%%%%%%%%%%%%%%%%%%%%%%%%%%%%%%%%%%%%%%%%%%%%%%%%%%%%%%%%%
			\addplot[PRESLIC] coordinates{
				(189.206,0.681779)
				(378.882,0.770994)
				(573.018,0.823581)
				(752.995,0.855059)
				(921.787,0.875113)
				(1110.43,0.895849)
				(1262.35,0.907571)
				(1445.35,0.920021)
				(1659.42,0.931816)
				(1954.02,0.943088)
				(2339.45,0.955823)
				(2601.8,0.963389)
				(3082.42,0.97254)
				(3505.78,0.978789)
				(3822.55,0.981418)
				(4402.4,0.986016)
				(5008.01,0.988914)
			};
	
			% W rec.mean[0] %%%%%%%%%%%%%%%%%%%%%%%%%%%%%%%%%%%%%%%%%%%%%%%%%%%%%%%%%%%%
			\addplot[W] coordinates{
				(199.607,0.656122)
				(302.642,0.708311)
				(408.6,0.742913)
				(608.068,0.78765)
				(813.78,0.82256)
				(999.44,0.844688)
				(1190.19,0.863895)
				(1373.25,0.881425)
				(1588.89,0.896086)
				(1820.39,0.909903)
				(2138.11,0.925489)
				(2539.07,0.94064)
				(2810.66,0.949171)
				(3313.99,0.961788)
				(3783.59,0.97068)
				(4142.1,0.974907)
				(4770.12,0.982773)
				(5365.69,0.986531)
			};
	
			% LSC rec.mean[0] %%%%%%%%%%%%%%%%%%%%%%%%%%%%%%%%%%%%%%%%%%%%%%%%%%%%%%%%%%%%
			\addplot[LSC] coordinates{
				(412.42,0.785885)
				(566.403,0.818698)
				(737.927,0.847607)
				(1100.36,0.885118)
				(1387.69,0.904746)
				(1658.32,0.920081)
				(1915.58,0.930294)
				(2087.95,0.936029)
				(2318.23,0.944384)
				(2558.32,0.951003)
				(2685.31,0.954153)
				(2950.94,0.959681)
				(3247.86,0.964933)
				(3645.17,0.970573)
				(3930.7,0.97376)
				(4322.76,0.979261)
				(4740.34,0.983666)
				(5250.39,0.988231)
			};
	
			% WP rec.mean[0] %%%%%%%%%%%%%%%%%%%%%%%%%%%%%%%%%%%%%%%%%%%%%%%%%%%%%%%%%%%%
			\addplot[WP] coordinates{
				(211.102,0.671269)
				(323.783,0.71679)
				(403.883,0.742619)
				(625.86,0.786929)
				(823.02,0.818262)
				(1022.26,0.841434)
				(1222.72,0.859571)
				(1385.35,0.87259)
				(1559.27,0.886249)
				(1797.57,0.899475)
				(2060.75,0.912854)
				(2438.15,0.928524)
				(2702.28,0.937775)
				(3226.35,0.951651)
				(3920.65,0.965404)
				(3980.05,0.966644)
				(4987.74,0.978903)
				(5069.48,0.979802)
			};
	
			% QS rec.mean[0] %%%%%%%%%%%%%%%%%%%%%%%%%%%%%%%%%%%%%%%%%%%%%%%%%%%%%%%%%%%%
			\addplot[QS] coordinates{
				(223.997,0.416027)
				(317.098,0.504533)
				(444.57,0.582111)
				(626.915,0.653316)
				(830.568,0.704259)
				(1003.65,0.735093)
				(1117.84,0.751634)
				(1401.9,0.784824)
				(1593.35,0.80155)
				(1834.32,0.819057)
				(2169.17,0.839225)
				(2619.48,0.860989)
				(3266.76,0.88431)
				(4204.98,0.90988)
				(5572.49,0.934218)
			};
	
			% VLSLIC rec.mean[0] %%%%%%%%%%%%%%%%%%%%%%%%%%%%%%%%%%%%%%%%%%%%%%%%%%%%%%%%%%%%
			\addplot[VLSLIC] coordinates{
				(744.14,0.822374)
				(900.468,0.848651)
				(1017.84,0.863268)
				(1278.77,0.889327)
				(1492.48,0.906491)
				(1671.62,0.918211)
				(1851.27,0.928896)
				(1992.46,0.93491)
				(2165.14,0.942988)
				(2318.73,0.948544)
				(2546.17,0.955549)
				(2829.93,0.96274)
				(3019.05,0.967557)
				(3328.6,0.973281)
				(3594.5,0.97732)
				(3778.65,0.979959)
				(4132.48,0.98346)
				(4422.49,0.985944)
			};
	
			% CIS rec.mean[0] %%%%%%%%%%%%%%%%%%%%%%%%%%%%%%%%%%%%%%%%%%%%%%%%%%%%%%%%%%%%
			\addplot[CIS] coordinates{
				(320.195,0.618476)
				(395.365,0.649219)
				(467.525,0.671494)
				(601.885,0.705438)
				(736.583,0.731883)
				(849.645,0.747677)
				(981.057,0.765535)
				(1091.31,0.779897)
				(1216.94,0.793547)
				(1380.64,0.80886)
				(1601.9,0.825167)
				(1883.81,0.84696)
				(2101.28,0.859823)
				(2894.19,0.892038)
				(3109.28,0.902071)
				(3543.38,0.915615)
				(4114.64,0.927022)
			};
	
			% RESEEDS3D rec.mean[0] %%%%%%%%%%%%%%%%%%%%%%%%%%%%%%%%%%%%%%%%%%%%%%%%%%%%%%%%%%%%
			\addplot[RESEEDS3D] coordinates{
				(199.258,0.766285)
				(300.057,0.822217)
				(398.795,0.853366)
				(602.237,0.895866)
				(801.247,0.9156)
				(1003.77,0.932664)
				(1203.26,0.944781)
				(1396.06,0.95433)
				(1601.57,0.961006)
				(1818.47,0.966848)
				(1991.77,0.971852)
				(2412.05,0.978273)
				(2809.62,0.984072)
				(3201.4,0.987158)
				(3613.59,0.989182)
				(4004.17,0.992422)
				(4612.92,0.994705)
				(5237.39,0.995762)
			};
	
			% ERS rec.mean[0] %%%%%%%%%%%%%%%%%%%%%%%%%%%%%%%%%%%%%%%%%%%%%%%%%%%%%%%%%%%%
			\addplot[ERS] coordinates{
				(200,0.754332)
				(300,0.796291)
				(400,0.824751)
				(600,0.86508)
				(800,0.891929)
				(1000,0.911982)
				(1200,0.927674)
				(1400,0.939406)
				(1600,0.949179)
				(1800,0.95689)
				(2000,0.96342)
				(2400,0.9733)
				(2800,0.980201)
				(3200,0.985089)
				(3600,0.988973)
				(4000,0.9915)
				(4600,0.994193)
				(5200,0.996116)
			};
	
			% DASP rec.mean[0] %%%%%%%%%%%%%%%%%%%%%%%%%%%%%%%%%%%%%%%%%%%%%%%%%%%%%%%%%%%%
			\addplot[DASP] coordinates{
				(470.547,0.738027)
				(571.287,0.775232)
				(670.5,0.798267)
				(860.343,0.831531)
				(1050.99,0.853257)
				(1241.78,0.870236)
				(1427.19,0.88402)
				(1612.46,0.894422)
				(1798.14,0.903986)
				(1980.08,0.911799)
				(2164.51,0.918259)
				(2524.9,0.929687)
				(2885.04,0.93946)
				(3244.73,0.946728)
				(3598.39,0.952252)
				(3947.44,0.957856)
				(4457.26,0.963517)
				(4965.73,0.968588)
			};
	
			% MSS rec.mean[0] %%%%%%%%%%%%%%%%%%%%%%%%%%%%%%%%%%%%%%%%%%%%%%%%%%%%%%%%%%%%
			\addplot[MSS] coordinates{
				(213.423,0.677816)
				(330.475,0.726373)
				(434.73,0.749431)
				(696.625,0.798988)
				(947.095,0.827867)
				(1166.71,0.847441)
				(1428.43,0.867019)
				(1631.67,0.877602)
				(1873.19,0.890861)
				(2161.03,0.903107)
				(2548.86,0.917068)
				(3080.65,0.932566)
				(3446.28,0.941017)
				(4062.02,0.953159)
				(4593.88,0.960613)
				(5065.06,0.966077)
				(5900.9,0.975348)
				(6661.89,0.980492)
			};
	
			% ETPS rec.mean[0] %%%%%%%%%%%%%%%%%%%%%%%%%%%%%%%%%%%%%%%%%%%%%%%%%%%%%%%%%%%%
			\addplot[ETPS] coordinates{
				(220.122,0.826332)
				(325.103,0.86017)
				(415.103,0.878354)
				(635.708,0.909146)
				(841.638,0.927123)
				(1022.26,0.936349)
				(1227.44,0.947248)
				(1388.07,0.952849)
				(1580.77,0.958558)
				(1797.57,0.96444)
				(2099.17,0.971676)
				(2500.64,0.977696)
				(2770.58,0.980283)
				(3226.35,0.986407)
				(3613.57,0.988099)
				(3961.65,0.989719)
				(4557.15,0.992734)
				(5106.62,0.994501)
			};

			\end{axis}
	\end{tikzpicture}
\end{subfigure}%
\begin{subfigure}[b]{\fullthreeone\textwidth}\phantomsubcaption\label{subfig:appendix-experiments-sunrgbd-ue_np.mean[0]}
	%%%%%%%%%%%%%%%%%%%%%%%%%%%%%%%%%%%%%%%%%%%%%%%%%%%%%%%%%%%%
	% ue_np.mean[0]
	%%%%%%%%%%%%%%%%%%%%%%%%%%%%%%%%%%%%%%%%%%%%%%%%%%%%%%%%%%%%
	\begin{tikzpicture}
		\begin{axis}[EQSUNRGBDUE,xmode=log]		
	
			% SLIC3D ue_np.mean[0] %%%%%%%%%%%%%%%%%%%%%%%%%%%%%%%%%%%%%%%%%%%%%%%%%%%%%%%%%%%%
			\addplot[SLIC3D] coordinates{
				(183.423,0.161008)
				(278.67,0.135863)
				(364.19,0.123257)
				(573.713,0.105413)
				(765.372,0.0953273)
				(943.992,0.0885091)
				(1139.02,0.083079)
				(1296.29,0.0795456)
				(1484.66,0.0759617)
				(1696.09,0.0724067)
				(1993.85,0.0684893)
				(2379.4,0.0645004)
				(2638.21,0.0620871)
				(3112.2,0.058488)
				(3533.8,0.0558954)
				(3850.01,0.0541864)
				(4402.95,0.0515583)
				(4982.22,0.0492436)
			};
	
			% CCS ue_np.mean[0] %%%%%%%%%%%%%%%%%%%%%%%%%%%%%%%%%%%%%%%%%%%%%%%%%%%%%%%%%%%%
			\addplot[CCS] coordinates{
				(195.753,0.174841)
				(297.047,0.144499)
				(384.925,0.128856)
				(613.1,0.105208)
				(806.407,0.094717)
				(983.122,0.0881199)
				(1186.45,0.082095)
				(1344.63,0.0783597)
				(1535.62,0.0748617)
				(1750.9,0.0714742)
				(2053.87,0.0674729)
				(2449.29,0.0635113)
				(2720.31,0.0612067)
				(3199.35,0.0577014)
				(3647.26,0.0550748)
				(3966.9,0.0532677)
				(4514.55,0.0507427)
				(5088.07,0.0484126)
			};
	
			% SEEDS ue_np.mean[0] %%%%%%%%%%%%%%%%%%%%%%%%%%%%%%%%%%%%%%%%%%%%%%%%%%%%%%%%%%%%
			\addplot[SEEDS] coordinates{
				(242.32,0.187256)
				(347.67,0.160961)
				(448.677,0.14485)
				(664.97,0.134443)
				(862.708,0.113565)
				(1070.34,0.108188)
				(1272.95,0.0985635)
				(1468.56,0.0959975)
				(1674.56,0.0879075)
				(1891.19,0.0867043)
				(2065.22,0.0818662)
				(2491.02,0.0787608)
				(2891.48,0.0785904)
				(3275.06,0.069129)
				(3683.12,0.0677606)
				(4069.33,0.0620394)
				(4676.44,0.0592444)
				(5291.02,0.0572088)
			};
	
			% SLIC ue_np.mean[0] %%%%%%%%%%%%%%%%%%%%%%%%%%%%%%%%%%%%%%%%%%%%%%%%%%%%%%%%%%%%
			\addplot[SLIC] coordinates{
				(185.398,0.163454)
				(282.175,0.137957)
				(367.93,0.125428)
				(579.227,0.107674)
				(773.253,0.096578)
				(953.495,0.089775)
				(1150.35,0.0845072)
				(1308.64,0.0808068)
				(1499.73,0.0770126)
				(1711.93,0.0735437)
				(2012.34,0.0696933)
				(2399.39,0.0653937)
				(2660.45,0.0630825)
				(3138.66,0.0593688)
				(3565.77,0.0567209)
				(3884.39,0.0549997)
				(4439.6,0.0522275)
				(5020.27,0.0499401)
			};
	
			% CW ue_np.mean[0] %%%%%%%%%%%%%%%%%%%%%%%%%%%%%%%%%%%%%%%%%%%%%%%%%%%%%%%%%%%%
			\addplot[CW] coordinates{
				(198.477,0.165805)
				(300.233,0.142224)
				(408.348,0.125593)
				(613.44,0.109629)
				(818.883,0.099269)
				(1012.03,0.0926699)
				(1244.05,0.086788)
				(1427.65,0.0826989)
				(1623.65,0.0791811)
				(1876.75,0.0765152)
				(2098.07,0.07365)
				(2481.19,0.0695102)
				(2859.2,0.066015)
				(3434.78,0.0617706)
				(3867.11,0.0595668)
				(4348.77,0.0576053)
				(4833.84,0.055716)
				(5520.31,0.052651)
			};
	
			% TP ue_np.mean[0] %%%%%%%%%%%%%%%%%%%%%%%%%%%%%%%%%%%%%%%%%%%%%%%%%%%%%%%%%%%%
			\addplot[TP] coordinates{
				(306.073,0.13515)
				(430.46,0.123594)
				(526.722,0.114095)
				(783.747,0.103954)
				(1003.49,0.093829)
				(1410.02,0.0829499)
				(1771.18,0.0763595)
				(2287.71,0.0700045)
				(2650.15,0.067176)
				(2914.68,0.064948)
				(3348.8,0.0620356)
				(3709.81,0.0601128)
			};
	
			% POISE ue_np.mean[0] %%%%%%%%%%%%%%%%%%%%%%%%%%%%%%%%%%%%%%%%%%%%%%%%%%%%%%%%%%%%
			\addplot[POISE] coordinates{
				(204.91,0.148669)
				(306.737,0.129705)
				(408.505,0.117152)
				(611.875,0.104528)
				(815.405,0.0973643)
				(1018.5,0.0924783)
				(1220.88,0.0886373)
				(1421.58,0.0855769)
				(1618.95,0.0834299)
				(1813.59,0.0814956)
				(2002.57,0.0800192)
				(2356.21,0.0775518)
				(2669.72,0.0759058)
				(2932.32,0.0747244)
				(3135.23,0.0740608)
				(3284.75,0.073593)
				(3421.31,0.0732431)
				(3494.31,0.0730851)
			};
			
			% FH ue_np.mean[0] %%%%%%%%%%%%%%%%%%%%%%%%%%%%%%%%%%%%%%%%%%%%%%%%%%%%%%%%%%%%
			\addplot[FH] coordinates{
				(841.75,0.123307)
				(1061.4,0.113177)
				(1277.38,0.103649)
				(1349.58,0.0999925)
				(1858.42,0.0919589)
				(2440.8,0.0799675)
				(2688.55,0.0812489)
				(2806.41,0.0793615)
				(3771.37,0.0691355)
				(4117.87,0.0697068)
				(4886.07,0.0673882)
				(5426.45,0.0615402)
				(6202.02,0.0593482)
			};
	
			% EAMS ue_np.mean[0] %%%%%%%%%%%%%%%%%%%%%%%%%%%%%%%%%%%%%%%%%%%%%%%%%%%%%%%%%%%%
			\addplot[EAMS] coordinates{
				(586.677,0.115973)
				(632.232,0.112999)
				(687.737,0.109732)
				(836.78,0.102607)
				(1027.26,0.0946259)
				(1478.62,0.0845185)
				(3028.83,0.066683)
				(3155.03,0.0658734)
				(3284.1,0.0649906)
				(3576.09,0.063236)
				(3934.32,0.0613072)
				(4483.62,0.0587445)
				(5080.25,0.0564324)
				(5840.7,0.0537807)
				(6948.21,0.0506649)
				(9692.98,0.0449129)
				(17957,0.0356251)
			};
	
			% CRS ue_np.mean[0] %%%%%%%%%%%%%%%%%%%%%%%%%%%%%%%%%%%%%%%%%%%%%%%%%%%%%%%%%%%%
			\addplot[CRS] coordinates{
				(283.467,0.140625)
				(399.272,0.120844)
				(530.128,0.108709)
				(798.34,0.0931541)
				(1037.19,0.0849892)
				(1259.96,0.0787983)
				(1495.45,0.0738168)
				(1694.58,0.0710281)
				(1947.02,0.0671718)
				(2248.88,0.0637567)
				(2455.29,0.0620045)
				(2859.61,0.0582491)
				(3297.01,0.0554)
				(3904.05,0.0519881)
				(4297.22,0.0500601)
				(4803.79,0.04803)
				(5386.35,0.0459029)
				(6068.33,0.0437911)
			};
	
			% RESEEDS ue_np.mean[0] %%%%%%%%%%%%%%%%%%%%%%%%%%%%%%%%%%%%%%%%%%%%%%%%%%%%%%%%%%%%
			\addplot[RESEEDS] coordinates{
				(199.387,0.182481)
				(300.128,0.146514)
				(399.04,0.129483)
				(602.485,0.120358)
				(801.48,0.101929)
				(1004.03,0.0975756)
				(1203.35,0.087277)
				(1396.1,0.0840598)
				(1601.84,0.0783054)
				(1818.54,0.076859)
				(1992.37,0.0731753)
				(2412.1,0.0692016)
				(2809.68,0.067027)
				(3201.43,0.061282)
				(3613.63,0.060189)
				(4004.66,0.055273)
				(4612.95,0.0525292)
				(5237.44,0.0515599)
			};
	
			% ERGC ue_np.mean[0] %%%%%%%%%%%%%%%%%%%%%%%%%%%%%%%%%%%%%%%%%%%%%%%%%%%%%%%%%%%%
			\addplot[ERGC] coordinates{
				(196,0.150901)
				(296.905,0.129202)
				(401.35,0.11474)
				(600,0.099458)
				(818.51,0.0894945)
				(982.7,0.0843174)
				(1206.2,0.0792874)
				(1373.27,0.0758033)
				(1564.46,0.0725334)
				(1821.57,0.0693707)
				(2003.95,0.0671765)
				(2374.46,0.0632826)
				(2728.29,0.0603994)
				(3264.02,0.0569426)
				(3663.22,0.0548656)
				(4101.08,0.0528209)
				(4545.42,0.0510136)
				(5170.11,0.0485406)
			};
	
			% PF ue_np.mean[0] %%%%%%%%%%%%%%%%%%%%%%%%%%%%%%%%%%%%%%%%%%%%%%%%%%%%%%%%%%%%
			\addplot[PF] coordinates{
				(297.922,0.3014)
				(446.965,0.262736)
				(576.292,0.242616)
				(890.323,0.211982)
				(1179.05,0.193446)
				(1446.47,0.181061)
				(1709.54,0.171241)
				(1918.02,0.164721)
				(2158.64,0.158472)
				(2441.51,0.152904)
				(2816.51,0.146024)
				(3392.35,0.14069)
				(4335.67,0.131005)
				(6371.69,0.117867)
			};
	
			% TPS ue_np.mean[0] %%%%%%%%%%%%%%%%%%%%%%%%%%%%%%%%%%%%%%%%%%%%%%%%%%%%%%%%%%%%
			\addplot[TPS] coordinates{
				(232.273,0.170937)
				(332.385,0.148348)
				(444.122,0.132617)
				(657.893,0.11457)
				(889.677,0.103054)
				(1069.44,0.0965218)
				(1277.73,0.0904506)
				(1526.67,0.0857883)
				(1751.96,0.0815025)
				(1970.22,0.0782737)
				(2152.95,0.0761908)
				(2501.68,0.0724919)
				(3072.82,0.0663136)
				(3436.22,0.0636813)
				(3853.6,0.0609035)
				(4314.69,0.0586673)
				(4837.54,0.0567925)
				(5569.37,0.0530743)
			};
	
			% VC ue_np.mean[0] %%%%%%%%%%%%%%%%%%%%%%%%%%%%%%%%%%%%%%%%%%%%%%%%%%%%%%%%%%%%
			\addplot[VC] coordinates{
				(264.483,0.190909)
				(419.552,0.144514)
				(553.5,0.122813)
				(681.035,0.110868)
				(922.085,0.0962267)
				(1156.54,0.0872858)
				(1382.11,0.0815191)
				(1597.11,0.0769238)
				(1816.61,0.0736639)
				(2034.93,0.0708092)
				(2248.42,0.0683105)
				(2454.14,0.0661483)
				(2869.61,0.0627104)
				(3276.23,0.0601446)
				(3673.47,0.0576538)
				(4059.96,0.0557693)
				(4435.44,0.054016)
				(4955.44,0.0518903)
				(5425.75,0.050269)
			};
	
			% PB ue_np.mean[0] %%%%%%%%%%%%%%%%%%%%%%%%%%%%%%%%%%%%%%%%%%%%%%%%%%%%%%%%%%%%
			\addplot[PB] coordinates{
				(298.603,0.205471)
				(397.12,0.171198)
				(479.885,0.153357)
				(680.155,0.127984)
				(870.438,0.113744)
				(1047.06,0.105186)
				(1237.46,0.0974037)
				(1390.23,0.0926066)
				(1569.82,0.0881893)
				(1783.13,0.0832061)
				(2072.57,0.079283)
				(2461.61,0.0740669)
				(2716.23,0.071092)
				(3159.05,0.0671)
				(3547.41,0.0645232)
				(3882.6,0.0626076)
				(4461.39,0.0592362)
				(4997.73,0.0565176)
			};
	
			% VCCS ue_np.mean[0] %%%%%%%%%%%%%%%%%%%%%%%%%%%%%%%%%%%%%%%%%%%%%%%%%%%%%%%%%%%%
			\addplot[VCCS] coordinates{
				(600.688,0.302359)
				(850.68,0.255183)
				(947.458,0.241146)
				(1254.7,0.223646)
				(1340.03,0.214707)
				(1426.25,0.208607)
				(1495.03,0.201661)
				(1622.35,0.193122)
				(1724.65,0.184236)
				(1876.49,0.175684)
				(2020.57,0.169167)
				(2184.95,0.160646)
				(2394.23,0.149558)
				(2551.26,0.146701)
				(2841.62,0.135994)
				(3185.59,0.124734)
				(3477.67,0.117648)
				(3842.61,0.110692)
				(4566.27,0.0916286)
				(5200.63,0.0795888)
				(5463.7,0.0827249)
			};
	
			% PRESLIC ue_np.mean[0] %%%%%%%%%%%%%%%%%%%%%%%%%%%%%%%%%%%%%%%%%%%%%%%%%%%%%%%%%%%%
			\addplot[PRESLIC] coordinates{
				(189.206,0.182233)
				(378.882,0.136438)
				(573.018,0.115164)
				(752.995,0.104907)
				(921.787,0.0983144)
				(1110.43,0.0921892)
				(1262.35,0.0884828)
				(1445.35,0.0841745)
				(1659.42,0.0803268)
				(1954.02,0.0760685)
				(2339.45,0.0713717)
				(2601.8,0.0685349)
				(3082.42,0.0642932)
				(3505.78,0.0613182)
				(3822.55,0.0593989)
				(4402.4,0.0567204)
				(5008.01,0.054031)
			};
	
			% W ue_np.mean[0] %%%%%%%%%%%%%%%%%%%%%%%%%%%%%%%%%%%%%%%%%%%%%%%%%%%%%%%%%%%%
			\addplot[W] coordinates{
				(199.607,0.177312)
				(302.642,0.152098)
				(408.6,0.134411)
				(608.068,0.116228)
				(813.78,0.105033)
				(999.44,0.0979182)
				(1190.19,0.0922547)
				(1373.25,0.0875726)
				(1588.89,0.0832006)
				(1820.39,0.0795707)
				(2138.11,0.0750359)
				(2539.07,0.0707117)
				(2810.66,0.0688009)
				(3313.99,0.0649385)
				(3783.59,0.061867)
				(4142.1,0.0594275)
				(4770.12,0.0562173)
				(5365.69,0.0542005)
			};
	
			% LSC ue_np.mean[0] %%%%%%%%%%%%%%%%%%%%%%%%%%%%%%%%%%%%%%%%%%%%%%%%%%%%%%%%%%%%
			\addplot[LSC] coordinates{
				(412.42,0.128419)
				(566.403,0.113239)
				(737.927,0.101974)
				(1100.36,0.0883409)
				(1387.69,0.0812315)
				(1658.32,0.0762689)
				(1915.58,0.0721951)
				(2087.95,0.0696646)
				(2318.23,0.0667228)
				(2558.32,0.063847)
				(2685.31,0.0621342)
				(2950.94,0.0595834)
				(3247.86,0.0574121)
				(3645.17,0.0548861)
				(3930.7,0.0533937)
				(4322.76,0.0517247)
				(4740.34,0.0504507)
				(5250.39,0.0490217)
			};
	
			% WP ue_np.mean[0] %%%%%%%%%%%%%%%%%%%%%%%%%%%%%%%%%%%%%%%%%%%%%%%%%%%%%%%%%%%%
			\addplot[WP] coordinates{
				(211.102,0.152233)
				(323.783,0.130433)
				(403.883,0.119864)
				(625.86,0.103783)
				(823.02,0.095855)
				(1022.26,0.0888571)
				(1222.72,0.0849017)
				(1385.35,0.0817872)
				(1559.27,0.0791377)
				(1797.57,0.0756164)
				(2060.75,0.0722459)
				(2438.15,0.0686383)
				(2702.28,0.0671307)
				(3226.35,0.0629448)
				(3920.65,0.058923)
				(3980.05,0.0586596)
				(4987.74,0.0549078)
				(5069.48,0.054629)
			};
	
			% QS ue_np.mean[0] %%%%%%%%%%%%%%%%%%%%%%%%%%%%%%%%%%%%%%%%%%%%%%%%%%%%%%%%%%%%
			\addplot[QS] coordinates{
				(223.997,0.393866)
				(317.098,0.29214)
				(444.57,0.223152)
				(626.915,0.172328)
				(830.568,0.143434)
				(1003.65,0.128618)
				(1117.84,0.121388)
				(1401.9,0.107661)
				(1593.35,0.101389)
				(1834.32,0.0952508)
				(2169.17,0.0887458)
				(2619.48,0.0822409)
				(3266.76,0.0756327)
				(4204.98,0.0684508)
				(5572.49,0.0613054)
			};
	
			% VLSLIC ue_np.mean[0] %%%%%%%%%%%%%%%%%%%%%%%%%%%%%%%%%%%%%%%%%%%%%%%%%%%%%%%%%%%%
			\addplot[VLSLIC] coordinates{
				(744.14,0.126229)
				(900.468,0.110445)
				(1017.84,0.102631)
				(1278.77,0.0906124)
				(1492.48,0.0838366)
				(1671.62,0.0793555)
				(1851.27,0.0758422)
				(1992.46,0.0735704)
				(2165.14,0.0708996)
				(2318.73,0.0688146)
				(2546.17,0.0662193)
				(2829.93,0.0638047)
				(3019.05,0.0624413)
				(3328.6,0.0603205)
				(3594.5,0.0588544)
				(3778.65,0.0578989)
				(4132.48,0.0567518)
				(4422.49,0.0557849)
			};
	
			% CIS ue_np.mean[0] %%%%%%%%%%%%%%%%%%%%%%%%%%%%%%%%%%%%%%%%%%%%%%%%%%%%%%%%%%%%
			\addplot[CIS] coordinates{
				(320.195,0.143102)
				(395.365,0.127757)
				(467.525,0.118629)
				(601.885,0.10642)
				(736.583,0.0984432)
				(849.645,0.0939075)
				(981.057,0.0894691)
				(1091.31,0.0862073)
				(1216.94,0.0833034)
				(1380.64,0.0801579)
				(1601.9,0.0767435)
				(1883.81,0.0726092)
				(2101.28,0.0702561)
				(2894.19,0.0641904)
				(3109.28,0.0626068)
				(3543.38,0.0597684)
				(4114.64,0.0568807)
			};
	
			% RESEEDS3D ue_np.mean[0] %%%%%%%%%%%%%%%%%%%%%%%%%%%%%%%%%%%%%%%%%%%%%%%%%%%%%%%%%%%%
			\addplot[RESEEDS3D] coordinates{
				(199.258,0.168764)
				(300.057,0.137395)
				(398.795,0.123124)
				(602.237,0.109662)
				(801.247,0.0942846)
				(1003.77,0.0886489)
				(1203.26,0.0815615)
				(1396.06,0.0786008)
				(1601.57,0.0733852)
				(1818.47,0.0709058)
				(1991.77,0.0683851)
				(2412.05,0.0646122)
				(2809.62,0.062687)
				(3201.4,0.0578274)
				(3613.59,0.0560303)
				(4004.17,0.052585)
				(4612.92,0.0499202)
				(5237.39,0.0483833)
			};
	
			% ERS ue_np.mean[0] %%%%%%%%%%%%%%%%%%%%%%%%%%%%%%%%%%%%%%%%%%%%%%%%%%%%%%%%%%%%
			\addplot[ERS] coordinates{
				(200,0.137797)
				(300,0.120505)
				(400,0.110213)
				(600,0.0975389)
				(800,0.0894522)
				(1000,0.0835158)
				(1200,0.0791256)
				(1400,0.0753014)
				(1600,0.072107)
				(1800,0.0693437)
				(2000,0.0670742)
				(2400,0.0629918)
				(2800,0.0594888)
				(3200,0.0565891)
				(3600,0.0541811)
				(4000,0.0521271)
				(4600,0.0493403)
				(5200,0.0469996)
			};
	
			% DASP ue_np.mean[0] %%%%%%%%%%%%%%%%%%%%%%%%%%%%%%%%%%%%%%%%%%%%%%%%%%%%%%%%%%%%
			\addplot[DASP] coordinates{
				(470.547,0.144782)
				(571.287,0.124497)
				(670.5,0.112782)
				(860.343,0.0991609)
				(1050.99,0.0910172)
				(1241.78,0.0849679)
				(1427.19,0.0803124)
				(1612.46,0.0770059)
				(1798.14,0.0741537)
				(1980.08,0.0715178)
				(2164.51,0.069517)
				(2524.9,0.0655272)
				(2885.04,0.0626438)
				(3244.73,0.0600775)
				(3598.39,0.0581254)
				(3947.44,0.0561428)
				(4457.26,0.0538059)
				(4965.73,0.0516938)
			};
	
			% MSS ue_np.mean[0] %%%%%%%%%%%%%%%%%%%%%%%%%%%%%%%%%%%%%%%%%%%%%%%%%%%%%%%%%%%%
			\addplot[MSS] coordinates{
				(213.423,0.168097)
				(330.475,0.141298)
				(434.73,0.130465)
				(696.625,0.108083)
				(947.095,0.0979161)
				(1166.71,0.0920277)
				(1428.43,0.0863505)
				(1631.67,0.0831407)
				(1873.19,0.0791205)
				(2161.03,0.0764407)
				(2548.86,0.0726916)
				(3080.65,0.0686651)
				(3446.28,0.0661954)
				(4062.02,0.0630131)
				(4593.88,0.0610524)
				(5065.06,0.0588967)
				(5900.9,0.0560133)
				(6661.89,0.0538693)
			};
	
			% ETPS ue_np.mean[0] %%%%%%%%%%%%%%%%%%%%%%%%%%%%%%%%%%%%%%%%%%%%%%%%%%%%%%%%%%%%
			\addplot[ETPS] coordinates{
				(220.122,0.146412)
				(325.103,0.124843)
				(415.103,0.113518)
				(635.708,0.0975135)
				(841.638,0.0887203)
				(1022.26,0.082339)
				(1227.44,0.0776748)
				(1388.07,0.0744082)
				(1580.77,0.0711228)
				(1797.57,0.0681865)
				(2099.17,0.0646774)
				(2500.64,0.0608641)
				(2770.58,0.0587573)
				(3226.35,0.0556289)
				(3613.57,0.0532569)
				(3961.65,0.0513892)
				(4557.15,0.0489743)
				(5106.62,0.0468722)
			};

			\end{axis}
	\end{tikzpicture}
\end{subfigure}%
\begin{subfigure}[b]{\fullthreeone\textwidth}\phantomsubcaption\label{subfig:appendix-experiments-sunrgbd-ev.mean[0]}
	%%%%%%%%%%%%%%%%%%%%%%%%%%%%%%%%%%%%%%%%%%%%%%%%%%%%%%%%%%%%
	% ev.mean[0]
	%%%%%%%%%%%%%%%%%%%%%%%%%%%%%%%%%%%%%%%%%%%%%%%%%%%%%%%%%%%%
	\begin{tikzpicture}
		\begin{axis}[EQSUNRGBDEV,xmode=log]		
	
			% SLIC3D ev.mean[0] %%%%%%%%%%%%%%%%%%%%%%%%%%%%%%%%%%%%%%%%%%%%%%%%%%%%%%%%%%%%
			\addplot[SLIC3D] coordinates{
				(183.423,0.889986)
				(278.67,0.9084)
				(364.19,0.918244)
				(573.713,0.930141)
				(765.372,0.938868)
				(943.992,0.945026)
				(1139.02,0.948693)
				(1296.29,0.95146)
				(1484.66,0.95416)
				(1696.09,0.95673)
				(1993.85,0.959176)
				(2379.4,0.961709)
				(2638.21,0.963233)
				(3112.2,0.965613)
				(3533.8,0.967033)
				(3850.01,0.968131)
				(4402.95,0.970133)
				(4982.22,0.971566)
			};
	
			% CCS ev.mean[0] %%%%%%%%%%%%%%%%%%%%%%%%%%%%%%%%%%%%%%%%%%%%%%%%%%%%%%%%%%%%
			\addplot[CCS] coordinates{
				(195.753,0.899118)
				(297.047,0.92021)
				(384.925,0.932742)
				(613.1,0.949383)
				(806.407,0.956495)
				(983.122,0.960679)
				(1186.45,0.964505)
				(1344.63,0.966592)
				(1535.62,0.968477)
				(1750.9,0.970517)
				(2053.87,0.972512)
				(2449.29,0.974799)
				(2720.31,0.975971)
				(3199.35,0.977658)
				(3647.26,0.978867)
				(3966.9,0.979576)
				(4514.55,0.980857)
				(5088.07,0.981753)
			};
	
			% SEEDS ev.mean[0] %%%%%%%%%%%%%%%%%%%%%%%%%%%%%%%%%%%%%%%%%%%%%%%%%%%%%%%%%%%%
			\addplot[SEEDS] coordinates{
				(242.32,0.942084)
				(347.67,0.949986)
				(448.677,0.956068)
				(664.97,0.961482)
				(862.708,0.967082)
				(1070.34,0.969364)
				(1272.95,0.971981)
				(1468.56,0.973257)
				(1674.56,0.975797)
				(1891.19,0.974864)
				(2065.22,0.978241)
				(2491.02,0.977924)
				(2891.48,0.977779)
				(3275.06,0.981283)
				(3683.12,0.980435)
				(4069.33,0.984366)
				(4676.44,0.984531)
				(5291.02,0.983731)
			};
	
			% SLIC ev.mean[0] %%%%%%%%%%%%%%%%%%%%%%%%%%%%%%%%%%%%%%%%%%%%%%%%%%%%%%%%%%%%
			\addplot[SLIC] coordinates{
				(185.398,0.889124)
				(282.175,0.907009)
				(367.93,0.918034)
				(579.227,0.929306)
				(773.253,0.938356)
				(953.495,0.944045)
				(1150.35,0.947926)
				(1308.64,0.950573)
				(1499.73,0.953329)
				(1711.93,0.955715)
				(2012.34,0.958413)
				(2399.39,0.96117)
				(2660.45,0.962586)
				(3138.66,0.965126)
				(3565.77,0.96667)
				(3884.39,0.96784)
				(4439.6,0.969716)
				(5020.27,0.971079)
			};
	
			% CW ev.mean[0] %%%%%%%%%%%%%%%%%%%%%%%%%%%%%%%%%%%%%%%%%%%%%%%%%%%%%%%%%%%%
			\addplot[CW] coordinates{
				(198.477,0.882344)
				(300.233,0.90071)
				(408.348,0.912929)
				(613.44,0.925638)
				(818.883,0.932859)
				(1012.03,0.938291)
				(1244.05,0.942847)
				(1427.65,0.944661)
				(1623.65,0.946785)
				(1876.75,0.94929)
				(2098.07,0.951212)
				(2481.19,0.953967)
				(2859.2,0.955762)
				(3434.78,0.958058)
				(3867.11,0.959368)
				(4348.77,0.960713)
				(4833.84,0.961711)
				(5520.31,0.963233)
			};
	
			% TP ev.mean[0] %%%%%%%%%%%%%%%%%%%%%%%%%%%%%%%%%%%%%%%%%%%%%%%%%%%%%%%%%%%%
			\addplot[TP] coordinates{
				(306.073,0.912614)
				(430.46,0.916346)
				(526.722,0.923462)
				(783.747,0.927615)
				(1003.49,0.934238)
				(1410.02,0.942528)
				(1771.18,0.946713)
				(2287.71,0.951018)
				(2650.15,0.952947)
				(2914.68,0.954505)
				(3348.8,0.956472)
				(3709.81,0.957556)
			};
	
			% POISE ev.mean[0] %%%%%%%%%%%%%%%%%%%%%%%%%%%%%%%%%%%%%%%%%%%%%%%%%%%%%%%%%%%%
			\addplot[POISE] coordinates{
				(204.91,0.895543)
				(306.737,0.909523)
				(408.505,0.917052)
				(611.875,0.924323)
				(815.405,0.927705)
				(1018.5,0.929799)
				(1220.88,0.931352)
				(1421.58,0.932385)
				(1618.95,0.933299)
				(1813.59,0.933928)
				(2002.57,0.934404)
				(2356.21,0.93515)
				(2669.72,0.935696)
				(2932.32,0.93595)
				(3135.23,0.936167)
				(3284.75,0.936268)
				(3421.31,0.936371)
				(3494.31,0.936409)
			};
			
			% FH ev.mean[0] %%%%%%%%%%%%%%%%%%%%%%%%%%%%%%%%%%%%%%%%%%%%%%%%%%%%%%%%%%%%
			\addplot[FH] coordinates{
				(841.75,0.915251)
				(1061.4,0.927384)
				(1277.38,0.936284)
				(1349.58,0.938902)
				(1858.42,0.955215)
				(2440.8,0.959082)
				(2688.55,0.963862)
				(2806.41,0.963973)
				(3771.37,0.96952)
				(4117.87,0.971676)
				(4886.07,0.974411)
				(5426.45,0.975732)
				(6202.02,0.979637)
			};
	
			% EAMS ev.mean[0] %%%%%%%%%%%%%%%%%%%%%%%%%%%%%%%%%%%%%%%%%%%%%%%%%%%%%%%%%%%%
			\addplot[EAMS] coordinates{
				(586.677,0.957041)
				(632.232,0.958523)
				(687.737,0.960226)
				(836.78,0.963839)
				(1027.26,0.967444)
				(1478.62,0.97229)
				(3028.83,0.981659)
				(3155.03,0.981982)
				(3284.1,0.982303)
				(3576.09,0.982982)
				(3934.32,0.983696)
				(4483.62,0.984623)
				(5080.25,0.985427)
				(5840.7,0.986313)
				(6948.21,0.987359)
				(9692.98,0.989331)
				(17957,0.992454)
			};
	
			% CRS ev.mean[0] %%%%%%%%%%%%%%%%%%%%%%%%%%%%%%%%%%%%%%%%%%%%%%%%%%%%%%%%%%%%
			\addplot[CRS] coordinates{
				(283.467,0.902029)
				(399.272,0.916519)
				(530.128,0.927038)
				(798.34,0.938515)
				(1037.19,0.944188)
				(1259.96,0.948731)
				(1495.45,0.951845)
				(1694.58,0.95324)
				(1947.02,0.956303)
				(2248.88,0.958958)
				(2455.29,0.959621)
				(2859.61,0.962593)
				(3297.01,0.964497)
				(3904.05,0.966878)
				(4297.22,0.968328)
				(4803.79,0.969477)
				(5386.35,0.970975)
				(6068.33,0.972477)
			};
	
			% RESEEDS ev.mean[0] %%%%%%%%%%%%%%%%%%%%%%%%%%%%%%%%%%%%%%%%%%%%%%%%%%%%%%%%%%%%
			\addplot[RESEEDS] coordinates{
				(199.387,0.945655)
				(300.128,0.959934)
				(399.04,0.964931)
				(602.485,0.967107)
				(801.48,0.97248)
				(1004.03,0.973515)
				(1203.35,0.977176)
				(1396.1,0.977994)
				(1601.84,0.979655)
				(1818.54,0.979755)
				(1992.37,0.980812)
				(2412.1,0.982056)
				(2809.68,0.982147)
				(3201.43,0.984396)
				(3613.63,0.984353)
				(4004.66,0.98605)
				(4612.95,0.986837)
				(5237.44,0.986845)
			};
	
			% ERGC ev.mean[0] %%%%%%%%%%%%%%%%%%%%%%%%%%%%%%%%%%%%%%%%%%%%%%%%%%%%%%%%%%%%
			\addplot[ERGC] coordinates{
				(196,0.912035)
				(296.905,0.928367)
				(401.35,0.938623)
				(600,0.94951)
				(818.51,0.9568)
				(982.7,0.960331)
				(1206.2,0.964101)
				(1373.27,0.966403)
				(1564.46,0.968468)
				(1821.57,0.970386)
				(2003.95,0.971745)
				(2374.46,0.973984)
				(2728.29,0.975898)
				(3264.02,0.97791)
				(3663.22,0.97903)
				(4101.08,0.980015)
				(4545.42,0.98097)
				(5170.11,0.982163)
			};
	
			% PF ev.mean[0] %%%%%%%%%%%%%%%%%%%%%%%%%%%%%%%%%%%%%%%%%%%%%%%%%%%%%%%%%%%%
			\addplot[PF] coordinates{
				(297.922,0.749092)
				(446.965,0.782095)
				(576.292,0.799818)
				(890.323,0.825919)
				(1179.05,0.842503)
				(1446.47,0.853817)
				(1709.54,0.862155)
				(1918.02,0.867406)
				(2158.64,0.873104)
				(2441.51,0.878485)
				(2816.51,0.884405)
				(3392.35,0.891832)
				(4335.67,0.900223)
				(6371.69,0.912762)
			};
	
			% TPS ev.mean[0] %%%%%%%%%%%%%%%%%%%%%%%%%%%%%%%%%%%%%%%%%%%%%%%%%%%%%%%%%%%%
			\addplot[TPS] coordinates{
				(232.273,0.846776)
				(332.385,0.868382)
				(444.122,0.883165)
				(657.893,0.901247)
				(889.677,0.913603)
				(1069.44,0.919852)
				(1277.73,0.925714)
				(1526.67,0.930758)
				(1751.96,0.934559)
				(1970.22,0.937654)
				(2152.95,0.940081)
				(2501.68,0.943458)
				(3072.82,0.947805)
				(3436.22,0.950458)
				(3853.6,0.952433)
				(4314.69,0.954857)
				(4837.54,0.956857)
				(5569.37,0.958812)
			};
	
			% VC ev.mean[0] %%%%%%%%%%%%%%%%%%%%%%%%%%%%%%%%%%%%%%%%%%%%%%%%%%%%%%%%%%%%
			\addplot[VC] coordinates{
				(264.483,0.899636)
				(419.552,0.929012)
				(553.5,0.941861)
				(681.035,0.949774)
				(922.085,0.959104)
				(1156.54,0.964316)
				(1382.11,0.967681)
				(1597.11,0.970285)
				(1816.61,0.972022)
				(2034.93,0.973358)
				(2248.42,0.974609)
				(2454.14,0.975618)
				(2869.61,0.977309)
				(3276.23,0.978515)
				(3673.47,0.97955)
				(4059.96,0.980473)
				(4435.44,0.98127)
				(4955.44,0.982245)
				(5425.75,0.982977)
			};
	
			% PB ev.mean[0] %%%%%%%%%%%%%%%%%%%%%%%%%%%%%%%%%%%%%%%%%%%%%%%%%%%%%%%%%%%%
			\addplot[PB] coordinates{
				(298.603,0.812203)
				(397.12,0.841803)
				(479.885,0.859807)
				(680.155,0.882054)
				(870.438,0.898533)
				(1047.06,0.906947)
				(1237.46,0.912953)
				(1390.23,0.920366)
				(1569.82,0.924566)
				(1783.13,0.9282)
				(2072.57,0.932544)
				(2461.61,0.939592)
				(2716.23,0.943484)
				(3159.05,0.947982)
				(3547.41,0.950269)
				(3882.6,0.952565)
				(4461.39,0.95626)
				(4997.73,0.958861)
			};
	
			% VCCS ev.mean[0] %%%%%%%%%%%%%%%%%%%%%%%%%%%%%%%%%%%%%%%%%%%%%%%%%%%%%%%%%%%%
			\addplot[VCCS] coordinates{
				(600.688,0.732202)
				(850.68,0.772888)
				(947.458,0.784583)
				(1254.7,0.80088)
				(1340.03,0.807017)
				(1426.25,0.814706)
				(1495.03,0.818361)
				(1622.35,0.826568)
				(1724.65,0.833559)
				(1876.49,0.841913)
				(2020.57,0.847095)
				(2184.95,0.855823)
				(2394.23,0.865158)
				(2551.26,0.868651)
				(2841.62,0.877552)
				(3185.59,0.889366)
				(3477.67,0.894901)
				(3842.61,0.901377)
				(4566.27,0.919628)
				(5200.63,0.93091)
				(5463.7,0.927898)
			};
	
			% PRESLIC ev.mean[0] %%%%%%%%%%%%%%%%%%%%%%%%%%%%%%%%%%%%%%%%%%%%%%%%%%%%%%%%%%%%
			\addplot[PRESLIC] coordinates{
				(189.206,0.874021)
				(378.882,0.895221)
				(573.018,0.920425)
				(752.995,0.927154)
				(921.787,0.930926)
				(1110.43,0.935448)
				(1262.35,0.937996)
				(1445.35,0.940928)
				(1659.42,0.943533)
				(1954.02,0.945866)
				(2339.45,0.949746)
				(2601.8,0.952328)
				(3082.42,0.955274)
				(3505.78,0.95748)
				(3822.55,0.958594)
				(4402.4,0.959953)
				(5008.01,0.961976)
			};
	
			% W ev.mean[0] %%%%%%%%%%%%%%%%%%%%%%%%%%%%%%%%%%%%%%%%%%%%%%%%%%%%%%%%%%%%
			\addplot[W] coordinates{
				(199.607,0.870469)
				(302.642,0.891552)
				(408.6,0.906483)
				(608.068,0.920345)
				(813.78,0.929052)
				(999.44,0.934414)
				(1190.19,0.938779)
				(1373.25,0.942103)
				(1588.89,0.945609)
				(1820.39,0.947991)
				(2138.11,0.950307)
				(2539.07,0.952851)
				(2810.66,0.954731)
				(3313.99,0.956914)
				(3783.59,0.958712)
				(4142.1,0.959915)
				(4770.12,0.961482)
				(5365.69,0.962544)
			};
	
			% LSC ev.mean[0] %%%%%%%%%%%%%%%%%%%%%%%%%%%%%%%%%%%%%%%%%%%%%%%%%%%%%%%%%%%%
			\addplot[LSC] coordinates{
				(412.42,0.94816)
				(566.403,0.954252)
				(737.927,0.958842)
				(1100.36,0.963805)
				(1387.69,0.966446)
				(1658.32,0.96778)
				(1915.58,0.969143)
				(2087.95,0.96995)
				(2318.23,0.970491)
				(2558.32,0.971217)
				(2685.31,0.971632)
				(2950.94,0.972347)
				(3247.86,0.97278)
				(3645.17,0.973419)
				(3930.7,0.9739)
				(4322.76,0.973906)
				(4740.34,0.973212)
				(5250.39,0.972771)
			};
	
			% WP ev.mean[0] %%%%%%%%%%%%%%%%%%%%%%%%%%%%%%%%%%%%%%%%%%%%%%%%%%%%%%%%%%%%
			\addplot[WP] coordinates{
				(211.102,0.900379)
				(323.783,0.918615)
				(403.883,0.927047)
				(625.86,0.939717)
				(823.02,0.946341)
				(1022.26,0.951045)
				(1222.72,0.954321)
				(1385.35,0.956629)
				(1559.27,0.958651)
				(1797.57,0.960734)
				(2060.75,0.963173)
				(2438.15,0.965332)
				(2702.28,0.966643)
				(3226.35,0.968913)
				(3920.65,0.971118)
				(3980.05,0.971242)
				(4987.74,0.973293)
				(5069.48,0.973421)
			};
	
			% QS ev.mean[0] %%%%%%%%%%%%%%%%%%%%%%%%%%%%%%%%%%%%%%%%%%%%%%%%%%%%%%%%%%%%
			\addplot[QS] coordinates{
				(223.997,0.715941)
				(317.098,0.812036)
				(444.57,0.871653)
				(626.915,0.910308)
				(830.568,0.931722)
				(1003.65,0.941901)
				(1117.84,0.94676)
				(1401.9,0.955842)
				(1593.35,0.96006)
				(1834.32,0.964191)
				(2169.17,0.968453)
				(2619.48,0.972691)
				(3266.76,0.976769)
				(4204.98,0.980864)
				(5572.49,0.984777)
			};
	
			% VLSLIC ev.mean[0] %%%%%%%%%%%%%%%%%%%%%%%%%%%%%%%%%%%%%%%%%%%%%%%%%%%%%%%%%%%%
			\addplot[VLSLIC] coordinates{
				(744.14,0.955101)
				(900.468,0.958588)
				(1017.84,0.960324)
				(1278.77,0.962674)
				(1492.48,0.963745)
				(1671.62,0.964593)
				(1851.27,0.964999)
				(1992.46,0.965129)
				(2165.14,0.965461)
				(2318.73,0.965492)
				(2546.17,0.965678)
				(2829.93,0.965029)
				(3019.05,0.964769)
				(3328.6,0.964192)
				(3594.5,0.964225)
				(3778.65,0.9636)
				(4132.48,0.961944)
				(4422.49,0.960921)
			};
	
			% CIS ev.mean[0] %%%%%%%%%%%%%%%%%%%%%%%%%%%%%%%%%%%%%%%%%%%%%%%%%%%%%%%%%%%%
			\addplot[CIS] coordinates{
				(320.195,0.907164)
				(395.365,0.917598)
				(467.525,0.925516)
				(601.885,0.934592)
				(736.583,0.940729)
				(849.645,0.940942)
				(981.057,0.945714)
				(1091.31,0.95003)
				(1216.94,0.951845)
				(1380.64,0.953631)
				(1601.9,0.956767)
				(1883.81,0.960191)
				(2101.28,0.961334)
				(2894.19,0.963736)
				(3109.28,0.964282)
				(3543.38,0.968927)
				(4114.64,0.971746)
			};
	
			% RESEEDS3D ev.mean[0] %%%%%%%%%%%%%%%%%%%%%%%%%%%%%%%%%%%%%%%%%%%%%%%%%%%%%%%%%%%%
			\addplot[RESEEDS3D] coordinates{
				(199.258,0.926915)
				(300.057,0.946768)
				(398.795,0.954535)
				(602.237,0.961512)
				(801.247,0.967894)
				(1003.77,0.970168)
				(1203.26,0.974164)
				(1396.06,0.975253)
				(1601.57,0.977216)
				(1818.47,0.977802)
				(1991.77,0.978961)
				(2412.05,0.980529)
				(2809.62,0.980903)
				(3201.4,0.983154)
				(3613.59,0.983417)
				(4004.17,0.984997)
				(4612.92,0.985894)
				(5237.39,0.98612)
			};
	
			% ERS ev.mean[0] %%%%%%%%%%%%%%%%%%%%%%%%%%%%%%%%%%%%%%%%%%%%%%%%%%%%%%%%%%%%
			\addplot[ERS] coordinates{
				(200,0.883201)
				(300,0.897053)
				(400,0.90485)
				(600,0.914545)
				(800,0.920643)
				(1000,0.924906)
				(1200,0.928175)
				(1400,0.930997)
				(1600,0.933398)
				(1800,0.935516)
				(2000,0.937323)
				(2400,0.940729)
				(2800,0.943476)
				(3200,0.945929)
				(3600,0.948181)
				(4000,0.950175)
				(4600,0.952842)
				(5200,0.955069)
			};
	
			% DASP ev.mean[0] %%%%%%%%%%%%%%%%%%%%%%%%%%%%%%%%%%%%%%%%%%%%%%%%%%%%%%%%%%%%
			\addplot[DASP] coordinates{
				(470.547,0.9224)
				(571.287,0.936153)
				(670.5,0.943702)
				(860.343,0.95236)
				(1050.99,0.957461)
				(1241.78,0.961293)
				(1427.19,0.963912)
				(1612.46,0.965817)
				(1798.14,0.967468)
				(1980.08,0.968901)
				(2164.51,0.970012)
				(2524.9,0.972044)
				(2885.04,0.973603)
				(3244.73,0.97484)
				(3598.39,0.975791)
				(3947.44,0.97675)
				(4457.26,0.977716)
				(4965.73,0.978634)
			};
	
			% MSS ev.mean[0] %%%%%%%%%%%%%%%%%%%%%%%%%%%%%%%%%%%%%%%%%%%%%%%%%%%%%%%%%%%%
			\addplot[MSS] coordinates{
				(213.423,0.883086)
				(330.475,0.905775)
				(434.73,0.914518)
				(696.625,0.932053)
				(947.095,0.939624)
				(1166.71,0.943977)
				(1428.43,0.947597)
				(1631.67,0.949804)
				(1873.19,0.951996)
				(2161.03,0.953657)
				(2548.86,0.955889)
				(3080.65,0.957925)
				(3446.28,0.958974)
				(4062.02,0.960507)
				(4593.88,0.961334)
				(5065.06,0.962174)
				(5900.9,0.963572)
				(6661.89,0.964331)
			};
	
			% ETPS ev.mean[0] %%%%%%%%%%%%%%%%%%%%%%%%%%%%%%%%%%%%%%%%%%%%%%%%%%%%%%%%%%%%
			\addplot[ETPS] coordinates{
				(220.122,0.962295)
				(325.103,0.968567)
				(415.103,0.971609)
				(635.708,0.976169)
				(841.638,0.978709)
				(1022.26,0.980357)
				(1227.44,0.981506)
				(1388.07,0.982328)
				(1580.77,0.983235)
				(1797.57,0.984056)
				(2099.17,0.984873)
				(2500.64,0.985889)
				(2770.58,0.986508)
				(3226.35,0.987192)
				(3613.57,0.987871)
				(3961.65,0.988302)
				(4557.15,0.988859)
				(5106.62,0.989341)
			};

			\end{axis}
	\end{tikzpicture}
\end{subfigure}
	\begin{subfigure}[b]{\fullthreeone\textwidth}\phantomsubcaption\label{subfig:appendix-experiments-fash-rec.mean[0]}
	%%%%%%%%%%%%%%%%%%%%%%%%%%%%%%%%%%%%%%%%%%%%%%%%%%%%%%%%%%%%
	% rec.mean[0]
	%%%%%%%%%%%%%%%%%%%%%%%%%%%%%%%%%%%%%%%%%%%%%%%%%%%%%%%%%%%%
	\begin{tikzpicture}
		\begin{axis}[EQFashRec,xmode=log]

			% CCS rec.mean[0] %%%%%%%%%%%%%%%%%%%%%%%%%%%%%%%%%%%%%%%%%%%%%%%%%%%%%%%%%%%%
			\addplot[CCS] coordinates{
				(205.099,0.85355)
				(323.803,0.90682)
				(442.991,0.934521)
				(663.605,0.959466)
				(888.464,0.973971)
				(1145.83,0.982251)
				(1290.6,0.985429)
				(1514.56,0.989193)
				(1804.75,0.991905)
				(1804.75,0.991905)
				(2118.27,0.994419)
				(2607.61,0.996407)
				(3136.79,0.997892)
				(3136.79,0.997892)
				(4030.41,0.998933)
				(4030.41,0.998933)
				(5170.28,0.999667)
				(5170.28,0.999667)
			};

			% SEEDS rec.mean[0] %%%%%%%%%%%%%%%%%%%%%%%%%%%%%%%%%%%%%%%%%%%%%%%%%%%%%%%%%%%%
			\addplot[SEEDS] coordinates{
				(260.22,0.970545)
				(363.724,0.981078)
				(472.406,0.98641)
				(675.082,0.992123)
				(875.184,0.994411)
				(1078.34,0.996107)
				(1272.02,0.997026)
				(1466.01,0.997581)
				(1686.85,0.998467)
				(1886.09,0.99886)
				(2084.5,0.999153)
				(2465.45,0.999041)
				(2862.45,0.999516)
				(3376.59,0.999696)
				(3773,0.999781)
				(4056.65,0.99978)
				(5005.27,0.999892)
				(5061.29,0.999909)
			};

			% SLIC rec.mean[0] %%%%%%%%%%%%%%%%%%%%%%%%%%%%%%%%%%%%%%%%%%%%%%%%%%%%%%%%%%%%
			\addplot[SLIC] coordinates{
				(183.616,0.907833)
				(287.775,0.933933)
				(393.184,0.950086)
				(587.981,0.964532)
				(796.192,0.974008)
				(1031.22,0.980186)
				(1172.04,0.9817)
				(1377.87,0.984048)
				(1648.17,0.985926)
				(1650.3,0.984744)
				(1945.79,0.988209)
				(2399.7,0.992025)
				(2906.47,0.994518)
				(3748.69,0.997689)
				(3748.69,0.997689)
				(4847.82,0.999338)
				(4847.82,0.999338)
			};

			% RW rec.mean[0] %%%%%%%%%%%%%%%%%%%%%%%%%%%%%%%%%%%%%%%%%%%%%%%%%%%%%%%%%%%%
			\addplot[RW] coordinates{
				(406.784,0.855619)
				(1348.01,0.956106)
				(4048.94,0.996038)
				(6058.17,0.999462)
			};

			% CW rec.mean[0] %%%%%%%%%%%%%%%%%%%%%%%%%%%%%%%%%%%%%%%%%%%%%%%%%%%%%%%%%%%%
			\addplot[CW] coordinates{
				(199.177,0.846128)
				(293.937,0.883302)
				(409.708,0.906853)
				(591.276,0.933372)
				(839.903,0.95283)
				(1030.22,0.961909)
				(1205.46,0.970927)
				(1396.52,0.97597)
				(1668.84,0.982387)
				(1796.95,0.983556)
				(2121.13,0.988643)
				(2647.64,0.99319)
				(2861.41,0.994349)
				(3267.16,0.995999)
				(3643.52,0.99715)
				(4081.58,0.998024)
				(4756.03,0.999035)
				(5360.93,0.999465)
			};

			% TP rec.mean[0] %%%%%%%%%%%%%%%%%%%%%%%%%%%%%%%%%%%%%%%%%%%%%%%%%%%%%%%%%%%%
			\addplot[TP] coordinates{
				(280.927,0.840122)
				(452.384,0.911513)
				(555.011,0.925071)
				(738.32,0.94207)
				%(1028.6,0.938922)
				(1275.51,0.954787)
				(1441.55,0.961148)
				(1930.85,0.975407)
				(2280.38,0.98155)
				(3080.77,0.986027)
				(3683.73,0.989913)
				(4359.43,0.994609)
			};

			% POISE rec.mean[0] %%%%%%%%%%%%%%%%%%%%%%%%%%%%%%%%%%%%%%%%%%%%%%%%%%%%%%%%%%%%
			\addplot[POISE] coordinates{
				(204.877,0.952193)
				(306.665,0.969354)
				(408.393,0.977673)
				(611.996,0.985971)
				(815.078,0.990074)
				(1016.78,0.992346)
				(1215.99,0.99372)
				(1408.03,0.994555)
				(1590.31,0.995161)
				(1759.67,0.99554)
				(1917.33,0.995803)
				(2193.21,0.99609)
				(2416.91,0.996206)
				(2585.33,0.996269)
				(2709.65,0.99631)
				(2795.14,0.996334)
				(2857.91,0.996353)
				(2872.78,0.996355)
			};

			% FH rec.mean[0] %%%%%%%%%%%%%%%%%%%%%%%%%%%%%%%%%%%%%%%%%%%%%%%%%%%%%%%%%%%%
			\addplot[FH] coordinates{
				(820.24,0.908862)
				(1017.45,0.926451)
				(1255.95,0.949133)
				(1422.06,0.952666)
				(1626.81,0.965357)
				(2095.7,0.971862)
				(2264.94,0.980056)
				(3618.47,0.992237)
				(3968.9,0.991488)
				(4191.68,0.99236)
				(4226.29,0.9927)
				(5065.34,0.993639)
				(6378.82,0.996746)
			};

			% EAMS rec.mean[0] %%%%%%%%%%%%%%%%%%%%%%%%%%%%%%%%%%%%%%%%%%%%%%%%%%%%%%%%%%%%
			\addplot[EAMS] coordinates{
				(334.743,0.954785)
				(356.009,0.957613)
				(382.151,0.960487)
				(452.27,0.967071)
				(532.631,0.974812)
				(734.402,0.982711)
				(1235.54,0.991578)
				(1302.6,0.991945)
				(1404.02,0.992457)
				(1491.76,0.992922)
				(1591.75,0.993313)
				(1888.34,0.994033)
				(2325.36,0.994579)
				(2903.95,0.994947)
				(4169.4,0.995267)
			};

			% CRS rec.mean[0] %%%%%%%%%%%%%%%%%%%%%%%%%%%%%%%%%%%%%%%%%%%%%%%%%%%%%%%%%%%%
			\addplot[CRS] coordinates{
				(273.927,0.915767)
				(425.171,0.947137)
				(524.505,0.960509)
				(773.732,0.975922)
				(1066.8,0.984769)
				(1248.1,0.988055)
				(1520.54,0.991153)
				(1728.48,0.992666)
				(1966.6,0.995153)
				(2098.87,0.995839)
				(2558.84,0.997254)
				(3022.16,0.998262)
				(3303.81,0.998677)
				(3808.31,0.999194)
				(4129.2,0.999398)
				(4584.01,0.999603)
				(5320.33,0.999818)
				(5908.32,0.999892)
			};

			% SEAW rec.mean[0] %%%%%%%%%%%%%%%%%%%%%%%%%%%%%%%%%%%%%%%%%%%%%%%%%%%%%%%%%%%%
			\addplot[SEAW] coordinates{
				(187.497,0.682196)
				(425.186,0.82432)
				(1234.06,0.937215)
				(4222.63,0.995033)
			};

			% RESEEDS rec.mean[0] %%%%%%%%%%%%%%%%%%%%%%%%%%%%%%%%%%%%%%%%%%%%%%%%%%%%%%%%%%%%
			\addplot[RESEEDS] coordinates{
				(200.773,0.977733)
				(301.121,0.986639)
				(400.423,0.990966)
				(600.713,0.994914)
				(800.771,0.996539)
				(1000.07,0.99744)
				(1200.08,0.997886)
				(1400.11,0.998405)
				(1600.07,0.998888)
				(1800.06,0.998959)
				(2000.06,0.999289)
				(2400.12,0.999379)
				(2800.13,0.999565)
				(3300.09,0.999721)
				(3700.18,0.999819)
				(4000.16,0.999835)
				(4950.18,0.999925)
				(5000.16,0.999935)
			};

			% ERGC rec.mean[0] %%%%%%%%%%%%%%%%%%%%%%%%%%%%%%%%%%%%%%%%%%%%%%%%%%%%%%%%%%%%
			\addplot[ERGC] coordinates{
				(196,0.885806)
				(289,0.923998)
				(400,0.945192)
				(600,0.965268)
				(841,0.977469)
				(992,0.982737)
				(1155,0.987044)
				(1332,0.990067)
				(1600,0.992799)
				(1720,0.993649)
				(2024,0.99565)
				(2500,0.997313)
				(2750,0.997817)
				(3135,0.998636)
				(3420,0.998968)
				(3819,0.999255)
				(4489,0.999646)
				(5025,0.999751)
			};

			% PF rec.mean[0] %%%%%%%%%%%%%%%%%%%%%%%%%%%%%%%%%%%%%%%%%%%%%%%%%%%%%%%%%%%%
			\addplot[PF] coordinates{
				(287.639,0.565267)
				(447.933,0.627232)
				(606.404,0.670503)
				(875.881,0.722089)
				(1211.94,0.763493)
				(1542.84,0.797247)
				(1746.93,0.814693)
				(2019.1,0.833265)
				(2334.11,0.850653)
				(2737.01,0.869)
				(3309.32,0.905469)
				(4281.63,0.929821)
				(5790.62,0.954359)
			};

			% TPS rec.mean[0] %%%%%%%%%%%%%%%%%%%%%%%%%%%%%%%%%%%%%%%%%%%%%%%%%%%%%%%%%%%%
			\addplot[TPS] coordinates{
				(240.968,0.763736)
				(345,0.810473)
				(425.052,0.834644)
				(658.648,0.880865)
				(885.365,0.909666)
				(1100.77,0.925933)
				(1378.51,0.943206)
				(1575.89,0.95316)
				(1661.41,0.95665)
				(1992.65,0.966236)
				(2184.11,0.971281)
				(2554.44,0.97915)
				(3222.27,0.987158)
				(3524.76,0.99022)
				(4065.15,0.993305)
				(4552.29,0.995057)
				(5048.88,0.997493)
				(6022.01,0.998628)
			};

			% NC rec.mean[0] %%%%%%%%%%%%%%%%%%%%%%%%%%%%%%%%%%%%%%%%%%%%%%%%%%%%%%%%%%%%
			\addplot[NC] coordinates{
				(430.693,0.932057)
				(1041.48,0.953414)
				(2965.42,0.991496)
				(4028.44,0.997335)
			};

			% VC rec.mean[0] %%%%%%%%%%%%%%%%%%%%%%%%%%%%%%%%%%%%%%%%%%%%%%%%%%%%%%%%%%%%
			\addplot[VC] coordinates{
				(229.812,0.737424)
				(383.836,0.82541)
				(584.402,0.897168)
				(868.048,0.944215)
				(1084.39,0.96371)
				(1256.67,0.973556)
				(1572.27,0.98265)
				(1854.83,0.98749)
				(2121.29,0.990191)
				(2367.94,0.991944)
				(2613.23,0.993321)
				(2851.13,0.99409)
				(3082.91,0.994912)
				(3303.61,0.995511)
				(3743.63,0.996366)
				(4176.97,0.9969)
				(4574.03,0.997378)
				(4966.49,0.997829)
				(5304.49,0.99818)
				(5782.63,0.998411)
			};

			% PB rec.mean[0] %%%%%%%%%%%%%%%%%%%%%%%%%%%%%%%%%%%%%%%%%%%%%%%%%%%%%%%%%%%%
			\addplot[PB] coordinates{
				(346.801,0.796961)
				(446.687,0.838124)
				(537.451,0.86293)
				(691.737,0.899523)
				(923.156,0.92977)
				(1132.22,0.947836)
				(1286.56,0.956995)
				(1480.41,0.966374)
				(1708.32,0.973383)
				(1708.32,0.973383)
				(2029.09,0.981663)
				(2389.29,0.987864)
				(2974.92,0.993396)
				(2974.92,0.993396)
				(3693.23,0.997197)
				(3693.23,0.997197)
				(4884.82,0.998837)
				(4884.82,0.998837)
			};

			% PRESLIC rec.mean[0] %%%%%%%%%%%%%%%%%%%%%%%%%%%%%%%%%%%%%%%%%%%%%%%%%%%%%%%%%%%%
			\addplot[PRESLIC] coordinates{
				(184.222,0.864162)
				(289.592,0.90372)
				(393.639,0.924067)
				(589.462,0.945515)
				(795.786,0.961311)
				(1031.99,0.96959)
				(1171.21,0.972409)
				(1377.24,0.975846)
				(1646.02,0.979362)
				(1647.89,0.977378)
				(1943.21,0.983018)
				(2396.45,0.987903)
				(2901.84,0.992474)
				(3744.27,0.996362)
				(3744.27,0.996362)
				(4842.14,0.998925)
				(4842.14,0.998925)
			};

			% W rec.mean[0] %%%%%%%%%%%%%%%%%%%%%%%%%%%%%%%%%%%%%%%%%%%%%%%%%%%%%%%%%%%%
			\addplot[W] coordinates{
				(189.026,0.835136)
				(298.726,0.877385)
				(433.637,0.90513)
				(615.082,0.931992)
				(827.998,0.951345)
				(1116.68,0.965142)
				(1246.48,0.971139)
				(1482.3,0.977114)
				(1722.38,0.983138)
				(1722.38,0.983138)
				(2033.96,0.987757)
				(2525.92,0.992139)
				(3118.03,0.995656)
				(3118.03,0.995656)
				(4018.14,0.998028)
				(4018.14,0.998028)
				(5283.66,0.999318)
				(5283.66,0.999318)
			};

			% LSC rec.mean[0] %%%%%%%%%%%%%%%%%%%%%%%%%%%%%%%%%%%%%%%%%%%%%%%%%%%%%%%%%%%%
			\addplot[LSC] coordinates{
				(534.754,0.962471)
				(777.454,0.976391)
				(1013.2,0.982903)
				(1335.02,0.98888)
				(1675.81,0.99195)
				(1904.02,0.993267)
				(2226.93,0.994728)
				(2345.62,0.995098)
				(2566.38,0.995874)
				(2565.18,0.996009)
				(2831.62,0.996823)
				(3174.62,0.997351)
				(3270.2,0.997566)
				(3545.63,0.998193)
				(3796.98,0.998539)
				(4091.52,0.998763)
				(4607.06,0.99936)
				(5116.96,0.999519)
			};

			% WP rec.mean[0] %%%%%%%%%%%%%%%%%%%%%%%%%%%%%%%%%%%%%%%%%%%%%%%%%%%%%%%%%%%%
			\addplot[WP] coordinates{
				(204,0.83797)
				(330,0.876852)
				(425,0.899702)
				(600,0.924773)
				(864,0.944366)
				(1080,0.95726)
				(1247,0.963869)
				(1426,0.969944)
				(1700,0.976161)
				(1700,0.976163)
				(2035,0.982211)
				(2400,0.987444)
				(3015,0.992099)
				(3015,0.992087)
				(3750,0.995798)
				(3750,0.995798)
				(4902,0.998282)
				(4902,0.998282)
			};

			% QS rec.mean[0] %%%%%%%%%%%%%%%%%%%%%%%%%%%%%%%%%%%%%%%%%%%%%%%%%%%%%%%%%%%%
			\addplot[QS] coordinates{
				(310.248,0.5576)
				(452.998,0.704583)
				(692.458,0.825498)
				(836.002,0.866054)
				(1059,0.904381)
				(1219.6,0.92124)
				(1454.73,0.93874)
				(1608.81,0.946616)
				(1787.15,0.95407)
				(2009.01,0.960819)
				(2279.07,0.967266)
				(2622.46,0.973065)
				(3082.46,0.978533)
				(3703.75,0.983515)
				(4580.52,0.987975)
				(5831.35,0.991889)
			};

			% VLSLIC rec.mean[0] %%%%%%%%%%%%%%%%%%%%%%%%%%%%%%%%%%%%%%%%%%%%%%%%%%%%%%%%%%%%
			\addplot[VLSLIC] coordinates{
				(658.287,0.961536)
				(783.374,0.969772)
				(876.521,0.97503)
				(1023.6,0.980442)
				(1181.12,0.983892)
				(1320.08,0.987266)
				(1429.35,0.987695)
				(1588.18,0.989037)
				(1788.1,0.990565)
				(1783.07,0.989927)
				(2077.75,0.991433)
				(2418.96,0.99321)
				(2993.57,0.995238)
				(3003.13,0.994577)
				(3723.2,0.997137)
				(3723.2,0.997137)
				(4895.39,0.999393)
				(4895.39,0.999393)
			};

			% CIS rec.mean[0] %%%%%%%%%%%%%%%%%%%%%%%%%%%%%%%%%%%%%%%%%%%%%%%%%%%%%%%%%%%%
			\addplot[CIS] coordinates{
				(346.605,0.860293)
				(416.704,0.878212)
				(498.555,0.893224)
				(629.289,0.912974)
				(800.998,0.931286)
				(978.771,0.943031)
				(1078.75,0.947951)
				(1215.26,0.955072)
				(1393.65,0.961546)
				(1411.64,0.96194)
				(1712,0.970414)
				(2042.96,0.977047)
				(2534.86,0.983481)
				(3324.5,0.98986)
				(3324.5,0.98986)
				(5000.77,0.994281)
				(5000.77,0.994281)
			};

			% ERS rec.mean[0] %%%%%%%%%%%%%%%%%%%%%%%%%%%%%%%%%%%%%%%%%%%%%%%%%%%%%%%%%%%%
			\addplot[ERS] coordinates{
				(200,0.917615)
				(300,0.942257)
				(400,0.955818)
				(600,0.971937)
				(800,0.980813)
				(1000,0.986144)
				(1200,0.990152)
				(1400,0.992796)
				(1600,0.994594)
				(1800,0.995914)
				(2000,0.996892)
				(2400,0.998094)
				(2800,0.998881)
				(3200,0.999275)
				(3600,0.999565)
				(4000,0.999697)
				(4600,0.999861)
				(5200,0.999918)
			};

			% MSS rec.mean[0] %%%%%%%%%%%%%%%%%%%%%%%%%%%%%%%%%%%%%%%%%%%%%%%%%%%%%%%%%%%%
			\addplot[MSS] coordinates{
				(205.549,0.877088)
				(330.577,0.912898)
				(436.387,0.930103)
				(637.976,0.950026)
				(945.067,0.965712)
				(1204.04,0.974046)
				(1406.86,0.978083)
				(1661.87,0.982013)
				(1968.29,0.985543)
				(1968.29,0.985543)
				(2383.46,0.989274)
				(2847.26,0.992241)
				(3639.98,0.995049)
				(3640.21,0.994926)
				(4598.5,0.997338)
				(4598.5,0.997338)
				(6276.05,0.998949)
			};

			% ETPS rec.mean[0] %%%%%%%%%%%%%%%%%%%%%%%%%%%%%%%%%%%%%%%%%%%%%%%%%%%%%%%%%%%%
			\addplot[ETPS] coordinates{
				(216,0.978518)
				(330,0.986379)
				(425,0.990415)
				(600,0.993678)
				(864,0.99568)
				(1080,0.996958)
				(1247,0.997337)
				(1457,0.997894)
				(1700,0.998184)
				(1700,0.998184)
				(2035,0.998913)
				(2400,0.999131)
				(3015,0.999406)
				(3015,0.999406)
				(3750,0.999588)
				(3750,0.999588)
				(4988,0.999823)
				(4988,0.999823)
			};

			\end{axis}
	\end{tikzpicture}
\end{subfigure}%
\begin{subfigure}[b]{\fullthreeone\textwidth}\phantomsubcaption\label{subfig:appendix-experiments-fash-ue_np.mean[0]}
	%%%%%%%%%%%%%%%%%%%%%%%%%%%%%%%%%%%%%%%%%%%%%%%%%%%%%%%%%%%%
	% ue_np.mean[0]
	%%%%%%%%%%%%%%%%%%%%%%%%%%%%%%%%%%%%%%%%%%%%%%%%%%%%%%%%%%%%
	\begin{tikzpicture}
		\begin{axis}[EQFashUE,xmode=log]

			% CCS ue_np.mean[0] %%%%%%%%%%%%%%%%%%%%%%%%%%%%%%%%%%%%%%%%%%%%%%%%%%%%%%%%%%%%
			\addplot[CCS] coordinates{
				(205.099,0.0679732)
				(323.803,0.052016)
				(442.991,0.0439188)
				(663.605,0.0372172)
				(888.464,0.0334284)
				(1145.83,0.0309594)
				(1290.6,0.0300051)
				(1514.56,0.0286398)
				(1804.75,0.0276082)
				(1804.75,0.0276082)
				(2118.27,0.0264104)
				(2607.61,0.0251542)
				(3136.79,0.024072)
				(3136.79,0.024072)
				(4030.41,0.0227406)
				(4030.41,0.0227406)
				(5170.28,0.0212861)
				(5170.28,0.0212861)
			};

			% SEEDS ue_np.mean[0] %%%%%%%%%%%%%%%%%%%%%%%%%%%%%%%%%%%%%%%%%%%%%%%%%%%%%%%%%%%%
			\addplot[SEEDS] coordinates{
				(260.22,0.0917434)
				(363.724,0.0732111)
				(472.406,0.0752194)
				(675.082,0.0590722)
				(875.184,0.0502368)
				(1078.34,0.0453725)
				(1272.02,0.0403669)
				(1466.01,0.0388076)
				(1686.85,0.0426621)
				(1886.09,0.0400691)
				(2084.5,0.0369804)
				(2465.45,0.0308019)
				(2862.45,0.0315566)
				(3376.59,0.0284799)
				(3773,0.0271998)
				(4056.65,0.0263493)
				(5005.27,0.0232201)
				(5061.29,0.0237617)
			};

			% SLIC ue_np.mean[0] %%%%%%%%%%%%%%%%%%%%%%%%%%%%%%%%%%%%%%%%%%%%%%%%%%%%%%%%%%%%
			\addplot[SLIC] coordinates{
				(183.616,0.0641222)
				(287.775,0.0524292)
				(393.184,0.045541)
				(587.981,0.0395632)
				(796.192,0.0348006)
				(1031.22,0.0316991)
				(1172.04,0.0299941)
				(1377.87,0.028498)
				(1648.17,0.0272456)
				(1650.3,0.0271827)
				(1945.79,0.0261162)
				(2399.7,0.0249588)
				(2906.47,0.0240154)
				(3748.69,0.0228624)
				(3748.69,0.0228624)
				(4847.82,0.0216896)
				(4847.82,0.0216896)
			};

			% RW ue_np.mean[0] %%%%%%%%%%%%%%%%%%%%%%%%%%%%%%%%%%%%%%%%%%%%%%%%%%%%%%%%%%%%
			\addplot[RW] coordinates{
				(406.784,0.0706038)
				(1348.01,0.0391731)
				(4048.94,0.0263299)
				(6058.17,0.0266076)
			};

			% CW ue_np.mean[0] %%%%%%%%%%%%%%%%%%%%%%%%%%%%%%%%%%%%%%%%%%%%%%%%%%%%%%%%%%%%
			\addplot[CW] coordinates{
				(199.177,0.0901565)
				(293.937,0.074528)
				(409.708,0.0648879)
				(591.276,0.0542755)
				(839.903,0.0477438)
				(1030.22,0.0445162)
				(1205.46,0.0410447)
				(1396.52,0.0390591)
				(1668.84,0.0364586)
				(1796.95,0.0358503)
				(2121.13,0.0336509)
				(2647.64,0.0314914)
				(2861.41,0.0307747)
				(3267.16,0.029578)
				(3643.52,0.0286481)
				(4081.58,0.0276912)
				(4756.03,0.0265878)
				(5360.93,0.0255524)
			};

			% TP ue_np.mean[0] %%%%%%%%%%%%%%%%%%%%%%%%%%%%%%%%%%%%%%%%%%%%%%%%%%%%%%%%%%%%
			\addplot[TP] coordinates{
				(280.927,0.0737446)
				(452.384,0.0558335)
				(555.011,0.0488216)
				(738.32,0.0423146)
				%(1028.6,0.0694451)
				(1275.51,0.0566608)
				(1441.55,0.0527271)
				(1930.85,0.042114)
				(2280.38,0.0371407)
				(3080.77,0.0329815)
				(3683.73,0.0303659)
				(4359.43,0.0277898)
			};

			% POISE ue_np.mean[0] %%%%%%%%%%%%%%%%%%%%%%%%%%%%%%%%%%%%%%%%%%%%%%%%%%%%%%%%%%%%
			\addplot[POISE] coordinates{
				(204.877,0.056491)
				(306.665,0.0446834)
				(408.393,0.0386464)
				(611.996,0.0329634)
				(815.078,0.0297257)
				(1016.78,0.027711)
				(1215.99,0.0265818)
				(1408.03,0.0256841)
				(1590.31,0.025028)
				(1759.67,0.0246132)
				(1917.33,0.0242573)
				(2193.21,0.0238208)
				(2416.91,0.0235353)
				(2585.33,0.0233978)
				(2709.65,0.0233323)
				(2795.14,0.023298)
				(2857.91,0.0232801)
				(2872.78,0.0232763)
			};

			% FH ue_np.mean[0] %%%%%%%%%%%%%%%%%%%%%%%%%%%%%%%%%%%%%%%%%%%%%%%%%%%%%%%%%%%%
			\addplot[FH] coordinates{
				(820.24,0.0645562)
				(1017.45,0.0568712)
				(1255.95,0.0496846)
				(1422.06,0.0447739)
				(1626.81,0.0429386)
				(2095.7,0.0382475)
				(2264.94,0.0368575)
				(3618.47,0.0300979)
				(3968.9,0.0297393)
				(4191.68,0.0289077)
				(4226.29,0.0293531)
				(5065.34,0.0272481)
				(6378.82,0.0237998)
			};

			% EAMS ue_np.mean[0] %%%%%%%%%%%%%%%%%%%%%%%%%%%%%%%%%%%%%%%%%%%%%%%%%%%%%%%%%%%%
			\addplot[EAMS] coordinates{
				(334.743,0.0466931)
				(356.009,0.0454499)
				(382.151,0.0440304)
				(452.27,0.0409065)
				(532.631,0.0378365)
				(734.402,0.0339473)
				(1235.54,0.0293815)
				(1302.6,0.0289974)
				(1404.02,0.0284229)
				(1491.76,0.0279613)
				(1591.75,0.0274785)
				(1888.34,0.0263156)
				(2325.36,0.0250072)
				(2903.95,0.0236603)
				(4169.4,0.0215842)
			};

			% CRS ue_np.mean[0] %%%%%%%%%%%%%%%%%%%%%%%%%%%%%%%%%%%%%%%%%%%%%%%%%%%%%%%%%%%%
			\addplot[CRS] coordinates{
				(273.927,0.0609635)
				(425.171,0.0507625)
				(524.505,0.0462761)
				(773.732,0.0408107)
				(1066.8,0.036809)
				(1248.1,0.0350087)
				(1520.54,0.0329096)
				(1728.48,0.0318066)
				(1966.6,0.0301311)
				(2098.87,0.0295106)
				(2558.84,0.0276736)
				(3022.16,0.0265654)
				(3303.81,0.0258126)
				(3808.31,0.024415)
				(4129.2,0.0237546)
				(4584.01,0.0229146)
				(5320.33,0.0219685)
				(5908.32,0.0212905)
			};

			% SEAW ue_np.mean[0] %%%%%%%%%%%%%%%%%%%%%%%%%%%%%%%%%%%%%%%%%%%%%%%%%%%%%%%%%%%%
			\addplot[SEAW] coordinates{
				(187.497,0.169008)
				(425.186,0.0877033)
				(1234.06,0.043661)
				(4222.63,0.02615)
			};

			% RESEEDS ue_np.mean[0] %%%%%%%%%%%%%%%%%%%%%%%%%%%%%%%%%%%%%%%%%%%%%%%%%%%%%%%%%%%%
			\addplot[RESEEDS] coordinates{
				(200.773,0.0804503)
				(301.121,0.0553946)
				(400.423,0.0603223)
				(600.713,0.046857)
				(800.771,0.0399535)
				(1000.07,0.035619)
				(1200.08,0.032056)
				(1400.11,0.0303259)
				(1600.07,0.0331177)
				(1800.06,0.0313354)
				(2000.06,0.0294)
				(2400.12,0.025435)
				(2800.13,0.0247926)
				(3300.09,0.0239177)
				(3700.18,0.0230626)
				(4000.16,0.0219552)
				(4950.18,0.0203439)
				(5000.16,0.0207573)
			};

			% ERGC ue_np.mean[0] %%%%%%%%%%%%%%%%%%%%%%%%%%%%%%%%%%%%%%%%%%%%%%%%%%%%%%%%%%%%
			\addplot[ERGC] coordinates{
				(196,0.0650601)
				(289,0.0527884)
				(400,0.0463421)
				(600,0.0397564)
				(841,0.0357345)
				(992,0.0338358)
				(1155,0.0319287)
				(1332,0.0306532)
				(1600,0.0292892)
				(1720,0.0287343)
				(2024,0.0274069)
				(2500,0.0259708)
				(2750,0.0254409)
				(3135,0.0246119)
				(3420,0.0240124)
				(3819,0.0233598)
				(4489,0.0225574)
				(5025,0.0218768)
			};

			% PF ue_np.mean[0] %%%%%%%%%%%%%%%%%%%%%%%%%%%%%%%%%%%%%%%%%%%%%%%%%%%%%%%%%%%%
			\addplot[PF] coordinates{
				(287.639,0.179106)
				(447.933,0.150927)
				(606.404,0.132719)
				(875.881,0.113291)
				(1211.94,0.0996103)
				(1542.84,0.0890575)
				(1746.93,0.0838664)
				(2019.1,0.0789583)
				(2334.11,0.0741886)
				(2737.01,0.0692184)
				(3309.32,0.0637657)
				(4281.63,0.0577981)
				(5790.62,0.0511081)
			};

			% TPS ue_np.mean[0] %%%%%%%%%%%%%%%%%%%%%%%%%%%%%%%%%%%%%%%%%%%%%%%%%%%%%%%%%%%%
			\addplot[TPS] coordinates{
				(240.968,0.078582)
				(345,0.062665)
				(425.052,0.054953)
				(658.648,0.0422032)
				(885.365,0.0355121)
				(1100.77,0.0325131)
				(1378.51,0.0287965)
				(1575.89,0.0266053)
				(1661.41,0.0258134)
				(1992.65,0.0241454)
				(2184.11,0.0230953)
				(2554.44,0.022064)
				(3222.27,0.021116)
				(3524.76,0.020123)
				(4065.15,0.0184898)
				(4552.29,0.0174741)
				(5048.88,0.0190173)
				(6022.01,0.0170758)
			};

			% NC ue_np.mean[0] %%%%%%%%%%%%%%%%%%%%%%%%%%%%%%%%%%%%%%%%%%%%%%%%%%%%%%%%%%%%
			\addplot[NC] coordinates{
				(430.693,0.039707)
				(1041.48,0.0335876)
				(2965.42,0.0267946)
				(4028.44,0.0256997)
			};

			% VC ue_np.mean[0] %%%%%%%%%%%%%%%%%%%%%%%%%%%%%%%%%%%%%%%%%%%%%%%%%%%%%%%%%%%%
			\addplot[VC] coordinates{
				(229.812,0.184267)
				(383.836,0.128875)
				(584.402,0.0885546)
				(868.048,0.0614734)
				(1084.39,0.0503802)
				(1256.67,0.0443683)
				(1572.27,0.0381485)
				(1854.83,0.0342373)
				(2121.29,0.0317916)
				(2367.94,0.0302551)
				(2613.23,0.0288743)
				(2851.13,0.0278974)
				(3082.91,0.0267861)
				(3303.61,0.0260582)
				(3743.63,0.0248503)
				(4176.97,0.0238322)
				(4574.03,0.0230424)
				(4966.49,0.0222687)
				(5304.49,0.021721)
				(5782.63,0.0209646)
			};

			% PB ue_np.mean[0] %%%%%%%%%%%%%%%%%%%%%%%%%%%%%%%%%%%%%%%%%%%%%%%%%%%%%%%%%%%%
			\addplot[PB] coordinates{
				(346.801,0.125268)
				(446.687,0.0986945)
				(537.451,0.0853236)
				(691.737,0.0683735)
				(923.156,0.0564933)
				(1132.22,0.0497422)
				(1286.56,0.0461131)
				(1480.41,0.0426147)
				(1708.32,0.0399019)
				(1708.32,0.0399019)
				(2029.09,0.0362617)
				(2389.29,0.0336546)
				(2974.92,0.0304203)
				(2974.92,0.0304203)
				(3693.23,0.0278361)
				(3693.23,0.0278361)
				(4884.82,0.0248665)
				(4884.82,0.0248665)
			};

			% PRESLIC ue_np.mean[0] %%%%%%%%%%%%%%%%%%%%%%%%%%%%%%%%%%%%%%%%%%%%%%%%%%%%%%%%%%%%
			\addplot[PRESLIC] coordinates{
				(184.222,0.0860039)
				(289.592,0.068754)
				(393.639,0.0591687)
				(589.462,0.0497552)
				(795.786,0.042833)
				(1031.99,0.0387423)
				(1171.21,0.0367596)
				(1377.24,0.034382)
				(1646.02,0.0324789)
				(1647.89,0.0325549)
				(1943.21,0.0309143)
				(2396.45,0.0290043)
				(2901.84,0.0272479)
				(3744.27,0.0253277)
				(3744.27,0.0253277)
				(4842.14,0.0236074)
				(4842.14,0.0236074)
			};

			% W ue_np.mean[0] %%%%%%%%%%%%%%%%%%%%%%%%%%%%%%%%%%%%%%%%%%%%%%%%%%%%%%%%%%%%
			\addplot[W] coordinates{
				(189.026,0.104282)
				(298.726,0.0821862)
				(433.637,0.0697173)
				(615.082,0.0585036)
				(827.998,0.0502654)
				(1116.68,0.0452672)
				(1246.48,0.0428116)
				(1482.3,0.0402504)
				(1722.38,0.037543)
				(1722.38,0.037543)
				(2033.96,0.035164)
				(2525.92,0.0327855)
				(3118.03,0.0305563)
				(3118.03,0.0305563)
				(4018.14,0.028175)
				(4018.14,0.028175)
				(5283.66,0.0261079)
				(5283.66,0.0261079)
			};

			% LSC ue_np.mean[0] %%%%%%%%%%%%%%%%%%%%%%%%%%%%%%%%%%%%%%%%%%%%%%%%%%%%%%%%%%%%
			\addplot[LSC] coordinates{
				(534.754,0.0472616)
				(777.454,0.0395625)
				(1013.2,0.0359645)
				(1335.02,0.0325427)
				(1675.81,0.0301275)
				(1904.02,0.0290933)
				(2226.93,0.0279949)
				(2345.62,0.0273794)
				(2566.38,0.0266183)
				(2565.18,0.0262834)
				(2831.62,0.0255574)
				(3174.62,0.0246657)
				(3270.2,0.0240651)
				(3545.63,0.0234781)
				(3796.98,0.0228938)
				(4091.52,0.0225681)
				(4607.06,0.0223626)
				(5116.96,0.0220844)
			};

			% WP ue_np.mean[0] %%%%%%%%%%%%%%%%%%%%%%%%%%%%%%%%%%%%%%%%%%%%%%%%%%%%%%%%%%%%
			\addplot[WP] coordinates{
				(204,0.0840244)
				(330,0.0667068)
				(425,0.0580243)
				(600,0.0494597)
				(864,0.0430897)
				(1080,0.0391348)
				(1247,0.0372376)
				(1426,0.0353486)
				(1700,0.0334915)
				(1700,0.033494)
				(2035,0.0314658)
				(2400,0.0295748)
				(3015,0.0276904)
				(3015,0.0276902)
				(3750,0.0257514)
				(3750,0.0257514)
				(4902,0.0242823)
				(4902,0.0242823)
			};

			% QS ue_np.mean[0] %%%%%%%%%%%%%%%%%%%%%%%%%%%%%%%%%%%%%%%%%%%%%%%%%%%%%%%%%%%%
			\addplot[QS] coordinates{
				(310.248,0.294773)
				(452.998,0.196618)
				(692.458,0.115002)
				(836.002,0.0893088)
				(1059,0.0670436)
				(1219.6,0.0578153)
				(1454.73,0.0490597)
				(1608.81,0.0449531)
				(1787.15,0.0415334)
				(2009.01,0.0381731)
				(2279.07,0.0350551)
				(2622.46,0.0321147)
				(3082.46,0.0293801)
				(3703.75,0.0267533)
				(4580.52,0.0241692)
				(5831.35,0.0214139)
			};

			% VLSLIC ue_np.mean[0] %%%%%%%%%%%%%%%%%%%%%%%%%%%%%%%%%%%%%%%%%%%%%%%%%%%%%%%%%%%%
			\addplot[VLSLIC] coordinates{
				(658.287,0.0542932)
				(783.374,0.0469887)
				(876.521,0.0426489)
				(1023.6,0.0382561)
				(1181.12,0.0352334)
				(1320.08,0.0328026)
				(1429.35,0.0315082)
				(1588.18,0.0300446)
				(1788.1,0.0287683)
				(1783.07,0.028442)
				(2077.75,0.0272548)
				(2418.96,0.0259596)
				(2993.57,0.0247993)
				(3003.13,0.0248046)
				(3723.2,0.0237311)
				(3723.2,0.0237311)
				(4895.39,0.0225704)
				(4895.39,0.0225704)
			};

			% CIS ue_np.mean[0] %%%%%%%%%%%%%%%%%%%%%%%%%%%%%%%%%%%%%%%%%%%%%%%%%%%%%%%%%%%%
			\addplot[CIS] coordinates{
				(346.605,0.0588337)
				(416.704,0.051435)
				(498.555,0.0464455)
				(629.289,0.040862)
				(800.998,0.0364286)
				(978.771,0.0337084)
				(1078.75,0.0327401)
				(1215.26,0.0314663)
				(1393.65,0.0302647)
				(1411.64,0.0301974)
				(1712,0.0283586)
				(2042.96,0.0271735)
				(2534.86,0.0256878)
				(3324.5,0.0243093)
				(3324.5,0.0243093)
				(5000.77,0.0215964)
				(5000.77,0.0215964)
			};

			% ERS ue_np.mean[0] %%%%%%%%%%%%%%%%%%%%%%%%%%%%%%%%%%%%%%%%%%%%%%%%%%%%%%%%%%%%
			\addplot[ERS] coordinates{
				(200,0.0640079)
				(300,0.0542473)
				(400,0.0486543)
				(600,0.0427847)
				(800,0.0393748)
				(1000,0.037205)
				(1200,0.0353502)
				(1400,0.0338978)
				(1600,0.0326999)
				(1800,0.0316567)
				(2000,0.0307179)
				(2400,0.0291883)
				(2800,0.027889)
				(3200,0.0268876)
				(3600,0.0258441)
				(4000,0.0250276)
				(4600,0.0239331)
				(5200,0.0229686)
			};

			% MSS ue_np.mean[0] %%%%%%%%%%%%%%%%%%%%%%%%%%%%%%%%%%%%%%%%%%%%%%%%%%%%%%%%%%%%
			\addplot[MSS] coordinates{
				(205.549,0.0862862)
				(330.577,0.0679002)
				(436.387,0.0596761)
				(637.976,0.0509215)
				(945.067,0.0447051)
				(1204.04,0.0409043)
				(1406.86,0.039381)
				(1661.87,0.0377301)
				(1968.29,0.0361841)
				(1968.29,0.0361841)
				(2383.46,0.0345185)
				(2847.26,0.0331063)
				(3639.98,0.0314821)
				(3640.21,0.0314511)
				(4598.5,0.0296113)
				(4598.5,0.0296113)
				(6276.05,0.0279381)
			};

			% ETPS ue_np.mean[0] %%%%%%%%%%%%%%%%%%%%%%%%%%%%%%%%%%%%%%%%%%%%%%%%%%%%%%%%%%%%
			\addplot[ETPS] coordinates{
				(216,0.0566241)
				(330,0.0454219)
				(425,0.0400688)
				(600,0.0348726)
				(864,0.0314334)
				(1080,0.0290884)
				(1247,0.0277965)
				(1457,0.026763)
				(1700,0.0252919)
				(1700,0.0252919)
				(2035,0.024419)
				(2400,0.0229673)
				(3015,0.02182)
				(3015,0.02182)
				(3750,0.0206171)
				(3750,0.0206171)
				(4988,0.0189705)
				(4988,0.0189705)
			};

			\end{axis}
	\end{tikzpicture}
\end{subfigure}%
\begin{subfigure}[b]{\fullthreeone\textwidth}\phantomsubcaption\label{subfig:appendix-experiments-fash-ev.mean[0]}
	%%%%%%%%%%%%%%%%%%%%%%%%%%%%%%%%%%%%%%%%%%%%%%%%%%%%%%%%%%%%
	% ev.mean[0]
	%%%%%%%%%%%%%%%%%%%%%%%%%%%%%%%%%%%%%%%%%%%%%%%%%%%%%%%%%%%%
	\begin{tikzpicture}
		\begin{axis}[EQFashEV,xmode=log]

			% CCS ev.mean[0] %%%%%%%%%%%%%%%%%%%%%%%%%%%%%%%%%%%%%%%%%%%%%%%%%%%%%%%%%%%%
			\addplot[CCS] coordinates{
				(205.099,0.870193)
				(323.803,0.894388)
				(442.991,0.906726)
				(663.605,0.921302)
				(888.464,0.929875)
				(1145.83,0.936234)
				(1290.6,0.938525)
				(1514.56,0.942478)
				(1804.75,0.945908)
				(1804.75,0.945908)
				(2118.27,0.94889)
				(2607.61,0.95256)
				(3136.79,0.95557)
				(3136.79,0.95557)
				(4030.41,0.959221)
				(4030.41,0.959221)
				(5170.28,0.962787)
				(5170.28,0.962787)
			};

			% SEEDS ev.mean[0] %%%%%%%%%%%%%%%%%%%%%%%%%%%%%%%%%%%%%%%%%%%%%%%%%%%%%%%%%%%%
			\addplot[SEEDS] coordinates{
				(260.22,0.918897)
				(363.724,0.931344)
				(472.406,0.927442)
				(675.082,0.939389)
				(875.184,0.946233)
				(1078.34,0.948976)
				(1272.02,0.953826)
				(1466.01,0.95632)
				(1686.85,0.948735)
				(1886.09,0.951272)
				(2084.5,0.954239)
				(2465.45,0.963852)
				(2862.45,0.963813)
				(3376.59,0.963709)
				(3773,0.965542)
				(4056.65,0.969166)
				(5005.27,0.972071)
				(5061.29,0.970297)
			};

			% SLIC ev.mean[0] %%%%%%%%%%%%%%%%%%%%%%%%%%%%%%%%%%%%%%%%%%%%%%%%%%%%%%%%%%%%
			\addplot[SLIC] coordinates{
				(183.616,0.867061)
				(287.775,0.884929)
				(393.184,0.897857)
				(587.981,0.91003)
				(796.192,0.920863)
				(1031.22,0.928117)
				(1172.04,0.932607)
				(1377.87,0.93642)
				(1648.17,0.940165)
				(1650.3,0.940025)
				(1945.79,0.943392)
				(2399.7,0.947201)
				(2906.47,0.950595)
				(3748.69,0.954679)
				(3748.69,0.954679)
				(4847.82,0.958705)
				(4847.82,0.958705)
			};

			% RW ev.mean[0] %%%%%%%%%%%%%%%%%%%%%%%%%%%%%%%%%%%%%%%%%%%%%%%%%%%%%%%%%%%%
			\addplot[RW] coordinates{
				(406.784,0.855464)
				(1348.01,0.909253)
				(4048.94,0.942654)
				(6058.17,0.946372)
			};

			% CW ev.mean[0] %%%%%%%%%%%%%%%%%%%%%%%%%%%%%%%%%%%%%%%%%%%%%%%%%%%%%%%%%%%%
			\addplot[CW] coordinates{
				(199.177,0.827166)
				(293.937,0.847103)
				(409.708,0.862287)
				(591.276,0.876934)
				(839.903,0.888279)
				(1030.22,0.894814)
				(1205.46,0.899688)
				(1396.52,0.903627)
				(1668.84,0.908464)
				(1796.95,0.910208)
				(2121.13,0.914036)
				(2647.64,0.918985)
				(2861.41,0.920846)
				(3267.16,0.923116)
				(3643.52,0.925598)
				(4081.58,0.927908)
				(4756.03,0.93039)
				(5360.93,0.932691)
			};

			% TP ev.mean[0] %%%%%%%%%%%%%%%%%%%%%%%%%%%%%%%%%%%%%%%%%%%%%%%%%%%%%%%%%%%%
			\addplot[TP] coordinates{
				(280.927,0.860195)
				(452.384,0.882146)
				(555.011,0.89106)
				(738.32,0.900484)
				%(1028.6,0.866285)
				(1275.51,0.882507)
				(1441.55,0.887004)
				(1930.85,0.902428)
				(2280.38,0.909632)
				(3080.77,0.916791)
				(3683.73,0.921935)
				(4359.43,0.926904)
			};

			% POISE ev.mean[0] %%%%%%%%%%%%%%%%%%%%%%%%%%%%%%%%%%%%%%%%%%%%%%%%%%%%%%%%%%%%
			\addplot[POISE] coordinates{
				(204.877,0.867412)
				(306.665,0.883595)
				(408.393,0.891931)
				(611.996,0.900681)
				(815.078,0.905714)
				(1016.78,0.908556)
				(1215.99,0.910556)
				(1408.03,0.91205)
				(1590.31,0.913145)
				(1759.67,0.913968)
				(1917.33,0.914566)
				(2193.21,0.915392)
				(2416.91,0.91599)
				(2585.33,0.91633)
				(2709.65,0.91651)
				(2795.14,0.916615)
				(2857.91,0.916675)
				(2872.78,0.916686)
			};

			% FH ev.mean[0] %%%%%%%%%%%%%%%%%%%%%%%%%%%%%%%%%%%%%%%%%%%%%%%%%%%%%%%%%%%%
			\addplot[FH] coordinates{
				(820.24,0.86446)
				(1017.45,0.882099)
				(1255.95,0.894577)
				(1422.06,0.910093)
				(1626.81,0.909203)
				(2095.7,0.924522)
				(2264.94,0.924723)
				(3618.47,0.944096)
				(3968.9,0.946199)
				(4191.68,0.951671)
				(4226.29,0.946853)
				(5065.34,0.957241)
				(6378.82,0.964946)
			};

			% EAMS ev.mean[0] %%%%%%%%%%%%%%%%%%%%%%%%%%%%%%%%%%%%%%%%%%%%%%%%%%%%%%%%%%%%
			\addplot[EAMS] coordinates{
				(334.743,0.885522)
				(356.009,0.888007)
				(382.151,0.890574)
				(452.27,0.896499)
				(532.631,0.903215)
				(734.402,0.911718)
				(1235.54,0.922911)
				(1302.6,0.923939)
				(1404.02,0.925337)
				(1491.76,0.926479)
				(1591.75,0.927668)
				(1888.34,0.930633)
				(2325.36,0.934105)
				(2903.95,0.937748)
				(4169.4,0.942987)
			};

			% CRS ev.mean[0] %%%%%%%%%%%%%%%%%%%%%%%%%%%%%%%%%%%%%%%%%%%%%%%%%%%%%%%%%%%%
			\addplot[CRS] coordinates{
				(273.927,0.863053)
				(425.171,0.883157)
				(524.505,0.891004)
				(773.732,0.903678)
				(1066.8,0.912807)
				(1248.1,0.917006)
				(1520.54,0.922012)
				(1728.48,0.923881)
				(1966.6,0.927869)
				(2098.87,0.929427)
				(2558.84,0.933961)
				(3022.16,0.936375)
				(3303.81,0.938257)
				(3808.31,0.942076)
				(4129.2,0.94369)
				(4584.01,0.945835)
				(5320.33,0.948263)
				(5908.32,0.949673)
			};

			% SEAW ev.mean[0] %%%%%%%%%%%%%%%%%%%%%%%%%%%%%%%%%%%%%%%%%%%%%%%%%%%%%%%%%%%%
			\addplot[SEAW] coordinates{
				(187.497,0.782984)
				(425.186,0.860498)
				(1234.06,0.911343)
				(4222.63,0.946846)
			};

			% RESEEDS ev.mean[0] %%%%%%%%%%%%%%%%%%%%%%%%%%%%%%%%%%%%%%%%%%%%%%%%%%%%%%%%%%%%
			\addplot[RESEEDS] coordinates{
				(200.773,0.91571)
				(301.121,0.934592)
				(400.423,0.929385)
				(600.713,0.94145)
				(800.771,0.947818)
				(1000.07,0.953352)
				(1200.08,0.957393)
				(1400.11,0.959648)
				(1600.07,0.954463)
				(1800.06,0.95678)
				(2000.06,0.958935)
				(2400.12,0.966193)
				(2800.13,0.966986)
				(3300.09,0.966971)
				(3700.18,0.968464)
				(4000.16,0.971357)
				(4950.18,0.973579)
				(5000.16,0.972515)
			};

			% ERGC ev.mean[0] %%%%%%%%%%%%%%%%%%%%%%%%%%%%%%%%%%%%%%%%%%%%%%%%%%%%%%%%%%%%
			\addplot[ERGC] coordinates{
				(196,0.870741)
				(289,0.888273)
				(400,0.901607)
				(600,0.915609)
				(841,0.925336)
				(992,0.930178)
				(1155,0.934184)
				(1332,0.937558)
				(1600,0.942182)
				(1720,0.943791)
				(2024,0.94711)
				(2500,0.951481)
				(2750,0.953549)
				(3135,0.955855)
				(3420,0.957543)
				(3819,0.959527)
				(4489,0.962244)
				(5025,0.964173)
			};

			% PF ev.mean[0] %%%%%%%%%%%%%%%%%%%%%%%%%%%%%%%%%%%%%%%%%%%%%%%%%%%%%%%%%%%%
			\addplot[PF] coordinates{
				(287.639,0.710536)
				(447.933,0.746441)
				(606.404,0.768948)
				(875.881,0.793715)
				(1211.94,0.81277)
				(1542.84,0.826653)
				(1746.93,0.834385)
				(2019.1,0.841278)
				(2334.11,0.848629)
				(2737.01,0.856158)
				(3309.32,0.866585)
				(4281.63,0.876681)
				(5790.62,0.88797)
			};

			% TPS ev.mean[0] %%%%%%%%%%%%%%%%%%%%%%%%%%%%%%%%%%%%%%%%%%%%%%%%%%%%%%%%%%%%
			\addplot[TPS] coordinates{
				(240.968,0.799367)
				(345,0.823904)
				(425.052,0.83632)
				(658.648,0.858876)
				(885.365,0.871731)
				(1100.77,0.878884)
				(1378.51,0.888426)
				(1575.89,0.893625)
				(1661.41,0.89532)
				(1992.65,0.901333)
				(2184.11,0.903519)
				(2554.44,0.908117)
				(3222.27,0.913565)
				(3524.76,0.915674)
				(4065.15,0.921172)
				(4552.29,0.923968)
				(5048.88,0.922421)
				(6022.01,0.929188)
			};

			% NC ev.mean[0] %%%%%%%%%%%%%%%%%%%%%%%%%%%%%%%%%%%%%%%%%%%%%%%%%%%%%%%%%%%%
			\addplot[NC] coordinates{
				(430.693,0.868124)
				(1041.48,0.884617)
				(2965.42,0.904653)
				(4028.44,0.899695)
			};

			% VC ev.mean[0] %%%%%%%%%%%%%%%%%%%%%%%%%%%%%%%%%%%%%%%%%%%%%%%%%%%%%%%%%%%%
			\addplot[VC] coordinates{
				(229.812,0.761031)
				(383.836,0.818208)
				(584.402,0.860831)
				(868.048,0.892622)
				(1084.39,0.908242)
				(1256.67,0.917945)
				(1572.27,0.928964)
				(1854.83,0.935612)
				(2121.29,0.940324)
				(2367.94,0.943661)
				(2613.23,0.946345)
				(2851.13,0.948477)
				(3082.91,0.950539)
				(3303.61,0.952002)
				(3743.63,0.954779)
				(4176.97,0.957213)
				(4574.03,0.959093)
				(4966.49,0.960847)
				(5304.49,0.961852)
				(5782.63,0.963816)
			};

			% PB ev.mean[0] %%%%%%%%%%%%%%%%%%%%%%%%%%%%%%%%%%%%%%%%%%%%%%%%%%%%%%%%%%%%
			\addplot[PB] coordinates{
				(346.801,0.73974)
				(446.687,0.78265)
				(537.451,0.803962)
				(691.737,0.828378)
				(923.156,0.853293)
				(1132.22,0.866521)
				(1286.56,0.874182)
				(1480.41,0.882147)
				(1708.32,0.888634)
				(1708.32,0.888634)
				(2029.09,0.897333)
				(2389.29,0.903336)
				(2974.92,0.912989)
				(2974.92,0.912989)
				(3693.23,0.920805)
				(3693.23,0.920805)
				(4884.82,0.931354)
				(4884.82,0.931354)
			};

			% PRESLIC ev.mean[0] %%%%%%%%%%%%%%%%%%%%%%%%%%%%%%%%%%%%%%%%%%%%%%%%%%%%%%%%%%%%
			\addplot[PRESLIC] coordinates{
				(184.222,0.840027)
				(289.592,0.864774)
				(393.639,0.876548)
				(589.462,0.89537)
				(795.786,0.904764)
				(1031.99,0.913506)
				(1171.21,0.917884)
				(1377.24,0.922329)
				(1646.02,0.928276)
				(1647.89,0.928139)
				(1943.21,0.931802)
				(2396.45,0.937178)
				(2901.84,0.94054)
				(3744.27,0.946149)
				(3744.27,0.946149)
				(4842.14,0.950723)
				(4842.14,0.950723)
			};

			% W ev.mean[0] %%%%%%%%%%%%%%%%%%%%%%%%%%%%%%%%%%%%%%%%%%%%%%%%%%%%%%%%%%%%
			\addplot[W] coordinates{
				(189.026,0.810381)
				(298.726,0.838055)
				(433.637,0.856732)
				(615.082,0.872941)
				(827.998,0.884106)
				(1116.68,0.894118)
				(1246.48,0.897845)
				(1482.3,0.90307)
				(1722.38,0.907176)
				(1722.38,0.907176)
				(2033.96,0.911914)
				(2525.92,0.917254)
				(3118.03,0.921694)
				(3118.03,0.921694)
				(4018.14,0.927354)
				(4018.14,0.927354)
				(5283.66,0.932147)
				(5283.66,0.932147)
			};

			% LSC ev.mean[0] %%%%%%%%%%%%%%%%%%%%%%%%%%%%%%%%%%%%%%%%%%%%%%%%%%%%%%%%%%%%
			\addplot[LSC] coordinates{
				(534.754,0.913197)
				(777.454,0.923251)
				(1013.2,0.928353)
				(1335.02,0.933644)
				(1675.81,0.937612)
				(1904.02,0.939492)
				(2226.93,0.940835)
				(2345.62,0.941796)
				(2566.38,0.942691)
				(2565.18,0.943989)
				(2831.62,0.944863)
				(3174.62,0.947216)
				(3270.2,0.949481)
				(3545.63,0.951379)
				(3796.98,0.95302)
				(4091.52,0.953126)
				(4607.06,0.953227)
				(5116.96,0.953105)
			};

			% WP ev.mean[0] %%%%%%%%%%%%%%%%%%%%%%%%%%%%%%%%%%%%%%%%%%%%%%%%%%%%%%%%%%%%
			\addplot[WP] coordinates{
				(204,0.845474)
				(330,0.867781)
				(425,0.879751)
				(600,0.893187)
				(864,0.905052)
				(1080,0.91201)
				(1247,0.916059)
				(1426,0.919691)
				(1700,0.923869)
				(1700,0.923875)
				(2035,0.928728)
				(2400,0.932658)
				(3015,0.937533)
				(3015,0.937536)
				(3750,0.942018)
				(3750,0.942018)
				(4902,0.946537)
				(4902,0.946537)
			};

			% QS ev.mean[0] %%%%%%%%%%%%%%%%%%%%%%%%%%%%%%%%%%%%%%%%%%%%%%%%%%%%%%%%%%%%
			\addplot[QS] coordinates{
				(310.248,0.529011)
				(452.998,0.709502)
				(692.458,0.821905)
				(836.002,0.852549)
				(1059,0.879953)
				(1219.6,0.892749)
				(1454.73,0.905603)
				(1608.81,0.912363)
				(1787.15,0.918823)
				(2009.01,0.925168)
				(2279.07,0.931496)
				(2622.46,0.938069)
				(3082.46,0.944845)
				(3703.75,0.95185)
				(4580.52,0.958994)
				(5831.35,0.966223)
			};

			% VLSLIC ev.mean[0] %%%%%%%%%%%%%%%%%%%%%%%%%%%%%%%%%%%%%%%%%%%%%%%%%%%%%%%%%%%%
			\addplot[VLSLIC] coordinates{
				(658.287,0.922641)
				(783.374,0.925855)
				(876.521,0.927218)
				(1023.6,0.928802)
				(1181.12,0.93109)
				(1320.08,0.933358)
				(1429.35,0.93544)
				(1588.18,0.937536)
				(1788.1,0.940295)
				(1783.07,0.941908)
				(2077.75,0.94401)
				(2418.96,0.947479)
				(2993.57,0.949828)
				(3003.13,0.950546)
				(3723.2,0.952672)
				(3723.2,0.952672)
				(4895.39,0.954431)
				(4895.39,0.954431)
			};

			% CIS ev.mean[0] %%%%%%%%%%%%%%%%%%%%%%%%%%%%%%%%%%%%%%%%%%%%%%%%%%%%%%%%%%%%
			\addplot[CIS] coordinates{
				(346.605,0.881824)
				(416.704,0.888943)
				(498.555,0.89572)
				(629.289,0.903657)
				(800.998,0.912646)
				(978.771,0.919959)
				(1078.75,0.92108)
				(1215.26,0.925002)
				(1393.65,0.92686)
				(1411.64,0.927196)
				(1712,0.934258)
				(2042.96,0.936979)
				(2534.86,0.942819)
				(3324.5,0.946349)
				(3324.5,0.946349)
				(5000.77,0.958048)
				(5000.77,0.958048)
			};

			% ERS ev.mean[0] %%%%%%%%%%%%%%%%%%%%%%%%%%%%%%%%%%%%%%%%%%%%%%%%%%%%%%%%%%%%
			\addplot[ERS] coordinates{
				(200,0.851324)
				(300,0.867773)
				(400,0.877525)
				(600,0.890214)
				(800,0.897888)
				(1000,0.903236)
				(1200,0.907454)
				(1400,0.910903)
				(1600,0.913894)
				(1800,0.916441)
				(2000,0.918754)
				(2400,0.922571)
				(2800,0.925939)
				(3200,0.928729)
				(3600,0.931221)
				(4000,0.933464)
				(4600,0.936565)
				(5200,0.939204)
			};

			% MSS ev.mean[0] %%%%%%%%%%%%%%%%%%%%%%%%%%%%%%%%%%%%%%%%%%%%%%%%%%%%%%%%%%%%
			\addplot[MSS] coordinates{
				(205.549,0.83712)
				(330.577,0.86406)
				(436.387,0.875234)
				(637.976,0.889585)
				(945.067,0.903879)
				(1204.04,0.909536)
				(1406.86,0.913383)
				(1661.87,0.916678)
				(1968.29,0.920078)
				(1968.29,0.920078)
				(2383.46,0.923385)
				(2847.26,0.92568)
				(3639.98,0.929554)
				(3640.21,0.929585)
				(4598.5,0.932548)
				(4598.5,0.932548)
				(6276.05,0.936584)
			};

			% ETPS ev.mean[0] %%%%%%%%%%%%%%%%%%%%%%%%%%%%%%%%%%%%%%%%%%%%%%%%%%%%%%%%%%%%
			\addplot[ETPS] coordinates{
				(216,0.938511)
				(330,0.946505)
				(425,0.951301)
				(600,0.956297)
				(864,0.960464)
				(1080,0.962924)
				(1247,0.964373)
				(1457,0.966178)
				(1700,0.968231)
				(1700,0.968231)
				(2035,0.969555)
				(2400,0.971335)
				(3015,0.973103)
				(3015,0.973103)
				(3750,0.975284)
				(3750,0.975284)
				(4988,0.977265)
				(4988,0.977265)
			};

			\end{axis}
	\end{tikzpicture}
\end{subfigure}

	\caption{\Rec, \UE and \EV on the \SBD, \SUNRGBD and \Fash datasets. Similar to the results
	presented for the \BSDS and \NYU datasets (compare Figures \ref{fig:experiments-quantitative-bsds500}
	and \ref{fig:experiments-quantitative-nyuv2}), \Rec and \UE give a roguh overview of
	algorithm performance with respect to ground truth. Concerning \Rec, we observe
	similar performance across the three datasets, while algorithms may show different
	behavior with respect to \UE. Similarly, \EV gives a ground truth independent
	overview of algorithm performance where algorithms show similar performance across datasets.
	\textbf{Best viewed in color.}}
	\label{fig:appendix-experiments-sbd-sunrgbd-fash}
	\vskip 12px
	\input{legends/full+depth}
\end{figure*}
\begin{figure*}
	\centering
	\begin{subfigure}[b]{0.325\textwidth}\phantomsubcaption\label{subfig:appendix-experiments-runtime-sbd}
	\begin{tikzpicture}
		\begin{axis}[AESBDt,ymode=log]
			% CCS %%%%%%%%%%%%%%%%%%%%%%%%%%%%%%%%%%%%%%%%%%%%%%%%%%%%%%%%%%%%
			\addplot[CCS] coordinates{
				(430.83,0.097861666666667)
				(1299.21,0.092760333333333)
				(3500.9,0.08232)
			};

				% CIS %%%%%%%%%%%%%%%%%%%%%%%%%%%%%%%%%%%%%%%%%%%%%%%%%%%%%%%%%%%%
			\addplot[CIS] coordinates{
				(401.874,1.192525)
				(943.665,1.12152)
				(3318.96,1.41482)
			};

				% CRS %%%%%%%%%%%%%%%%%%%%%%%%%%%%%%%%%%%%%%%%%%%%%%%%%%%%%%%%%%%%
			\addplot[CRS] coordinates{
				(497.593,0.2353985)
				(1467.36,0.385985)
				(4396.39,0.64592)
			};

				% CW %%%%%%%%%%%%%%%%%%%%%%%%%%%%%%%%%%%%%%%%%%%%%%%%%%%%%%%%%%%%
			\addplot[CW] coordinates{
				(411.784,0.00159329)
				(1258.45,0.001656185)
				(4239.54,0.00172956)
			};

				% EAMS %%%%%%%%%%%%%%%%%%%%%%%%%%%%%%%%%%%%%%%%%%%%%%%%%%%%%%%%%%%%
			\addplot[EAMS] coordinates{
				(692.407,0.0553535)
				(1352.02,0.054948)
				(3435.61,0.0551595)
			};

				% ERGC %%%%%%%%%%%%%%%%%%%%%%%%%%%%%%%%%%%%%%%%%%%%%%%%%%%%%%%%%%%%
			\addplot[ERGC] coordinates{
				(398.958,0.02488465)
				(1225.23,0.0257966)
				(3794.68,0.02593285)
			};

				% ERS %%%%%%%%%%%%%%%%%%%%%%%%%%%%%%%%%%%%%%%%%%%%%%%%%%%%%%%%%%%%
			\addplot[ERS] coordinates{
				(400,0.140765)
				(1200,0.146688)
				(3600,0.1430295)
			};

				% ETPS %%%%%%%%%%%%%%%%%%%%%%%%%%%%%%%%%%%%%%%%%%%%%%%%%%%%%%%%%%%%
			\addplot[ETPS] coordinates{
				(415.836,0.147411)
				(1185.61,0.1664885)
				(3309.82,0.2635745)
			};

				% FH %%%%%%%%%%%%%%%%%%%%%%%%%%%%%%%%%%%%%%%%%%%%%%%%%%%%%%%%%%%%
			\addplot[FH] coordinates{
				(303.964,0.01541925)
				(329.31,0.0172222)
				(1354.88,0.0159853)
			};

				% LSC %%%%%%%%%%%%%%%%%%%%%%%%%%%%%%%%%%%%%%%%%%%%%%%%%%%%%%%%%%%%
			\addplot[LSC] coordinates{
				(804.126,0.123627)
				(1404.39,0.1339515)
				(2138.92,0.14413)
			};

				% MSS %%%%%%%%%%%%%%%%%%%%%%%%%%%%%%%%%%%%%%%%%%%%%%%%%%%%%%%%%%%%
			\addplot[MSS] coordinates{
				(419.465,0.0118344)
				(1321.31,0.0123166)
				(4207.84,0.01366875)
			};

				% PB %%%%%%%%%%%%%%%%%%%%%%%%%%%%%%%%%%%%%%%%%%%%%%%%%%%%%%%%%%%%
			\addplot[PB] coordinates{
				(432.128,0.0126625)
				(1156.66,0.0124738)
				(3142.19,0.01264155)
			};

				% PF %%%%%%%%%%%%%%%%%%%%%%%%%%%%%%%%%%%%%%%%%%%%%%%%%%%%%%%%%%%%
			\addplot[PF] coordinates{
				(456.474,0.0015468991352201)
				(1285.37,0.0016359751498952)
				(17789.8,0.00186107760587)
			};

				% POISE %%%%%%%%%%%%%%%%%%%%%%%%%%%%%%%%%%%%%%%%%%%%%%%%%%%%%%%%%%%%
			\addplot[POISE] coordinates{
				(406.413,0.3016015)
				(1056.29,0.278544)
				(1132.36,0.266935)
			};

				% PRESLIC %%%%%%%%%%%%%%%%%%%%%%%%%%%%%%%%%%%%%%%%%%%%%%%%%%%%%%%%%%%%
			\addplot[PRESLIC] coordinates{
				(360.824,0.0078302)
				(1146.3,0.0077149)
				(3243.14,0.00751575)
			};

				% RESEEDS %%%%%%%%%%%%%%%%%%%%%%%%%%%%%%%%%%%%%%%%%%%%%%%%%%%%%%%%%%%%
			\addplot[RESEEDS] coordinates{
				(400.572,0.0162683)
				(1201.9,0.023522)
				(3373.89,0.02299795)
			};

				% RW %%%%%%%%%%%%%%%%%%%%%%%%%%%%%%%%%%%%%%%%%%%%%%%%%%%%%%%%%%%%
			\addplot[RW] coordinates{
				(417.935,0.69413)
				(1287.57,2.309636)
				(3875.47,7.5433615)
			};

				% SEAW %%%%%%%%%%%%%%%%%%%%%%%%%%%%%%%%%%%%%%%%%%%%%%%%%%%%%%%%%%%%
			\addplot[SEAW] coordinates{
				(154.597,0.069037)
				(414.981,0.2452075)
				(1371.13,0.9195975)
			};

				% SEEDS %%%%%%%%%%%%%%%%%%%%%%%%%%%%%%%%%%%%%%%%%%%%%%%%%%%%%%%%%%%%
			\addplot[SEEDS] coordinates{
				(444.631,0.670095)
				(1243.06,1.830155)
				(3399.34,1.837035)
			};

				% SLIC %%%%%%%%%%%%%%%%%%%%%%%%%%%%%%%%%%%%%%%%%%%%%%%%%%%%%%%%%%%%
			\addplot[SLIC] coordinates{
				(368.765,0.01880505)
				(1164.08,0.01905665)
				(3325.09,0.01927675)
			};

				% TP %%%%%%%%%%%%%%%%%%%%%%%%%%%%%%%%%%%%%%%%%%%%%%%%%%%%%%%%%%%%
			\addplot[TP] coordinates{
				(494.558,1.364714)
				(918.268,1.9865935)
				(1200.89,1.0990485)
			};

				% VC %%%%%%%%%%%%%%%%%%%%%%%%%%%%%%%%%%%%%%%%%%%%%%%%%%%%%%%%%%%%
			\addplot[VC] coordinates{
				(2605.56,0.259277)
				(3930.44,0.2417085)
				(5677.76,0.3709855)
			};

				% VLSLIC %%%%%%%%%%%%%%%%%%%%%%%%%%%%%%%%%%%%%%%%%%%%%%%%%%%%%%%%%%%%
			%\addplot[VLSLIC] coordinates{
			%	(0,0.02886785)
			%	(0,0.02805025)
			%	(0,0.0289727)
			%};

				% W %%%%%%%%%%%%%%%%%%%%%%%%%%%%%%%%%%%%%%%%%%%%%%%%%%%%%%%%%%%%
			\addplot[W] coordinates{
				(402.275,0.0019392)
				(1238.6,0.00230608)
				(3637.87,0.00206499)
			};

				% WP %%%%%%%%%%%%%%%%%%%%%%%%%%%%%%%%%%%%%%%%%%%%%%%%%%%%%%%%%%%%
			\addplot[WP] coordinates{
				(397.694,0.090983263598327)
				(1223.32,0.12464435146444)
				(2934.56,0.19884937238494)
			};

			\end{axis}
	\end{tikzpicture}
\end{subfigure}
\begin{subfigure}[b]{0.325\textwidth}\phantomsubcaption\label{subfig:appendix-experiments-runtime-sunrgbd}
	\begin{tikzpicture}
		\begin{axis}[AESUNRGBDt,ymode=log]
			% CCS %%%%%%%%%%%%%%%%%%%%%%%%%%%%%%%%%%%%%%%%%%%%%%%%%%%%%%%%%%%%
			\addplot[CCS] coordinates{
				(384.925,0.3278625)
				(1186.45,0.3158875)
				(3647.26,0.3240125)
			};

				% CIS %%%%%%%%%%%%%%%%%%%%%%%%%%%%%%%%%%%%%%%%%%%%%%%%%%%%%%%%%%%%
			\addplot[CIS] coordinates{
				(467.525,6.41695)
				(981.057,6.3328)
				(2894.19,5.63845)
			};

				% CRS %%%%%%%%%%%%%%%%%%%%%%%%%%%%%%%%%%%%%%%%%%%%%%%%%%%%%%%%%%%%
			\addplot[CRS] coordinates{
				(530.128,0.5849)
				(1495.45,0.897525)
				(4297.22,1.38979)
			};

				% CW %%%%%%%%%%%%%%%%%%%%%%%%%%%%%%%%%%%%%%%%%%%%%%%%%%%%%%%%%%%%
			\addplot[CW] coordinates{
				(408.348,0.00765)
				(1244.05,0.0079625)
				(3867.11,0.0077125)
			};

				% EAMS %%%%%%%%%%%%%%%%%%%%%%%%%%%%%%%%%%%%%%%%%%%%%%%%%%%%%%%%%%%%
			\addplot[EAMS] coordinates{
				(687.737,0.3563245)
				(3028.83,0.229814)
				(6948.21,0.2282215)
			};

				% ERGC %%%%%%%%%%%%%%%%%%%%%%%%%%%%%%%%%%%%%%%%%%%%%%%%%%%%%%%%%%%%
			\addplot[ERGC] coordinates{
				(401.35,0.1113375)
				(1206.2,0.1203625)
				(3663.22,0.1291375)
			};

				% ERS %%%%%%%%%%%%%%%%%%%%%%%%%%%%%%%%%%%%%%%%%%%%%%%%%%%%%%%%%%%%
			\addplot[ERS] coordinates{
				(400,0.690975)
				(1200,0.716835)
				(3600,0.759475)
			};

				% ETPS %%%%%%%%%%%%%%%%%%%%%%%%%%%%%%%%%%%%%%%%%%%%%%%%%%%%%%%%%%%%
			\addplot[ETPS] coordinates{
				(415.103,0.4326)
				(1227.44,0.663975)
				(3613.57,0.89831)
			};

				% FH %%%%%%%%%%%%%%%%%%%%%%%%%%%%%%%%%%%%%%%%%%%%%%%%%%%%%%%%%%%%
			\addplot[FH] coordinates{
				(1122.93,0.0826)
				(1248.22,0.04960015)
				(5426.45,0.083925)
			};

				% LSC %%%%%%%%%%%%%%%%%%%%%%%%%%%%%%%%%%%%%%%%%%%%%%%%%%%%%%%%%%%%
			\addplot[LSC] coordinates{
				(737.927,0.3253375)
				(1915.58,0.34095)
				(3930.7,0.35455)
			};

				% MSS %%%%%%%%%%%%%%%%%%%%%%%%%%%%%%%%%%%%%%%%%%%%%%%%%%%%%%%%%%%%
			\addplot[MSS] coordinates{
				(434.73,0.0425)
				(1428.43,0.04392505)
				(4593.88,0.04817515)
			};

				% PB %%%%%%%%%%%%%%%%%%%%%%%%%%%%%%%%%%%%%%%%%%%%%%%%%%%%%%%%%%%%
			\addplot[PB] coordinates{
				(479.885,0.0555125)
				(1237.46,0.0557375)
				(3547.41,0.0560375)
			};

				% PF %%%%%%%%%%%%%%%%%%%%%%%%%%%%%%%%%%%%%%%%%%%%%%%%%%%%%%%%%%%%
			\addplot[PF] coordinates{
				(576.292,0.00812595144125)
				(1709.54,0.00857078089125)
				(24640.7,0.00924426589)
			};

				% POISE %%%%%%%%%%%%%%%%%%%%%%%%%%%%%%%%%%%%%%%%%%%%%%%%%%%%%%%%%%%%
			\addplot[POISE] coordinates{
				(408.505,0.4904575)
				(1220.88,0.5138595)
				(3135.23,0.452092)
			};

				% PRESLIC %%%%%%%%%%%%%%%%%%%%%%%%%%%%%%%%%%%%%%%%%%%%%%%%%%%%%%%%%%%%
			\addplot[PRESLIC] coordinates{
				(573.018,0.03309995)
				(1262.35,0.03466245)
				(3822.55,0.03439995)
			};

				% RESEEDS %%%%%%%%%%%%%%%%%%%%%%%%%%%%%%%%%%%%%%%%%%%%%%%%%%%%%%%%%%%%
			\addplot[RESEEDS] coordinates{
				(399.04,0.05795)
				(1203.35,0.0582625)
				(3613.63,0.067875)
			};

				% RW %%%%%%%%%%%%%%%%%%%%%%%%%%%%%%%%%%%%%%%%%%%%%%%%%%%%%%%%%%%%
			\addplot[RW] coordinates{
				(418.008,4.4533525)
				(1237.46,13.5448045)
			};

				% SEEDS %%%%%%%%%%%%%%%%%%%%%%%%%%%%%%%%%%%%%%%%%%%%%%%%%%%%%%%%%%%%
			\addplot[SEEDS] coordinates{
				(448.677,1.0312)
				(1272.95,0.82956)
				(3683.12,1.986535)
			};

				% SLIC %%%%%%%%%%%%%%%%%%%%%%%%%%%%%%%%%%%%%%%%%%%%%%%%%%%%%%%%%%%%
			\addplot[SLIC] coordinates{
				(367.93,0.07615)
				(1150.35,0.0786125)
				(3565.77,0.0800375)
			};

				% TP %%%%%%%%%%%%%%%%%%%%%%%%%%%%%%%%%%%%%%%%%%%%%%%%%%%%%%%%%%%%
			\addplot[TP] coordinates{
				(526.722,8.4176275)
				(1410.02,7.379943)
				(3709.81,4.4517225)
			};

				% TPS %%%%%%%%%%%%%%%%%%%%%%%%%%%%%%%%%%%%%%%%%%%%%%%%%%%%%%%%%%%%
			\addplot[TPS] coordinates{
				(444.122,1.8130185)
				(1277.73,1.9906915)
				(3853.6,2.4631325)
			};

				% VC %%%%%%%%%%%%%%%%%%%%%%%%%%%%%%%%%%%%%%%%%%%%%%%%%%%%%%%%%%%%
			\addplot[VC] coordinates{
				(1853.82,1.950925)
				(3291.23,1.444615)
				(6366.5,1.250985)
			};

				% VLSLIC %%%%%%%%%%%%%%%%%%%%%%%%%%%%%%%%%%%%%%%%%%%%%%%%%%%%%%%%%%%%
			%\addplot[VLSLIC] coordinates{
			%	(0,0.1201375)
			%	(0,0.122825)
			%	(0,0.1242875)
			%};

				% W %%%%%%%%%%%%%%%%%%%%%%%%%%%%%%%%%%%%%%%%%%%%%%%%%%%%%%%%%%%%
			\addplot[W] coordinates{
				(408.6,0.007875)
				(1190.19,0.0085)
				(3783.59,0.00925)
			};

				% WP %%%%%%%%%%%%%%%%%%%%%%%%%%%%%%%%%%%%%%%%%%%%%%%%%%%%%%%%%%%%
			\addplot[WP] coordinates{
				(403.883,0.17930174563591)
				(1222.72,0.20900249376559)
				(3920.65,0.30491271820449)
			};

			\end{axis}
	\end{tikzpicture}
\end{subfigure}
\begin{subfigure}[b]{0.325\textwidth}\phantomsubcaption\label{subfig:appendix-experiments-runtime-fash}
	\begin{tikzpicture}
		\begin{axis}[AEFasht,ymode=log]
			% CCS %%%%%%%%%%%%%%%%%%%%%%%%%%%%%%%%%%%%%%%%%%%%%%%%%%%%%%%%%%%%
			\addplot[CCS] coordinates{
				(442.991,0.2348165)
				(1290.6,0.2341795)
				(4030.41,0.232106)
			};

				% CIS %%%%%%%%%%%%%%%%%%%%%%%%%%%%%%%%%%%%%%%%%%%%%%%%%%%%%%%%%%%%
			\addplot[CIS] coordinates{
				(498.555,4.225425)
				(1078.75,4.104835)
				(3324.5,3.608025)
			};

				% CRS %%%%%%%%%%%%%%%%%%%%%%%%%%%%%%%%%%%%%%%%%%%%%%%%%%%%%%%%%%%%
			\addplot[CRS] coordinates{
				(524.505,0.50214)
				(1520.54,0.77096)
				(4129.2,1.176845)
			};

				% CW %%%%%%%%%%%%%%%%%%%%%%%%%%%%%%%%%%%%%%%%%%%%%%%%%%%%%%%%%%%%
			\addplot[CW] coordinates{
				(409.708,0.00572355)
				(1205.46,0.00562635)
				(3643.52,0.00574515)
			};

				% EAMS %%%%%%%%%%%%%%%%%%%%%%%%%%%%%%%%%%%%%%%%%%%%%%%%%%%%%%%%%%%%
			\addplot[EAMS] coordinates{
				(382.151,0.451427)
				(1235.54,0.27248)
			};

				% ERGC %%%%%%%%%%%%%%%%%%%%%%%%%%%%%%%%%%%%%%%%%%%%%%%%%%%%%%%%%%%%
			\addplot[ERGC] coordinates{
				(400,0.079406)
				(1155,0.0860905)
				(3420,0.0898165)
			};

				% ERS %%%%%%%%%%%%%%%%%%%%%%%%%%%%%%%%%%%%%%%%%%%%%%%%%%%%%%%%%%%%
			\addplot[ERS] coordinates{
				(400,0.544525)
				(1200,0.532635)
				(3600,0.572115)
			};

				% ETPS %%%%%%%%%%%%%%%%%%%%%%%%%%%%%%%%%%%%%%%%%%%%%%%%%%%%%%%%%%%%
			\addplot[ETPS] coordinates{
				(425,0.3909935)
				(1247,0.4612745)
				(3750,0.527105)
			};

				% FH %%%%%%%%%%%%%%%%%%%%%%%%%%%%%%%%%%%%%%%%%%%%%%%%%%%%%%%%%%%%
			\addplot[FH] coordinates{
				(1017.45,0.057041)
				(1052.94,0.0596435)
				(4322.43,0.0373218)
			};

				% LSC %%%%%%%%%%%%%%%%%%%%%%%%%%%%%%%%%%%%%%%%%%%%%%%%%%%%%%%%%%%%
			\addplot[LSC] coordinates{
				(1013.2,0.400766)
				(2226.93,0.4251075)
				(3796.98,0.442819)
			};

				% MSS %%%%%%%%%%%%%%%%%%%%%%%%%%%%%%%%%%%%%%%%%%%%%%%%%%%%%%%%%%%%
			\addplot[MSS] coordinates{
				(436.387,0.03457875)
				(1406.86,0.03639305)
				(4598.5,0.04090715)
			};

				% PB %%%%%%%%%%%%%%%%%%%%%%%%%%%%%%%%%%%%%%%%%%%%%%%%%%%%%%%%%%%%
			\addplot[PB] coordinates{
				(537.451,0.0414148)
				(1286.56,0.0413824)
				(3693.23,0.04048605)
			};

				% PF %%%%%%%%%%%%%%%%%%%%%%%%%%%%%%%%%%%%%%%%%%%%%%%%%%%%%%%%%%%%
			\addplot[PF] coordinates{
				(606.404,0.0059945880799136)
				(1746.93,0.0063254501425486)
				(22669.1,0.0065265988131749)
			};

				% POISE %%%%%%%%%%%%%%%%%%%%%%%%%%%%%%%%%%%%%%%%%%%%%%%%%%%%%%%%%%%%
			\addplot[POISE] coordinates{
				(408.393,0.444891)
				(1215.99,0.4617155)
				(2709.65,0.391011)
			};

				% PRESLIC %%%%%%%%%%%%%%%%%%%%%%%%%%%%%%%%%%%%%%%%%%%%%%%%%%%%%%%%%%%%
			\addplot[PRESLIC] coordinates{
				(393.639,0.02224625)
				(1171.21,0.02068035)
				(3744.27,0.017635)
			};

				% RESEEDS %%%%%%%%%%%%%%%%%%%%%%%%%%%%%%%%%%%%%%%%%%%%%%%%%%%%%%%%%%%%
			\addplot[RESEEDS] coordinates{
				(400.423,0.04773215)
				(1200.08,0.04598275)
				(3700.18,0.062419)
			};

				% RW %%%%%%%%%%%%%%%%%%%%%%%%%%%%%%%%%%%%%%%%%%%%%%%%%%%%%%%%%%%%
			\addplot[RW] coordinates{
				(406.784,2.9520195)
				(1348.01,9.2815775)
				(4048.94,31.8738935)
			};

				% SEAW %%%%%%%%%%%%%%%%%%%%%%%%%%%%%%%%%%%%%%%%%%%%%%%%%%%%%%%%%%%%
			\addplot[SEAW] coordinates{
				(425.186,0.7322215)
				(1234.06,2.7089575)
				(4222.63,10.309943)
			};

				% SEEDS %%%%%%%%%%%%%%%%%%%%%%%%%%%%%%%%%%%%%%%%%%%%%%%%%%%%%%%%%%%%
			\addplot[SEEDS] coordinates{
				(472.406,0.7442)
				(1272.02,0.675315)
				(3773,2.038575)
			};

				% SLIC %%%%%%%%%%%%%%%%%%%%%%%%%%%%%%%%%%%%%%%%%%%%%%%%%%%%%%%%%%%%
			\addplot[SLIC] coordinates{
				(393.184,0.1090495)
				(1172.04,0.109957)
				(3748.69,0.1120415)
			};

				% TP %%%%%%%%%%%%%%%%%%%%%%%%%%%%%%%%%%%%%%%%%%%%%%%%%%%%%%%%%%%%
			\addplot[TP] coordinates{
				(555.011,7.0758865)
				(1441.55,7.144591)
				(3683.73,4.315332)
			};

				% TPS %%%%%%%%%%%%%%%%%%%%%%%%%%%%%%%%%%%%%%%%%%%%%%%%%%%%%%%%%%%%
			\addplot[TPS] coordinates{
				(425.052,1.2799025)
				(1378.51,1.4645955)
				(4065.15,1.988987)
			};

				% VC %%%%%%%%%%%%%%%%%%%%%%%%%%%%%%%%%%%%%%%%%%%%%%%%%%%%%%%%%%%%
			\addplot[VC] coordinates{
				(4299.32,1.29933)
				(6427.84,0.990845)
				(9930.68,0.960595)
			};

				% VLSLIC %%%%%%%%%%%%%%%%%%%%%%%%%%%%%%%%%%%%%%%%%%%%%%%%%%%%%%%%%%%%
			%\addplot[VLSLIC] coordinates{
			%	(0,0.62475)
			%	(0,0.7931)
			%	(0,0.662075)
			%};

				% W %%%%%%%%%%%%%%%%%%%%%%%%%%%%%%%%%%%%%%%%%%%%%%%%%%%%%%%%%%%%
			\addplot[W] coordinates{
				(433.637,0.00600435)
				(1246.48,0.0062311)
				(4018.14,0.0065875)
			};

				% WP %%%%%%%%%%%%%%%%%%%%%%%%%%%%%%%%%%%%%%%%%%%%%%%%%%%%%%%%%%%%
			\addplot[WP] coordinates{
				(425,0.15612068965517)
				(1247,0.18674568965517)
				(3750,0.27596982758621)
			};

			\end{axis}
	\end{tikzpicture}
\end{subfigure}

	\caption{Runtime in seconds $t$ on the \SBD, \SUNRGBD and \Fash datasets. The results allow
	to get an impression of how runtime of individual algorithms scales with the size
	of the image. In particular, we deduce that most algorithm's runtime scales linear
	in the input size, while the number of generated superpixels does have little influence.
	\textbf{Best viewed in color.}}
	\label{fig:appendix-experiments-runtime}
\end{figure*}
\begin{figure*}
	\centering
	\begin{subfigure}[b]{\fullthreeone\textwidth}\phantomsubcaption\label{subfig:appendix-experiments-bsds500-ue_np.mean_max}
	%%%%%%%%%%%%%%%%%%%%%%%%%%%%%%%%%%%%%%%%%%%%%%%%%%%%%%%%%%%%
	% ue_np.mean_max
	%%%%%%%%%%%%%%%%%%%%%%%%%%%%%%%%%%%%%%%%%%%%%%%%%%%%%%%%%%%%
	\begin{tikzpicture}
		\begin{axis}[EQBSDS500UE2,xmode=log]

			% CCS ue_np.mean_max %%%%%%%%%%%%%%%%%%%%%%%%%%%%%%%%%%%%%%%%%%%%%%%%%%%%%%%%%%%%
			\addplot[CCS] coordinates{
				(218.865,0.179397)
				(303.21,0.155508)
				(453.82,0.129268)
				(706.35,0.107203)
				(871.12,0.0988991)
				(1204.99,0.0876399)
				(1438.62,0.0821357)
				(1438.62,0.0821357)
				(1762.49,0.076469)
				(2107.24,0.0720313)
				(2107.24,0.0720313)
				(2702.61,0.0660912)
				(3406.7,0.0612614)
				(3406.7,0.0612614)
				(3406.7,0.0612614)
				(4654.44,0.0554056)
				(4654.44,0.0554056)
				(6619.1,0.0489194)
			};

			% SEEDS ue_np.mean_max %%%%%%%%%%%%%%%%%%%%%%%%%%%%%%%%%%%%%%%%%%%%%%%%%%%%%%%%%%%%
			\addplot[SEEDS] coordinates{
				(261.62,0.223076)
				(365.675,0.19367)
				(468.81,0.166744)
				(670.57,0.144635)
				(870.75,0.127624)
				(1087.4,0.120216)
				(1270.11,0.110512)
				(1451.85,0.104601)
				(1669.18,0.0958062)
				(1873.19,0.091841)
				%(2104.62,0.102696)
				(2462.77,0.0825942)
				(2793.43,0.0820788)
				(3260.86,0.0738429)
				(3895.78,0.0707603)
				(3895.78,0.0707603)
				(4846.12,0.0633063)
				(4846.12,0.0633063)
			};

			% SLIC ue_np.mean_max %%%%%%%%%%%%%%%%%%%%%%%%%%%%%%%%%%%%%%%%%%%%%%%%%%%%%%%%%%%%
			\addplot[SLIC] coordinates{
				(180.32,0.174507)
				(256.725,0.14981)
				(368.57,0.133617)
				(575.335,0.113918)
				(726.36,0.10487)
				(1002.87,0.0957898)
				(1203.61,0.0912283)
				(1203.61,0.0912283)
				(1475.69,0.0857781)
				(1814.8,0.0785351)
				(1814.8,0.0785351)
				(2334.56,0.0724657)
				(3038.13,0.0657068)
				(3038.13,0.0657068)
				(3038.13,0.0657068)
				(4188.97,0.0589157)
				(4188.97,0.0589157)
				(6139.41,0.0512524)
			};

			% RW ue_np.mean_max %%%%%%%%%%%%%%%%%%%%%%%%%%%%%%%%%%%%%%%%%%%%%%%%%%%%%%%%%%%%
			\addplot[RW] coordinates{
				(211.565,0.22099)
				(313.64,0.18514)
				(407.125,0.163871)
				(649.355,0.131395)
				(858.65,0.117308)
				(1032.29,0.10929)
				(1280.94,0.0999393)
				(1505.72,0.0932428)
				(1649.74,0.0900246)
				(1940.98,0.0854435)
				(2101.2,0.0821806)
				(2570.18,0.0765988)
				(2916.36,0.0728736)
				(3413.13,0.0694277)
				(3882.85,0.0661343)
				(4457.4,0.0624712)
				(5093.4,0.0609616)
				(5663.8,0.0527803)
			};

			% CW ue_np.mean_max %%%%%%%%%%%%%%%%%%%%%%%%%%%%%%%%%%%%%%%%%%%%%%%%%%%%%%%%%%%%
			\addplot[CW] coordinates{
				(197.775,0.193019)
				(309.29,0.160417)
				(405.935,0.144929)
				(609.18,0.125212)
				(807.005,0.112944)
				(1045.56,0.10316)
				(1252.72,0.0974855)
				(1492.83,0.0916924)
				(1649.62,0.0889166)
				(1816.43,0.0858197)
				(2091.79,0.0822643)
				(2565.8,0.0769704)
				(2917.57,0.0738519)
				(3319.52,0.0706267)
				(4023.98,0.0661988)
				(4023.98,0.0661988)
				(4639.83,0.0628539)
				(5846.73,0.0586372)
			};

			% TP ue_np.mean_max %%%%%%%%%%%%%%%%%%%%%%%%%%%%%%%%%%%%%%%%%%%%%%%%%%%%%%%%%%%%
			\addplot[TP] coordinates{
				(283.695,0.170092)
				(381.285,0.146071)
				(555.1,0.121786)
				(752.29,0.110206)
				(1007.04,0.0999947)
				(1292.4,0.09124)
				(1495.14,0.0874474)
				%(2017.73,0.0993508)
				(2478.02,0.0830604)
				(2831,0.0770876)
				(3018.75,0.079288)
			};

			% POISE ue_np.mean_max %%%%%%%%%%%%%%%%%%%%%%%%%%%%%%%%%%%%%%%%%%%%%%%%%%%%%%%%%%%%
			\addplot[POISE] coordinates{
				(204.855,0.165397)
				(306.68,0.140885)
				(408.385,0.128172)
				(611.79,0.112118)
				(814.975,0.103144)
				(1017.48,0.0972514)
				(1216.26,0.0924521)
				(1409.36,0.0894045)
				(1587.37,0.0869714)
				(1749.07,0.085415)
				(1887.06,0.0842683)
				(2090.32,0.08283)
				(2221.84,0.0821898)
				(2278.76,0.0819765)
				(2287.5,0.0819418)
				(2288.94,0.0819369)
				(2288.94,0.0819369)
				(2288.94,0.0819369)
			};

			% FH ue_np.mean_max %%%%%%%%%%%%%%%%%%%%%%%%%%%%%%%%%%%%%%%%%%%%%%%%%%%%%%%%%%%%
			\addplot[FH] coordinates{
				(628.745,0.182601)
				(799.09,0.150389)
				(963.36,0.140866)
				(1090.39,0.131649)
				(1187.04,0.125432)
				(1605.71,0.110407)
				(2533.01,0.0877137)
				(3000.74,0.0845623)
				(3219.63,0.0821065)
				(3814.42,0.0819198)
				(4746.96,0.0709097)
			};

			% EAMS ue_np.mean_max %%%%%%%%%%%%%%%%%%%%%%%%%%%%%%%%%%%%%%%%%%%%%%%%%%%%%%%%%%%%
			\addplot[EAMS] coordinates{
				(261.12,0.165201)
				(283.33,0.159604)
				(309.87,0.153536)
				(383.52,0.140069)
				(499.435,0.126177)
				(725.36,0.110582)
				(1417.35,0.0873781)
				(1473.6,0.0862626)
				(1534.33,0.0851062)
				(1883.59,0.0823172)
				(2063.65,0.079916)
				(2346.71,0.0765394)
				(2642.78,0.0737047)
				(3113.25,0.0724)
				(3640.44,0.0687855)
				(4914.12,0.0624623)
				(8189.58,0.0539269)
			};

			% CRS ue_np.mean_max %%%%%%%%%%%%%%%%%%%%%%%%%%%%%%%%%%%%%%%%%%%%%%%%%%%%%%%%%%%%
			\addplot[CRS] coordinates{
				(299.575,0.143179)
				(449.585,0.123216)
				(635.865,0.108806)
				(895.935,0.0960168)
				(1180.31,0.0873514)
				(1530.1,0.0799281)
				(1758.22,0.0763857)
				(2057.15,0.0722613)
				(2308.66,0.0694392)
				(2461.75,0.0678385)
				(2755.4,0.065272)
				(3401.31,0.0606573)
				(3879.51,0.057869)
				(4346.83,0.0557048)
				(5208.23,0.0521298)
				(5208.23,0.0521298)
				(5821.7,0.0499159)
				(7241.41,0.0460821)
			};

			% SEAW ue_np.mean_max %%%%%%%%%%%%%%%%%%%%%%%%%%%%%%%%%%%%%%%%%%%%%%%%%%%%%%%%%%%%
			\addplot[SEAW] coordinates{
				(165.505,0.374019)
				(356.805,0.224315)
				(923.49,0.130425)
				(2933.9,0.0762253)
				(10727.5,0.0448995)
			};

			% RESEEDS ue_np.mean_max %%%%%%%%%%%%%%%%%%%%%%%%%%%%%%%%%%%%%%%%%%%%%%%%%%%%%%%%%%%%
			\addplot[RESEEDS] coordinates{
				(200.795,0.185835)
				(301.525,0.163531)
				(401.465,0.132874)
				(602.33,0.113933)
				(800.99,0.10611)
				(1020.12,0.0994214)
				(1201.34,0.0923218)
				(1378.1,0.087877)
				(1601.55,0.0800663)
				(1802.11,0.0781975)
				(2040.11,0.0832874)
				(2402.13,0.0699599)
				(2720.13,0.0694846)
				(3200.2,0.0644983)
				(3840.22,0.0617858)
				(3840.22,0.0617858)
				(4800.37,0.0561646)
				(4800.37,0.0561646)
			};

			% ERGC ue_np.mean_max %%%%%%%%%%%%%%%%%%%%%%%%%%%%%%%%%%%%%%%%%%%%%%%%%%%%%%%%%%%%
			\addplot[ERGC] coordinates{
				(196,0.159735)
				(306,0.133944)
				(400,0.121812)
				(600,0.104949)
				(792.57,0.0954413)
				(1024,0.0878287)
				(1224,0.0831799)
				(1453.2,0.0784696)
				(1600,0.0761292)
				(1760,0.0737437)
				(2024,0.0707404)
				(2472.98,0.0666267)
				(2809,0.0641758)
				(3180,0.0612931)
				(3840,0.0581234)
				(3840,0.0581234)
				(4416,0.0553687)
				(5520,0.0518964)
			};

			% PF ue_np.mean_max %%%%%%%%%%%%%%%%%%%%%%%%%%%%%%%%%%%%%%%%%%%%%%%%%%%%%%%%%%%%
			\addplot[PF] coordinates{
				(281.06,0.333921)
				(428.82,0.293331)
				(585.08,0.265215)
				(928.805,0.227506)
				(1172.76,0.207163)
				(1602.65,0.186339)
				(1861.72,0.174457)
				(2267.11,0.16405)
				(2697.08,0.153131)
				(3382.15,0.144579)
				(4207.66,0.133761)
				(6051.74,0.116788)
			};

			% TPS ue_np.mean_max %%%%%%%%%%%%%%%%%%%%%%%%%%%%%%%%%%%%%%%%%%%%%%%%%%%%%%%%%%%%
			\addplot[TPS] coordinates{
				(224.26,0.196998)
				(903.595,0.115562)
				(1130.53,0.106762)
				(1332.49,0.100735)
				(1556.56,0.0961695)
				(1736.84,0.0913408)
				(1988.95,0.0869456)
				(2167.04,0.0838316)
				(2656.03,0.0790315)
				(2987.25,0.0744852)
				(3505.61,0.0708157)
				(4030.29,0.0663199)
				(4568.25,0.0642427)
				(5242.04,0.0611118)
				(5978.67,0.0581679)
			};

			% NC ue_np.mean_max %%%%%%%%%%%%%%%%%%%%%%%%%%%%%%%%%%%%%%%%%%%%%%%%%%%%%%%%%%%%
			\addplot[NC] coordinates{
				(223.505,0.146866)
				(321.69,0.130709)
				(416.03,0.121517)
				(594.185,0.111767)
				(763.865,0.105481)
				(922.645,0.101732)
				(1073.6,0.0986864)
				(1213.8,0.0961924)
				(1348.36,0.0944127)
				(1473.43,0.0930103)
				(1591.75,0.0915994)
				(2000.27,0.087882)
			};

			% VC ue_np.mean_max %%%%%%%%%%%%%%%%%%%%%%%%%%%%%%%%%%%%%%%%%%%%%%%%%%%%%%%%%%%%
			\addplot[VC] coordinates{
				(351.72,0.377812)
				(498.095,0.281205)
				(712.82,0.209932)
				(994.125,0.155886)
				(1201.27,0.13293)
				(1375.27,0.118151)
				(1685.08,0.101915)
				(1968.97,0.0920285)
				(2224.75,0.0855216)
				(2476.43,0.0806172)
				(2712.29,0.0769346)
				(2948.04,0.0739811)
				(3169.69,0.0713155)
				(3389.95,0.0689737)
				(3804.08,0.0655697)
				(4196.75,0.0628818)
				(4566.66,0.0605806)
				(4815.37,0.0591216)
				(5157.48,0.0573595)
				(5401.49,0.0558277)
			};

			% PB ue_np.mean_max %%%%%%%%%%%%%%%%%%%%%%%%%%%%%%%%%%%%%%%%%%%%%%%%%%%%%%%%%%%%
			\addplot[PB] coordinates{
				(324.39,0.243429)
				(398.425,0.209432)
				(494.17,0.180692)
				(692.845,0.147203)
				(853.905,0.131623)
				(1129.27,0.115345)
				(1323.88,0.10663)
				(1323.88,0.10663)
				(1601.37,0.0979916)
				(1929.58,0.0897623)
				(1929.58,0.0897623)
				(2453.31,0.0812174)
				(3126.91,0.0732926)
				(3126.91,0.0732926)
				(3126.91,0.0732926)
				(4270.56,0.0646975)
				(4270.56,0.0646975)
				(6122.31,0.0565567)
			};

			% PRESLIC ue_np.mean_max %%%%%%%%%%%%%%%%%%%%%%%%%%%%%%%%%%%%%%%%%%%%%%%%%%%%%%%%%%%%
			\addplot[PRESLIC] coordinates{
				(369,0.15315)
				(581.54,0.127686)
				(734.575,0.117583)
				(1020.3,0.103212)
				(1229.22,0.0957013)
				(1229.22,0.0957013)
				(1511.3,0.0880409)
				(1838.94,0.0822743)
				(1838.94,0.0822743)
				(2375.99,0.0748201)
				(3059.43,0.068826)
				(3059.43,0.068826)
				(3059.43,0.068826)
				(4218.88,0.0614387)
				(4218.88,0.0614387)
				(6137.8,0.0537188)
			};

			% W ue_np.mean_max %%%%%%%%%%%%%%%%%%%%%%%%%%%%%%%%%%%%%%%%%%%%%%%%%%%%%%%%%%%%
			\addplot[W] coordinates{
				(188.285,0.227112)
				(296.48,0.182095)
				(387.92,0.162436)
				(608.055,0.133801)
				(794.47,0.120306)
				(1100.87,0.10578)
				(1302.83,0.0990923)
				(1302.83,0.0990923)
				(1572.26,0.0928368)
				(1960.24,0.086063)
				(1960.24,0.086063)
				(2478.43,0.078717)
				(3292.41,0.0715909)
				(3292.41,0.0715909)
				(3292.41,0.0715909)
				(4439.31,0.0650595)
				(4439.31,0.0650595)
				(6526.67,0.0568372)
			};

			% LSC ue_np.mean_max %%%%%%%%%%%%%%%%%%%%%%%%%%%%%%%%%%%%%%%%%%%%%%%%%%%%%%%%%%%%
			\addplot[LSC] coordinates{
				(548.755,0.143054)
				(756.92,0.122502)
				(1045.31,0.107074)
				(1360.95,0.0969879)
				(1665.85,0.0900165)
				(1919.65,0.0855723)
				(2055.98,0.0836367)
				(2239.5,0.081482)
				(2376.71,0.0801275)
				(2441.63,0.0791931)
				(2603.55,0.0777619)
				(2909.2,0.0759016)
				(3147.7,0.0749653)
				(3373.12,0.0732703)
				(3762.88,0.0734845)
				(3762.88,0.0734845)
				(4057.29,0.0738871)
				(4401.67,0.0803919)
			};

			% WP ue_np.mean_max %%%%%%%%%%%%%%%%%%%%%%%%%%%%%%%%%%%%%%%%%%%%%%%%%%%%%%%%%%%%
			\addplot[WP] coordinates{
				(216,0.189755)
				(294,0.164562)
				(384,0.148291)
				(600,0.125216)
				(805,0.112475)
				(1080,0.101169)
				(1320,0.0945852)
				(1536,0.0892273)
				(1536,0.0889869)
				(1944,0.0826232)
				(1944,0.082478)
				(2400,0.0765399)
				(3174,0.0702651)
				(3174,0.0701382)
				(4320,0.0639556)
				(4320,0.0639556)
				(4320,0.0639556)
				(5836.85,0.057387)
			};

			% QS ue_np.mean_max %%%%%%%%%%%%%%%%%%%%%%%%%%%%%%%%%%%%%%%%%%%%%%%%%%%%%%%%%%%%
			\addplot[QS] coordinates{
				(324.725,0.631546)
				(468.655,0.481517)
				(615,0.381608)
				(884.79,0.27222)
				(984.505,0.24583)
				(1275.62,0.193321)
				(1500.06,0.168415)
				(1646.89,0.155863)
				(1824.2,0.143951)
				(2040.32,0.132493)
				(2303.69,0.121701)
				(2626.4,0.111059)
				(3066.62,0.10053)
				(3646.03,0.0906887)
				(4444.58,0.0812972)
				(5554.87,0.072386)
			};

			% VLSLIC ue_np.mean_max %%%%%%%%%%%%%%%%%%%%%%%%%%%%%%%%%%%%%%%%%%%%%%%%%%%%%%%%%%%%
			\addplot[VLSLIC] coordinates{
				(575.975,0.146745)
				(651.86,0.133883)
				(763.62,0.12431)
				(899,0.112635)
				(988.985,0.106043)
				(1193.67,0.0981205)
				(1348.08,0.0935383)
				(1349.01,0.0929579)
				(1579.08,0.0889043)
				(1849.65,0.0848381)
				(1857.27,0.0842664)
				(2307.73,0.0796918)
				(2890.56,0.0748332)
				(2924.52,0.0736624)
				(2954.18,0.0727692)
				(3858.98,0.0707121)
				(3858.98,0.0707121)
				(4809.57,0.077104)
			};

			% CIS ue_np.mean_max %%%%%%%%%%%%%%%%%%%%%%%%%%%%%%%%%%%%%%%%%%%%%%%%%%%%%%%%%%%%
			\addplot[CIS] coordinates{
				(317.63,0.163639)
				(411.595,0.14103)
				(472.685,0.132016)
				(633.345,0.115086)
				(777.05,0.106079)
				(969.085,0.0968127)
				(1150.5,0.0910555)
				(1168.82,0.0908149)
				(1371.27,0.085803)
				(1637.39,0.080614)
				(1667.02,0.0803034)
				(2056.01,0.0750905)
				(3126.04,0.0675849)
				(3625.6,0.0653815)
				(4558.92,0.0600433)
				(4558.92,0.0600433)
				(6448.65,0.0528125)
			};

			% ERS ue_np.mean_max %%%%%%%%%%%%%%%%%%%%%%%%%%%%%%%%%%%%%%%%%%%%%%%%%%%%%%%%%%%%
			\addplot[ERS] coordinates{
				(200,0.165125)
				(300,0.143245)
				(400,0.128528)
				(600,0.112156)
				(800,0.101957)
				(1000,0.0951928)
				(1200,0.0897034)
				(1400,0.0855294)
				(1600,0.0816775)
				(1800,0.0787337)
				(2000,0.0762525)
				(2400,0.071886)
				(2800,0.0683312)
				(3200,0.0654288)
				(3600,0.0629577)
				(4000,0.0606193)
				(4600,0.0576594)
				(5200,0.0552272)
			};

			% MSS ue_np.mean_max %%%%%%%%%%%%%%%%%%%%%%%%%%%%%%%%%%%%%%%%%%%%%%%%%%%%%%%%%%%%
			\addplot[MSS] coordinates{
				(200.765,0.191002)
				(282.33,0.169581)
				(420.66,0.135999)
				(664.115,0.114166)
				(837.885,0.107277)
				(1179.3,0.0953561)
				(1422.34,0.0892904)
				(1423.04,0.08919)
				(1768,0.0832694)
				(2156.96,0.079052)
				(2157.01,0.0788515)
				(2826.35,0.0723073)
				(3669.97,0.0682022)
				(3670.34,0.0682585)
				(3669.92,0.0681392)
				(5213.45,0.0616563)
				(5213.45,0.0616563)
				(7846.35,0.0546919)
			};

			% ETPS ue_np.mean_max %%%%%%%%%%%%%%%%%%%%%%%%%%%%%%%%%%%%%%%%%%%%%%%%%%%%%%%%%%%%
			\addplot[ETPS] coordinates{
				(216,0.155646)
				(294,0.136541)
				(425,0.120575)
				(651,0.102522)
				(805,0.0941807)
				(1107,0.0842104)
				(1320,0.0803088)
				(1320,0.0803088)
				(1617,0.0745378)
				(1944,0.0703869)
				(1944,0.0703869)
				(2501,0.0644105)
				(3174,0.059511)
				(3174,0.059511)
				(3174,0.059511)
				(4374,0.0532)
				(4374,0.0532)
				(6305,0.047044)
			};

		\end{axis}
	\end{tikzpicture}
\end{subfigure}
\begin{subfigure}[b]{0.325\textwidth}\phantomsubcaption\label{subfig:appendix-experiments-bsds500-asa.mean_min}
	%%%%%%%%%%%%%%%%%%%%%%%%%%%%%%%%%%%%%%%%%%%%%%%%%%%%%%%%%%%%
	% asa.mean_min
	%%%%%%%%%%%%%%%%%%%%%%%%%%%%%%%%%%%%%%%%%%%%%%%%%%%%%%%%%%%%
	\begin{tikzpicture}
		\begin{axis}[AEBSDS500ASAMeanMin,xmode=log]

			% CCS asa.mean_min %%%%%%%%%%%%%%%%%%%%%%%%%%%%%%%%%%%%%%%%%%%%%%%%%%%%%%%%%%%%
			\addplot[CCS] coordinates{
				(218.865,0.908208)
				(303.21,0.920773)
				(453.82,0.934442)
				(706.35,0.945844)
				(871.12,0.950106)
				(1204.99,0.955879)
				(1438.62,0.958669)
				(1438.62,0.958669)
				(1762.49,0.961559)
				(2107.24,0.963813)
				(2107.24,0.963813)
				(2702.61,0.966827)
				(3406.7,0.969259)
				(3406.7,0.969259)
				(3406.7,0.969259)
				(4654.44,0.972214)
				(4654.44,0.972214)
				(6619.1,0.975478)
			};

			% SEEDS asa.mean_min %%%%%%%%%%%%%%%%%%%%%%%%%%%%%%%%%%%%%%%%%%%%%%%%%%%%%%%%%%%%
			\addplot[SEEDS] coordinates{
				(261.62,0.884077)
				(365.675,0.900092)
				(468.81,0.914361)
				(670.57,0.926247)
				(870.75,0.935162)
				(1087.4,0.939089)
				(1270.11,0.944106)
				(1451.85,0.947074)
				(1669.18,0.951623)
				(1873.19,0.953645)
				%(2104.62,0.948101)
				(2462.77,0.958403)
				(2793.43,0.958627)
				(3260.86,0.962842)
				(3895.78,0.964413)
				(3895.78,0.964413)
				(4846.12,0.968203)
				(4846.12,0.968203)
			};

			% SLIC asa.mean_min %%%%%%%%%%%%%%%%%%%%%%%%%%%%%%%%%%%%%%%%%%%%%%%%%%%%%%%%%%%%
			\addplot[SLIC] coordinates{
				(180.32,0.910757)
				(256.725,0.92385)
				(368.57,0.932248)
				(575.335,0.942485)
				(726.36,0.947129)
				(1002.87,0.951751)
				(1203.61,0.954087)
				(1203.61,0.954087)
				(1475.69,0.956857)
				(1814.8,0.960531)
				(1814.8,0.960531)
				(2334.56,0.963595)
				(3038.13,0.967022)
				(3038.13,0.967022)
				(3038.13,0.967022)
				(4188.97,0.970441)
				(4188.97,0.970441)
				(6139.41,0.974301)
			};

			% RW asa.mean_min %%%%%%%%%%%%%%%%%%%%%%%%%%%%%%%%%%%%%%%%%%%%%%%%%%%%%%%%%%%%
			\addplot[RW] coordinates{
				(211.565,0.886994)
				(313.64,0.905628)
				(407.125,0.91683)
				(649.355,0.933607)
				(858.65,0.940794)
				(1032.29,0.944952)
				(1280.94,0.949692)
				(1505.72,0.95311)
				(1649.74,0.95475)
				(1940.98,0.957075)
				(2101.2,0.958719)
				(2570.18,0.961543)
				(2916.36,0.963429)
				(3413.13,0.965164)
				(3882.85,0.966806)
				(4457.4,0.96866)
				(5093.4,0.969419)
				(5663.8,0.973556)
			};

			% CW asa.mean_min %%%%%%%%%%%%%%%%%%%%%%%%%%%%%%%%%%%%%%%%%%%%%%%%%%%%%%%%%%%%
			\addplot[CW] coordinates{
				(197.775,0.901507)
				(309.29,0.91852)
				(405.935,0.926499)
				(609.18,0.936762)
				(807.005,0.943036)
				(1045.56,0.948021)
				(1252.72,0.950956)
				(1492.83,0.953901)
				(1649.62,0.955318)
				(1816.43,0.956875)
				(2091.79,0.958675)
				(2565.8,0.96135)
				(2917.57,0.962912)
				(3319.52,0.964558)
				(4023.98,0.966786)
				(4023.98,0.966786)
				(4639.83,0.968479)
				(5846.73,0.970608)
			};

			% TP asa.mean_min %%%%%%%%%%%%%%%%%%%%%%%%%%%%%%%%%%%%%%%%%%%%%%%%%%%%%%%%%%%%
			\addplot[TP] coordinates{
				(283.695,0.913616)
				(381.285,0.926023)
				(555.1,0.938535)
				(752.29,0.944488)
				(1007.04,0.949624)
				(1292.4,0.954141)
				(1495.14,0.956055)
				(1696.62,0.9342)
				%(2017.73,0.949655)
				(2478.02,0.958124)
				(2831,0.961212)
				(3018.75,0.960084)
			};

			% POISE asa.mean_min %%%%%%%%%%%%%%%%%%%%%%%%%%%%%%%%%%%%%%%%%%%%%%%%%%%%%%%%%%%%
			\addplot[POISE] coordinates{
				(204.855,0.91549)
				(306.68,0.92844)
				(408.385,0.934974)
				(611.79,0.943281)
				(814.975,0.947928)
				(1017.48,0.950919)
				(1216.26,0.953356)
				(1409.36,0.95492)
				(1587.37,0.956162)
				(1749.07,0.956956)
				(1887.06,0.957528)
				(2090.32,0.958254)
				(2221.84,0.958577)
				(2278.76,0.958683)
				(2287.5,0.9587)
				(2288.94,0.958703)
				(2288.94,0.958703)
				(2288.94,0.958703)
			};

			% FH asa.mean_min %%%%%%%%%%%%%%%%%%%%%%%%%%%%%%%%%%%%%%%%%%%%%%%%%%%%%%%%%%%%
			\addplot[FH] coordinates{
				(628.745,0.906702)
				(963.36,0.928572)
				(782.42,0.917747)
				(799.09,0.923939)
				(1090.39,0.933565)
				(1187.04,0.936596)
				(1605.71,0.944418)
				(2533.01,0.955877)
				(3000.74,0.957532)
				(3219.63,0.958764)
				(3814.42,0.958755)
				(4746.96,0.964367)
			};

			% EAMS asa.mean_min %%%%%%%%%%%%%%%%%%%%%%%%%%%%%%%%%%%%%%%%%%%%%%%%%%%%%%%%%%%%
			\addplot[EAMS] coordinates{
				(261.12,0.915669)
				(283.33,0.918584)
				(309.87,0.921742)
				(383.52,0.928822)
				(499.435,0.936068)
				(725.36,0.944139)
				(1417.35,0.955994)
				(1473.6,0.95656)
				(1534.33,0.957152)
				(1883.59,0.958419)
				(2063.65,0.959637)
				(2346.71,0.961342)
				(2642.78,0.962781)
				(3113.25,0.963149)
				(3640.44,0.964982)
				(4914.12,0.968165)
				(8189.58,0.972404)
			};

			% CRS asa.mean_min %%%%%%%%%%%%%%%%%%%%%%%%%%%%%%%%%%%%%%%%%%%%%%%%%%%%%%%%%%%%
			\addplot[CRS] coordinates{
				(299.575,0.92645)
				(449.585,0.937016)
				(635.865,0.944625)
				(895.935,0.951366)
				(1180.31,0.955836)
				(1530.1,0.959638)
				(1758.22,0.961485)
				(2057.15,0.963602)
				(2308.66,0.965031)
				(2461.75,0.965848)
				(2755.4,0.967154)
				(3401.31,0.969501)
				(3879.51,0.970913)
				(4346.83,0.97201)
				(5208.23,0.973821)
				(5208.23,0.973821)
				(5821.7,0.974937)
				(7241.41,0.976875)
			};

			% SEAW asa.mean_min %%%%%%%%%%%%%%%%%%%%%%%%%%%%%%%%%%%%%%%%%%%%%%%%%%%%%%%%%%%%
			\addplot[SEAW] coordinates{
				(165.505,0.801376)
				(356.805,0.884676)
				(923.49,0.934049)
				(2933.9,0.961714)
				(10727.5,0.977502)
			};

			% RESEEDS asa.mean_min %%%%%%%%%%%%%%%%%%%%%%%%%%%%%%%%%%%%%%%%%%%%%%%%%%%%%%%%%%%%
			\addplot[RESEEDS] coordinates{
				(200.795,0.904322)
				(301.525,0.916377)
				(401.465,0.932239)
				(602.33,0.94219)
				(800.99,0.946255)
				(1020.12,0.949689)
				(1201.34,0.953375)
				(1378.1,0.955598)
				(1601.55,0.959625)
				(1802.11,0.960559)
				(2040.11,0.957986)
				(2402.13,0.964796)
				(2720.13,0.964983)
				(3200.2,0.967564)
				(3840.22,0.968937)
				(3840.22,0.968937)
				(4800.37,0.971788)
				(4800.37,0.971788)
			};

			% ERGC asa.mean_min %%%%%%%%%%%%%%%%%%%%%%%%%%%%%%%%%%%%%%%%%%%%%%%%%%%%%%%%%%%%
			\addplot[ERGC] coordinates{
				(196,0.918495)
				(306,0.932118)
				(400,0.938316)
				(600,0.947015)
				(792.57,0.951865)
				(1024,0.955781)
				(1224,0.95813)
				(1453.2,0.960561)
				(1600,0.961732)
				(1760,0.962925)
				(2024,0.964459)
				(2472.98,0.96654)
				(2809,0.96778)
				(3180,0.969229)
				(3840,0.970828)
				(3840,0.970828)
				(4416,0.972222)
				(5520,0.973971)
			};

			% PF asa.mean_min %%%%%%%%%%%%%%%%%%%%%%%%%%%%%%%%%%%%%%%%%%%%%%%%%%%%%%%%%%%%
			\addplot[PF] coordinates{
				(281.06,0.825828)
				(428.82,0.848454)
				(585.08,0.863402)
				(928.805,0.883547)
				(1172.76,0.894387)
				(1602.65,0.905226)
				(1861.72,0.911338)
				(2267.11,0.916732)
				(2697.08,0.922384)
				(3382.15,0.926826)
				(4207.66,0.932373)
				(6051.74,0.941031)
			};

			% TPS asa.mean_min %%%%%%%%%%%%%%%%%%%%%%%%%%%%%%%%%%%%%%%%%%%%%%%%%%%%%%%%%%%%
			\addplot[TPS] coordinates{
				(224.26,0.898478)
				(903.595,0.941544)
				(1130.53,0.946109)
				(1332.49,0.949202)
				(1556.56,0.951527)
				(1736.84,0.953991)
				(1988.95,0.956229)
				(2167.04,0.957829)
				(2656.03,0.960255)
				(2987.25,0.962574)
				(3505.61,0.964433)
				(4030.29,0.9667)
				(4568.25,0.967752)
				(5242.04,0.969335)
				(5978.67,0.970815)
			};

			% NC asa.mean_min %%%%%%%%%%%%%%%%%%%%%%%%%%%%%%%%%%%%%%%%%%%%%%%%%%%%%%%%%%%%
			\addplot[NC] coordinates{
				(223.505,0.925043)
				(321.69,0.933535)
				(416.03,0.938303)
				(594.185,0.943387)
				(763.865,0.946675)
				(922.645,0.948591)
				(1073.6,0.950155)
				(1213.8,0.951389)
				(1348.36,0.952333)
				(1473.43,0.953033)
				(1591.75,0.953768)
				(2000.27,0.955673)
			};

			% VC asa.mean_min %%%%%%%%%%%%%%%%%%%%%%%%%%%%%%%%%%%%%%%%%%%%%%%%%%%%%%%%%%%%
			\addplot[VC] coordinates{
				(351.72,0.796433)
				(498.095,0.852401)
				(712.82,0.891693)
				(994.125,0.920447)
				(1201.27,0.932453)
				(1375.27,0.940167)
				(1685.08,0.948545)
				(1968.97,0.95364)
				(2224.75,0.95694)
				(2476.43,0.959464)
				(2712.29,0.961329)
				(2948.04,0.96283)
				(3169.69,0.964172)
				(3389.95,0.965361)
				(3804.08,0.967091)
				(4196.75,0.968446)
				(4566.66,0.969604)
				(4815.37,0.970341)
				(5157.48,0.97123)
				(5401.49,0.972003)
			};

			% PB asa.mean_min %%%%%%%%%%%%%%%%%%%%%%%%%%%%%%%%%%%%%%%%%%%%%%%%%%%%%%%%%%%%
			\addplot[PB] coordinates{
				(324.39,0.875534)
				(398.425,0.893461)
				(494.17,0.908327)
				(692.845,0.925643)
				(853.905,0.933595)
				(1129.27,0.941927)
				(1323.88,0.946346)
				(1323.88,0.946346)
				(1601.37,0.950722)
				(1929.58,0.954903)
				(1929.58,0.954903)
				(2453.31,0.959221)
				(3126.91,0.96322)
				(3126.91,0.96322)
				(3126.91,0.96322)
				(4270.56,0.967547)
				(4270.56,0.967547)
				(6122.31,0.971644)
			};

			% PRESLIC asa.mean_min %%%%%%%%%%%%%%%%%%%%%%%%%%%%%%%%%%%%%%%%%%%%%%%%%%%%%%%%%%%%
			\addplot[PRESLIC] coordinates{
				(369,0.922185)
				(581.54,0.935368)
				(734.575,0.940584)
				(1020.3,0.947965)
				(1229.22,0.951815)
				(1229.22,0.951815)
				(1511.3,0.95571)
				(1838.94,0.95865)
				(1838.94,0.95865)
				(2375.99,0.962417)
				(3059.43,0.965455)
				(3059.43,0.965455)
				(3059.43,0.965455)
				(4218.88,0.969175)
				(4218.88,0.969175)
				(6137.8,0.973067)
			};

			% W asa.mean_min %%%%%%%%%%%%%%%%%%%%%%%%%%%%%%%%%%%%%%%%%%%%%%%%%%%%%%%%%%%%
			\addplot[W] coordinates{
				(188.285,0.883158)
				(296.48,0.907246)
				(387.92,0.917563)
				(608.055,0.932282)
				(794.47,0.939215)
				(1100.87,0.946726)
				(1302.83,0.950111)
				(1302.83,0.950111)
				(1572.26,0.953284)
				(1960.24,0.956776)
				(1960.24,0.956776)
				(2478.43,0.960444)
				(3292.41,0.96407)
				(3292.41,0.96407)
				(3292.41,0.96407)
				(4439.31,0.967356)
				(4439.31,0.967356)
				(6526.67,0.971495)
			};

			% LSC asa.mean_min %%%%%%%%%%%%%%%%%%%%%%%%%%%%%%%%%%%%%%%%%%%%%%%%%%%%%%%%%%%%
			\addplot[LSC] coordinates{
				(548.755,0.927231)
				(756.92,0.937919)
				(1045.31,0.945915)
				(1360.95,0.951092)
				(1665.85,0.954629)
				(1919.65,0.956918)
				(2055.98,0.957899)
				(2239.5,0.959009)
				(2376.71,0.959695)
				(2441.63,0.960189)
				(2603.55,0.960892)
				(2909.2,0.961847)
				(3147.7,0.962324)
				(3373.12,0.963174)
				(3762.88,0.963057)
				(3762.88,0.963057)
				(4057.29,0.962842)
				(4401.67,0.959479)
			};

			% WP asa.mean_min %%%%%%%%%%%%%%%%%%%%%%%%%%%%%%%%%%%%%%%%%%%%%%%%%%%%%%%%%%%%
			\addplot[WP] coordinates{
				(216,0.90359)
				(294,0.916563)
				(384,0.924955)
				(600,0.936866)
				(805,0.943363)
				(1080,0.949141)
				(1320,0.952474)
				(1536,0.955168)
				(1536,0.955284)
				(1944,0.958516)
				(1944,0.95859)
				(2400,0.961584)
				(3174,0.964758)
				(3174,0.964822)
				(4320,0.967928)
				(4320,0.967928)
				(4320,0.967928)
				(5836.85,0.971237)
			};

			% QS asa.mean_min %%%%%%%%%%%%%%%%%%%%%%%%%%%%%%%%%%%%%%%%%%%%%%%%%%%%%%%%%%%%
			\addplot[QS] coordinates{
				(324.725,0.64773)
				(468.655,0.741238)
				(615,0.799129)
				(884.79,0.859309)
				(984.505,0.873528)
				(1275.62,0.90151)
				(1500.06,0.914516)
				(1646.89,0.921017)
				(1824.2,0.927193)
				(2040.32,0.933099)
				(2303.69,0.938612)
				(2626.4,0.944071)
				(3066.62,0.949442)
				(3646.03,0.954445)
				(4444.58,0.959196)
				(5554.87,0.963693)
			};

			% VLSLIC asa.mean_min %%%%%%%%%%%%%%%%%%%%%%%%%%%%%%%%%%%%%%%%%%%%%%%%%%%%%%%%%%%%
			\addplot[VLSLIC] coordinates{
				(575.975,0.925226)
				(651.86,0.932016)
				(763.62,0.937005)
				(899,0.943044)
				(988.985,0.946462)
				(1193.67,0.950565)
				(1348.08,0.952897)
				(1349.01,0.953187)
				(1579.08,0.955263)
				(1849.65,0.95733)
				(1857.27,0.957624)
				(2307.73,0.959942)
				(2890.56,0.9624)
				(2924.52,0.962998)
				(2954.18,0.963446)
				(3858.98,0.964469)
				(3858.98,0.964469)
				(4809.57,0.961208)
			};

			% CIS asa.mean_min %%%%%%%%%%%%%%%%%%%%%%%%%%%%%%%%%%%%%%%%%%%%%%%%%%%%%%%%%%%%
			\addplot[CIS] coordinates{
				(317.63,0.916769)
				(411.595,0.928525)
				(472.685,0.933186)
				(633.345,0.941965)
				(777.05,0.946575)
				(969.085,0.951274)
				(1150.5,0.954214)
				(1168.82,0.954333)
				(1371.27,0.956888)
				(1637.39,0.959523)
				(1667.02,0.959684)
				(2056.01,0.962313)
				(3126.04,0.966093)
				(3625.6,0.967204)
				(4558.92,0.969893)
				(4558.92,0.969893)
				(6448.65,0.973531)
			};

			% ERS asa.mean_min %%%%%%%%%%%%%%%%%%%%%%%%%%%%%%%%%%%%%%%%%%%%%%%%%%%%%%%%%%%%
			\addplot[ERS] coordinates{
				(200,0.915777)
				(300,0.927261)
				(400,0.93493)
				(600,0.943358)
				(800,0.948618)
				(1000,0.952044)
				(1200,0.954859)
				(1400,0.956969)
				(1600,0.958925)
				(1800,0.960419)
				(2000,0.961677)
				(2400,0.963886)
				(2800,0.965689)
				(3200,0.967152)
				(3600,0.968402)
				(4000,0.969584)
				(4600,0.971076)
				(5200,0.972306)
			};

			% MSS asa.mean_min %%%%%%%%%%%%%%%%%%%%%%%%%%%%%%%%%%%%%%%%%%%%%%%%%%%%%%%%%%%%
			\addplot[MSS] coordinates{
				(200.765,0.902226)
				(282.33,0.913223)
				(420.66,0.931066)
				(664.115,0.942331)
				(837.885,0.945851)
				(1179.3,0.951972)
				(1422.34,0.95505)
				(1423.04,0.955102)
				(1768,0.958118)
				(2156.96,0.960257)
				(2157.01,0.960345)
				(2826.35,0.963661)
				(3669.97,0.965734)
				(3670.34,0.965704)
				(3669.92,0.965769)
				(5213.45,0.969049)
				(5213.45,0.969049)
				(7846.35,0.972574)
			};

			% ETPS asa.mean_min %%%%%%%%%%%%%%%%%%%%%%%%%%%%%%%%%%%%%%%%%%%%%%%%%%%%%%%%%%%%
			\addplot[ETPS] coordinates{
				(216,0.919542)
				(294,0.93005)
				(425,0.938469)
				(651,0.947941)
				(805,0.952281)
				(1107,0.95744)
				(1320,0.959446)
				(1320,0.959446)
				(1617,0.962407)
				(1944,0.964546)
				(1944,0.964546)
				(2501,0.967605)
				(3174,0.970099)
				(3174,0.970099)
				(3174,0.970099)
				(4374,0.973288)
				(4374,0.973288)
				(6305,0.976394)
			};

			\end{axis}
	\end{tikzpicture}
\end{subfigure}%
\begin{subfigure}[b]{0.325\textwidth}\phantomsubcaption\label{subfig:appendix-experiments-bsds500-ue_levin.mean_max}
	%%%%%%%%%%%%%%%%%%%%%%%%%%%%%%%%%%%%%%%%%%%%%%%%%%%%%%%%%%%%
	% ue_levin.mean_max
	%%%%%%%%%%%%%%%%%%%%%%%%%%%%%%%%%%%%%%%%%%%%%%%%%%%%%%%%%%%%
	\begin{tikzpicture}
		\begin{axis}[AEBSDS500UELevinMeanMax,xmode=log]

			% CCS ue_levin.mean_max %%%%%%%%%%%%%%%%%%%%%%%%%%%%%%%%%%%%%%%%%%%%%%%%%%%%%%%%%%%%
			\addplot[CCS] coordinates{
				(218.865,159.582)
				(303.21,116.339)
				(453.82,78.4627)
				(706.35,49.4687)
				(871.12,41.5995)
				(1204.99,29.3757)
				(1438.62,24.5127)
				(1438.62,24.5127)
				(1762.49,20.3766)
				(2107.24,16.7246)
				(2107.24,16.7246)
				(2702.61,13.2118)
				(3406.7,10.6033)
				(3406.7,10.6033)
				(3406.7,10.6033)
				(4654.44,7.5929)
				(4654.44,7.5929)
				(6619.1,5.40553)
			};

			% SEEDS ue_levin.mean_max %%%%%%%%%%%%%%%%%%%%%%%%%%%%%%%%%%%%%%%%%%%%%%%%%%%%%%%%%%%%
			\addplot[SEEDS] coordinates{
				(261.62,225.403)
				(365.675,162.885)
				(468.81,118.874)
				(670.57,82.0287)
				(870.75,56.9778)
				(1087.4,40.7727)
				(1270.11,36.7068)
				(1451.85,29.0983)
				(1669.18,27.6516)
				(1873.19,22.5098)
				(2104.62,19.8286)
				(2462.77,18.5872)
				(2793.43,15.3166)
				(3260.86,12.4763)
				(3895.78,10.5427)
				(3895.78,10.5427)
				(4846.12,8.68279)
				(4846.12,8.68279)
			};

			% SLIC ue_levin.mean_max %%%%%%%%%%%%%%%%%%%%%%%%%%%%%%%%%%%%%%%%%%%%%%%%%%%%%%%%%%%%
			\addplot[SLIC] coordinates{
				(180.32,194.675)
				(256.725,137.369)
				(368.57,92.1886)
				(575.335,60.2154)
				(726.36,48.2287)
				(1002.87,36.4507)
				(1203.61,29.684)
				(1203.61,29.684)
				(1475.69,25.1665)
				(1814.8,19.9439)
				(1814.8,19.9439)
				(2334.56,15.2932)
				(3038.13,11.641)
				(3038.13,11.641)
				(3038.13,11.641)
				(4188.97,8.66537)
				(4188.97,8.66537)
				(6139.41,5.70553)
			};

			% RW ue_levin.mean_max %%%%%%%%%%%%%%%%%%%%%%%%%%%%%%%%%%%%%%%%%%%%%%%%%%%%%%%%%%%%
			\addplot[RW] coordinates{
				(211.565,247.979)
				(313.64,169.704)
				(407.125,130.569)
				(649.355,75.5453)
				(858.65,56.12)
				(1032.29,49.368)
				(1280.94,36.2976)
				(1505.72,32.2215)
				(1649.74,29.5681)
				(1940.98,24.8508)
				(2101.2,22.185)
				(2570.18,18.2866)
				(2916.36,15.8124)
				(3413.13,13.4888)
				(3882.85,11.9268)
				(4457.4,10.2425)
				(5093.4,9.36366)
				(5663.8,4.9509)
			};

			% CW ue_levin.mean_max %%%%%%%%%%%%%%%%%%%%%%%%%%%%%%%%%%%%%%%%%%%%%%%%%%%%%%%%%%%%
			\addplot[CW] coordinates{
				(197.775,214.487)
				(309.29,154.261)
				(405.935,114.634)
				(609.18,79.2617)
				(807.005,57.6897)
				(1045.56,44.5698)
				(1252.72,37.8514)
				(1492.83,31.3786)
				(1649.62,27.2503)
				(1816.43,25.4745)
				(2091.79,22.1489)
				(2565.8,17.8893)
				(2917.57,14.9884)
				(3319.52,13.5929)
				(4023.98,10.7495)
				(4023.98,10.7495)
				(4639.83,9.27471)
				(5846.73,7.56935)
			};

			% TP ue_levin.mean_max %%%%%%%%%%%%%%%%%%%%%%%%%%%%%%%%%%%%%%%%%%%%%%%%%%%%%%%%%%%%
			\addplot[TP] coordinates{
				(283.695,130.979)
				(381.285,92.4009)
				(555.1,64.0041)
				(752.29,45.3632)
				(1007.04,39.5633)
				(1292.4,28.2864)
				(1495.14,23.6813)
				(1696.62,284.305)
				%(2017.73,66.1442)
				(2478.02,22.7466)
				(2831,15.5469)
				(3018.75,17.2363)
			};

			% POISE ue_levin.mean_max %%%%%%%%%%%%%%%%%%%%%%%%%%%%%%%%%%%%%%%%%%%%%%%%%%%%%%%%%%%%
			\addplot[POISE] coordinates{
				(204.855,806.769)
				(306.68,652.138)
				(408.385,595.438)
				(611.79,526.771)
				(814.975,483.964)
				(1017.48,454.286)
				(1216.26,438.175)
				(1409.36,420.405)
				(1587.37,395.092)
				(1749.07,394.402)
				(1887.06,393.726)
				(2090.32,391.318)
				(2221.84,390.163)
				(2278.76,390.018)
				(2287.5,389.981)
				(2288.94,389.978)
				(2288.94,389.978)
				(2288.94,389.978)
			};

			% FH ue_levin.mean_max %%%%%%%%%%%%%%%%%%%%%%%%%%%%%%%%%%%%%%%%%%%%%%%%%%%%%%%%%%%%
			\addplot[FH] coordinates{
				(628.745,329.837)
				(782.42,292.398)
				(799.09,189.636)
				(963.36,220.917)
				(1090.39,125.078)
				(1187.04,170.65)
				(1605.71,81.7802)
				(2533.01,62.0058)
				(3000.74,38.092)
				(3219.63,34.1336)
				(3814.42,48.7399)
				(4746.96,31.0179)
			};

			% EAMS ue_levin.mean_max %%%%%%%%%%%%%%%%%%%%%%%%%%%%%%%%%%%%%%%%%%%%%%%%%%%%%%%%%%%%
			\addplot[EAMS] coordinates{
				(261.12,218.44)
				(283.33,206.884)
				(309.87,193.968)
				(383.52,160.05)
				(499.435,135.615)
				(725.36,105.038)
				(1417.35,50.892)
				(1473.6,49.7686)
				(1534.33,48.7878)
				(1883.59,141.041)
				(2063.65,135.458)
				(2346.71,130.427)
				(2642.78,127.129)
				(3113.25,124.706)
				(3640.44,121.117)
				(4914.12,80.9761)
				(8189.58,91.4042)
			};

			% CRS ue_levin.mean_max %%%%%%%%%%%%%%%%%%%%%%%%%%%%%%%%%%%%%%%%%%%%%%%%%%%%%%%%%%%%
			\addplot[CRS] coordinates{
				(299.575,150.024)
				(449.585,93.7506)
				(635.865,70.7525)
				(895.935,50.2567)
				(1180.31,36.463)
				(1530.1,29.4375)
				(1758.22,25.231)
				(2057.15,22.1311)
				(2308.66,19.6314)
				(2461.75,17.8479)
				(2755.4,15.782)
				(3401.31,12.7428)
				(3879.51,11.1697)
				(4346.83,9.90994)
				(5208.23,8.43965)
				(5208.23,8.43965)
				(5821.7,7.41601)
				(7241.41,5.89774)
			};

			% SEAW ue_levin.mean_max %%%%%%%%%%%%%%%%%%%%%%%%%%%%%%%%%%%%%%%%%%%%%%%%%%%%%%%%%%%%
			\addplot[SEAW] coordinates{
				(165.505,822.654)
				(356.805,220.559)
				(923.49,53.5968)
				(2933.9,14.0727)
				(10727.5,3.48437)
			};

			% RESEEDS ue_levin.mean_max %%%%%%%%%%%%%%%%%%%%%%%%%%%%%%%%%%%%%%%%%%%%%%%%%%%%%%%%%%%%
			\addplot[RESEEDS] coordinates{
				(200.795,165.699)
				(301.525,112.687)
				(401.465,86.2463)
				(602.33,59.3425)
				(800.99,43.1575)
				(1020.12,37.1863)
				(1201.34,28.7346)
				(1378.1,27.6531)
				(1601.55,22.0135)
				(1802.11,20.9201)
				(2040.11,19.082)
				(2402.13,14.4105)
				(2720.13,14.0401)
				(3200.2,11.5603)
				(3840.22,9.95772)
				(3840.22,9.95772)
				(4800.37,8.10632)
				(4800.37,8.10632)
			};

			% ERGC ue_levin.mean_max %%%%%%%%%%%%%%%%%%%%%%%%%%%%%%%%%%%%%%%%%%%%%%%%%%%%%%%%%%%%
			\addplot[ERGC] coordinates{
				(196,201.826)
				(306,134.663)
				(400,100.783)
				(600,66.0981)
				(792.57,51.9306)
				(1024,39.8925)
				(1224,33.6387)
				(1453.2,27.7689)
				(1600,25.2838)
				(1760,23.9015)
				(2024,21.5594)
				(2472.98,17.4647)
				(2809,15.1238)
				(3180,13.0899)
				(3840,10.8515)
				(3840,10.8515)
				(4416,9.74104)
				(5520,7.91424)
			};

			% PF ue_levin.mean_max %%%%%%%%%%%%%%%%%%%%%%%%%%%%%%%%%%%%%%%%%%%%%%%%%%%%%%%%%%%%
			\addplot[PF] coordinates{
				(281.06,455.89)
				(428.82,314.649)
				(585.08,237.445)
				(928.805,157.958)
				(1172.76,126.337)
				(1602.65,91.1665)
				(1861.72,79.974)
				(2267.11,69.8833)
				(2697.08,59.632)
				(3382.15,46.3995)
				(4207.66,39.8588)
				(6051.74,25.5322)
			};

			% TPS ue_levin.mean_max %%%%%%%%%%%%%%%%%%%%%%%%%%%%%%%%%%%%%%%%%%%%%%%%%%%%%%%%%%%%
			\addplot[TPS] coordinates{
				(224.26,215.424)
				(903.595,52.4085)
				(1130.53,42.355)
				(1332.49,34.2524)
				(1556.56,30.0361)
				(1736.84,26.0867)
				(1988.95,22.9717)
				(2167.04,19.7553)
				(2656.03,16.6735)
				(2987.25,14.2908)
				(3505.61,12.1001)
				(4030.29,9.63941)
				(4568.25,8.84278)
				(5242.04,8.11249)
				(5978.67,7.21019)
			};

			% NC ue_levin.mean_max %%%%%%%%%%%%%%%%%%%%%%%%%%%%%%%%%%%%%%%%%%%%%%%%%%%%%%%%%%%%
			\addplot[NC] coordinates{
				(223.505,178.281)
				(321.69,123.899)
				(416.03,100.537)
				(594.185,71.3917)
				(763.865,56.7747)
				(922.645,47.5887)
				(1073.6,42.3065)
				(1213.8,36.9599)
				(1348.36,32.8932)
				(1473.43,29.7216)
				(1591.75,27.2345)
				(2000.27,21.9982)
			};

			% VC ue_levin.mean_max %%%%%%%%%%%%%%%%%%%%%%%%%%%%%%%%%%%%%%%%%%%%%%%%%%%%%%%%%%%%
			\addplot[VC] coordinates{
				(351.72,1572.75)
				(498.095,801.516)
				(712.82,398.372)
				(994.125,189.101)
				(1201.27,128.076)
				(1375.27,94.4683)
				(1685.08,64.1576)
				(1968.97,45.7357)
				(2224.75,36.6431)
				(2476.43,30.4898)
				(2712.29,27.1565)
				(2948.04,22.6887)
				(3169.69,20.6271)
				(3389.95,18.7817)
				(3804.08,16.6301)
				(4196.75,14.2828)
				(4566.66,13.1841)
				(4815.37,11.8998)
				(5157.48,11.5168)
				(5401.49,11.3471)
			};

			% PB ue_levin.mean_max %%%%%%%%%%%%%%%%%%%%%%%%%%%%%%%%%%%%%%%%%%%%%%%%%%%%%%%%%%%%
			\addplot[PB] coordinates{
				(324.39,165.296)
				(398.425,111.111)
				(494.17,83.4718)
				(692.845,52.0232)
				(853.905,41.5818)
				(1129.27,30.1236)
				(1323.88,25.3209)
				(1323.88,25.3209)
				(1601.37,21.4151)
				(1929.58,17.8733)
				(1929.58,17.8733)
				(2453.31,14.0758)
				(3126.91,11.0137)
				(3126.91,11.0137)
				(3126.91,11.0137)
				(4270.56,8.21592)
				(4270.56,8.21592)
				(6122.31,5.82638)
			};

			% PRESLIC ue_levin.mean_max %%%%%%%%%%%%%%%%%%%%%%%%%%%%%%%%%%%%%%%%%%%%%%%%%%%%%%%%%%%%
			\addplot[PRESLIC] coordinates{
				(369,100.044)
				(581.54,62.2626)
				(734.575,48.4541)
				(1020.3,33.9666)
				(1229.22,29.2742)
				(1229.22,29.2742)
				(1511.3,23.0241)
				(1838.94,19.0786)
				(1838.94,19.0786)
				(2375.99,14.7504)
				(3059.43,11.3221)
				(3059.43,11.3221)
				(3059.43,11.3221)
				(4218.88,8.19598)
				(4218.88,8.19598)
				(6137.8,5.47659)
			};

			% W ue_levin.mean_max %%%%%%%%%%%%%%%%%%%%%%%%%%%%%%%%%%%%%%%%%%%%%%%%%%%%%%%%%%%%
			\addplot[W] coordinates{
				(188.285,399.844)
				(296.48,239.704)
				(387.92,185.847)
				(608.055,104.746)
				(794.47,76.2477)
				(1100.87,54.8899)
				(1302.83,43.848)
				(1302.83,43.848)
				(1572.26,33.1778)
				(1960.24,27.3054)
				(1960.24,27.3054)
				(2478.43,19.9851)
				(3292.41,14.7352)
				(3292.41,14.7352)
				(3292.41,14.7352)
				(4439.31,10.3933)
				(4439.31,10.3933)
				(6526.67,6.51016)
			};

			% LSC ue_levin.mean_max %%%%%%%%%%%%%%%%%%%%%%%%%%%%%%%%%%%%%%%%%%%%%%%%%%%%%%%%%%%%
			\addplot[LSC] coordinates{
				(548.755,154.271)
				(756.92,94.5319)
				(1045.31,64.1432)
				(1360.95,46.9977)
				(1665.85,35.4387)
				(1919.65,28.7269)
				(2055.98,25.5604)
				(2239.5,23.2694)
				(2376.71,20.9289)
				(2441.63,20.0638)
				(2603.55,19.1744)
				(2909.2,16.1607)
				(3147.7,14.455)
				(3373.12,13.855)
				(3762.88,13.3432)
				(3762.88,13.3432)
				(4057.29,12.5747)
				(4401.67,13.064)
			};

			% WP ue_levin.mean_max %%%%%%%%%%%%%%%%%%%%%%%%%%%%%%%%%%%%%%%%%%%%%%%%%%%%%%%%%%%%
			\addplot[WP] coordinates{
				(216,214.885)
				(294,143.017)
				(384,113.197)
				(600,67.5333)
				(805,51.8405)
				(1080,37.3711)
				(1320,32.3265)
				(1536,27.1577)
				(1536,26.863)
				(1944,20.0041)
				(1944,19.6695)
				(2400,15.166)
				(3174,11.93)
				(3174,11.7032)
				(4320,8.51191)
				(4320,8.51191)
				(4320,8.51191)
				(5836.85,11.0885)
			};

			% QS ue_levin.mean_max %%%%%%%%%%%%%%%%%%%%%%%%%%%%%%%%%%%%%%%%%%%%%%%%%%%%%%%%%%%%
			\addplot[QS] coordinates{
				(324.725,3095.52)
				(468.655,1544.63)
				(615,981.583)
				(884.79,479.033)
				(984.505,408.694)
				(1275.62,278.065)
				(1500.06,212.631)
				(1646.89,183.775)
				(1824.2,160.366)
				(2040.32,137.392)
				(2303.69,116.403)
				(2626.4,97.6427)
				(3066.62,79.4778)
				(3646.03,60.2454)
				(4444.58,46.5347)
				(5554.87,34.693)
			};

			% VLSLIC ue_levin.mean_max %%%%%%%%%%%%%%%%%%%%%%%%%%%%%%%%%%%%%%%%%%%%%%%%%%%%%%%%%%%%
			\addplot[VLSLIC] coordinates{
				(575.975,125.101)
				(651.86,99.1247)
				(763.62,75.5409)
				(899,53.2833)
				(988.985,41.4081)
				(1193.67,33.2505)
				(1348.08,27.5115)
				(1349.01,27.6473)
				(1579.08,23.6854)
				(1849.65,19.3841)
				(1857.27,19.2574)
				(2307.73,16.0623)
				(2890.56,13.0109)
				(2924.52,12.89)
				(2954.18,12.5675)
				(3858.98,10.4772)
				(3858.98,10.4772)
				(4809.57,10.8633)
			};

			% CIS ue_levin.mean_max %%%%%%%%%%%%%%%%%%%%%%%%%%%%%%%%%%%%%%%%%%%%%%%%%%%%%%%%%%%%
			\addplot[CIS] coordinates{
				(317.63,165.885)
				(411.595,118.959)
				(472.685,96.1431)
				(633.345,67.4239)
				(777.05,52.0536)
				(969.085,40.898)
				(1150.5,34.6952)
				(1168.82,33.991)
				(1371.27,28.9083)
				(1637.39,23.0246)
				(1667.02,23.5628)
				(2056.01,18.6658)
				(3126.04,14.1192)
				(3625.6,13.6194)
				(4558.92,10.1091)
				(4558.92,10.1091)
				(6448.65,6.74503)
			};

			% ERS ue_levin.mean_max %%%%%%%%%%%%%%%%%%%%%%%%%%%%%%%%%%%%%%%%%%%%%%%%%%%%%%%%%%%%
			\addplot[ERS] coordinates{
				(200,152.941)
				(300,102.622)
				(400,75.4535)
				(600,50.09)
				(800,38.6438)
				(1000,30.5983)
				(1200,25.3664)
				(1400,21.1723)
				(1600,18.3716)
				(1800,16.9086)
				(2000,15.0314)
				(2400,12.692)
				(2800,10.5555)
				(3200,9.25821)
				(3600,8.24971)
				(4000,7.48667)
				(4600,6.53492)
				(5200,5.79087)
			};

			% MSS ue_levin.mean_max %%%%%%%%%%%%%%%%%%%%%%%%%%%%%%%%%%%%%%%%%%%%%%%%%%%%%%%%%%%%
			\addplot[MSS] coordinates{
				(200.765,319.129)
				(282.33,251.163)
				(420.66,121.929)
				(664.115,76.6503)
				(837.885,66.5475)
				(1179.3,44.0803)
				(1422.34,33.8745)
				(1423.04,34.6477)
				(1768,27.4589)
				(2156.96,22.0757)
				(2157.01,22.5082)
				(2826.35,16.9418)
				(3669.97,13.6637)
				(3670.34,13.8189)
				(3669.92,13.5691)
				(5213.45,9.65202)
				(5213.45,9.65202)
				(7846.35,6.37186)
			};

			% ETPS ue_levin.mean_max %%%%%%%%%%%%%%%%%%%%%%%%%%%%%%%%%%%%%%%%%%%%%%%%%%%%%%%%%%%%
			\addplot[ETPS] coordinates{
				(216,170.582)
				(294,120.356)
				(425,89.0489)
				(651,59.5504)
				(805,47.2251)
				(1107,34.329)
				(1320,28.1307)
				(1320,28.1307)
				(1617,23.8758)
				(1944,18.7579)
				(1944,18.7579)
				(2501,15.5921)
				(3174,11.3354)
				(3174,11.3354)
				(3174,11.3354)
				(4374,9.25586)
				(4374,9.25586)
				(6305,5.89335)
			};

			\end{axis}
	\end{tikzpicture}
\end{subfigure}
\begin{subfigure}[b]{0.325\textwidth}
\end{subfigure}
\\
	\begin{subfigure}[b]{0.325\textwidth}\phantomsubcaption\label{subfig:appendix-experiments-nyuv2-ue_np.mean[0]}
	%%%%%%%%%%%%%%%%%%%%%%%%%%%%%%%%%%%%%%%%%%%%%%%%%%%%%%%%%%%%
	% ue_np.mean[0]
	%%%%%%%%%%%%%%%%%%%%%%%%%%%%%%%%%%%%%%%%%%%%%%%%%%%%%%%%%%%%
	\begin{tikzpicture}
		\begin{axis}[EQNYUV2UE2,xmode=log]

			% CCS ue_np.mean[0] %%%%%%%%%%%%%%%%%%%%%%%%%%%%%%%%%%%%%%%%%%%%%%%%%%%%%%%%%%%%
			\addplot[CCS] coordinates{
				(194.01,0.216993)
				(282.682,0.181444)
				(397.098,0.154589)
				(597.153,0.129512)
				(803.471,0.115026)
				(925.103,0.108389)
				(1177.26,0.0988794)
				(1397.83,0.0931286)
				(1588.43,0.0888312)
				(1878.46,0.0834146)
				(1878.46,0.0834146)
				(2231.81,0.0781559)
				(2677.06,0.0730322)
				(3328.01,0.066989)
				(3328.01,0.066989)
				(4313.98,0.0606271)
				(4313.98,0.0606271)
				(5573.99,0.0544813)
			};

			% SEEDS ue_np.mean[0] %%%%%%%%%%%%%%%%%%%%%%%%%%%%%%%%%%%%%%%%%%%%%%%%%%%%%%%%%%%%
			\addplot[SEEDS] coordinates{
				(246.043,0.255681)
				(347.734,0.238093)
				(450.607,0.205785)
				(654.489,0.169993)
				(857.654,0.149621)
				(1057.59,0.133604)
				(1258.91,0.125471)
				(1462.49,0.119259)
				(1661.42,0.1105)
				(2066.14,0.11113)
				(2473.07,0.0964617)
				(2864.58,0.0923406)
				(3245.7,0.0878599)
				(3711.17,0.08469)
				(4314.51,0.076744)
				(4499.92,0.0736838)
				(4922.37,0.0731233)
			};

			% SLIC ue_np.mean[0] %%%%%%%%%%%%%%%%%%%%%%%%%%%%%%%%%%%%%%%%%%%%%%%%%%%%%%%%%%%%
			\addplot[SLIC] coordinates{
				(184.484,0.20446)
				(279.316,0.173027)
				(385.211,0.15327)
				(596.942,0.1304)
				(819.424,0.116049)
				(939.241,0.110653)
				(1211.08,0.100956)
				(1425.96,0.0956362)
				(1642.44,0.0906285)
				(1940.44,0.0854002)
				(1940.44,0.0854002)
				(2316.71,0.0798519)
				(2774.73,0.0745244)
				(3441.91,0.0685611)
				(3441.91,0.0685611)
				(4395.74,0.0624498)
				(4395.74,0.0624498)
				(5687.73,0.05599)
			};

			% RW ue_np.mean[0] %%%%%%%%%%%%%%%%%%%%%%%%%%%%%%%%%%%%%%%%%%%%%%%%%%%%%%%%%%%%
			\addplot[RW] coordinates{
				(215.912,0.224623)
				(306.055,0.19181)
				(440.393,0.162739)
				(635.679,0.140094)
				(846.511,0.12565)
				(1060.11,0.115735)
				(1306.99,0.106652)
				(1464.51,0.102514)
				(1685.01,0.0972265)
				(1868.17,0.0939696)
				(2109.97,0.0916813)
				(3967.51,0.0720991)
				(4401.89,0.0693895)
				(5110.46,0.0660025)
			};

			% CW ue_np.mean[0] %%%%%%%%%%%%%%%%%%%%%%%%%%%%%%%%%%%%%%%%%%%%%%%%%%%%%%%%%%%%
			\addplot[CW] coordinates{
				(198.614,0.213808)
				(293.173,0.182339)
				(407.073,0.160195)
				(613.509,0.138853)
				(834.84,0.12352)
				(1056.32,0.113553)
				(1266,0.106607)
				(1460.01,0.101244)
				(1700.24,0.0967138)
				(1828.56,0.094263)
				(2009.84,0.0913538)
				(2449.61,0.0852219)
				(2991.17,0.0795497)
				(3233.9,0.077063)
				(3707.85,0.0739416)
				(4124.98,0.0708589)
				(5174.47,0.0651113)
				(5281.06,0.0653677)
			};

			% TP ue_np.mean[0] %%%%%%%%%%%%%%%%%%%%%%%%%%%%%%%%%%%%%%%%%%%%%%%%%%%%%%%%%%%%
			\addplot[TP] coordinates{
				(295.99,0.173268)
				(395.489,0.152659)
				(551.496,0.143646)
				(775.682,0.124888)
				(1000.76,0.113298)
				(1130.49,0.107902)
				(1401.47,0.098847)
				(1572.01,0.0956769)
				(1815.33,0.0908693)
				(2110.15,0.0862166)
			};

			% POISE ue_np.mean[0] %%%%%%%%%%%%%%%%%%%%%%%%%%%%%%%%%%%%%%%%%%%%%%%%%%%%%%%%%%%%
			\addplot[POISE] coordinates{
				(204.942,0.199433)
				(306.83,0.167724)
				(408.614,0.150395)
				(612.1,0.131616)
				(815.501,0.120815)
				(1019.01,0.114183)
				(1222.5,0.108918)
				(1425.91,0.104877)
				(1628.65,0.101643)
				(1830.61,0.0989652)
				(2031.68,0.0967829)
				(2427.44,0.0934294)
				(2807.4,0.0908592)
				(3158.64,0.0892863)
				(3466.96,0.0879902)
				(3736.5,0.0871307)
				(4042.51,0.0864045)
				(4219.03,0.0860905)
			};

			% FH ue_np.mean[0] %%%%%%%%%%%%%%%%%%%%%%%%%%%%%%%%%%%%%%%%%%%%%%%%%%%%%%%%%%%%
			\addplot[FH] coordinates{
				(689.802,0.158879)
				(763.792,0.151115)
				(813.815,0.147271)
				(988.712,0.131736)
				(1077.66,0.125833)
				(1359.76,0.113046)
				(1408.18,0.117883)
				(1559.12,0.115032)
				(1923.5,0.0978674)
				(2206.32,0.100858)
				(2448.19,0.0971935)
				(2699.96,0.0850741)
				(3024.27,0.0872945)
				(3568.29,0.0847671)
				(4199.72,0.079623)
				(4594.6,0.0736738)
			};

			% EAMS ue_np.mean[0] %%%%%%%%%%%%%%%%%%%%%%%%%%%%%%%%%%%%%%%%%%%%%%%%%%%%%%%%%%%%
			\addplot[EAMS] coordinates{
				(419.068,0.144734)
				(455.657,0.140388)
				(499.416,0.135681)
				(641.188,0.125544)
				(842.328,0.114359)
				(1268.71,0.101801)
				(2584.6,0.0832607)
				(2683.21,0.0822668)
				(2779.09,0.0827376)
				(2997.7,0.0808278)
				(3180.29,0.080521)
				(3421.45,0.0793529)
				(3646.16,0.0787786)
				(3943.04,0.0783151)
				(4448.4,0.0755675)
				(5663.91,0.0704212)
				(9098.39,0.0619494)
			};

			% CRS ue_np.mean[0] %%%%%%%%%%%%%%%%%%%%%%%%%%%%%%%%%%%%%%%%%%%%%%%%%%%%%%%%%%%%
			\addplot[CRS] coordinates{
				(254.263,0.174765)
				(396.694,0.145756)
				(514.895,0.130322)
				(764.997,0.111728)
				(988.589,0.101933)
				(1232.03,0.0933611)
				(1498.25,0.0871983)
				(1707.65,0.0835093)
				(1968.19,0.0790225)
				(2099.81,0.0768389)
				(2299.23,0.0743375)
				(2706.1,0.0701011)
				(3259.07,0.0654525)
				(3558.75,0.0633238)
				(4055.02,0.0602152)
				(4394.15,0.0583992)
				(5418.11,0.0537133)
				(5695.52,0.0527932)
			};

			% SEAW ue_np.mean[0] %%%%%%%%%%%%%%%%%%%%%%%%%%%%%%%%%%%%%%%%%%%%%%%%%%%%%%%%%%%%
			\addplot[SEAW] coordinates{
				(124.667,0.35668)
				(349.414,0.211234)
				(1192.36,0.119281)
				(4497.65,0.0695662)
			};

			% RESEEDS ue_np.mean[0] %%%%%%%%%%%%%%%%%%%%%%%%%%%%%%%%%%%%%%%%%%%%%%%%%%%%%%%%%%%%
			\addplot[RESEEDS] coordinates{
				(200.441,0.214926)
				(300.263,0.207865)
				(400.323,0.17559)
				(600.361,0.140873)
				(800.526,0.122851)
				(1000.06,0.108806)
				(1200.06,0.102369)
				(1400.96,0.096608)
				(1600.21,0.0903118)
				(1792.08,0.101512)
				(1998.13,0.0869894)
				(2408.17,0.0776285)
				(2801.44,0.0743663)
				(3182.16,0.0703948)
				(3648.12,0.0689841)
				(4257.44,0.0624645)
				(4444.16,0.0609398)
				(4864.15,0.0601597)
			};

			% ERGC ue_np.mean[0] %%%%%%%%%%%%%%%%%%%%%%%%%%%%%%%%%%%%%%%%%%%%%%%%%%%%%%%%%%%%
			\addplot[ERGC] coordinates{
				(196,0.194213)
				(289,0.165664)
				(400,0.145713)
				(600,0.126109)
				(812,0.112044)
				(1024,0.103719)
				(1224,0.0976167)
				(1406,0.0927796)
				(1681,0.0878355)
				(1763,0.0865362)
				(1935,0.0839173)
				(2350,0.0784336)
				(2856,0.0734824)
				(3080,0.0714722)
				(3520,0.0685269)
				(3904,0.0656898)
				(4864,0.0602308)
				(5100,0.0600402)
			};

			% PF ue_np.mean[0] %%%%%%%%%%%%%%%%%%%%%%%%%%%%%%%%%%%%%%%%%%%%%%%%%%%%%%%%%%%%
			\addplot[PF] coordinates{
				(281.19,0.36559)
				(430.376,0.320023)
				(586.371,0.291422)
				(907.19,0.255785)
				(1201.57,0.232596)
				(1354.07,0.22401)
				(1718.72,0.208817)
				(1939.43,0.200007)
				(2251.38,0.191596)
				(2592.38,0.182522)
				(3186.03,0.174335)
				(4180.57,0.160473)
				(6182.51,0.14268)
			};

			% TPS ue_np.mean[0] %%%%%%%%%%%%%%%%%%%%%%%%%%%%%%%%%%%%%%%%%%%%%%%%%%%%%%%%%%%%
			\addplot[TPS] coordinates{
				(230.419,0.215002)
				(313.739,0.188408)
				(450.436,0.163971)
				(638.008,0.142489)
				(857.559,0.126421)
				(1043.09,0.117862)
				(1317.64,0.107356)
				(1503.8,0.101886)
				(1703.97,0.0975202)
				(1873.98,0.0938476)
				(2144.93,0.0909679)
				(2580.57,0.0843347)
				(2894.48,0.0808023)
				(3314.47,0.0773751)
				(3924.13,0.0708422)
				(4380.02,0.0687939)
				(5132.34,0.0642085)
				(5738.91,0.0623916)
			};

			% NC ue_np.mean[0] %%%%%%%%%%%%%%%%%%%%%%%%%%%%%%%%%%%%%%%%%%%%%%%%%%%%%%%%%%%%
			\addplot[NC] coordinates{
				(439.952,0.142744)
				(1154.47,0.114141)
				(2631.49,0.0836629)
				(3874.82,0.072947)
			};

			% VC ue_np.mean[0] %%%%%%%%%%%%%%%%%%%%%%%%%%%%%%%%%%%%%%%%%%%%%%%%%%%%%%%%%%%%
			\addplot[VC] coordinates{
				(243.744,0.239429)
				(400.291,0.177749)
				(531.672,0.150768)
				(660.84,0.134662)
				(895.915,0.115686)
				(1125.96,0.104743)
				(1348.09,0.0975676)
				(1564.34,0.09205)
				(1781.1,0.0876126)
				(1995.21,0.0842898)
				(2205.86,0.0813086)
				(2416.36,0.0785084)
				(2820.76,0.0743309)
				(3224.25,0.0708998)
				(3610.4,0.0680529)
				(3988.04,0.0657843)
				(4351.59,0.0636596)
				(4847.23,0.0611841)
				(5262.69,0.0592493)
			};

			% PB ue_np.mean[0] %%%%%%%%%%%%%%%%%%%%%%%%%%%%%%%%%%%%%%%%%%%%%%%%%%%%%%%%%%%%
			\addplot[PB] coordinates{
				(273.248,0.250511)
				(360.188,0.211851)
				(463.193,0.185017)
				(656.962,0.155462)
				(857.526,0.137242)
				(970.915,0.129242)
				(1217.27,0.117229)
				(1387.49,0.111035)
				(1616.5,0.104393)
				(1896.95,0.0984057)
				(1896.95,0.0984057)
				(2243.04,0.0917164)
				(2681.85,0.0860795)
				(3316.29,0.0792165)
				(3316.29,0.0792165)
				(4151.92,0.0726646)
				(4151.92,0.0726646)
				(5425.24,0.0661418)
			};

			% VCCS ue_np.mean[0] %%%%%%%%%%%%%%%%%%%%%%%%%%%%%%%%%%%%%%%%%%%%%%%%%%%%%%%%%%%%
			\addplot[VCCS] coordinates{
				(564.123,0.199063)
				(600.283,0.191901)
				(648.629,0.181498)
				(755.632,0.167167)
				(836.05,0.158543)
				(1051.64,0.148687)
				(1109.07,0.144169)
				(1179.74,0.140147)
				(1237.23,0.137731)
				(1322.17,0.133494)
				(1400.72,0.130995)
				(1526.22,0.124518)
				(1630.95,0.121193)
				(1756.59,0.116931)
				(1914.09,0.112234)
				(2024.07,0.111118)
				(2258.58,0.104267)
				(2558.8,0.096803)
				(2780.39,0.0943227)
				(3044.22,0.090837)
				(3687.7,0.078021)
				(4171.76,0.0730024)
				(4255.91,0.0774884)
			};

			% PRESLIC ue_np.mean[0] %%%%%%%%%%%%%%%%%%%%%%%%%%%%%%%%%%%%%%%%%%%%%%%%%%%%%%%%%%%%
			\addplot[PRESLIC] coordinates{
				(189.89,0.217645)
				(388.539,0.161778)
				(589.772,0.140151)
				(798.767,0.125066)
				(920.201,0.118191)
				(1188.65,0.107943)
				(1401.21,0.101064)
				(1612.25,0.096223)
				(1904.93,0.0904268)
				(1904.93,0.0904268)
				(2282.12,0.0843557)
				(2738.74,0.0794571)
				(3422.72,0.0727227)
				(3422.72,0.0727227)
				(4393.15,0.0658819)
				(4393.15,0.0658819)
				(5724.22,0.0596778)
			};

			% W ue_np.mean[0] %%%%%%%%%%%%%%%%%%%%%%%%%%%%%%%%%%%%%%%%%%%%%%%%%%%%%%%%%%%%
			\addplot[W] coordinates{
				(193.87,0.230706)
				(303.997,0.192795)
				(397.103,0.17269)
				(621.108,0.144759)
				(870.236,0.127816)
				(959.915,0.122907)
				(1234.38,0.112302)
				(1418.74,0.106601)
				(1652.83,0.100844)
				(1957.01,0.0947564)
				(1957.01,0.0947564)
				(2345.3,0.0889117)
				(2867.42,0.0822838)
				(3518.42,0.076682)
				(3518.42,0.076682)
				(4497.46,0.0696664)
				(4497.46,0.0696664)
				(5940.33,0.0629077)
			};

			% LSC ue_np.mean[0] %%%%%%%%%%%%%%%%%%%%%%%%%%%%%%%%%%%%%%%%%%%%%%%%%%%%%%%%%%%%
			\addplot[LSC] coordinates{
				(383.095,0.161783)
				(552.409,0.140294)
				(718.163,0.126949)
				(1059.33,0.110189)
				(1414.95,0.0988616)
				(1816.61,0.0898628)
				(2009.32,0.086118)
				(2263.42,0.0819968)
				(2398.35,0.0792781)
				(2446.8,0.0779983)
				(2588.34,0.0758377)
				(2830.36,0.0724911)
				(3264.3,0.0682116)
				(3438.52,0.0668361)
				(3802.19,0.0641309)
				(4059.6,0.0629008)
				(4900.83,0.0597229)
				(5001.09,0.058997)
			};

			% WP ue_np.mean[0] %%%%%%%%%%%%%%%%%%%%%%%%%%%%%%%%%%%%%%%%%%%%%%%%%%%%%%%%%%%%
			\addplot[WP] coordinates{
				(204,0.190887)
				(315,0.162058)
				(432,0.145964)
				(638,0.127101)
				(850,0.11559)
				(1064,0.106748)
				(1230,0.102756)
				(1408,0.0984739)
				(1645,0.0941458)
				(1938,0.0898955)
				(1938,0.0897323)
				(2296,0.0848078)
				(2745,0.0799514)
				(3400,0.0751955)
				(3400,0.0751955)
				(4256,0.0701715)
				(4256,0.0701715)
				(5568,0.064637)
			};

			% QS ue_np.mean[0] %%%%%%%%%%%%%%%%%%%%%%%%%%%%%%%%%%%%%%%%%%%%%%%%%%%%%%%%%%%%
			\addplot[QS] coordinates{
				(223.521,0.407091)
				(325.303,0.309692)
				(414.609,0.262196)
				(663.504,0.195078)
				(828.486,0.172383)
				(1077.18,0.150497)
				(1252.44,0.139966)
				(1475.56,0.130129)
				(1767.37,0.120231)
				(2165.34,0.110591)
				(2703.49,0.101247)
				(3460.91,0.0915782)
				(4543.63,0.082224)
				(6144.37,0.0728167)
			};

			% CIS ue_np.mean[0] %%%%%%%%%%%%%%%%%%%%%%%%%%%%%%%%%%%%%%%%%%%%%%%%%%%%%%%%%%%%
			\addplot[CIS] coordinates{
				(292.103,0.18675)
				(366.787,0.164434)
				(436.398,0.151046)
				(575.972,0.133225)
				(721.491,0.122199)
				(783.672,0.118371)
				(963.875,0.110277)
				(1083.42,0.106211)
				(1227.51,0.101446)
				(1408.35,0.097497)
				(1424.97,0.0971798)
				(1751.64,0.0914291)
				(2118.16,0.0856974)
				(2717.81,0.0794659)
				(3240.8,0.0747416)
				(3240.8,0.0747416)
				(4824.68,0.0667908)
			};

			% RESEEDS3D ue_np.mean[0] %%%%%%%%%%%%%%%%%%%%%%%%%%%%%%%%%%%%%%%%%%%%%%%%%%%%%%%%%%%%
			\addplot[RESEEDS3D] coordinates{
				(200.035,0.300818)
				(300.105,0.172161)
				(400.128,0.149737)
				(600.233,0.125157)
				(800.351,0.111215)
				(1000.04,0.100143)
				(1200.03,0.0939219)
				(1400.65,0.0895527)
				(1600.08,0.0841377)
				(1792.08,0.0907046)
				(1998.07,0.0806915)
				(2408.16,0.0731884)
				(2801.12,0.0698269)
				(3182.14,0.0663)
				(3648.05,0.0636726)
				(4257.09,0.0593419)
				(4444.11,0.0578795)
				(4864.04,0.0566276)
			};

			% ERS ue_np.mean[0] %%%%%%%%%%%%%%%%%%%%%%%%%%%%%%%%%%%%%%%%%%%%%%%%%%%%%%%%%%%%
			\addplot[ERS] coordinates{
				(200,0.164743)
				(300,0.142188)
				(400,0.129084)
				(600,0.113761)
				(800,0.103738)
				(1000,0.0970346)
				(1200,0.0915101)
				(1400,0.0872521)
				(1600,0.0834982)
				(1800,0.0804217)
				(2000,0.0777025)
				(2400,0.0728562)
				(2800,0.0690113)
				(3200,0.065847)
				(3600,0.0629311)
				(4000,0.0605456)
				(4600,0.057255)
				(5200,0.0545465)
			};

			% DASP ue_np.mean[0] %%%%%%%%%%%%%%%%%%%%%%%%%%%%%%%%%%%%%%%%%%%%%%%%%%%%%%%%%%%%
			\addplot[DASP] coordinates{
				(417.987,0.173612)
				(521.05,0.14796)
				(620.704,0.132322)
				(812.654,0.115135)
				(1006.72,0.104708)
				(1197.12,0.0974706)
				(1388.06,0.0921737)
				(1576.93,0.0877052)
				(1767.2,0.0841756)
				(1947.7,0.0811193)
				(2131.65,0.0788187)
				(2494.37,0.0743063)
				(2860.93,0.07091)
				(3211.97,0.0678386)
				(3564.56,0.06541)
				(3909.24,0.0632915)
				(4418.09,0.0605536)
				(4919.67,0.0582789)
			};

			% MSS ue_np.mean[0] %%%%%%%%%%%%%%%%%%%%%%%%%%%%%%%%%%%%%%%%%%%%%%%%%%%%%%%%%%%%
			\addplot[MSS] coordinates{
				(207.576,0.211847)
				(308.962,0.184244)
				(436.368,0.157)
				(668.784,0.136278)
				(912.997,0.122561)
				(1055.23,0.115873)
				(1363.6,0.106897)
				(1575.7,0.102165)
				(1862.23,0.0969694)
				(2222.47,0.0918802)
				(2222.47,0.0918802)
				(2666.45,0.086867)
				(3236.21,0.08235)
				(4079.47,0.0762429)
				(4078.85,0.0763146)
				(5209.16,0.0709364)
				(5209.16,0.0709364)
				(6994.34,0.0642183)
			};

			% ETPS ue_np.mean[0] %%%%%%%%%%%%%%%%%%%%%%%%%%%%%%%%%%%%%%%%%%%%%%%%%%%%%%%%%%%%
			\addplot[ETPS] coordinates{
				(221,0.185724)
				(315,0.157975)
				(432,0.141052)
				(638,0.120615)
				(850,0.107936)
				(972,0.103176)
				(1230,0.0942883)
				(1408,0.0897313)
				(1645,0.0850996)
				(1938,0.079887)
				(1938,0.079887)
				(2296,0.0752435)
				(2745,0.0701251)
				(3400,0.0650547)
				(3400,0.0650547)
				(4256,0.0592612)
				(4256,0.0592612)
				(5568,0.053455)
			};

			\end{axis}
	\end{tikzpicture}
\end{subfigure}
\begin{subfigure}[b]{0.325\textwidth}\phantomsubcaption\label{subfig:appendix-experiments-nyuv2-asa.mean[0]}
	%%%%%%%%%%%%%%%%%%%%%%%%%%%%%%%%%%%%%%%%%%%%%%%%%%%%%%%%%%%%
	% asa.mean[0]
	%%%%%%%%%%%%%%%%%%%%%%%%%%%%%%%%%%%%%%%%%%%%%%%%%%%%%%%%%%%%
	\begin{tikzpicture}
		\begin{axis}[AENYUV2ASA,xmode=log]

			% CCS asa.mean[0] %%%%%%%%%%%%%%%%%%%%%%%%%%%%%%%%%%%%%%%%%%%%%%%%%%%%%%%%%%%%
			\addplot[CCS] coordinates{
				(194.01,0.889222)
				(282.682,0.90789)
				(397.098,0.921828)
				(597.153,0.934664)
				(803.471,0.942059)
				(925.103,0.945433)
				(1177.26,0.950267)
				(1397.83,0.95318)
				(1588.43,0.955353)
				(1878.46,0.958093)
				(1878.46,0.958093)
				(2231.81,0.960745)
				(2677.06,0.963334)
				(3328.01,0.966384)
				(3328.01,0.966384)
				(4313.98,0.969586)
				(4313.98,0.969586)
				(5573.99,0.972682)
			};

			% SEEDS asa.mean[0] %%%%%%%%%%%%%%%%%%%%%%%%%%%%%%%%%%%%%%%%%%%%%%%%%%%%%%%%%%%%
			\addplot[SEEDS] coordinates{
				(246.043,0.867269)
				(347.734,0.87732)
				(450.607,0.894521)
				(654.489,0.913412)
				(857.654,0.924021)
				(1057.59,0.932382)
				(1258.91,0.936564)
				(1462.49,0.939718)
				(1661.42,0.944238)
				(1870.25,0.931143)
				(2066.14,0.943878)
				(2473.07,0.951404)
				(2864.58,0.953499)
				(3245.7,0.955771)
				(3711.17,0.957401)
				(4314.51,0.961413)
				(4499.92,0.962977)
				(4922.37,0.963256)
			};

			% SLIC asa.mean[0] %%%%%%%%%%%%%%%%%%%%%%%%%%%%%%%%%%%%%%%%%%%%%%%%%%%%%%%%%%%%
			\addplot[SLIC] coordinates{
				(184.484,0.89532)
				(279.316,0.911946)
				(385.211,0.922245)
				(596.942,0.934054)
				(819.424,0.941438)
				(939.241,0.944202)
				(1211.08,0.94916)
				(1425.96,0.95186)
				(1642.44,0.954394)
				(1940.44,0.957056)
				(1940.44,0.957056)
				(2316.71,0.959857)
				(2774.73,0.962553)
				(3441.91,0.965571)
				(3441.91,0.965571)
				(4395.74,0.968657)
				(4395.74,0.968657)
				(5687.73,0.971915)
			};

			% RW asa.mean[0] %%%%%%%%%%%%%%%%%%%%%%%%%%%%%%%%%%%%%%%%%%%%%%%%%%%%%%%%%%%%
			\addplot[RW] coordinates{
				(215.912,0.884709)
				(306.055,0.902118)
				(440.393,0.917277)
				(635.679,0.929108)
				(846.511,0.936561)
				(1060.11,0.941641)
				(1306.99,0.94627)
				(1464.51,0.948372)
				(1685.01,0.951055)
				(1868.17,0.952716)
				(2109.97,0.953866)
				(3967.51,0.963789)
				(4401.89,0.965157)
				(5110.46,0.966861)
			};

			% CW asa.mean[0] %%%%%%%%%%%%%%%%%%%%%%%%%%%%%%%%%%%%%%%%%%%%%%%%%%%%%%%%%%%%
			\addplot[CW] coordinates{
				(198.614,0.890618)
				(293.173,0.907061)
				(407.073,0.918694)
				(613.509,0.929767)
				(834.84,0.937665)
				(1056.32,0.942735)
				(1266,0.946265)
				(1460.01,0.949018)
				(1700.24,0.951315)
				(1828.56,0.952562)
				(2009.84,0.954036)
				(2449.61,0.957148)
				(2991.17,0.960039)
				(3233.9,0.961285)
				(3707.85,0.96286)
				(4124.98,0.964418)
				(5174.47,0.967319)
				(5281.06,0.967192)
			};

			% TP asa.mean[0] %%%%%%%%%%%%%%%%%%%%%%%%%%%%%%%%%%%%%%%%%%%%%%%%%%%%%%%%%%%%
			\addplot[TP] coordinates{
				(295.99,0.911872)
				(395.489,0.922643)
				(551.496,0.925813)
				(775.682,0.936055)
				(1000.76,0.942251)
				(1130.49,0.945195)
				(1401.47,0.950059)
				(1572.01,0.951658)
				(1815.33,0.954202)
				(2110.15,0.956594)
			};

			% POISE asa.mean[0] %%%%%%%%%%%%%%%%%%%%%%%%%%%%%%%%%%%%%%%%%%%%%%%%%%%%%%%%%%%%
			\addplot[POISE] coordinates{
				(204.942,0.898252)
				(306.83,0.914952)
				(408.614,0.923945)
				(612.1,0.9336)
				(815.501,0.93913)
				(1019.01,0.942517)
				(1222.5,0.945191)
				(1425.91,0.947246)
				(1628.65,0.94889)
				(1830.61,0.950247)
				(2031.68,0.95135)
				(2427.44,0.953051)
				(2807.4,0.954347)
				(3158.64,0.955145)
				(3466.96,0.9558)
				(3736.5,0.956233)
				(4042.51,0.956597)
				(4219.03,0.956754)
			};

			% FH asa.mean[0] %%%%%%%%%%%%%%%%%%%%%%%%%%%%%%%%%%%%%%%%%%%%%%%%%%%%%%%%%%%%
			\addplot[FH] coordinates{
				(689.802,0.91909)
				(763.792,0.923145)
				(813.815,0.925117)
				(988.712,0.933213)
				(1077.66,0.936359)
				(1359.76,0.942901)
				(1923.5,0.950667)
				(2699.96,0.957151)
				(1559.12,0.941999)
				(1408.18,0.940555)
				(3024.27,0.956008)
				(2206.32,0.949212)
				(2448.19,0.951058)
				(4594.6,0.962981)
				(3568.29,0.957288)
				(4199.72,0.959973)
			};

			% EAMS asa.mean[0] %%%%%%%%%%%%%%%%%%%%%%%%%%%%%%%%%%%%%%%%%%%%%%%%%%%%%%%%%%%%
			\addplot[EAMS] coordinates{
				(419.068,0.926506)
				(455.657,0.928768)
				(499.416,0.931234)
				(641.188,0.936468)
				(842.328,0.942234)
				(1268.71,0.948648)
				(2584.6,0.958076)
				(2683.21,0.95858)
				(2779.09,0.958336)
				(2997.7,0.959309)
				(3180.29,0.959452)
				(3421.45,0.960064)
				(3646.16,0.960357)
				(3943.04,0.960581)
				(4448.4,0.961972)
				(5663.91,0.964575)
				(9098.39,0.968841)
			};

			% CRS asa.mean[0] %%%%%%%%%%%%%%%%%%%%%%%%%%%%%%%%%%%%%%%%%%%%%%%%%%%%%%%%%%%%
			\addplot[CRS] coordinates{
				(254.263,0.910368)
				(396.694,0.92572)
				(514.895,0.933848)
				(764.997,0.943451)
				(988.589,0.948492)
				(1232.03,0.952882)
				(1498.25,0.956048)
				(1707.65,0.957934)
				(1968.19,0.960205)
				(2099.81,0.961315)
				(2299.23,0.962589)
				(2706.1,0.964751)
				(3259.07,0.967105)
				(3558.75,0.968184)
				(4055.02,0.969752)
				(4394.15,0.970667)
				(5418.11,0.973041)
				(5695.52,0.973501)
			};

			% SEAW asa.mean[0] %%%%%%%%%%%%%%%%%%%%%%%%%%%%%%%%%%%%%%%%%%%%%%%%%%%%%%%%%%%%
			\addplot[SEAW] coordinates{
				(124.667,0.813304)
				(349.414,0.892239)
				(1192.36,0.939927)
				(4497.65,0.9651)
			};

			% RESEEDS asa.mean[0] %%%%%%%%%%%%%%%%%%%%%%%%%%%%%%%%%%%%%%%%%%%%%%%%%%%%%%%%%%%%
			\addplot[RESEEDS] coordinates{
				(200.441,0.889564)
				(300.263,0.893706)
				(400.323,0.910551)
				(600.361,0.928539)
				(800.526,0.937858)
				(1000.06,0.94505)
				(1200.06,0.948316)
				(1400.96,0.951286)
				(1600.21,0.954492)
				(1792.08,0.94874)
				(1998.13,0.956157)
				(2408.17,0.96094)
				(2801.44,0.962596)
				(3182.16,0.964606)
				(3648.12,0.965319)
				(4257.44,0.968621)
				(4444.16,0.969384)
				(4864.15,0.969784)
			};

			% ERGC asa.mean[0] %%%%%%%%%%%%%%%%%%%%%%%%%%%%%%%%%%%%%%%%%%%%%%%%%%%%%%%%%%%%
			\addplot[ERGC] coordinates{
				(196,0.900935)
				(289,0.915886)
				(400,0.926248)
				(600,0.936301)
				(812,0.943496)
				(1024,0.947749)
				(1224,0.950843)
				(1406,0.953304)
				(1681,0.955816)
				(1763,0.956468)
				(1935,0.957797)
				(2350,0.960583)
				(2856,0.963082)
				(3080,0.964099)
				(3520,0.965587)
				(3904,0.967018)
				(4864,0.969771)
				(5100,0.969865)
			};

			% PF asa.mean[0] %%%%%%%%%%%%%%%%%%%%%%%%%%%%%%%%%%%%%%%%%%%%%%%%%%%%%%%%%%%%
			\addplot[PF] coordinates{
				(281.19,0.808478)
				(430.376,0.833667)
				(586.371,0.849405)
				(907.19,0.868703)
				(1201.57,0.88108)
				(1354.07,0.885598)
				(1718.72,0.893555)
				(1939.43,0.89815)
				(2251.38,0.902523)
				(2592.38,0.907244)
				(3186.03,0.911467)
				(4180.57,0.918609)
				(6182.51,0.927766)
			};

			% TPS asa.mean[0] %%%%%%%%%%%%%%%%%%%%%%%%%%%%%%%%%%%%%%%%%%%%%%%%%%%%%%%%%%%%
			\addplot[TPS] coordinates{
				(230.419,0.890024)
				(313.739,0.904103)
				(450.436,0.916883)
				(638.008,0.927982)
				(857.559,0.936243)
				(1043.09,0.940608)
				(1317.64,0.945988)
				(1503.8,0.948758)
				(1703.97,0.950977)
				(1873.98,0.952831)
				(2144.93,0.954292)
				(2580.57,0.957648)
				(2894.48,0.959425)
				(3314.47,0.961154)
				(3924.13,0.964456)
				(4380.02,0.96549)
				(5132.34,0.967797)
				(5738.91,0.96871)
			};

			% NC asa.mean[0] %%%%%%%%%%%%%%%%%%%%%%%%%%%%%%%%%%%%%%%%%%%%%%%%%%%%%%%%%%%%
			\addplot[NC] coordinates{
				(439.952,0.927752)
				(1154.47,0.942489)
				(2631.49,0.957972)
				(3874.82,0.963383)
			};

			% VC asa.mean[0] %%%%%%%%%%%%%%%%%%%%%%%%%%%%%%%%%%%%%%%%%%%%%%%%%%%%%%%%%%%%
			\addplot[VC] coordinates{
				(243.744,0.876105)
				(400.291,0.909232)
				(531.672,0.923479)
				(660.84,0.931825)
				(895.915,0.941603)
				(1125.96,0.947215)
				(1348.09,0.950878)
				(1564.34,0.953678)
				(1781.1,0.955934)
				(1995.21,0.957609)
				(2205.86,0.959126)
				(2416.36,0.960544)
				(2820.76,0.962659)
				(3224.25,0.964394)
				(3610.4,0.965825)
				(3988.04,0.966977)
				(4351.59,0.968046)
				(4847.23,0.969299)
				(5262.69,0.970268)
			};

			% PB asa.mean[0] %%%%%%%%%%%%%%%%%%%%%%%%%%%%%%%%%%%%%%%%%%%%%%%%%%%%%%%%%%%%
			\addplot[PB] coordinates{
				(273.248,0.872094)
				(360.188,0.892409)
				(463.193,0.906324)
				(656.962,0.921556)
				(857.526,0.930873)
				(970.915,0.934942)
				(1217.27,0.941024)
				(1387.49,0.944173)
				(1616.5,0.947539)
				(1896.95,0.950562)
				(1896.95,0.950562)
				(2243.04,0.953944)
				(2681.85,0.956787)
				(3316.29,0.96024)
				(3316.29,0.96024)
				(4151.92,0.963548)
				(4151.92,0.963548)
				(5425.24,0.96683)
			};

			% VCCS asa.mean[0] %%%%%%%%%%%%%%%%%%%%%%%%%%%%%%%%%%%%%%%%%%%%%%%%%%%%%%%%%%%%
			\addplot[VCCS] coordinates{
				(564.123,0.896995)
				(600.283,0.900933)
				(648.629,0.906426)
				(755.632,0.913981)
				(836.05,0.918357)
				(1051.64,0.923593)
				(1109.07,0.925824)
				(1179.74,0.927997)
				(1237.23,0.929165)
				(1322.17,0.931458)
				(1400.72,0.932467)
				(1526.22,0.93593)
				(1630.95,0.937695)
				(1756.59,0.939863)
				(1914.09,0.942437)
				(2024.07,0.942875)
				(2258.58,0.946475)
				(2558.8,0.950505)
				(2780.39,0.951675)
				(3044.22,0.953398)
				(3687.7,0.960513)
				(4171.76,0.963205)
				(4255.91,0.960559)
			};

			% PRESLIC asa.mean[0] %%%%%%%%%%%%%%%%%%%%%%%%%%%%%%%%%%%%%%%%%%%%%%%%%%%%%%%%%%%%
			\addplot[PRESLIC] coordinates{
				(189.89,0.888186)
				(388.539,0.917794)
				(589.772,0.929085)
				(798.767,0.936812)
				(920.201,0.940352)
				(1188.65,0.945593)
				(1401.21,0.949107)
				(1612.25,0.951558)
				(1904.93,0.954491)
				(1904.93,0.954491)
				(2282.12,0.957583)
				(2738.74,0.960059)
				(3422.72,0.963469)
				(3422.72,0.963469)
				(4393.15,0.966923)
				(4393.15,0.966923)
				(5724.22,0.970058)
			};

			% W asa.mean[0] %%%%%%%%%%%%%%%%%%%%%%%%%%%%%%%%%%%%%%%%%%%%%%%%%%%%%%%%%%%%
			\addplot[W] coordinates{
				(193.87,0.881217)
				(303.997,0.901373)
				(397.103,0.912058)
				(621.108,0.92663)
				(870.236,0.93537)
				(959.915,0.93791)
				(1234.38,0.943356)
				(1418.74,0.946284)
				(1652.83,0.949212)
				(1957.01,0.952296)
				(1957.01,0.952296)
				(2345.3,0.955271)
				(2867.42,0.958631)
				(3518.42,0.96146)
				(3518.42,0.96146)
				(4497.46,0.965015)
				(4497.46,0.965015)
				(5940.33,0.96843)
			};

			% LSC asa.mean[0] %%%%%%%%%%%%%%%%%%%%%%%%%%%%%%%%%%%%%%%%%%%%%%%%%%%%%%%%%%%%
			\addplot[LSC] coordinates{
				(383.095,0.917744)
				(552.409,0.92894)
				(718.163,0.935805)
				(1059.33,0.944416)
				(1414.95,0.950201)
				(1816.61,0.954774)
				(2009.32,0.956674)
				(2263.42,0.958759)
				(2398.35,0.960144)
				(2446.8,0.96079)
				(2588.34,0.961881)
				(2830.36,0.963581)
				(3264.3,0.965745)
				(3438.52,0.966439)
				(3802.19,0.967805)
				(4059.6,0.968429)
				(4900.83,0.970028)
				(5001.09,0.970394)
			};

			% WP asa.mean[0] %%%%%%%%%%%%%%%%%%%%%%%%%%%%%%%%%%%%%%%%%%%%%%%%%%%%%%%%%%%%
			\addplot[WP] coordinates{
				(204,0.902606)
				(315,0.91771)
				(432,0.926117)
				(638,0.935858)
				(850,0.941746)
				(1064,0.946254)
				(1230,0.948293)
				(1408,0.950477)
				(1645,0.952659)
				(1938,0.954812)
				(1938,0.954896)
				(2296,0.957399)
				(2745,0.959848)
				(3400,0.962258)
				(3400,0.962258)
				(4256,0.964788)
				(4256,0.964788)
				(5568,0.967574)
			};

			% QS asa.mean[0] %%%%%%%%%%%%%%%%%%%%%%%%%%%%%%%%%%%%%%%%%%%%%%%%%%%%%%%%%%%%
			\addplot[QS] coordinates{
				(223.521,0.785765)
				(325.303,0.839673)
				(414.609,0.865393)
				(663.504,0.900756)
				(828.486,0.91261)
				(1077.18,0.923907)
				(1252.44,0.929328)
				(1475.56,0.934356)
				(1767.37,0.939401)
				(2165.34,0.944312)
				(2703.49,0.949045)
				(3460.91,0.953951)
				(4543.63,0.958689)
				(6144.37,0.963437)
			};

			% CIS asa.mean[0] %%%%%%%%%%%%%%%%%%%%%%%%%%%%%%%%%%%%%%%%%%%%%%%%%%%%%%%%%%%%
			\addplot[CIS] coordinates{
				(292.103,0.90514)
				(366.787,0.916739)
				(436.398,0.923615)
				(575.972,0.932828)
				(721.491,0.938457)
				(783.672,0.94041)
				(963.875,0.944524)
				(1083.42,0.946581)
				(1227.51,0.948992)
				(1408.35,0.951003)
				(1424.97,0.951163)
				(1751.64,0.954062)
				(2118.16,0.956964)
				(2717.81,0.960106)
				(3240.8,0.962489)
				(3240.8,0.962489)
				(4824.68,0.966493)
			};

			% RESEEDS3D asa.mean[0] %%%%%%%%%%%%%%%%%%%%%%%%%%%%%%%%%%%%%%%%%%%%%%%%%%%%%%%%%%%%
			\addplot[RESEEDS3D] coordinates{
				(200.035,0.831298)
				(300.105,0.912464)
				(400.128,0.92406)
				(600.233,0.936713)
				(800.351,0.943857)
				(1000.04,0.949509)
				(1200.03,0.952673)
				(1400.65,0.954895)
				(1600.08,0.957645)
				(1792.08,0.954278)
				(1998.07,0.959386)
				(2408.16,0.963197)
				(2801.12,0.964914)
				(3182.14,0.966685)
				(3648.05,0.968018)
				(4257.09,0.970205)
				(4444.11,0.970938)
				(4864.04,0.971567)
			};

			% ERS asa.mean[0] %%%%%%%%%%%%%%%%%%%%%%%%%%%%%%%%%%%%%%%%%%%%%%%%%%%%%%%%%%%%
			\addplot[ERS] coordinates{
				(200,0.915888)
				(300,0.927743)
				(400,0.934561)
				(600,0.942505)
				(800,0.947656)
				(1000,0.951093)
				(1200,0.953923)
				(1400,0.956097)
				(1600,0.958003)
				(1800,0.959568)
				(2000,0.960947)
				(2400,0.963396)
				(2800,0.965346)
				(3200,0.966946)
				(3600,0.968415)
				(4000,0.96962)
				(4600,0.971281)
				(5200,0.972643)
			};

			% DASP asa.mean[0] %%%%%%%%%%%%%%%%%%%%%%%%%%%%%%%%%%%%%%%%%%%%%%%%%%%%%%%%%%%%
			\addplot[DASP] coordinates{
				(417.987,0.911734)
				(521.05,0.925038)
				(620.704,0.933157)
				(812.654,0.941931)
				(1006.72,0.947251)
				(1197.12,0.950966)
				(1388.06,0.953641)
				(1576.93,0.955897)
				(1767.2,0.957697)
				(1947.7,0.959228)
				(2131.65,0.960405)
				(2494.37,0.962693)
				(2860.93,0.964404)
				(3211.97,0.965958)
				(3564.56,0.967179)
				(3909.24,0.968247)
				(4418.09,0.969628)
				(4919.67,0.970772)
			};

			% MSS asa.mean[0] %%%%%%%%%%%%%%%%%%%%%%%%%%%%%%%%%%%%%%%%%%%%%%%%%%%%%%%%%%%%
			\addplot[MSS] coordinates{
				(207.576,0.891131)
				(308.962,0.905765)
				(436.368,0.920153)
				(668.784,0.931024)
				(912.997,0.938063)
				(1055.23,0.941494)
				(1363.6,0.946115)
				(1575.7,0.948536)
				(1862.23,0.951176)
				(2222.47,0.953766)
				(2222.47,0.953766)
				(2666.45,0.956308)
				(3236.21,0.958593)
				(4079.47,0.961699)
				(4078.85,0.961666)
				(5209.16,0.964387)
				(5209.16,0.964387)
				(6994.34,0.967777)
			};

			% ETPS asa.mean[0] %%%%%%%%%%%%%%%%%%%%%%%%%%%%%%%%%%%%%%%%%%%%%%%%%%%%%%%%%%%%
			\addplot[ETPS] coordinates{
				(221,0.904719)
				(315,0.919494)
				(432,0.928317)
				(638,0.938929)
				(850,0.945462)
				(972,0.947897)
				(1230,0.95246)
				(1408,0.954766)
				(1645,0.957127)
				(1938,0.959772)
				(1938,0.959772)
				(2296,0.962146)
				(2745,0.964734)
				(3400,0.967306)
				(3400,0.967306)
				(4256,0.970236)
				(4256,0.970236)
				(5568,0.973162)
			};

			\end{axis}
	\end{tikzpicture}
\end{subfigure}
\begin{subfigure}[b]{0.325\textwidth}\phantomsubcaption\label{subfig:appendix-experiments-nyuv2-ue_levin.mean[0]}
	%%%%%%%%%%%%%%%%%%%%%%%%%%%%%%%%%%%%%%%%%%%%%%%%%%%%%%%%%%%%
	% ue_levin.mean[0]
	%%%%%%%%%%%%%%%%%%%%%%%%%%%%%%%%%%%%%%%%%%%%%%%%%%%%%%%%%%%%
	\begin{tikzpicture}
		\begin{axis}[AENYUV2UELevin,xmode=log]

			% CCS ue_levin.mean[0] %%%%%%%%%%%%%%%%%%%%%%%%%%%%%%%%%%%%%%%%%%%%%%%%%%%%%%%%%%%%
			\addplot[CCS] coordinates{
				(194.01,3.91577)
				(282.682,3.01928)
				(397.098,2.42005)
				(597.153,1.86908)
				(803.471,1.49104)
				(925.103,1.35213)
				(1177.26,1.12952)
				(1397.83,1.00736)
				(1588.43,0.940367)
				(1878.46,0.827513)
				(1878.46,0.827513)
				(2231.81,0.751377)
				(2677.06,0.654786)
				(3328.01,0.565929)
				(3328.01,0.565929)
				(4313.98,0.473263)
				(4313.98,0.473263)
				(5573.99,0.401435)
			};

			% SEEDS ue_levin.mean[0] %%%%%%%%%%%%%%%%%%%%%%%%%%%%%%%%%%%%%%%%%%%%%%%%%%%%%%%%%%%%
			\addplot[SEEDS] coordinates{
				(246.043,6.65855)
				(347.734,4.92645)
				(450.607,3.89629)
				(654.489,2.929)
				(857.654,2.3427)
				(1057.59,1.84325)
				(1258.91,1.62152)
				(1462.49,1.56729)
				(1661.42,1.3442)
				(1870.25,1.525)
				(2066.14,1.23394)
				(2473.07,1.03069)
				(2864.58,0.981529)
				(3245.7,0.851624)
				(3711.17,0.766183)
				(4314.51,0.725848)
				(4499.92,0.650191)
				(4922.37,0.620863)
			};

			% SLIC ue_levin.mean[0] %%%%%%%%%%%%%%%%%%%%%%%%%%%%%%%%%%%%%%%%%%%%%%%%%%%%%%%%%%%%
			\addplot[SLIC] coordinates{
				(184.484,4.80852)
				(279.316,3.47515)
				(385.211,2.792)
				(596.942,2.06215)
				(819.424,1.68097)
				(939.241,1.54048)
				(1211.08,1.30037)
				(1425.96,1.12845)
				(1642.44,1.05623)
				(1940.44,0.915922)
				(1940.44,0.915922)
				(2316.71,0.818338)
				(2774.73,0.718648)
				(3441.91,0.614133)
				(3441.91,0.614133)
				(4395.74,0.522387)
				(4395.74,0.522387)
				(5687.73,0.436733)
			};

			% RW ue_levin.mean[0] %%%%%%%%%%%%%%%%%%%%%%%%%%%%%%%%%%%%%%%%%%%%%%%%%%%%%%%%%%%%
			\addplot[RW] coordinates{
				(215.912,6.46879)
				(306.055,4.66746)
				(440.393,3.5738)
				(635.679,2.68791)
				(846.511,2.26093)
				(1060.11,1.88923)
				(1306.99,1.58432)
				(1464.51,1.48384)
				(1685.01,1.32772)
				(1868.17,1.2438)
				(2109.97,1.11551)
				(3967.51,0.714067)
				(4401.89,0.661793)
				(5110.46,0.600812)
			};

			% CW ue_levin.mean[0] %%%%%%%%%%%%%%%%%%%%%%%%%%%%%%%%%%%%%%%%%%%%%%%%%%%%%%%%%%%%
			\addplot[CW] coordinates{
				(198.614,5.35943)
				(293.173,4.21349)
				(407.073,3.48215)
				(613.509,2.53891)
				(834.84,2.07648)
				(1056.32,1.74168)
				(1266,1.51534)
				(1460.01,1.36166)
				(1700.24,1.24407)
				(1828.56,1.16769)
				(2009.84,1.09775)
				(2449.61,0.946088)
				(2991.17,0.842423)
				(3233.9,0.794448)
				(3707.85,0.737172)
				(4124.98,0.681392)
				(5174.47,0.583505)
				(5281.06,0.594636)
			};

			% TP ue_levin.mean[0] %%%%%%%%%%%%%%%%%%%%%%%%%%%%%%%%%%%%%%%%%%%%%%%%%%%%%%%%%%%%
			\addplot[TP] coordinates{
				(295.99,3.61652)
				(395.489,2.83845)
				(551.496,3.84414)
				(775.682,2.97243)
				(1000.76,2.11368)
				(1130.49,1.91742)
				(1401.47,1.35829)
				(1572.01,1.26513)
				(1815.33,1.08436)
				(2110.15,0.929264)
			};

			% POISE ue_levin.mean[0] %%%%%%%%%%%%%%%%%%%%%%%%%%%%%%%%%%%%%%%%%%%%%%%%%%%%%%%%%%%%
			\addplot[POISE] coordinates{
				(204.942,9.02524)
				(306.83,7.73633)
				(408.614,6.76073)
				(612.1,5.67059)
				(815.501,5.11705)
				(1019.01,4.89865)
				(1222.5,4.63891)
				(1425.91,4.39734)
				(1628.65,4.20217)
				(1830.61,4.1137)
				(2031.68,3.84279)
				(2427.44,3.80618)
				(2807.4,3.5683)
				(3158.64,3.49253)
				(3466.96,3.45516)
				(3736.5,3.41242)
				(4042.51,3.39378)
				(4219.03,3.38776)
			};

			% FH ue_levin.mean[0] %%%%%%%%%%%%%%%%%%%%%%%%%%%%%%%%%%%%%%%%%%%%%%%%%%%%%%%%%%%%
			\addplot[FH] coordinates{
				(689.802,3.91423)
				(763.792,3.60314)
				(813.815,3.45314)
				(988.712,2.84132)
				(1077.66,2.61526)
				(1359.76,2.08112)
				(1923.5,1.57667)
				(2699.96,1.21077)
				(1559.12,2.13123)
				(1408.18,2.31481)
				(3024.27,1.31065)
				(2206.32,1.62264)
				(2448.19,1.4234)
				(4594.6,0.840397)
				(3568.29,1.24339)
				(4199.72,1.01641)
			};

			% EAMS ue_levin.mean[0] %%%%%%%%%%%%%%%%%%%%%%%%%%%%%%%%%%%%%%%%%%%%%%%%%%%%%%%%%%%%
			\addplot[EAMS] coordinates{
				(419.068,3.20696)
				(455.657,3.02446)
				(499.416,2.8631)
				(641.188,3.1362)
				(842.328,2.62355)
				(1268.71,2.19326)
				(2584.6,1.56583)
				(2683.21,1.5364)
				(2779.09,1.54757)
				(2997.7,1.50371)
				(3180.29,1.51676)
				(3421.45,1.56792)
				(3646.16,1.52457)
				(3943.04,1.55557)
				(4448.4,1.5028)
				(5663.91,1.42054)
				(9098.39,1.27068)
			};

			% CRS ue_levin.mean[0] %%%%%%%%%%%%%%%%%%%%%%%%%%%%%%%%%%%%%%%%%%%%%%%%%%%%%%%%%%%%
			\addplot[CRS] coordinates{
				(254.263,3.83242)
				(396.694,2.66787)
				(514.895,2.2693)
				(764.997,1.69703)
				(988.589,1.3989)
				(1232.03,1.1811)
				(1498.25,1.02174)
				(1707.65,0.933716)
				(1968.19,0.843616)
				(2099.81,0.806428)
				(2299.23,0.765823)
				(2706.1,0.688991)
				(3259.07,0.604899)
				(3558.75,0.571249)
				(4055.02,0.526816)
				(4394.15,0.496337)
				(5418.11,0.434619)
				(5695.52,0.416573)
			};

			% SEAW ue_levin.mean[0] %%%%%%%%%%%%%%%%%%%%%%%%%%%%%%%%%%%%%%%%%%%%%%%%%%%%%%%%%%%%
			\addplot[SEAW] coordinates{
				(124.667,10.0064)
				(349.414,3.47617)
				(1192.36,1.32196)
				(4497.65,0.519633)
			};

			% RESEEDS ue_levin.mean[0] %%%%%%%%%%%%%%%%%%%%%%%%%%%%%%%%%%%%%%%%%%%%%%%%%%%%%%%%%%%%
			\addplot[RESEEDS] coordinates{
				(200.441,5.63491)
				(300.263,4.37898)
				(400.323,3.38812)
				(600.361,2.48277)
				(800.526,1.96004)
				(1000.06,1.67527)
				(1200.06,1.48912)
				(1400.96,1.30038)
				(1600.21,1.18857)
				(1792.08,1.27455)
				(1998.13,1.05969)
				(2408.17,0.892227)
				(2801.44,0.776052)
				(3182.16,0.733669)
				(3648.12,0.690672)
				(4257.44,0.578974)
				(4444.16,0.581865)
				(4864.15,0.549836)
			};

			% ERGC ue_levin.mean[0] %%%%%%%%%%%%%%%%%%%%%%%%%%%%%%%%%%%%%%%%%%%%%%%%%%%%%%%%%%%%
			\addplot[ERGC] coordinates{
				(196,4.99977)
				(289,3.78614)
				(400,3.06871)
				(600,2.2531)
				(812,1.81499)
				(1024,1.54439)
				(1224,1.37267)
				(1406,1.24366)
				(1681,1.06877)
				(1763,1.04674)
				(1935,0.985752)
				(2350,0.857052)
				(2856,0.74604)
				(3080,0.708492)
				(3520,0.648779)
				(3904,0.598898)
				(4864,0.510812)
				(5100,0.501663)
			};

			% PF ue_levin.mean[0] %%%%%%%%%%%%%%%%%%%%%%%%%%%%%%%%%%%%%%%%%%%%%%%%%%%%%%%%%%%%
			\addplot[PF] coordinates{
				(281.19,9.55009)
				(430.376,6.62426)
				(586.371,5.58109)
				(907.19,4.27204)
				(1201.57,3.63909)
				(1354.07,3.41611)
				(1718.72,3.00166)
				(1939.43,2.74764)
				(2251.38,2.57113)
				(2592.38,2.34659)
				(3186.03,2.19107)
				(4180.57,1.90313)
				(6182.51,1.6012)
			};

			% TPS ue_levin.mean[0] %%%%%%%%%%%%%%%%%%%%%%%%%%%%%%%%%%%%%%%%%%%%%%%%%%%%%%%%%%%%
			\addplot[TPS] coordinates{
				(230.419,4.14703)
				(313.739,3.33587)
				(450.436,2.57594)
				(638.008,2.07323)
				(857.559,1.6731)
				(1043.09,1.48472)
				(1317.64,1.24027)
				(1503.8,1.12959)
				(1703.97,1.00992)
				(1873.98,0.961111)
				(2144.93,0.907534)
				(2580.57,0.778027)
				(2894.48,0.725163)
				(3314.47,0.674171)
				(3924.13,0.534619)
				(4380.02,0.507184)
				(5132.34,0.468456)
				(5738.91,0.439113)
			};

			% NC ue_levin.mean[0] %%%%%%%%%%%%%%%%%%%%%%%%%%%%%%%%%%%%%%%%%%%%%%%%%%%%%%%%%%%%
			\addplot[NC] coordinates{
				(439.952,2.8756)
				(1154.47,1.58079)
				(2631.49,0.682427)
				(3874.82,0.543358)
			};

			% VC ue_levin.mean[0] %%%%%%%%%%%%%%%%%%%%%%%%%%%%%%%%%%%%%%%%%%%%%%%%%%%%%%%%%%%%
			\addplot[VC] coordinates{
				(243.744,8.15606)
				(400.291,4.73834)
				(531.672,3.46425)
				(660.84,2.86684)
				(895.915,2.11626)
				(1125.96,1.6612)
				(1348.09,1.41495)
				(1564.34,1.24415)
				(1781.1,1.11751)
				(1995.21,1.0473)
				(2205.86,0.975182)
				(2416.36,0.891759)
				(2820.76,0.796538)
				(3224.25,0.724801)
				(3610.4,0.66591)
				(3988.04,0.602591)
				(4351.59,0.583042)
				(4847.23,0.526601)
				(5262.69,0.500457)
			};

			% PB ue_levin.mean[0] %%%%%%%%%%%%%%%%%%%%%%%%%%%%%%%%%%%%%%%%%%%%%%%%%%%%%%%%%%%%
			\addplot[PB] coordinates{
				(273.248,4.89497)
				(360.188,3.46834)
				(463.193,2.78699)
				(656.962,2.09335)
				(857.526,1.71613)
				(970.915,1.57476)
				(1217.27,1.33194)
				(1387.49,1.22829)
				(1616.5,1.1155)
				(1896.95,0.983964)
				(1896.95,0.983964)
				(2243.04,0.869225)
				(2681.85,0.787094)
				(3316.29,0.676468)
				(3316.29,0.676468)
				(4151.92,0.595118)
				(4151.92,0.595118)
				(5425.24,0.506506)
			};

			% VCCS ue_levin.mean[0] %%%%%%%%%%%%%%%%%%%%%%%%%%%%%%%%%%%%%%%%%%%%%%%%%%%%%%%%%%%%
			\addplot[VCCS] coordinates{
				(564.123,6.27629)
				(600.283,6.42428)
				(648.629,5.9934)
				(755.632,5.49606)
				(836.05,5.19999)
				(1051.64,4.8774)
				(1109.07,4.74258)
				(1179.74,4.5299)
				(1237.23,4.56638)
				(1322.17,4.43126)
				(1400.72,4.49467)
				(1526.22,4.13389)
				(1630.95,4.00047)
				(1756.59,4.09164)
				(1914.09,3.62321)
				(2024.07,3.86301)
				(2258.58,3.67744)
				(2558.8,3.28211)
				(2780.39,3.09585)
				(3044.22,2.67908)
				(3687.7,1.44625)
				(4171.76,1.06265)
				(4255.91,1.78806)
			};

			% PRESLIC ue_levin.mean[0] %%%%%%%%%%%%%%%%%%%%%%%%%%%%%%%%%%%%%%%%%%%%%%%%%%%%%%%%%%%%
			\addplot[PRESLIC] coordinates{
				(189.89,4.69806)
				(388.539,2.87914)
				(589.772,2.12998)
				(798.767,1.75619)
				(920.201,1.55215)
				(1188.65,1.37135)
				(1401.21,1.16994)
				(1612.25,1.07261)
				(1904.93,0.939768)
				(1904.93,0.939768)
				(2282.12,0.835363)
				(2738.74,0.732645)
				(3422.72,0.65185)
				(3422.72,0.65185)
				(4393.15,0.540121)
				(4393.15,0.540121)
				(5724.22,0.45814)
			};

			% W ue_levin.mean[0] %%%%%%%%%%%%%%%%%%%%%%%%%%%%%%%%%%%%%%%%%%%%%%%%%%%%%%%%%%%%
			\addplot[W] coordinates{
				(193.87,7.86)
				(303.997,5.57358)
				(397.103,4.52642)
				(621.108,3.16498)
				(870.236,2.42727)
				(959.915,2.27403)
				(1234.38,1.90284)
				(1418.74,1.68153)
				(1652.83,1.49357)
				(1957.01,1.3271)
				(1957.01,1.3271)
				(2345.3,1.11753)
				(2867.42,0.97244)
				(3518.42,0.833628)
				(3518.42,0.833628)
				(4497.46,0.687001)
				(4497.46,0.687001)
				(5940.33,0.564241)
			};

			% LSC ue_levin.mean[0] %%%%%%%%%%%%%%%%%%%%%%%%%%%%%%%%%%%%%%%%%%%%%%%%%%%%%%%%%%%%
			\addplot[LSC] coordinates{
				(383.095,3.82477)
				(552.409,2.95569)
				(718.163,2.35824)
				(1059.33,1.81567)
				(1414.95,1.43041)
				(1816.61,1.16397)
				(2009.32,1.06431)
				(2263.42,0.951139)
				(2398.35,0.877511)
				(2446.8,0.852917)
				(2588.34,0.818619)
				(2830.36,0.737547)
				(3264.3,0.63571)
				(3438.52,0.607693)
				(3802.19,0.549299)
				(4059.6,0.528703)
				(4900.83,0.475641)
				(5001.09,0.467495)
			};

			% WP ue_levin.mean[0] %%%%%%%%%%%%%%%%%%%%%%%%%%%%%%%%%%%%%%%%%%%%%%%%%%%%%%%%%%%%
			\addplot[WP] coordinates{
				(204,5.08379)
				(315,3.66777)
				(432,2.91933)
				(638,2.18799)
				(850,1.84674)
				(1064,1.5707)
				(1230,1.46376)
				(1408,1.309)
				(1645,1.17195)
				(1938,1.06859)
				(1938,1.05927)
				(2296,0.919075)
				(2745,0.813658)
				(3400,0.71392)
				(3400,0.71392)
				(4256,0.623039)
				(4256,0.623039)
				(5568,0.529035)
			};

			% QS ue_levin.mean[0] %%%%%%%%%%%%%%%%%%%%%%%%%%%%%%%%%%%%%%%%%%%%%%%%%%%%%%%%%%%%
			\addplot[QS] coordinates{
				(223.521,12.8841)
				(325.303,8.67246)
				(414.609,6.83794)
				(663.504,4.33612)
				(828.486,3.63032)
				(1077.18,2.92314)
				(1252.44,2.6209)
				(1475.56,2.33781)
				(1767.37,2.05483)
				(2165.34,1.77673)
				(2703.49,1.50095)
				(3460.91,1.25313)
				(4543.63,1.00328)
				(6144.37,0.796871)
			};

			% CIS ue_levin.mean[0] %%%%%%%%%%%%%%%%%%%%%%%%%%%%%%%%%%%%%%%%%%%%%%%%%%%%%%%%%%%%
			\addplot[CIS] coordinates{
				(292.103,4.28099)
				(366.787,3.40926)
				(436.398,2.79584)
				(575.972,2.21606)
				(721.491,1.80131)
				(783.672,1.72565)
				(963.875,1.53867)
				(1083.42,1.377)
				(1227.51,1.24299)
				(1408.35,1.11434)
				(1424.97,1.10698)
				(1751.64,1.00368)
				(2118.16,0.862681)
				(2717.81,0.736146)
				(3240.8,0.63364)
				(3240.8,0.63364)
				(4824.68,0.529905)
			};

			% RESEEDS3D ue_levin.mean[0] %%%%%%%%%%%%%%%%%%%%%%%%%%%%%%%%%%%%%%%%%%%%%%%%%%%%%%%%%%%%
			\addplot[RESEEDS3D] coordinates{
				(200.035,17.1184)
				(300.105,3.64643)
				(400.128,2.97001)
				(600.233,2.1759)
				(800.351,1.74414)
				(1000.04,1.51794)
				(1200.03,1.3357)
				(1400.65,1.19468)
				(1600.08,1.09294)
				(1792.08,1.13891)
				(1998.07,0.970465)
				(2408.16,0.825425)
				(2801.12,0.736224)
				(3182.14,0.687863)
				(3648.05,0.626073)
				(4257.09,0.543273)
				(4444.11,0.546446)
				(4864.04,0.512544)
			};

			% ERS ue_levin.mean[0] %%%%%%%%%%%%%%%%%%%%%%%%%%%%%%%%%%%%%%%%%%%%%%%%%%%%%%%%%%%%
			\addplot[ERS] coordinates{
				(200,3.78557)
				(300,2.82683)
				(400,2.30971)
				(600,1.7321)
				(800,1.40501)
				(1000,1.18794)
				(1200,1.05637)
				(1400,0.953011)
				(1600,0.871952)
				(1800,0.804297)
				(2000,0.752113)
				(2400,0.660186)
				(2800,0.596871)
				(3200,0.549367)
				(3600,0.509004)
				(4000,0.472602)
				(4600,0.433166)
				(5200,0.39837)
			};

			% DASP ue_levin.mean[0] %%%%%%%%%%%%%%%%%%%%%%%%%%%%%%%%%%%%%%%%%%%%%%%%%%%%%%%%%%%%
			\addplot[DASP] coordinates{
				(417.987,3.7766)
				(521.05,2.94016)
				(620.704,2.42638)
				(812.654,1.80684)
				(1006.72,1.55271)
				(1197.12,1.35137)
				(1388.06,1.19995)
				(1576.93,1.09833)
				(1767.2,1.02014)
				(1947.7,0.962488)
				(2131.65,0.916439)
				(2494.37,0.828212)
				(2860.93,0.693708)
				(3211.97,0.708175)
				(3564.56,0.605335)
				(3909.24,0.571036)
				(4418.09,0.5307)
				(4919.67,0.51381)
			};

			% MSS ue_levin.mean[0] %%%%%%%%%%%%%%%%%%%%%%%%%%%%%%%%%%%%%%%%%%%%%%%%%%%%%%%%%%%%
			\addplot[MSS] coordinates{
				(207.576,6.08191)
				(308.962,4.88333)
				(436.368,3.70442)
				(668.784,2.67622)
				(912.997,2.10484)
				(1055.23,1.89752)
				(1363.6,1.61152)
				(1575.7,1.46942)
				(1862.23,1.28416)
				(2222.47,1.13879)
				(2222.47,1.13879)
				(2666.45,0.990103)
				(3236.21,0.856208)
				(4079.47,0.730288)
				(4078.85,0.730166)
				(5209.16,0.610699)
				(5209.16,0.610699)
				(6994.34,0.50518)
			};

			% ETPS ue_levin.mean[0] %%%%%%%%%%%%%%%%%%%%%%%%%%%%%%%%%%%%%%%%%%%%%%%%%%%%%%%%%%%%
			\addplot[ETPS] coordinates{
				(221,4.89921)
				(315,3.67723)
				(432,3.01505)
				(638,2.24195)
				(850,1.8758)
				(972,1.65996)
				(1230,1.42214)
				(1408,1.28811)
				(1645,1.14344)
				(1938,1.02182)
				(1938,1.02182)
				(2296,0.895739)
				(2745,0.786985)
				(3400,0.67195)
				(3400,0.67195)
				(4256,0.573473)
				(4256,0.573473)
				(5568,0.47969)
			};

			\end{axis}
	\end{tikzpicture}
\end{subfigure}
\begin{subfigure}[b]{0.325\textwidth}
\end{subfigure}

	\caption{\UE, \ASA and \UEL on the \BSDS and \NYU datasets. We find that \ASA does not
	provide new insights compared to \UE, as it closely reflects $(1 - \UE)$ except
	for a minor absolute offset. \UEL, in contrast, provides a different point view
	compared to \UE. However, \UEL is harder to interpret and strongly varies across datasets.
	\textbf{Best viewed in color.}}
	\label{fig:appendix-experiments-bsds500-nyuv2}
	\vskip 12px
	\input{legends/full+depth}
\end{figure*}

The following experiments complement the discussion in Section \ref{subsec:experiments-quantitative}
in two regards. First, we present additional experiments considering both \ASA and
\UEL on the \BSDS and \NYU datasets. Then, we consider \Rec, \UE and \EV in more details for
the remaining datasets, \ie the \SBD, \SUNRGBD and \Fash datasets. We begin by discussing \ASA
and \UEL, also in regard to the observations made in Sections \ref{subsec:benchmark-correlation} and \ref{sec:appendix-benchmark}.

As observed on the \BSDS and \NYU datasets in Section \ref{subsec:experiments-quantitative},
\Rec and \UE can be used to roughly asses superpixel algorithms based on ground
truth. However, for large \K, these metrics are not necessarily
sufficient to discriminate between the superpixel algorithms. Considering Figure
\ref{fig:appendix-experiments-sbd-sunrgbd-fash}, in particular with regard to
\Rec, we can identify algorithms showing above-average performance such as \ETPS
and \SEEDS. These algorithms perform well on all three datasets. Similarly,
\PF, \QS, \SEAW and \TPS perform poorly on all three datasets. Regarding \UE,
in contrast, top-performer across all three algorithms are not identified as easily.
For example, \POISE demonstrates low \UE on the \SBD and \Fash datasets, while performing
poorly on the \SUNRGBD dataset. Similarly, \ERS shows excellent performance on the \SUNRGBD dataset,
while being outperformed by \POISE as well as \ETPS on the \SBD and \Fash datasets.
Overall, \Rec and \UE do not necessarily give a consistent
view on the performance of the superpixel algorithms across datasets. This may also
be explained by the ground truth quality as already discussed in Section \ref{subsec:experiments-quantitative}.

The above observations also justify the use of \EV to judge superpixel algorithms
independent of ground truth. Considering Figure \ref{fig:appendix-experiments-sbd-sunrgbd-fash},
in particular, with regard to \EV, we can observe a more consistent view across the datasets.
Both, top-performing algorithms such as \ETPS and \SEEDS, as well as poorly performing
algorithms such as \PF, \PB or \TPS can easily be identified. In between these two extremes,
superpixel algorithms are easier to discriminate compared to \Rec and \UE. Furthermore,
some superpixel algorithms such as \QS, \FH or \CIS are performing better compared to
\Rec or \UE. This confirms the observations that ground truth independent assessment
is beneficial but cannot replace \Rec or \UE.

We find that \ASA closely mimicks the behavior of $(1 - \UE)$ while \UEL may
complement our discussion with an additional viewpoint which is, however, hard to interpret.
We consider Figure \ref{fig:appendix-experiments-bsds500-nyuv2}
showing \UE, \ASA and \UEL for both the \BSDS and \NYU datasets. Focussing on \UE and
\ASA, we easily see that \ASA nearly reflects $(1 - \UE)$ while being a small
constant off. In particular, all algorithms exhibit nearly the same behavior,
while absolutely the algorithms show higher \ASA
compared to $(1 - \UE)$. This demonstrates that \ASA does not give new insights
with respect to the quantitative comparison of superpixel algorithms. In contrast,
the algorithms show different behavior considering \UEL. This is mainly due to the
unconstrained range of \UEL (compared to $\UE \in [0,1]$). In particular,
for algorithms such as \EAMS and \FH, \UEL reflects the behavior of $\max\UE$
as shown in Figure \ref{subfig:experiments-quantitative-bsds500-ue_np.max_max}.
The remaining algorithms lie more closely together.
Still, algorithms such as \ERS, \SEEDS or \PB show better \UEL than \UE (seen relatively to the remaining algorithms).
In the case of \EAMS and \FH, high \UEL may indeed
be explained by the considerations of Neubert and Protzel \cite{NeubertProtzel:2012}
arguing that \UEL unjustly penalizes large superpixels slightly overlapping with multiple ground truth segments.
For the remaining algorithms,
the same argument can only be applied in smaller scale as these algorithms usually
do not generate large superpixels. In this line of throught, the excellent
performance of \ERS may be explained by the employed regularizer for enforcing
uniform superpixel size. Overall, \ASA does not contribute to an insightful discussion,
while \UEL may be considered in addition to \UE to complete the picture of algorithm performance.
