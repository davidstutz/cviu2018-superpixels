\subsection{Runtime}
\label{subsec:appendix-experiments-robustness}

We briefly discuss runtime on the \SBD, \SUNRGBD and \Fash datasets allowing to get
more insights on how the algorithms scale with respect to image size and the
number of generated superpixels.

We find that the runtime of most algorithms scales roughly linear in the input size, while the
number of generated superpixels has little influence. We first remember that
the average image size of the \SBD, \SUNRGBD and \Fash datasets is: $314 \times 242 = 75988$,
$660 \times 488 = 322080$ and $400 \times 600 = 240000$. For $\K \approx 400$, \W
runs in roughly $1.9\text{ms}$ and $7.9\text{ms}$ on the \SBD and \SUNRGBD datasets, respectively.
As the input size for the \SUNRGBD dataset is roughly $4.24$ times larger compared to the
\SBD dataset, this results in roughly linear scaling of runtime with respect to the input size.
Similar reasoning can be applied to most of the remaining algorithms, especially
fast algorithms such as \CW, \PF, \preSLIC, \MSS or \SLIC. Except for \RW, \QS and \SEAW
we also notice that the number of generated superpixels does not influence runtime significantly.
Overall, the results confirm the claim of many authors that algorithms scale
linear in the input size, while the number of generated superpixels has little influence.
